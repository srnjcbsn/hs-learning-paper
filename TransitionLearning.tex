\documentclass[Master.tex]{subfiles}
\begin{document}

<\texttt{TODO:} rewrite based on \cite{ghallab2004a}>

STRIPS(Standford instritute research problem solver) is schema used
to define a domain and a problem.

STRIPS contain four components

\begin{itemize}
    \item An initial state $I$
    \item A set of conditions P which contain all possible predicates
    \item A goal state $G$
    \item A set $O$ of all available actionschemas
\end{itemize}

Actionschemas in STRIPS are defined as $\text{\ensuremath{\left(P^{+},P^{-},E^{+},E^{-}\right)}}$.
Futhermore, an actionschema also contain a list of variable $(x_{1},x_{2},\dots,x_{n-1},x_{n})$
which are the input parameters for an action. The preconditions define
when an actions is invalid meaning it can not be used, and the postconditions
define what predicates are present/not present in the state after
it has been applied. Predicates in the pre/post- conditins of an actionschema
are ungrounded meaning the object which they refer to are variables
and can change dependent on the input to an action.

To apply an action to a state all predicates in an action must be
grounded. To ground a predicate $p(x_{1},\dots,x_{n})$ a mapping
$M$ from the variables to the real objects in the environment is
required, for instance $\{x_{1}\mapsto red,x_{2}\mapsto blue, \dots\}$.
The mapping is then applied to the variables of the predicate, after
which the predicate is considered grounded. Thus we define the grouding
as a function $g:p(x_{1},\dots,x_{n})\mapsto p\left(M\left(x_{1}\right),,\dots,M\left(x_{n}\right)\right)$.

After predicates in an action has been grounded the preconditions
are checked if they hold for the current state $s$., \\
\[
g\left[P^{+}\right]\subseteq s\,\land\, g\left[P^{-}\right]\cap s=\emptyset
\]
.

If they hold then the postconditions can be applied to the state to
form the new state $s'$.

\[
s'=\left(s\setminus g\left[E^{-}\right]\right)\cup g\left[E^{+}\right]
\]


grounded predicates that are identical and reside in both $E^{+}$
and $E^{-}$ are called spurious effects (see \cite{Russell}), if they are ignored the
then because of the definition of $s'$ then the positive effects
in $E^{+}$will always take precedence.

Thus a state transition is defined as $(s,a,s')$ where $s$ is the
current state, $a$ is the grounded action and $s'$ is the new state.
Lastly it is important to note that all actions in STRIPS are absolute,
meaning they contain no conditional effects.

\begin{enumerate}
    \item[R1] An actions preconditions and effects are conjunctions of predicates.
    \item[R2] Any predicates occurring in the preconditions or effects of an action schema can only reference literals from the action schema's parameter list.
\end{enumerate}

% \section{Sokoban example}
% \subfile{TransitionLearning/SokobanExample}
%
% \section{Learning Effects}
% \subfile{TransitionLearning/Effects}
%
% \section{Learning Preconditions}
% \subfile{TransitionLearning/Preconditions}
%
% \section{Learning Conditional Effects}
% \subfile{TransitionLearning/Conditionals}

\end{document}
