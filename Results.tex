\documentclass[Master.tex]{subfiles}
\providecommand{\master}{.}
\begin{document}

When the scientific learning algorithm is applied through the main program, a statistics file is produced. This file contains, in order:
\begin{itemize}
    \item A header containing the name of each action schema in the domain, along with the action's $\mathbb{F}_A$.
    \item The string \texttt{``- - RUNNING - -''}, signifying the end of the header.
    \item For each experiment (plan) the agent conducted:
        \begin{itemize}
            \item The action which did not produce the expected outcome when applied (as explained in Section~\ref{sec:PDDLAlgo})
            \item The amount of problems the agent had solved before execution of this action.
            \item Eight integers, signifying how many predicates the agent has proven to be positive and negative effects and preconditions for the action, and how many are neither proven nor disproven to be positive and negative effects and preconditions for the action. Lastly, the number of candidate sets for the action is listed.
        \end{itemize}
\end{itemize}

In the following, we will show the results of applying the optimistic non-conditional learning algorithm to the sokoban domain described previously. In order to ease the agent's understanding of the  six actions (see Section~\ref{sec:SokobanPDDL}), we have provided two simple sokoban problems, depicted in Figures~\ref{fig:results:train1} and~\ref{fig:results:train2}. 

These problems serve as a training ground for the agent; in order to solve the first problem, the agent must utilize all three actions that operate on the horizontal axis, and to solve the second it must use the ones operating on the vertical axis. It then reuses the knowledge obtained from the training problems to solve the one in Figure~\ref{fig:results:train3}, which requires use of at least five of the six sokoban actions (it can be solved without use of $\texttt{move-h}$ or without use of $\texttt{move-v}$, but not without both).

From inspecting the generated statistics file, it can be seen that the agent conducts a total of 1793 experiments (i.e.\ failed plans) in order to solve the the large problem without any prior knowledge.

In contrast, solving the two training problems followed by the large one requires a total of 710 experiments (516 for the first problem, 151 for the second, and 43 for the third).

\begin{figure}
    \begin{subfigure}{0.3\textwidth}
        \resizebox{\linewidth}{!}{\begin{pgfpicture}
  \pgfpathrectangle{\pgfpointorigin}{\pgfqpoint{200.0000bp}{200.0000bp}}
  \pgfusepath{use as bounding box}
  \begin{pgfscope}
    \definecolor{fc}{rgb}{0.0000,0.0000,0.0000}
    \pgfsetfillcolor{fc}
    \pgfsetfillopacity{0.0000}
    \pgfsetlinewidth{2.0000bp}
    \definecolor{sc}{rgb}{0.0000,0.0000,0.0000}
    \pgfsetstrokecolor{sc}
    \pgfsetmiterjoin
    \pgfsetbuttcap
    \pgfpathqmoveto{192.0000bp}{100.0000bp}
    \pgfpathqcurveto{192.0000bp}{106.6274bp}{186.6274bp}{112.0000bp}{180.0000bp}{112.0000bp}
    \pgfpathqcurveto{173.3726bp}{112.0000bp}{168.0000bp}{106.6274bp}{168.0000bp}{100.0000bp}
    \pgfpathqcurveto{168.0000bp}{93.3726bp}{173.3726bp}{88.0000bp}{180.0000bp}{88.0000bp}
    \pgfpathqcurveto{186.6274bp}{88.0000bp}{192.0000bp}{93.3726bp}{192.0000bp}{100.0000bp}
    \pgfpathclose
    \pgfusepathqfillstroke
  \end{pgfscope}
  \begin{pgfscope}
    \definecolor{fc}{rgb}{0.0000,0.0000,0.0000}
    \pgfsetfillcolor{fc}
    \pgfsetfillopacity{0.0000}
    \pgfsetlinewidth{2.0000bp}
    \definecolor{sc}{rgb}{0.0000,0.0000,0.0000}
    \pgfsetstrokecolor{sc}
    \pgfsetmiterjoin
    \pgfsetbuttcap
    \pgfpathqmoveto{200.0000bp}{80.0000bp}
    \pgfpathqlineto{200.0000bp}{120.0000bp}
    \pgfpathqlineto{160.0000bp}{120.0000bp}
    \pgfpathqlineto{160.0000bp}{80.0000bp}
    \pgfpathqlineto{200.0000bp}{80.0000bp}
    \pgfpathclose
    \pgfusepathqfillstroke
  \end{pgfscope}
  \begin{pgfscope}
    \definecolor{fc}{rgb}{0.0000,0.0000,0.0000}
    \pgfsetfillcolor{fc}
    \pgfsetfillopacity{0.0000}
    \pgfsetlinewidth{2.0000bp}
    \definecolor{sc}{rgb}{0.0000,0.0000,0.0000}
    \pgfsetstrokecolor{sc}
    \pgfsetmiterjoin
    \pgfsetbuttcap
    \pgfpathqmoveto{160.0000bp}{80.0000bp}
    \pgfpathqlineto{160.0000bp}{120.0000bp}
    \pgfpathqlineto{120.0000bp}{120.0000bp}
    \pgfpathqlineto{120.0000bp}{80.0000bp}
    \pgfpathqlineto{160.0000bp}{80.0000bp}
    \pgfpathclose
    \pgfusepathqfillstroke
  \end{pgfscope}
  \begin{pgfscope}
    \definecolor{fc}{rgb}{0.0000,0.0000,0.0000}
    \pgfsetfillcolor{fc}
    \pgftransformshift{\pgfqpoint{100.0000bp}{100.0000bp}}
    \pgftransformscale{1.0000}
    \pgftext[base,left]{$c$}
  \end{pgfscope}
  \begin{pgfscope}
    \definecolor{fc}{rgb}{0.0000,0.0000,0.0000}
    \pgfsetfillcolor{fc}
    \pgfsetfillopacity{0.0000}
    \pgfsetlinewidth{2.0000bp}
    \definecolor{sc}{rgb}{0.0000,0.0000,0.0000}
    \pgfsetstrokecolor{sc}
    \pgfsetmiterjoin
    \pgfsetbuttcap
    \pgfpathqmoveto{112.0000bp}{88.0000bp}
    \pgfpathqlineto{112.0000bp}{112.0000bp}
    \pgfpathqlineto{88.0000bp}{112.0000bp}
    \pgfpathqlineto{88.0000bp}{88.0000bp}
    \pgfpathqlineto{112.0000bp}{88.0000bp}
    \pgfpathclose
    \pgfusepathqfillstroke
  \end{pgfscope}
  \begin{pgfscope}
    \definecolor{fc}{rgb}{0.0000,0.0000,0.0000}
    \pgfsetfillcolor{fc}
    \pgfsetfillopacity{0.0000}
    \pgfsetlinewidth{2.0000bp}
    \definecolor{sc}{rgb}{0.0000,0.0000,0.0000}
    \pgfsetstrokecolor{sc}
    \pgfsetmiterjoin
    \pgfsetbuttcap
    \pgfpathqmoveto{120.0000bp}{80.0000bp}
    \pgfpathqlineto{120.0000bp}{120.0000bp}
    \pgfpathqlineto{80.0000bp}{120.0000bp}
    \pgfpathqlineto{80.0000bp}{80.0000bp}
    \pgfpathqlineto{120.0000bp}{80.0000bp}
    \pgfpathclose
    \pgfusepathqfillstroke
  \end{pgfscope}
  \begin{pgfscope}
    \definecolor{fc}{rgb}{0.0000,0.0000,0.0000}
    \pgfsetfillcolor{fc}
    \pgfsetfillopacity{0.0000}
    \pgfsetlinewidth{2.0000bp}
    \definecolor{sc}{rgb}{0.0000,0.0000,0.0000}
    \pgfsetstrokecolor{sc}
    \pgfsetmiterjoin
    \pgfsetbuttcap
    \pgfpathqmoveto{80.0000bp}{80.0000bp}
    \pgfpathqlineto{80.0000bp}{120.0000bp}
    \pgfpathqlineto{40.0000bp}{120.0000bp}
    \pgfpathqlineto{40.0000bp}{80.0000bp}
    \pgfpathqlineto{80.0000bp}{80.0000bp}
    \pgfpathclose
    \pgfusepathqfillstroke
  \end{pgfscope}
  \begin{pgfscope}
    \definecolor{fc}{rgb}{0.0000,0.0000,0.0000}
    \pgfsetfillcolor{fc}
    \pgfsetfillopacity{0.0000}
    \pgfsetlinewidth{2.0000bp}
    \definecolor{sc}{rgb}{0.0000,0.0000,0.0000}
    \pgfsetstrokecolor{sc}
    \pgfsetmiterjoin
    \pgfsetbuttcap
    \pgfsetdash{{2.6833bp}{2.6833bp}}{0.0000bp}
    \pgfpathqmoveto{34.0000bp}{86.0000bp}
    \pgfpathqlineto{34.0000bp}{114.0000bp}
    \pgfpathqlineto{6.0000bp}{114.0000bp}
    \pgfpathqlineto{6.0000bp}{86.0000bp}
    \pgfpathqlineto{34.0000bp}{86.0000bp}
    \pgfpathclose
    \pgfusepathqfillstroke
  \end{pgfscope}
  \begin{pgfscope}
    \definecolor{fc}{rgb}{0.0000,0.0000,0.0000}
    \pgfsetfillcolor{fc}
    \pgfsetfillopacity{0.0000}
    \pgfsetlinewidth{2.0000bp}
    \definecolor{sc}{rgb}{0.0000,0.0000,0.0000}
    \pgfsetstrokecolor{sc}
    \pgfsetmiterjoin
    \pgfsetbuttcap
    \pgfpathqmoveto{40.0000bp}{80.0000bp}
    \pgfpathqlineto{40.0000bp}{120.0000bp}
    \pgfpathqlineto{-0.0000bp}{120.0000bp}
    \pgfpathqlineto{-0.0000bp}{80.0000bp}
    \pgfpathqlineto{40.0000bp}{80.0000bp}
    \pgfpathclose
    \pgfusepathqfillstroke
  \end{pgfscope}
\end{pgfpicture}
}
        \caption{Horizontal training problem for the sokoban agent.}\label{fig:results:train1}
    \end{subfigure}
    \begin{subfigure}{0.3\textwidth}
        \resizebox{\linewidth}{!}{\begin{pgfpicture}
  \pgfpathrectangle{\pgfpointorigin}{\pgfqpoint{200.0000bp}{200.0000bp}}
  \pgfusepath{use as bounding box}
  \begin{pgfscope}
    \definecolor{fc}{rgb}{0.0000,0.0000,0.0000}
    \pgfsetfillcolor{fc}
    \pgfsetfillopacity{0.0000}
    \pgfsetlinewidth{2.0000bp}
    \definecolor{sc}{rgb}{0.0000,0.0000,0.0000}
    \pgfsetstrokecolor{sc}
    \pgfsetmiterjoin
    \pgfsetbuttcap
    \pgfpathqmoveto{112.0000bp}{20.0000bp}
    \pgfpathqcurveto{112.0000bp}{26.6274bp}{106.6274bp}{32.0000bp}{100.0000bp}{32.0000bp}
    \pgfpathqcurveto{93.3726bp}{32.0000bp}{88.0000bp}{26.6274bp}{88.0000bp}{20.0000bp}
    \pgfpathqcurveto{88.0000bp}{13.3726bp}{93.3726bp}{8.0000bp}{100.0000bp}{8.0000bp}
    \pgfpathqcurveto{106.6274bp}{8.0000bp}{112.0000bp}{13.3726bp}{112.0000bp}{20.0000bp}
    \pgfpathclose
    \pgfusepathqfillstroke
  \end{pgfscope}
  \begin{pgfscope}
    \definecolor{fc}{rgb}{0.0000,0.0000,0.0000}
    \pgfsetfillcolor{fc}
    \pgfsetfillopacity{0.0000}
    \pgfsetlinewidth{2.0000bp}
    \definecolor{sc}{rgb}{0.0000,0.0000,0.0000}
    \pgfsetstrokecolor{sc}
    \pgfsetmiterjoin
    \pgfsetbuttcap
    \pgfpathqmoveto{120.0000bp}{0.0000bp}
    \pgfpathqlineto{120.0000bp}{40.0000bp}
    \pgfpathqlineto{80.0000bp}{40.0000bp}
    \pgfpathqlineto{80.0000bp}{0.0000bp}
    \pgfpathqlineto{120.0000bp}{0.0000bp}
    \pgfpathclose
    \pgfusepathqfillstroke
  \end{pgfscope}
  \begin{pgfscope}
    \definecolor{fc}{rgb}{0.0000,0.0000,0.0000}
    \pgfsetfillcolor{fc}
    \pgfsetfillopacity{0.0000}
    \pgfsetlinewidth{2.0000bp}
    \definecolor{sc}{rgb}{0.0000,0.0000,0.0000}
    \pgfsetstrokecolor{sc}
    \pgfsetmiterjoin
    \pgfsetbuttcap
    \pgfpathqmoveto{120.0000bp}{40.0000bp}
    \pgfpathqlineto{120.0000bp}{80.0000bp}
    \pgfpathqlineto{80.0000bp}{80.0000bp}
    \pgfpathqlineto{80.0000bp}{40.0000bp}
    \pgfpathqlineto{120.0000bp}{40.0000bp}
    \pgfpathclose
    \pgfusepathqfillstroke
  \end{pgfscope}
  \begin{pgfscope}
    \definecolor{fc}{rgb}{0.0000,0.0000,0.0000}
    \pgfsetfillcolor{fc}
    \pgftransformshift{\pgfqpoint{100.0000bp}{100.0000bp}}
    \pgftransformscale{1.0000}
    \pgftext[base,left]{$c$}
  \end{pgfscope}
  \begin{pgfscope}
    \definecolor{fc}{rgb}{0.0000,0.0000,0.0000}
    \pgfsetfillcolor{fc}
    \pgfsetfillopacity{0.0000}
    \pgfsetlinewidth{2.0000bp}
    \definecolor{sc}{rgb}{0.0000,0.0000,0.0000}
    \pgfsetstrokecolor{sc}
    \pgfsetmiterjoin
    \pgfsetbuttcap
    \pgfpathqmoveto{112.0000bp}{88.0000bp}
    \pgfpathqlineto{112.0000bp}{112.0000bp}
    \pgfpathqlineto{88.0000bp}{112.0000bp}
    \pgfpathqlineto{88.0000bp}{88.0000bp}
    \pgfpathqlineto{112.0000bp}{88.0000bp}
    \pgfpathclose
    \pgfusepathqfillstroke
  \end{pgfscope}
  \begin{pgfscope}
    \definecolor{fc}{rgb}{0.0000,0.0000,0.0000}
    \pgfsetfillcolor{fc}
    \pgfsetfillopacity{0.0000}
    \pgfsetlinewidth{2.0000bp}
    \definecolor{sc}{rgb}{0.0000,0.0000,0.0000}
    \pgfsetstrokecolor{sc}
    \pgfsetmiterjoin
    \pgfsetbuttcap
    \pgfpathqmoveto{120.0000bp}{80.0000bp}
    \pgfpathqlineto{120.0000bp}{120.0000bp}
    \pgfpathqlineto{80.0000bp}{120.0000bp}
    \pgfpathqlineto{80.0000bp}{80.0000bp}
    \pgfpathqlineto{120.0000bp}{80.0000bp}
    \pgfpathclose
    \pgfusepathqfillstroke
  \end{pgfscope}
  \begin{pgfscope}
    \definecolor{fc}{rgb}{0.0000,0.0000,0.0000}
    \pgfsetfillcolor{fc}
    \pgfsetfillopacity{0.0000}
    \pgfsetlinewidth{2.0000bp}
    \definecolor{sc}{rgb}{0.0000,0.0000,0.0000}
    \pgfsetstrokecolor{sc}
    \pgfsetmiterjoin
    \pgfsetbuttcap
    \pgfpathqmoveto{120.0000bp}{120.0000bp}
    \pgfpathqlineto{120.0000bp}{160.0000bp}
    \pgfpathqlineto{80.0000bp}{160.0000bp}
    \pgfpathqlineto{80.0000bp}{120.0000bp}
    \pgfpathqlineto{120.0000bp}{120.0000bp}
    \pgfpathclose
    \pgfusepathqfillstroke
  \end{pgfscope}
  \begin{pgfscope}
    \definecolor{fc}{rgb}{0.0000,0.0000,0.0000}
    \pgfsetfillcolor{fc}
    \pgfsetfillopacity{0.0000}
    \pgfsetlinewidth{2.0000bp}
    \definecolor{sc}{rgb}{0.0000,0.0000,0.0000}
    \pgfsetstrokecolor{sc}
    \pgfsetmiterjoin
    \pgfsetbuttcap
    \pgfsetdash{{2.6833bp}{2.6833bp}}{0.0000bp}
    \pgfpathqmoveto{114.0000bp}{166.0000bp}
    \pgfpathqlineto{114.0000bp}{194.0000bp}
    \pgfpathqlineto{86.0000bp}{194.0000bp}
    \pgfpathqlineto{86.0000bp}{166.0000bp}
    \pgfpathqlineto{114.0000bp}{166.0000bp}
    \pgfpathclose
    \pgfusepathqfillstroke
  \end{pgfscope}
  \begin{pgfscope}
    \definecolor{fc}{rgb}{0.0000,0.0000,0.0000}
    \pgfsetfillcolor{fc}
    \pgfsetfillopacity{0.0000}
    \pgfsetlinewidth{2.0000bp}
    \definecolor{sc}{rgb}{0.0000,0.0000,0.0000}
    \pgfsetstrokecolor{sc}
    \pgfsetmiterjoin
    \pgfsetbuttcap
    \pgfpathqmoveto{120.0000bp}{160.0000bp}
    \pgfpathqlineto{120.0000bp}{200.0000bp}
    \pgfpathqlineto{80.0000bp}{200.0000bp}
    \pgfpathqlineto{80.0000bp}{160.0000bp}
    \pgfpathqlineto{120.0000bp}{160.0000bp}
    \pgfpathclose
    \pgfusepathqfillstroke
  \end{pgfscope}
\end{pgfpicture}
}
        \caption{Vertical training problem for the sokoban agent.}\label{fig:results:train2}
    \end{subfigure}
    \begin{subfigure}{0.3\textwidth}
        \resizebox{\linewidth}{!}{\begin{pgfpicture}
  \pgfpathrectangle{\pgfpointorigin}{\pgfqpoint{200.0000bp}{200.0000bp}}
  \pgfusepath{use as bounding box}
  \begin{pgfscope}
    \definecolor{fc}{rgb}{0.0000,0.0000,0.0000}
    \pgfsetfillcolor{fc}
    \pgfsetfillopacity{0.0000}
    \pgfsetlinewidth{2.0000bp}
    \definecolor{sc}{rgb}{0.0000,0.0000,0.0000}
    \pgfsetstrokecolor{sc}
    \pgfsetmiterjoin
    \pgfsetbuttcap
    \pgfsetdash{{6.0000bp}{6.0000bp}}{0.0000bp}
    \pgfpathqmoveto{42.5000bp}{7.5000bp}
    \pgfpathqlineto{42.5000bp}{42.5000bp}
    \pgfpathqlineto{7.5000bp}{42.5000bp}
    \pgfpathqlineto{7.5000bp}{7.5000bp}
    \pgfpathqlineto{42.5000bp}{7.5000bp}
    \pgfpathclose
    \pgfusepathqfillstroke
  \end{pgfscope}
  \begin{pgfscope}
    \definecolor{fc}{rgb}{0.0000,0.0000,0.0000}
    \pgfsetfillcolor{fc}
    \pgfsetfillopacity{0.0000}
    \pgfsetlinewidth{2.0000bp}
    \definecolor{sc}{rgb}{0.0000,0.0000,0.0000}
    \pgfsetstrokecolor{sc}
    \pgfsetmiterjoin
    \pgfsetbuttcap
    \pgfpathqmoveto{50.0000bp}{0.0000bp}
    \pgfpathqlineto{50.0000bp}{50.0000bp}
    \pgfpathqlineto{0.0000bp}{50.0000bp}
    \pgfpathqlineto{-0.0000bp}{0.0000bp}
    \pgfpathqlineto{50.0000bp}{0.0000bp}
    \pgfpathclose
    \pgfusepathqfillstroke
  \end{pgfscope}
  \begin{pgfscope}
    \definecolor{fc}{rgb}{0.0000,0.0000,0.0000}
    \pgfsetfillcolor{fc}
    \pgfsetfillopacity{0.0000}
    \pgfsetlinewidth{2.0000bp}
    \definecolor{sc}{rgb}{0.0000,0.0000,0.0000}
    \pgfsetstrokecolor{sc}
    \pgfsetmiterjoin
    \pgfsetbuttcap
    \pgfpathqmoveto{50.0000bp}{50.0000bp}
    \pgfpathqlineto{50.0000bp}{100.0000bp}
    \pgfpathqlineto{0.0000bp}{100.0000bp}
    \pgfpathqlineto{-0.0000bp}{50.0000bp}
    \pgfpathqlineto{50.0000bp}{50.0000bp}
    \pgfpathclose
    \pgfusepathqfillstroke
  \end{pgfscope}
  \begin{pgfscope}
    \definecolor{fc}{rgb}{0.0000,0.0000,0.0000}
    \pgfsetfillcolor{fc}
    \pgftransformshift{\pgfqpoint{25.0000bp}{125.0000bp}}
    \pgftransformscale{1.2500}
    \pgftext[base,left]{$c_2$}
  \end{pgfscope}
  \begin{pgfscope}
    \definecolor{fc}{rgb}{0.0000,0.0000,0.0000}
    \pgfsetfillcolor{fc}
    \pgfsetfillopacity{0.0000}
    \pgfsetlinewidth{2.0000bp}
    \definecolor{sc}{rgb}{0.0000,0.0000,0.0000}
    \pgfsetstrokecolor{sc}
    \pgfsetmiterjoin
    \pgfsetbuttcap
    \pgfpathqmoveto{40.0000bp}{110.0000bp}
    \pgfpathqlineto{40.0000bp}{140.0000bp}
    \pgfpathqlineto{10.0000bp}{140.0000bp}
    \pgfpathqlineto{10.0000bp}{110.0000bp}
    \pgfpathqlineto{40.0000bp}{110.0000bp}
    \pgfpathclose
    \pgfusepathqfillstroke
  \end{pgfscope}
  \begin{pgfscope}
    \definecolor{fc}{rgb}{0.0000,0.0000,0.0000}
    \pgfsetfillcolor{fc}
    \pgfsetfillopacity{0.0000}
    \pgfsetlinewidth{2.0000bp}
    \definecolor{sc}{rgb}{0.0000,0.0000,0.0000}
    \pgfsetstrokecolor{sc}
    \pgfsetmiterjoin
    \pgfsetbuttcap
    \pgfpathqmoveto{50.0000bp}{100.0000bp}
    \pgfpathqlineto{50.0000bp}{150.0000bp}
    \pgfpathqlineto{0.0000bp}{150.0000bp}
    \pgfpathqlineto{-0.0000bp}{100.0000bp}
    \pgfpathqlineto{50.0000bp}{100.0000bp}
    \pgfpathclose
    \pgfusepathqfillstroke
  \end{pgfscope}
  \begin{pgfscope}
    \definecolor{fc}{rgb}{0.0000,0.0000,0.0000}
    \pgfsetfillcolor{fc}
    \pgfsetfillopacity{0.0000}
    \pgfsetlinewidth{2.0000bp}
    \definecolor{sc}{rgb}{0.0000,0.0000,0.0000}
    \pgfsetstrokecolor{sc}
    \pgfsetmiterjoin
    \pgfsetbuttcap
    \pgfsetdash{{6.0000bp}{6.0000bp}}{0.0000bp}
    \pgfpathqmoveto{192.5000bp}{157.5000bp}
    \pgfpathqlineto{192.5000bp}{192.5000bp}
    \pgfpathqlineto{157.5000bp}{192.5000bp}
    \pgfpathqlineto{157.5000bp}{157.5000bp}
    \pgfpathqlineto{192.5000bp}{157.5000bp}
    \pgfpathclose
    \pgfusepathqfillstroke
  \end{pgfscope}
  \begin{pgfscope}
    \definecolor{fc}{rgb}{0.0000,0.0000,0.0000}
    \pgfsetfillcolor{fc}
    \pgfsetfillopacity{0.0000}
    \pgfsetlinewidth{2.0000bp}
    \definecolor{sc}{rgb}{0.0000,0.0000,0.0000}
    \pgfsetstrokecolor{sc}
    \pgfsetmiterjoin
    \pgfsetbuttcap
    \pgfpathqmoveto{200.0000bp}{150.0000bp}
    \pgfpathqlineto{200.0000bp}{200.0000bp}
    \pgfpathqlineto{150.0000bp}{200.0000bp}
    \pgfpathqlineto{150.0000bp}{150.0000bp}
    \pgfpathqlineto{200.0000bp}{150.0000bp}
    \pgfpathclose
    \pgfusepathqfillstroke
  \end{pgfscope}
  \begin{pgfscope}
    \definecolor{fc}{rgb}{0.0000,0.0000,0.0000}
    \pgfsetfillcolor{fc}
    \pgfsetfillopacity{0.0000}
    \pgfsetlinewidth{2.0000bp}
    \definecolor{sc}{rgb}{0.0000,0.0000,0.0000}
    \pgfsetstrokecolor{sc}
    \pgfsetmiterjoin
    \pgfsetbuttcap
    \pgfpathqmoveto{150.0000bp}{150.0000bp}
    \pgfpathqlineto{150.0000bp}{200.0000bp}
    \pgfpathqlineto{100.0000bp}{200.0000bp}
    \pgfpathqlineto{100.0000bp}{150.0000bp}
    \pgfpathqlineto{150.0000bp}{150.0000bp}
    \pgfpathclose
    \pgfusepathqfillstroke
  \end{pgfscope}
  \begin{pgfscope}
    \definecolor{fc}{rgb}{0.0000,0.0000,0.0000}
    \pgfsetfillcolor{fc}
    \pgftransformshift{\pgfqpoint{75.0000bp}{175.0000bp}}
    \pgftransformscale{1.2500}
    \pgftext[base,left]{$c_1$}
  \end{pgfscope}
  \begin{pgfscope}
    \definecolor{fc}{rgb}{0.0000,0.0000,0.0000}
    \pgfsetfillcolor{fc}
    \pgfsetfillopacity{0.0000}
    \pgfsetlinewidth{2.0000bp}
    \definecolor{sc}{rgb}{0.0000,0.0000,0.0000}
    \pgfsetstrokecolor{sc}
    \pgfsetmiterjoin
    \pgfsetbuttcap
    \pgfpathqmoveto{90.0000bp}{160.0000bp}
    \pgfpathqlineto{90.0000bp}{190.0000bp}
    \pgfpathqlineto{60.0000bp}{190.0000bp}
    \pgfpathqlineto{60.0000bp}{160.0000bp}
    \pgfpathqlineto{90.0000bp}{160.0000bp}
    \pgfpathclose
    \pgfusepathqfillstroke
  \end{pgfscope}
  \begin{pgfscope}
    \definecolor{fc}{rgb}{0.0000,0.0000,0.0000}
    \pgfsetfillcolor{fc}
    \pgfsetfillopacity{0.0000}
    \pgfsetlinewidth{2.0000bp}
    \definecolor{sc}{rgb}{0.0000,0.0000,0.0000}
    \pgfsetstrokecolor{sc}
    \pgfsetmiterjoin
    \pgfsetbuttcap
    \pgfpathqmoveto{100.0000bp}{150.0000bp}
    \pgfpathqlineto{100.0000bp}{200.0000bp}
    \pgfpathqlineto{50.0000bp}{200.0000bp}
    \pgfpathqlineto{50.0000bp}{150.0000bp}
    \pgfpathqlineto{100.0000bp}{150.0000bp}
    \pgfpathclose
    \pgfusepathqfillstroke
  \end{pgfscope}
  \begin{pgfscope}
    \definecolor{fc}{rgb}{0.0000,0.0000,0.0000}
    \pgfsetfillcolor{fc}
    \pgfsetfillopacity{0.0000}
    \pgfsetlinewidth{2.0000bp}
    \definecolor{sc}{rgb}{0.0000,0.0000,0.0000}
    \pgfsetstrokecolor{sc}
    \pgfsetmiterjoin
    \pgfsetbuttcap
    \pgfpathqmoveto{40.0000bp}{175.0000bp}
    \pgfpathqcurveto{40.0000bp}{183.2843bp}{33.2843bp}{190.0000bp}{25.0000bp}{190.0000bp}
    \pgfpathqcurveto{16.7157bp}{190.0000bp}{10.0000bp}{183.2843bp}{10.0000bp}{175.0000bp}
    \pgfpathqcurveto{10.0000bp}{166.7157bp}{16.7157bp}{160.0000bp}{25.0000bp}{160.0000bp}
    \pgfpathqcurveto{33.2843bp}{160.0000bp}{40.0000bp}{166.7157bp}{40.0000bp}{175.0000bp}
    \pgfpathclose
    \pgfusepathqfillstroke
  \end{pgfscope}
  \begin{pgfscope}
    \definecolor{fc}{rgb}{0.0000,0.0000,0.0000}
    \pgfsetfillcolor{fc}
    \pgfsetfillopacity{0.0000}
    \pgfsetlinewidth{2.0000bp}
    \definecolor{sc}{rgb}{0.0000,0.0000,0.0000}
    \pgfsetstrokecolor{sc}
    \pgfsetmiterjoin
    \pgfsetbuttcap
    \pgfpathqmoveto{50.0000bp}{150.0000bp}
    \pgfpathqlineto{50.0000bp}{200.0000bp}
    \pgfpathqlineto{0.0000bp}{200.0000bp}
    \pgfpathqlineto{-0.0000bp}{150.0000bp}
    \pgfpathqlineto{50.0000bp}{150.0000bp}
    \pgfpathclose
    \pgfusepathqfillstroke
  \end{pgfscope}
\end{pgfpicture}
}
        \caption{A problem that requires use of all six actions in the sokoban domain.}\label{fig:results:train3}
    \end{subfigure}
    \caption{Sokoban worlds used for training.}\label{fig:results:sokoTraining}
\end{figure}

In the source code bundled with this thesis, the program \texttt{Plotting/Statistics.hs} for interpreting and visually representing statistics files are included  (along with source code for generating most diagrams used in this thesis). 

\begin{figure}
    \centering
    \begin{subfigure}{0.45\linewidth}
        \resizebox{\linewidth}{!}{\begin{pgfpicture}
  \pgfpathrectangle{\pgfpointorigin}{\pgfqpoint{200.0000bp}{200.0000bp}}
  \pgfusepath{use as bounding box}
  \begin{pgfscope}
    \definecolor{fc}{rgb}{0.0000,0.0000,0.0000}
    \pgfsetfillcolor{fc}
    \pgftransformshift{\pgfqpoint{28.3333bp}{25.8333bp}}
    \pgftransformscale{1.0417}
    \pgftext[base,left]{candidates}
  \end{pgfscope}
  \begin{pgfscope}
    \definecolor{fc}{rgb}{0.0000,0.0000,0.0000}
    \pgfsetfillcolor{fc}
    \pgfsetlinewidth{0.6928bp}
    \definecolor{sc}{rgb}{0.0000,0.0000,0.0000}
    \pgfsetstrokecolor{sc}
    \pgfsetmiterjoin
    \pgfsetbuttcap
    \pgfpathqmoveto{20.0000bp}{28.3333bp}
    \pgfpathqcurveto{20.0000bp}{30.1743bp}{18.5076bp}{31.6667bp}{16.6667bp}{31.6667bp}
    \pgfpathqcurveto{14.8257bp}{31.6667bp}{13.3333bp}{30.1743bp}{13.3333bp}{28.3333bp}
    \pgfpathqcurveto{13.3333bp}{26.4924bp}{14.8257bp}{25.0000bp}{16.6667bp}{25.0000bp}
    \pgfpathqcurveto{18.5076bp}{25.0000bp}{20.0000bp}{26.4924bp}{20.0000bp}{28.3333bp}
    \pgfpathclose
    \pgfusepathqfillstroke
  \end{pgfscope}
  \begin{pgfscope}
    \definecolor{fc}{rgb}{0.0000,0.0000,0.0000}
    \pgfsetfillcolor{fc}
    \pgftransformshift{\pgfqpoint{28.3333bp}{36.6667bp}}
    \pgftransformscale{1.0417}
    \pgftext[base,left]{negative unproven}
  \end{pgfscope}
  \begin{pgfscope}
    \definecolor{fc}{rgb}{1.0000,1.0000,0.0000}
    \pgfsetfillcolor{fc}
    \pgfsetlinewidth{0.6928bp}
    \definecolor{sc}{rgb}{1.0000,1.0000,0.0000}
    \pgfsetstrokecolor{sc}
    \pgfsetmiterjoin
    \pgfsetbuttcap
    \pgfpathqmoveto{20.0000bp}{39.1667bp}
    \pgfpathqcurveto{20.0000bp}{41.0076bp}{18.5076bp}{42.5000bp}{16.6667bp}{42.5000bp}
    \pgfpathqcurveto{14.8257bp}{42.5000bp}{13.3333bp}{41.0076bp}{13.3333bp}{39.1667bp}
    \pgfpathqcurveto{13.3333bp}{37.3257bp}{14.8257bp}{35.8333bp}{16.6667bp}{35.8333bp}
    \pgfpathqcurveto{18.5076bp}{35.8333bp}{20.0000bp}{37.3257bp}{20.0000bp}{39.1667bp}
    \pgfpathclose
    \pgfusepathqfillstroke
  \end{pgfscope}
  \begin{pgfscope}
    \definecolor{fc}{rgb}{0.0000,0.0000,0.0000}
    \pgfsetfillcolor{fc}
    \pgftransformshift{\pgfqpoint{28.3333bp}{47.5000bp}}
    \pgftransformscale{1.0417}
    \pgftext[base,left]{negative proven}
  \end{pgfscope}
  \begin{pgfscope}
    \definecolor{fc}{rgb}{0.0000,0.5020,0.0000}
    \pgfsetfillcolor{fc}
    \pgfsetlinewidth{0.6928bp}
    \definecolor{sc}{rgb}{0.0000,0.5020,0.0000}
    \pgfsetstrokecolor{sc}
    \pgfsetmiterjoin
    \pgfsetbuttcap
    \pgfpathqmoveto{20.0000bp}{50.0000bp}
    \pgfpathqcurveto{20.0000bp}{51.8409bp}{18.5076bp}{53.3333bp}{16.6667bp}{53.3333bp}
    \pgfpathqcurveto{14.8257bp}{53.3333bp}{13.3333bp}{51.8409bp}{13.3333bp}{50.0000bp}
    \pgfpathqcurveto{13.3333bp}{48.1591bp}{14.8257bp}{46.6667bp}{16.6667bp}{46.6667bp}
    \pgfpathqcurveto{18.5076bp}{46.6667bp}{20.0000bp}{48.1591bp}{20.0000bp}{50.0000bp}
    \pgfpathclose
    \pgfusepathqfillstroke
  \end{pgfscope}
  \begin{pgfscope}
    \definecolor{fc}{rgb}{0.0000,0.0000,0.0000}
    \pgfsetfillcolor{fc}
    \pgftransformshift{\pgfqpoint{28.3333bp}{58.3333bp}}
    \pgftransformscale{1.0417}
    \pgftext[base,left]{positive unproven}
  \end{pgfscope}
  \begin{pgfscope}
    \definecolor{fc}{rgb}{1.0000,0.0000,0.0000}
    \pgfsetfillcolor{fc}
    \pgfsetlinewidth{0.6928bp}
    \definecolor{sc}{rgb}{1.0000,0.0000,0.0000}
    \pgfsetstrokecolor{sc}
    \pgfsetmiterjoin
    \pgfsetbuttcap
    \pgfpathqmoveto{20.0000bp}{60.8333bp}
    \pgfpathqcurveto{20.0000bp}{62.6743bp}{18.5076bp}{64.1667bp}{16.6667bp}{64.1667bp}
    \pgfpathqcurveto{14.8257bp}{64.1667bp}{13.3333bp}{62.6743bp}{13.3333bp}{60.8333bp}
    \pgfpathqcurveto{13.3333bp}{58.9924bp}{14.8257bp}{57.5000bp}{16.6667bp}{57.5000bp}
    \pgfpathqcurveto{18.5076bp}{57.5000bp}{20.0000bp}{58.9924bp}{20.0000bp}{60.8333bp}
    \pgfpathclose
    \pgfusepathqfillstroke
  \end{pgfscope}
  \begin{pgfscope}
    \definecolor{fc}{rgb}{0.0000,0.0000,0.0000}
    \pgfsetfillcolor{fc}
    \pgftransformshift{\pgfqpoint{28.3333bp}{69.1667bp}}
    \pgftransformscale{1.0417}
    \pgftext[base,left]{positive proven}
  \end{pgfscope}
  \begin{pgfscope}
    \definecolor{fc}{rgb}{0.0000,0.0000,1.0000}
    \pgfsetfillcolor{fc}
    \pgfsetlinewidth{0.6928bp}
    \definecolor{sc}{rgb}{0.0000,0.0000,1.0000}
    \pgfsetstrokecolor{sc}
    \pgfsetmiterjoin
    \pgfsetbuttcap
    \pgfpathqmoveto{20.0000bp}{71.6667bp}
    \pgfpathqcurveto{20.0000bp}{73.5076bp}{18.5076bp}{75.0000bp}{16.6667bp}{75.0000bp}
    \pgfpathqcurveto{14.8257bp}{75.0000bp}{13.3333bp}{73.5076bp}{13.3333bp}{71.6667bp}
    \pgfpathqcurveto{13.3333bp}{69.8257bp}{14.8257bp}{68.3333bp}{16.6667bp}{68.3333bp}
    \pgfpathqcurveto{18.5076bp}{68.3333bp}{20.0000bp}{69.8257bp}{20.0000bp}{71.6667bp}
    \pgfpathclose
    \pgfusepathqfillstroke
  \end{pgfscope}
  \begin{pgfscope}
    \pgfsetlinewidth{0.6928bp}
    \definecolor{sc}{rgb}{1.0000,1.0000,0.0000}
    \pgfsetstrokecolor{sc}
    \pgfsetmiterjoin
    \pgfsetbuttcap
    \pgfpathqmoveto{25.0000bp}{175.0000bp}
    \pgfpathqlineto{33.3333bp}{175.0000bp}
    \pgfpathqlineto{41.6667bp}{175.0000bp}
    \pgfpathqlineto{50.0000bp}{175.0000bp}
    \pgfpathqlineto{58.3333bp}{175.0000bp}
    \pgfpathqlineto{66.6667bp}{175.0000bp}
    \pgfpathqlineto{75.0000bp}{175.0000bp}
    \pgfpathqlineto{83.3333bp}{175.0000bp}
    \pgfpathqlineto{91.6667bp}{175.0000bp}
    \pgfpathqlineto{100.0000bp}{175.0000bp}
    \pgfpathqlineto{108.3333bp}{175.0000bp}
    \pgfpathqlineto{116.6667bp}{175.0000bp}
    \pgfpathqlineto{125.0000bp}{175.0000bp}
    \pgfpathqlineto{133.3333bp}{175.0000bp}
    \pgfpathqlineto{141.6667bp}{175.0000bp}
    \pgfpathqlineto{150.0000bp}{175.0000bp}
    \pgfpathqlineto{158.3333bp}{175.0000bp}
    \pgfpathqlineto{166.6667bp}{91.6667bp}
    \pgfpathqlineto{175.0000bp}{91.6667bp}
    \pgfpathqlineto{183.3333bp}{91.6667bp}
    \pgfpathqlineto{191.6667bp}{91.6667bp}
    \pgfpathqlineto{200.0000bp}{91.6667bp}
    \pgfusepathqstroke
  \end{pgfscope}
  \begin{pgfscope}
    \pgfsetlinewidth{0.6928bp}
    \definecolor{sc}{rgb}{0.0000,0.5020,0.0000}
    \pgfsetstrokecolor{sc}
    \pgfsetmiterjoin
    \pgfsetbuttcap
    \pgfpathqmoveto{25.0000bp}{91.6667bp}
    \pgfpathqlineto{33.3333bp}{91.6667bp}
    \pgfpathqlineto{41.6667bp}{91.6667bp}
    \pgfpathqlineto{50.0000bp}{91.6667bp}
    \pgfpathqlineto{58.3333bp}{91.6667bp}
    \pgfpathqlineto{66.6667bp}{91.6667bp}
    \pgfpathqlineto{75.0000bp}{91.6667bp}
    \pgfpathqlineto{83.3333bp}{91.6667bp}
    \pgfpathqlineto{91.6667bp}{91.6667bp}
    \pgfpathqlineto{100.0000bp}{91.6667bp}
    \pgfpathqlineto{108.3333bp}{91.6667bp}
    \pgfpathqlineto{116.6667bp}{91.6667bp}
    \pgfpathqlineto{125.0000bp}{91.6667bp}
    \pgfpathqlineto{133.3333bp}{91.6667bp}
    \pgfpathqlineto{141.6667bp}{91.6667bp}
    \pgfpathqlineto{150.0000bp}{91.6667bp}
    \pgfpathqlineto{158.3333bp}{91.6667bp}
    \pgfpathqlineto{166.6667bp}{100.0000bp}
    \pgfpathqlineto{175.0000bp}{100.0000bp}
    \pgfpathqlineto{183.3333bp}{100.0000bp}
    \pgfpathqlineto{191.6667bp}{100.0000bp}
    \pgfpathqlineto{200.0000bp}{100.0000bp}
    \pgfusepathqstroke
  \end{pgfscope}
  \begin{pgfscope}
    \pgfsetlinewidth{0.6928bp}
    \definecolor{sc}{rgb}{1.0000,0.0000,0.0000}
    \pgfsetstrokecolor{sc}
    \pgfsetmiterjoin
    \pgfsetbuttcap
    \pgfpathqmoveto{25.0000bp}{175.0000bp}
    \pgfpathqlineto{33.3333bp}{175.0000bp}
    \pgfpathqlineto{41.6667bp}{175.0000bp}
    \pgfpathqlineto{50.0000bp}{175.0000bp}
    \pgfpathqlineto{58.3333bp}{175.0000bp}
    \pgfpathqlineto{66.6667bp}{175.0000bp}
    \pgfpathqlineto{75.0000bp}{175.0000bp}
    \pgfpathqlineto{83.3333bp}{175.0000bp}
    \pgfpathqlineto{91.6667bp}{175.0000bp}
    \pgfpathqlineto{100.0000bp}{175.0000bp}
    \pgfpathqlineto{108.3333bp}{175.0000bp}
    \pgfpathqlineto{116.6667bp}{175.0000bp}
    \pgfpathqlineto{125.0000bp}{175.0000bp}
    \pgfpathqlineto{133.3333bp}{175.0000bp}
    \pgfpathqlineto{141.6667bp}{175.0000bp}
    \pgfpathqlineto{150.0000bp}{175.0000bp}
    \pgfpathqlineto{158.3333bp}{175.0000bp}
    \pgfpathqlineto{166.6667bp}{91.6667bp}
    \pgfpathqlineto{175.0000bp}{91.6667bp}
    \pgfpathqlineto{183.3333bp}{91.6667bp}
    \pgfpathqlineto{191.6667bp}{91.6667bp}
    \pgfpathqlineto{200.0000bp}{91.6667bp}
    \pgfusepathqstroke
  \end{pgfscope}
  \begin{pgfscope}
    \pgfsetlinewidth{0.6928bp}
    \definecolor{sc}{rgb}{0.0000,0.0000,1.0000}
    \pgfsetstrokecolor{sc}
    \pgfsetmiterjoin
    \pgfsetbuttcap
    \pgfpathqmoveto{25.0000bp}{91.6667bp}
    \pgfpathqlineto{33.3333bp}{91.6667bp}
    \pgfpathqlineto{41.6667bp}{91.6667bp}
    \pgfpathqlineto{50.0000bp}{91.6667bp}
    \pgfpathqlineto{58.3333bp}{91.6667bp}
    \pgfpathqlineto{66.6667bp}{91.6667bp}
    \pgfpathqlineto{75.0000bp}{91.6667bp}
    \pgfpathqlineto{83.3333bp}{91.6667bp}
    \pgfpathqlineto{91.6667bp}{91.6667bp}
    \pgfpathqlineto{100.0000bp}{91.6667bp}
    \pgfpathqlineto{108.3333bp}{91.6667bp}
    \pgfpathqlineto{116.6667bp}{91.6667bp}
    \pgfpathqlineto{125.0000bp}{91.6667bp}
    \pgfpathqlineto{133.3333bp}{91.6667bp}
    \pgfpathqlineto{141.6667bp}{91.6667bp}
    \pgfpathqlineto{150.0000bp}{91.6667bp}
    \pgfpathqlineto{158.3333bp}{91.6667bp}
    \pgfpathqlineto{166.6667bp}{100.0000bp}
    \pgfpathqlineto{175.0000bp}{100.0000bp}
    \pgfpathqlineto{183.3333bp}{100.0000bp}
    \pgfpathqlineto{191.6667bp}{100.0000bp}
    \pgfpathqlineto{200.0000bp}{100.0000bp}
    \pgfusepathqstroke
  \end{pgfscope}
  \begin{pgfscope}
    \pgfsetlinewidth{0.6928bp}
    \definecolor{sc}{rgb}{1.0000,0.0000,0.0000}
    \pgfsetstrokecolor{sc}
    \pgfsetmiterjoin
    \pgfsetbuttcap
    \pgfpathqmoveto{50.0000bp}{91.6667bp}
    \pgfpathqlineto{50.0000bp}{83.3333bp}
    \pgfusepathqstroke
  \end{pgfscope}
  \begin{pgfscope}
    \pgfsetlinewidth{0.6928bp}
    \definecolor{sc}{rgb}{1.0000,0.0000,0.0000}
    \pgfsetstrokecolor{sc}
    \pgfsetmiterjoin
    \pgfsetbuttcap
    \pgfpathqmoveto{166.6667bp}{91.6667bp}
    \pgfpathqlineto{166.6667bp}{83.3333bp}
    \pgfusepathqstroke
  \end{pgfscope}
  \begin{pgfscope}
    \pgfsetlinewidth{0.6928bp}
    \definecolor{sc}{rgb}{0.0000,0.0000,0.0000}
    \pgfsetstrokecolor{sc}
    \pgfsetmiterjoin
    \pgfsetbuttcap
    \pgfpathqmoveto{183.3333bp}{91.6667bp}
    \pgfpathqlineto{183.3333bp}{87.5000bp}
    \pgfusepathqstroke
  \end{pgfscope}
  \begin{pgfscope}
    \pgfsetlinewidth{0.6928bp}
    \definecolor{sc}{rgb}{0.0000,0.0000,0.0000}
    \pgfsetstrokecolor{sc}
    \pgfsetmiterjoin
    \pgfsetbuttcap
    \pgfpathqmoveto{141.6667bp}{91.6667bp}
    \pgfpathqlineto{141.6667bp}{87.5000bp}
    \pgfusepathqstroke
  \end{pgfscope}
  \begin{pgfscope}
    \pgfsetlinewidth{0.6928bp}
    \definecolor{sc}{rgb}{0.0000,0.0000,0.0000}
    \pgfsetstrokecolor{sc}
    \pgfsetmiterjoin
    \pgfsetbuttcap
    \pgfpathqmoveto{100.0000bp}{91.6667bp}
    \pgfpathqlineto{100.0000bp}{87.5000bp}
    \pgfusepathqstroke
  \end{pgfscope}
  \begin{pgfscope}
    \pgfsetlinewidth{0.6928bp}
    \definecolor{sc}{rgb}{0.0000,0.0000,0.0000}
    \pgfsetstrokecolor{sc}
    \pgfsetmiterjoin
    \pgfsetbuttcap
    \pgfpathqmoveto{58.3333bp}{91.6667bp}
    \pgfpathqlineto{58.3333bp}{87.5000bp}
    \pgfusepathqstroke
  \end{pgfscope}
  \begin{pgfscope}
    \definecolor{fc}{rgb}{0.0000,0.0000,0.0000}
    \pgfsetfillcolor{fc}
    \pgftransformshift{\pgfqpoint{-0.0000bp}{172.5000bp}}
    \pgftransformscale{1.0417}
    \pgftext[base,left]{$\mathbb{F}_A$}
  \end{pgfscope}
  \begin{pgfscope}
    \pgfsetlinewidth{0.6928bp}
    \definecolor{sc}{rgb}{0.0000,0.0000,0.0000}
    \pgfsetstrokecolor{sc}
    \pgfsetmiterjoin
    \pgfsetbuttcap
    \pgfpathqmoveto{16.6667bp}{175.0000bp}
    \pgfpathqlineto{15.0000bp}{175.0000bp}
    \pgfusepathqstroke
  \end{pgfscope}
  \begin{pgfscope}
    \pgfsetlinewidth{0.6928bp}
    \definecolor{sc}{rgb}{0.0000,0.0000,0.0000}
    \pgfsetstrokecolor{sc}
    \pgfsetmiterjoin
    \pgfsetbuttcap
    \pgfpathqmoveto{16.6667bp}{91.6667bp}
    \pgfpathqlineto{16.6667bp}{175.0000bp}
    \pgfusepathqstroke
  \end{pgfscope}
  \begin{pgfscope}
    \pgfsetlinewidth{0.6928bp}
    \definecolor{sc}{rgb}{0.0000,0.0000,0.0000}
    \pgfsetstrokecolor{sc}
    \pgfsetmiterjoin
    \pgfsetbuttcap
    \pgfpathqmoveto{16.6667bp}{91.6667bp}
    \pgfpathqlineto{200.0000bp}{91.6667bp}
    \pgfusepathqstroke
  \end{pgfscope}
\end{pgfpicture}
}
        \caption{move-h effects}\label{fig:res:ekmoveh}
    \end{subfigure}
    \begin{subfigure}{0.45\linewidth}
        \resizebox{\linewidth}{!}{\input{\master/Graphics/statistics2-ek-move-v.pgf}}
        \caption{move-v effects}\label{fig:res:ekmovev}
    \end{subfigure}
\end{figure}

\begin{figure}
    \centering
    \begin{subfigure}{0.42\linewidth}
        \resizebox{\linewidth}{!}{\input{\master/Graphics/statistics2-pk-move-h.pgf}}
        \caption{move-h effects}\label{fig:res:pkmoveh}
    \end{subfigure}
    \begin{subfigure}{0.42\linewidth}
        \resizebox{\linewidth}{!}{\input{\master/Graphics/statistics2-pk-move-v.pgf}}
        \caption{move-v effects}\label{fig:res:pkmovev}
    \end{subfigure}
\end{figure}

\end{document}
