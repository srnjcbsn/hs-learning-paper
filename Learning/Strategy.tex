\providecommand{\master}{..}
\documentclass[\master/Master.tex]{subfiles}
\begin{document}

	In our analysis of agent learning strategies we have found there are different traits that exists between all strategies, and some of the traits do tend to oppose each other thus by looking at the extreme cases we can get an understanding of where they are useful and in what situations. In~\cite{Walsh2008}, two such strategies are provided:
    \begin{itemize}
    \item An \emph{optimistic} strategy, that will always assume all actions
    does as much as possible, ensuring that a plan can always be found.
    \item A \emph{pessimistic} strategy which always assumes that actions undoes
    as much as possible, and will therefore never be able to achieve any (positive)
    goals.
    \end{itemize}
    Both of these strategies are equally valid approaches to learning.
    However, they show two different traits which we will define and analyze.
    \begin{definition}
    [{Explorative/Exploitative}] An explorative strategy, like an explorative graph search will prefer to try and test unknown territory in the search for informations which provide it with a better solution but only in cases where there is no upside decide not to explore.
    An exploitative strategy however will avoid the unexplored and only learn something when necessary otherwise it will use what it has already learned. Much akin to how the dynamic between a Breath-First-Search and a Depth-First-Search works. The advantage of a Explorative strategy is that once it has learned something, it will be very efficient at doing that thing, however for it to get to that point it will spend a long time learning. Opposite an exploitative strategy will be very fast to learn something but it will very likely be an inefficient way.
	\end{definition}
	\begin{example}
		Imagine an agent trying to learn how to sail a ship around an island, from harbor to harbor.
		
		Initially, it can either sail \textbf{clockwise} or \textbf{counter-clockwise}. Because it is new and inexperienced, it does not know what the outcome of either of these actions will be. 
		
		Let us assume that it requires seven \textbf{counter-clockwise} moves for the agent to reach its destination, but only two \textbf{clockwise} moves from its current location.
		
		Because the agent does not know anything, both the \textbf{clockwise} and the \textbf{counter-clockwise} option is initially equally good to it.
		
		\begin{itemize}
			\item If it was exploitative and initially tried a \textbf{counter-clockwise} move, then it would deem the problem solvable and no longer in need of learning. This is because an agent can simply stick to \textbf{counter-clockwise} moves around the island until it reaches its destination. 
			
			\item However if it used a strategy based explorativity, then no matter what option it tried it would have to test both actions, meaning it would take a minimum of four actions total.
			
			\item Lastly If it was exploitative but initially tried a \textbf{clockwise} move, then the agent would solve the problem in just 2 actions.
		\end{itemize}
		
		Thus we see that both extremes have advantages and disadvantages.
		
	\end{example}
	
	\begin{definition}
    [{Self-sufficient/Help-seeking}] We define a strategy which favors
    or completely relies on a teacher assisting it as a Help-seeking\footnote{The term help-seeking is coined by \cite{Gall1981224}} strategy
    and a strategy which does not as a self-sufficient strategy. While
    the term Help-seeking carry a negative connotation, it is not necessarily
    a bad strategy, for instance if the agent operates in an environment
    which is irreversible or dangerous, then getting assistance would
    be preferable over trying things out for itself. 

    \end{definition}
    
    \begin{example}
    	Imagine an agent that is put into a control room of nuclear powerplant.
    	 If the agent approached the problem of learning the environment in a self-sufficient manor, 
    	 this would result in the agent pressing the buttons until he figured out the system.
    	 Obviously this is dangerous and is inadvisable. 
    	 A more preferable approach would be to look through the manuals or receive guidance.
    \end{example}
	By labeling the traits unique to learning strategies we can now apply them to the learning strategies which are provided in~\cite{Walsh2008} as part of the algorithm. Furthermore we can also define new strategies which use a different mixture of the traits:
    \begin{example}
		[{Optimisic}] The \emph{optimistic} strategy as described in~\cite{Walsh2008} assumes that actions
	    can always be applied (unless proven otherwise) and contain \emph{all}
	    positive effects (unless proven otherwise). Using this strategy, a
	    plan can always be found (unless action schemas or goals have negative
	    predicates). As described the algorithm and strategy is purely based on the explorative trait, and has no mentions of reliance on a teacher or other external entity.
	    \begin{itemize}
	    \item Explorative\\
	    The strategy is using a explorative approach because the actions
	    which are unknown or have not been tested will always be chosen over
	    actions that has, as these actions are assumed to do more than they
	    are supposed to.
	    \item Self-sufficient\\
	    The strategy always assumes it can solve all problems even if that
	    is not the case, thus it will never ask a teacher unless the problem
	    is actually unsolvable.
	    \end{itemize}
	\end{example}
    \begin{example}
    [{Pessimistic}] A \emph{pessimistic} strategy assumes that actions
    can \emph{never} be applied (unless proven otherwise), and that they
    have all \emph{negative} effects (unless proven otherwise).
    \begin{itemize}
    \item Exploitative\\
    The strategy used will purposely assume that things that it does not
    know about does less than they actually do thus it will never try
    things it does not know about.
    \item Help-seeking\\
    The strategy will ask for help as much as possible, meaning that even
    if it is only one action it does not know about. It will ask a teacher
    to explain what it does over trying it out for itself.
    \end{itemize}
    \end{example}
    As we can see both of these strategies are extremes of the different
    traits, thus by changing the traits we can define new strategies.
    \begin{example}
    [{Confident}] This strategy about learning only
    what is necessary to complete a task and learn it as quickly as possible,
    it also needs to be self-reliant and simply test things for itself. We have chosen to name this an confident strategy, as these are the traits of a confident person. It also shows that there is a gradiant for instance if it is only self-sufficient and it never explores, would be an arrogant agent. However if it was overly help-seeking and explorative then would be an insecure agent, which doubt everything it has learned.
    
    \begin{itemize}
    \item Mostly exploitative\\
    The strategy relies on being confident in what it has learned and
    thus does not try out new things except in cases where it deems things it has learned too inefficient.
    \item Mostly Self-sufficient\\
    It believes that the agent is capable of learning on its own and thus avoids seeking help.
    \end{itemize}

	\end{example}



\end{document}
