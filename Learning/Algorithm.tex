\providecommand{\master}{..}
\documentclass[\master/Master.tex]{subfiles}

\begin{document}

In the following, we propose a general algorithm for learning in a virtual world. The algorithm bears resemblance to, and is derived from, the generally accepted scientific method, which can be summarized as follows:

\begin{enumerate}
    \item Propose a question
    \item Collect existing knowledge about the question
    \item Form a hypothesis answering the question
    \item Conduct an experiment serving to corroborate or disprove the hypothesis
    \item Analyze the results of the experiment and update the knowledge base accordingly.
    \item Repeat from step 2, taking into account the newly obtained knowledge
\end{enumerate}

Note that the above summation does not describe an algorithm, as there is no termination condition in the cycle formed by steps 2--6. However in the following we will algorithmize it.

\section{Algorithm based on the scientific method}

The algorithm presented in \Cref{algo:science} takes as input a question, a strategy, some initial knowledge, and a world in which the agent acts. The algorithm essentially repeats the cycle formed by steps 2--6 in the summation given above, until the question can be satisfactorily answered.


\begin{algorithm}
    \caption{Abstract learning algorithm based on the scientific method.}\label{algo:science}

    \begin{algorithmic}
        \Function{$\textsc{Learn}$}{\texttt{Question} $q$, \texttt{Strategy} $s$, \texttt{Knowledge} $k$, \texttt{World} $w$}
            \While{$\neg canAnswer(q, k, w)$}
                %\State{$k \gets k + inquire(w, q)$}
                \State{\texttt{Hypothesis} $h \gets form(q, k, s)$}
                \State{\texttt{Experiment} $e \gets design(w, h, s)$}
                \If{an experiment could be found}
                    \State{$(\texttt{Result} \, r, w) \gets conduct(e, w)$}
                    \State{$k \gets k \cup analyze(r, k)$}
                \Else%
                    \State{$k \gets k \cup inquire(w, q)$}
                \EndIf%
                \State{$s \gets update(s, k)$}
            \EndWhile%
            \State~\Return~$k$
        \EndFunction%
    \end{algorithmic}
\end{algorithm}

In the following we will elaborate on the types and functions used in the algorithmization of the scientific method, and their relevance to learning.

\paragraph*{World}
The world is something in which the agent can act. In a typical scientific experiment, this is the known universe, or --- more likely --- an appropriate abstraction thereof. As conducting an experiment yields an enirely new world, it is assumed that the world is discrete and static, ie.\ that it is only changed by the actions of the learning agent (specifically when the agent performs experiments). For an elaboration on these terms, see e.g.~\cite{Russell}. 

\paragraph*{Question}
In the context of learning, the formulation of the question will typically be posed as a problem that must be solved (``How do we get from $A$ to $B$?'').
However it can also take the more general form ``How does this concept work?''.
If the question is in form of a problem then it can typically be answered when that problem is solved, 
as the agent then has the knowledge required to solve the problem given the same start state (or any state on the path to the solution). 
However, if the question is more broadly defined it should be answerable when the agent has an good understanding of the concept. 
Note that for these broad question a hypothesis that answers the question, is generally not a satisfactory answer until the hypothesis has been proven to be true.


\paragraph*{Strategy}
The strategy determines how the agent forms hypotheses and chooses experiments to perform, and is as such essential to how the agent learns. Different learning strategies and their traits are described in Section~\ref{sec:Strategy}.

\paragraph*{Hypothesis}
After an experiment has been designed, the hypothesis used is discarded, since we are only interested in the knowledge it allowed us to obtain. It is therefore presumed that given the knowledge derived from performing experiments based on the hypothesis, the question can be answered satisfactorily without knowledge of the hypothesis.

\paragraph*{Experiment and Result}
An experiment is designed by a strategy and carried out in the world, which it changes. It is assumed that these changes can be observed, and obtained as part of the \texttt{Result} of the experiment. Additionally, the result may contain information on how the experiment was run, which parts of it succeeded and which did not.

\begin{description}
    \item[$canAnswer( \texttt{Question} \; q
                    , \texttt{Knowledge} \; k
                    , \texttt{World} \; w
                    )
         $]
        Whether the question can be satisfactorily answered, as described above.

    \item[$inquire( \texttt{World} \; w
                  , \texttt{Question} \; q
                  )
         $]
        Obtains knowledge useful for answering the question $q$ from external sources (given the state of the world $w$). This can be used to query a teacher.

    \item[$form( \texttt{Question} \; q
               , \texttt{Knowledge} \; k
               , \texttt{Strategy} \; s
               )
         $]
        Forms a \texttt{Hypothesis} from $s$ that seeks to answer $q$ given $k$.

    \item[$design( \texttt{World} \; w
                 , \texttt{Hypothesis} \; h
                 , \texttt{Strategy} \; s
                 )
         $]
        Construct an experiment which --- when conducted --- will yield new data that may verify or falsify the current hypothesis.

    \item[$conduct( \texttt{Experiment} \; e
                  , \texttt{World} \; w
                  )
         $]
        carries out the experiment $e$ in the world $w$, producing an updated world and a \texttt{Result}.

    \item[$analyze( \texttt{Result} \; r
                  , \texttt{Knowledge} \; k
                  )
         $]
        Analyzes the results of the conducted experiment with respect to existing knowledge.

    \item[$update( \texttt{Strategy} \; s
                 , \texttt{Knowledge} \; k
                 )
         $]
        Based on the newly formed knowledge and the strategy that was used to obtain it, a new strategy is formed.
\end{description}
\end{document}
