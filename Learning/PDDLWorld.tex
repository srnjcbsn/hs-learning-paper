\documentclass[../Master.tex]{subfiles}
\begin{document}

In order to discuss learning we must define a model for the domain which we operate within. Thus we choose to use the generally accepted PDDL or Planning Domain Definition Language (see~\cite{PDDL}). 

This section will give a minimal definition of PDDL, and provide context for syntax used in later chapter. We will also explain functions used in state manipulation such as the grounding and state-transition functions.


PDDL is based on set-theoretic planning (see~\cite{ghallab2004a}) popularized by the STRIPS language (see~\cite{STRIPS}). By default states in PDDL are based on closed-world-assumption, meaning that if something is not defined in the state it is assumed to be false. 
In PDDL and set-theoretic planning, a state is set of predicates defining the properties of the objects in an environment.

\begin{definition} 
	An intensional definition of a state $S$ in PDDL is a set of grounded predicates:
    \begin{equation*}
        S=\{p(o_1,\dots,o_n) \mid  p \in \mathbb{P} \land \{o_1,\dots,o_n\} \subseteq \mathcal{O} \land p(o_1,\dots,o_n) = \texttt{TRUE}\}
    \end{equation*}
    Where $\mathcal{O}$ is the set of all objects in the world, and $\mathbb{P}$ is the set of all predicates that can be used to describe the objects in the world.
\end{definition}

To change the state an agent must perform action on the state, in PDDL action schemas describes what changes the action will have to the state and under what conditions. 

	
\begin{definition} \label{def:lrn:action-spec-def}
		An action definition or action schema, defines how and when an action is to be applied. 
		
		$
		\begin{array}{ll}
		Action~schema~A: & \\
		
		\begin{array}{ll}
		Variables & = \{v_1,\dots,v_n\}			 \\  
		Pre	& = Goal Description(GD) \\
		Eff & = Effect(EFF) 
		\end{array}
		\end{array}$
		
		The BNF for GD and EFF is as follows:
		
	%DONT USE TABS on the following listing as it is directly affected by it
		\begin{lstlisting}
	<GD> -> (and <GD>*) 
	<GD> -> LITERAL
	
	<EFF> -> (and <EFF>*)
	<EFF> -> LITERAL
	
	<LITERAL> -> <predicate name> (TERM*)
	<LITERAL> -> (not <LITERAL>)
	
	<TERM> -> <object>
	<TERM> -> <variable>
		\end{lstlisting}
		
		The distinction between GD and EFF is important to PDDL action schemas to allow either to receive additional features without the other gaining the same.
	\end{definition}
	
In PDDL a set of action schemas, along with properties about always present objects and specifications of all Predicates are defined as a domain.

\begin{definition} A domain in PDDL, defines what predicates can be utilized, object that are constant between all instantiations of the domain, and the action schemas available to the planner.
	
$
\begin{array}{ll}
Domain~D: & \\

\begin{array}{ll}
Constants & = \{o_1,\dots,o_n\}			 \\  
Predicate-Specs = \mathbb{P} \\
Action-Specs & = A~set~of~action~schemas~avaiable~to~the~agent \\
\end{array}
\end{array}$
\end{definition}


	

An application of an action contain the name of its corresponding action schema, and a list of objects that are to be mapped to its variables.
\begin{definition} 
	An action $A(o_1,\dots,o_n)$ explains how action schema $A$ is to be applied. The mapping to variables works like normal function application. 
	
	$\sigma = \{ v_1 \mapsto o_1,\dots,v_n \mapsto o_n \}$.
	
	For notational purposes, an application of an action schema $A$ can be written as:
	
	$a = A(o_1,\dots,o_n)$ 
	
	I.e. the name of the action schema in lower case, where the input to action is some combination of objects.
	
	
	
\end{definition}


Predicates which contain Terms in its arguments are referred to as fluent predicates, while grounded predicates refer to predicates that only contain objects in its arguments. Given a mapping $\sigma$ from variables to objects any fluent predicate can become grounded.
\begin{definition} 
	Any fluent predicate cab be mapped to a grounded predicate, by a applying a mapping $\sigma$ to each of its arguments. As such the \texttt{grond} function takes a predicate $p$ and a mapping $\sigma$
	
		$\texttt{ground}(p(x_1,\dots,x_n),\sigma) = p(\sigma(x_1),\dots,\sigma(x_n)$ where $p$ is the name of the predicate.
		
	If the mapping is given in context, i.e. there is an action application in context of the discussion, then the shorthand $\texttt{ground}(p)$ can be used instead.
\end{definition}


To explain how an action is applied to a state we define a State-Transition function in accordance with set-theoretic planning.


\begin{definition} 
	State-Transition function $\gamma$ takes two inputs a State $S$ and an Action $A$. Its exact definition is:

	$\gamma(S,A) \triangleq 
	\left\{
	\begin{array}{ll}
		(S \setminus effect^-(A)) \cup effect^+(A) & \mbox{if } A~is~applicable~to~S  \\		
		undefined & otherwise 
	\end{array}
	\right.$ 
	
	where $effect^+$ are the positive changes, meaning the grounded predicates that are to be set \texttt{TRUE} in the new state $S'$. Conversely $effect^-$ are the negatives changes, which are the predicates to be set \texttt{FALSE} in $S'$. 
	
	\textbf{NB} For more feature rich PDDL versions, the $effect$ functions require both the domain, problem and state in order to get the actual effects; however we assume these are given from context unless specified otherwise.
	
\end{definition}

Furthermore, we define a single Transition as follows:

\begin{definition} 
	Transition $T$ refers to a triple (S,A,S'), where S is the original state, A is the action that was applied S and S' is the resulting state.
\end{definition}

Lastly in order for an agent to know its purpose when planning it must have a defined goal. Such a goal is defined within a PDDL Problem.

\begin{definition} A problem contains the knowledge of the domain it applies to, the objects that is specific to the problem, the initial state of the planner and the goal expressed as a goal description (see Def.  \ref{def:lrn:action-spec-def}).
	
	$
	\begin{array}{ll}
	Problem ~ P: & \\
	
	\begin{array}{ll}
	Doman & = Name ~ of ~ the ~ domain ~ it ~ is ~ based ~ on \\
	Objects ~ \mathcal{O} & = \{o_1,\dots,o_n\}			 \\  
	Init-State& = A ~ Set ~ of ~ Grounded ~ Predicates \\
	Goal ~ \mathcal{G} & = Goal ~ Description(GD)
	\end{array}
	\end{array}$
\end{definition}

To use a PDDL specification an agent can with a domain and an accompanying problem generate a plan, which solves that specific problem.

\begin{definition} 
	A plan $P$ is defined as a list of action, solving a given problem.
	$P = (a_1,\dots,a_n)$
\end{definition}

Given this machinery an agent is thus capable of modeling the world and define actions that is applicable in the world. These definitions provides a framework for learning of action schema, which we will go into detail with, for both non-conditional and conditional action schemas.


	
	


\end{document}
