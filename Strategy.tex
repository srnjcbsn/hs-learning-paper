\documentclass[master.tex]{subfiles}
\begin{document}




	\begin{comment}
	
	    \section*{Sec1}
	
	    Learning has two important steps to it, the first is figuring out
	    what to do in order to learn about something and the second is analysis
	    of what occured to understand and actually learn about it.
	
	    
	    Analysing an outcome is about figuring out what occured and what did
	    not occur. For instance if one pressed
	
	    In a STRIPS domain, all actions are absolute thus we know that if
	    an action changed the state then the action's preconditions was satisfied
	    and all predicates added
    \end{comment}



    \section*{Learning Strategy}

    Learning is a gradual process of trial and error, testing and analysis.
    In the initial state of an agent it will know nothing, thus it must
    test and try its available actions, in order to understand what it
    is capable of. For instance, Imagine that you were given a two buttons
    you would have no way of knowning what occurs once you press them
    and if you have no teacher that can inform you, then only trough testing
    by pressing the buttons can you find out.
    
    In order for an agent to learn its action-schema it must use a strategy
    which informs it on what actions it must take in order to achieve
    its goals or learn something that allows it to achieve its goals.
    In the {[}Walsh paper{]} two such strategies are provided:
    \begin{itemize}
    \item An \emph{optimistic} strategy, that will always assume all actions
    does as much as possible thus a plan can always be found.
    \item A \emph{pessimistic} strategy which always assumes that actions undoes
    as much as possible thus it will never be able to achieve any (positive)
    goals.
    \end{itemize}
    Both of these strategies are equally valid approaches to learning.
    However they have different traits which we will define and analyse.
    \begin{description}
    \item [{Curious/Incurious}] A curious strategy will prefer to try unknown
    methods if they provide potentially better solutions while an incurious
    strategy will avoid the unknown and use what it knows. The curiosity
    level of a strategy is dependent on how it values unproven knowledge
    in relation to proven knowledge. \\
    Curious: $value(u)=value(k)$ \\
    Incurious: $value(u)<value(k)$ \\
    \emph{where u is unproven knownledge and k is proven knowledge.}\\
    A curious strategy has the advantage that it obtains knowledge from
    which a planner produce generally shorter plans. Opposite an incurious
    will generally quickly provide some plan to solve a problem even if
    it might be a slower solution. For instance in the button example
    imagine that the first button solved half of the all the goals each
    time it was pressed where as the other button solved all the goals,
    the incurious agent would press the first button and would now never
    consider to press the other button since it is already capable of
    solving all the goals (with two button presses). The curious agent
    would try both approaches and thus learn that it can use the second
    button for it to achieve the smallest plan, however imagine that there
    was hundreds of buttons and only the two first did anything, in this
    case the curious agent would have to press all of them before it could
    begin solve the problem.
    \item [{Self-sufficient/Help-seeking}] We define a strategy which favors
    or completely relies on a teacher assisting it as a Help-seeking strategy
    and a strategy which does not as a self-sufficient strategy. While
    the term Help-seeking carry a negative connotation, it is not necessarily
    a bad strategy, for instance if the agent operates in an environment
    which is irreversible or dangerous, then getting assistence would
    be preferable over trying things out for itself. For instance imagine
    a person that is put into a control room of nuclear powerplant, his
    learning strategy would not be to press the buttons until he figured
    out the system, but rather look through the manuals.
    \item [{Goal~Driven}] A goal-driven strategy uses a sequence of problems
    that the agent should learn to solve. Thus, for each problem, the
    agent will gain enough knowledge to either solve it or correctly declare
    it unsolvable. Note that the knowledge obtained with this strategy
    is not necessarily sufficient to solve problems similar to the training
    problems, it only guarantees that the agent can correctly solve the
    given problems.
    \item [{Hypothesis~Driven}] A hypothesis driven strategy does not require
    a specific goal. Instead, it strives to make its hypothesis as precise
    as possible. For inductive actionschema learning this means to prove
    or disprove as many precondition and/or effects as possible. Another
    example is academic research which is predominantly a hypothesis driven
    learning strategy, as there is not always a clear problem to be solved
    but rather the goal of to maximize the knowledge and understanding
    of the subject. As evident this type of strategy might lead the agent
    to learn something that is irrelevant to all problems it will ever
    encounter, furthermore if the hypothesis space is large the agent
    might never learn knowledge which is relevant to the problems given
    to it.
    \end{description}
    By defining the different traits of the strategies we are capable
    of analysing the algorithms given in the {[}Walsh paper{]}.
    \begin{description}
    \item [{Optimisic}] The \emph{optimistic} strategy that assumes that actions
    can always be applied (unless proven otherwise) and contain \emph{all}
    positive effects (unless proven otherwise). Using this strategy, a
    plan can always be found (unless action schemas or goals have negative
    predicates).\end{description}
    \begin{itemize}
    \item Curious\\
    The strategy is using a curious approach because the that actions
    which are unknown or have not been tested will always be chosen over
    actions that has as these actions are assumed to do more than they
    are supposed to.
    \item Self-sufficient\\
    The strategy always assumes it can solve all problems even if that
    is not the case, thus it will never ask a teacher unless the problem
    is actually unsolvable.
    \item Goal Driven\\
    The algoritm is based on providing it a goal it will try to solve\end{itemize}
    \begin{description}
    \item [{Pessimistic}] A \emph{pessimistic} strategy assumes that actions
    can \emph{never} be applied (unless proven otherwise), and that they
    have all \emph{negative} effects (unless proven otherwise). \end{description}
    \begin{itemize}
    \item Incurious\\
    The strategy used will purposely assume that things that it does not
    know about does less than they actually do thus it will never try
    things it does not know about.
    \item Help-seeking\\
    The strategy will ask for help as much as possible, meaning that even
    if it is only one action it does not know about. It will ask a teacher
    to explain what it does over trying it out for itself.
    \item Goal Driven\\
    Both of the algorithm given in the paper are Goal Driven.
    \end{itemize}
    As we can see both of these strategies are extremes of the different
    traits, thus by changing the traits we can define new strategies.
    \begin{description}
    \item [{Confident~(worker~agent)}] This strategy about learning only
    what is necessary to complete a task and learn it as quickly as possible,
    it also needs to be self-reliant and simply test things for itself.
    We see this in our world in the workplace where people are given a
    job and is expected to find a solution as quickly as possible.\end{description}
    \begin{itemize}
    \item Incurious\\
    The strategy relies on being confident in what it has learned and
    thus does not try out new things.
    \item Self-sufficient\\
    It solves without the requirement of a teacher as there might not
    exist information on the specific problem it is handling.
    \item Goal Driven\\
    It is given a task and is expected to solve it.\end{itemize}
    \begin{description}
    \item [{Researcher}] A learning strategy appropiate for researching a specific
    field the agent would not be time constrained nor would it necessarily
    have a specific goal in mind, rather it would be open ended.\end{description}
    \begin{itemize}
    \item Curious\\
    A research strategy is curious about all approaches to a problem and
    thus it will try to learn everything.
    \item Self-sufficient\\
    When researching something new and unexplored it there would not be
    any available teacher for the subject thus a strategy which relies
    on a teacher would be useless for research.
    \item Hypothesis Driven\\
    Unlike the other strategies a researh strategy is not focusing on
    only one specific goal but rather to understand the entire domain.
    \end{itemize}

    \section*{Algorithm}
    \begin{enumerate}
    \item Determine action based on strategy
    \item If no action is found ask teacher for help, repeat from step 1 if
    teacher can help. End if not.
    \item Perform action
    \item Analyse new state through inductive reasoning
    \item If reached goal: End
    \item Repeat\end{enumerate}


\end{document}
