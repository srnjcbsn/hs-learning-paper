\documentclass[Master.tex]{subfiles}
\begin{document}

<TODO: Introduction, explaining what non-conditional learning is about and what the structure of the section will be>

For an action schema $A$, $\mathbb{P}_A$ is the set of all fluent predicates in the domain that can be applied with the arguments of $A$.
\[
\mathbb{P}_A = \left\{
p \left( x_1, \dots, x_{|p|} \right)
\; | \; \left\{ x_1, \dots, x_{|p|} \right\} \subseteq params(A)
\right\}
\]


(TO Thomas: Also note we use $U$ for unproven predicates, $D$ for disproven predicates and $K$ for proven predicates, )

\section{Learning effects (Major rewrite required/Add examples/Add theorems/Use disproven etc../ all in all a dated file )}
    \subfile{NonConditional/Effects}

\section{Learning Preconditions}
    \subfile{NonConditional/Preconditions}

\section{Hypothesis construction}
	\subfile{NonConditional/HypothesisConstruction}

\end{document}
