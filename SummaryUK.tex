\chapter{Summary (English)}


This thesis attempts to solve the problem of action schema learning.
That is, learning actions in a domain without knowledge of those actions' effects and precondition, and only by analyzing the state-transitions. The solution to these problems should not use any probability or evolutionary theories, but rather infer the actions, based only on analyzing the state-transitions. 
This thesis extends the work of \cite{Walsh2008},
by exploring the underlying properties of their work.
Because of this we are able to remove some of the restrictions in their model.

The goal of this thesis is to elucidate what it means to learn. Specifically we will focus on non-conditional actions, and attempt to provide a framework to solve the problem of conditional action schema learning.

We solve these problems using graph theory, set theory and logic reasoning.

