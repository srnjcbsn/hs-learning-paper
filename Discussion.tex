\documentclass[Master.tex]{subfiles}
\begin{document}

\section{Limitations on conditional action learning}
As discussed in <ref>, we have imposed several restrictions on the conditional action schemas that we can learn, the most important of which we will discuss here.

\paragraph*{No action parameters}
In order to narrow our study on conditional effects, we elected to disallow parametric actions. Thus, the actions we support 


\paragraph{No disjoint preconditions}
This restriction requires that the effect of a conditional must in some way be transitively related to the objects used to determine whether it should occur. As explained in Section~\ref{chp:ca}, this is vital for describing preconditions as connected paths. However, certain domains are not constructible under this restriction. 


\section{Missing: Time/Space complexity of non-condition action learning algorithm}

\section{Missing: Advantage/disadvantage of negative goals solution}

\section{Missing: Solving Hypergraph Collapsing problem}

\section{Missing: Thoughts on Simulation}

%\section{Discussion of Results}
%    \subfile{Discussion/DiscussionOfResults}
%
%\section{Planning}\label{sec:disc:planning}
%    \subfile{Discussion/Planning}
%
%\section{Hypergraph Collapsing}
%    \subfile{Discussion/HGCollapsing}

\end{document}
