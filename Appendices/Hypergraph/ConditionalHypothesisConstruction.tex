\providecommand{\master}{../..}
\documentclass[\master/Master.tex]{subfiles}
\begin{document}

This section provides ideas of how planning with hypergraphs can be accomplished and shows the necessary changes to the planner. 


\begin{figure}
	\centering
	\begin{subfigure}{.5\textwidth}
		\centering
		\begin{pgfpicture}
  \pgfpathrectangle{\pgfpointorigin}{\pgfqpoint{200.0000bp}{200.0000bp}}
  \pgfusepath{use as bounding box}
  \begin{pgfscope}
    \definecolor{fc}{rgb}{0.0000,0.0000,0.0000}
    \pgfsetfillcolor{fc}
    \pgfsetlinewidth{0.5538bp}
    \definecolor{sc}{rgb}{0.0000,0.0000,0.0000}
    \pgfsetstrokecolor{sc}
    \pgfsetmiterjoin
    \pgfsetbuttcap
    \pgfpathqmoveto{147.9167bp}{141.6667bp}
    \pgfpathqcurveto{147.9167bp}{150.8714bp}{140.4547bp}{158.3333bp}{131.2500bp}{158.3333bp}
    \pgfpathqcurveto{122.0453bp}{158.3333bp}{114.5833bp}{150.8714bp}{114.5833bp}{141.6667bp}
    \pgfpathqcurveto{114.5833bp}{132.4619bp}{122.0453bp}{125.0000bp}{131.2500bp}{125.0000bp}
    \pgfpathqcurveto{140.4547bp}{125.0000bp}{147.9167bp}{132.4619bp}{147.9167bp}{141.6667bp}
    \pgfpathclose
    \pgfusepathqfillstroke
  \end{pgfscope}
  \begin{pgfscope}
    \definecolor{fc}{rgb}{1.0000,1.0000,1.0000}
    \pgfsetfillcolor{fc}
    \pgfsetlinewidth{0.5538bp}
    \definecolor{sc}{rgb}{1.0000,1.0000,1.0000}
    \pgfsetstrokecolor{sc}
    \pgfsetmiterjoin
    \pgfsetbuttcap
    \pgfpathqmoveto{147.0833bp}{141.6667bp}
    \pgfpathqcurveto{147.0833bp}{150.4112bp}{139.9945bp}{157.5000bp}{131.2500bp}{157.5000bp}
    \pgfpathqcurveto{122.5055bp}{157.5000bp}{115.4167bp}{150.4112bp}{115.4167bp}{141.6667bp}
    \pgfpathqcurveto{115.4167bp}{132.9222bp}{122.5055bp}{125.8333bp}{131.2500bp}{125.8333bp}
    \pgfpathqcurveto{139.9945bp}{125.8333bp}{147.0833bp}{132.9222bp}{147.0833bp}{141.6667bp}
    \pgfpathclose
    \pgfusepathqfillstroke
  \end{pgfscope}
  \begin{pgfscope}
    \definecolor{fc}{rgb}{0.0000,0.0000,0.0000}
    \pgfsetfillcolor{fc}
    \pgfsetlinewidth{0.5538bp}
    \definecolor{sc}{rgb}{0.0000,0.0000,0.0000}
    \pgfsetstrokecolor{sc}
    \pgfsetmiterjoin
    \pgfsetbuttcap
    \pgfpathqmoveto{85.4167bp}{16.6667bp}
    \pgfpathqcurveto{85.4167bp}{25.8714bp}{77.9547bp}{33.3333bp}{68.7500bp}{33.3333bp}
    \pgfpathqcurveto{59.5453bp}{33.3333bp}{52.0833bp}{25.8714bp}{52.0833bp}{16.6667bp}
    \pgfpathqcurveto{52.0833bp}{7.4619bp}{59.5453bp}{0.0000bp}{68.7500bp}{0.0000bp}
    \pgfpathqcurveto{77.9547bp}{0.0000bp}{85.4167bp}{7.4619bp}{85.4167bp}{16.6667bp}
    \pgfpathclose
    \pgfusepathqfillstroke
  \end{pgfscope}
  \begin{pgfscope}
    \definecolor{fc}{rgb}{1.0000,1.0000,1.0000}
    \pgfsetfillcolor{fc}
    \pgfsetlinewidth{0.5538bp}
    \definecolor{sc}{rgb}{1.0000,1.0000,1.0000}
    \pgfsetstrokecolor{sc}
    \pgfsetmiterjoin
    \pgfsetbuttcap
    \pgfpathqmoveto{84.5833bp}{16.6667bp}
    \pgfpathqcurveto{84.5833bp}{25.4112bp}{77.4945bp}{32.5000bp}{68.7500bp}{32.5000bp}
    \pgfpathqcurveto{60.0055bp}{32.5000bp}{52.9167bp}{25.4112bp}{52.9167bp}{16.6667bp}
    \pgfpathqcurveto{52.9167bp}{7.9222bp}{60.0055bp}{0.8333bp}{68.7500bp}{0.8333bp}
    \pgfpathqcurveto{77.4945bp}{0.8333bp}{84.5833bp}{7.9222bp}{84.5833bp}{16.6667bp}
    \pgfpathclose
    \pgfusepathqfillstroke
  \end{pgfscope}
  \begin{pgfscope}
    \definecolor{fc}{rgb}{0.0000,0.0000,0.0000}
    \pgfsetfillcolor{fc}
    \pgfsetlinewidth{0.5538bp}
    \definecolor{sc}{rgb}{0.0000,0.0000,0.0000}
    \pgfsetstrokecolor{sc}
    \pgfsetmiterjoin
    \pgfsetbuttcap
    \pgfpathqmoveto{85.4167bp}{100.0000bp}
    \pgfpathqcurveto{85.4167bp}{109.2047bp}{77.9547bp}{116.6667bp}{68.7500bp}{116.6667bp}
    \pgfpathqcurveto{59.5453bp}{116.6667bp}{52.0833bp}{109.2047bp}{52.0833bp}{100.0000bp}
    \pgfpathqlineto{52.0833bp}{58.3333bp}
    \pgfpathqcurveto{52.0833bp}{49.1286bp}{59.5453bp}{41.6667bp}{68.7500bp}{41.6667bp}
    \pgfpathqcurveto{77.9547bp}{41.6667bp}{85.4167bp}{49.1286bp}{85.4167bp}{58.3333bp}
    \pgfpathqlineto{85.4167bp}{100.0000bp}
    \pgfpathqcurveto{85.4167bp}{109.2047bp}{77.9547bp}{116.6667bp}{68.7500bp}{116.6667bp}
    \pgfpathqcurveto{59.5453bp}{116.6667bp}{52.0833bp}{109.2047bp}{52.0833bp}{100.0000bp}
    \pgfpathqlineto{52.0833bp}{58.3333bp}
    \pgfpathqcurveto{52.0833bp}{49.1286bp}{59.5453bp}{41.6667bp}{68.7500bp}{41.6667bp}
    \pgfpathqcurveto{77.9547bp}{41.6667bp}{85.4167bp}{49.1286bp}{85.4167bp}{58.3333bp}
    \pgfpathqlineto{85.4167bp}{100.0000bp}
    \pgfpathclose
    \pgfusepathqfillstroke
  \end{pgfscope}
  \begin{pgfscope}
    \definecolor{fc}{rgb}{1.0000,1.0000,1.0000}
    \pgfsetfillcolor{fc}
    \pgfsetlinewidth{0.5538bp}
    \definecolor{sc}{rgb}{1.0000,1.0000,1.0000}
    \pgfsetstrokecolor{sc}
    \pgfsetmiterjoin
    \pgfsetbuttcap
    \pgfpathqmoveto{84.5833bp}{100.0000bp}
    \pgfpathqcurveto{84.5833bp}{108.7445bp}{77.4945bp}{115.8333bp}{68.7500bp}{115.8333bp}
    \pgfpathqcurveto{60.0055bp}{115.8333bp}{52.9167bp}{108.7445bp}{52.9167bp}{100.0000bp}
    \pgfpathqlineto{52.9167bp}{58.3333bp}
    \pgfpathqcurveto{52.9167bp}{49.5888bp}{60.0055bp}{42.5000bp}{68.7500bp}{42.5000bp}
    \pgfpathqcurveto{77.4945bp}{42.5000bp}{84.5833bp}{49.5888bp}{84.5833bp}{58.3333bp}
    \pgfpathqlineto{84.5833bp}{100.0000bp}
    \pgfpathqcurveto{84.5833bp}{108.7445bp}{77.4945bp}{115.8333bp}{68.7500bp}{115.8333bp}
    \pgfpathqcurveto{60.0055bp}{115.8333bp}{52.9167bp}{108.7445bp}{52.9167bp}{100.0000bp}
    \pgfpathqlineto{52.9167bp}{58.3333bp}
    \pgfpathqcurveto{52.9167bp}{49.5888bp}{60.0055bp}{42.5000bp}{68.7500bp}{42.5000bp}
    \pgfpathqcurveto{77.4945bp}{42.5000bp}{84.5833bp}{49.5888bp}{84.5833bp}{58.3333bp}
    \pgfpathqlineto{84.5833bp}{100.0000bp}
    \pgfpathclose
    \pgfusepathqfillstroke
  \end{pgfscope}
  \begin{pgfscope}
    \definecolor{fc}{rgb}{0.0000,0.0000,0.0000}
    \pgfsetfillcolor{fc}
    \pgfsetlinewidth{0.5538bp}
    \definecolor{sc}{rgb}{0.0000,0.0000,0.0000}
    \pgfsetstrokecolor{sc}
    \pgfsetmiterjoin
    \pgfsetbuttcap
    \pgfpathqmoveto{68.7500bp}{200.0000bp}
    \pgfpathqcurveto{59.5453bp}{200.0000bp}{52.0833bp}{192.5381bp}{52.0833bp}{183.3333bp}
    \pgfpathqcurveto{52.0833bp}{174.1286bp}{59.5453bp}{166.6667bp}{68.7500bp}{166.6667bp}
    \pgfpathqlineto{110.4167bp}{166.6667bp}
    \pgfpathqcurveto{119.6214bp}{166.6667bp}{127.0833bp}{174.1286bp}{127.0833bp}{183.3333bp}
    \pgfpathqcurveto{127.0833bp}{192.5381bp}{119.6214bp}{200.0000bp}{110.4167bp}{200.0000bp}
    \pgfpathqcurveto{105.9964bp}{200.0000bp}{101.7572bp}{198.2441bp}{98.6316bp}{195.1184bp}
    \pgfpathqlineto{56.9649bp}{153.4518bp}
    \pgfpathqcurveto{50.4561bp}{146.9430bp}{50.4561bp}{136.3903bp}{56.9649bp}{129.8816bp}
    \pgfpathqcurveto{63.4736bp}{123.3728bp}{74.0264bp}{123.3728bp}{80.5351bp}{129.8816bp}
    \pgfpathqcurveto{83.6607bp}{133.0072bp}{85.4167bp}{137.2464bp}{85.4167bp}{141.6667bp}
    \pgfpathqlineto{85.4167bp}{183.3333bp}
    \pgfpathqcurveto{85.4167bp}{192.5381bp}{77.9547bp}{200.0000bp}{68.7500bp}{200.0000bp}
    \pgfpathqcurveto{59.5453bp}{200.0000bp}{52.0833bp}{192.5381bp}{52.0833bp}{183.3333bp}
    \pgfpathqlineto{52.0833bp}{141.6667bp}
    \pgfpathqcurveto{52.0833bp}{132.4619bp}{59.5453bp}{125.0000bp}{68.7500bp}{125.0000bp}
    \pgfpathqcurveto{73.1703bp}{125.0000bp}{77.4095bp}{126.7559bp}{80.5351bp}{129.8816bp}
    \pgfpathqlineto{122.2018bp}{171.5482bp}
    \pgfpathqcurveto{128.7105bp}{178.0570bp}{128.7105bp}{188.6097bp}{122.2018bp}{195.1184bp}
    \pgfpathqcurveto{119.0762bp}{198.2441bp}{114.8369bp}{200.0000bp}{110.4167bp}{200.0000bp}
    \pgfpathqlineto{68.7500bp}{200.0000bp}
    \pgfpathclose
    \pgfusepathqfillstroke
  \end{pgfscope}
  \begin{pgfscope}
    \definecolor{fc}{rgb}{1.0000,1.0000,1.0000}
    \pgfsetfillcolor{fc}
    \pgfsetlinewidth{0.5538bp}
    \definecolor{sc}{rgb}{1.0000,1.0000,1.0000}
    \pgfsetstrokecolor{sc}
    \pgfsetmiterjoin
    \pgfsetbuttcap
    \pgfpathqmoveto{68.7500bp}{199.1667bp}
    \pgfpathqcurveto{60.0055bp}{199.1667bp}{52.9167bp}{192.0778bp}{52.9167bp}{183.3333bp}
    \pgfpathqcurveto{52.9167bp}{174.5888bp}{60.0055bp}{167.5000bp}{68.7500bp}{167.5000bp}
    \pgfpathqlineto{110.4167bp}{167.5000bp}
    \pgfpathqcurveto{119.1612bp}{167.5000bp}{126.2500bp}{174.5888bp}{126.2500bp}{183.3333bp}
    \pgfpathqcurveto{126.2500bp}{192.0778bp}{119.1612bp}{199.1667bp}{110.4167bp}{199.1667bp}
    \pgfpathqcurveto{106.2174bp}{199.1667bp}{102.1901bp}{197.4985bp}{99.2208bp}{194.5292bp}
    \pgfpathqlineto{57.5541bp}{152.8625bp}
    \pgfpathqcurveto{51.3708bp}{146.6792bp}{51.3708bp}{136.6541bp}{57.5541bp}{130.4708bp}
    \pgfpathqcurveto{63.7374bp}{124.2875bp}{73.7626bp}{124.2875bp}{79.9459bp}{130.4708bp}
    \pgfpathqcurveto{82.9152bp}{133.4401bp}{84.5833bp}{137.4674bp}{84.5833bp}{141.6667bp}
    \pgfpathqlineto{84.5833bp}{183.3333bp}
    \pgfpathqcurveto{84.5833bp}{192.0778bp}{77.4945bp}{199.1667bp}{68.7500bp}{199.1667bp}
    \pgfpathqcurveto{60.0055bp}{199.1667bp}{52.9167bp}{192.0778bp}{52.9167bp}{183.3333bp}
    \pgfpathqlineto{52.9167bp}{141.6667bp}
    \pgfpathqcurveto{52.9167bp}{132.9222bp}{60.0055bp}{125.8333bp}{68.7500bp}{125.8333bp}
    \pgfpathqcurveto{72.9493bp}{125.8333bp}{76.9765bp}{127.5015bp}{79.9459bp}{130.4708bp}
    \pgfpathqlineto{121.6125bp}{172.1375bp}
    \pgfpathqcurveto{127.7958bp}{178.3208bp}{127.7958bp}{188.3459bp}{121.6125bp}{194.5292bp}
    \pgfpathqcurveto{118.6432bp}{197.4985bp}{114.6159bp}{199.1667bp}{110.4167bp}{199.1667bp}
    \pgfpathqlineto{68.7500bp}{199.1667bp}
    \pgfpathclose
    \pgfusepathqfillstroke
  \end{pgfscope}
  \begin{pgfscope}
    \definecolor{fc}{rgb}{0.0000,0.0000,0.0000}
    \pgfsetfillcolor{fc}
    \pgftransformcm{1.0000}{0.0000}{0.0000}{1.0000}{\pgfqpoint{135.4167bp}{145.8333bp}}
    \pgftransformscale{1.0417}
    \pgftext[base,left]{$g_2$}
  \end{pgfscope}
  \begin{pgfscope}
    \definecolor{fc}{rgb}{0.0000,0.0000,0.0000}
    \pgfsetfillcolor{fc}
    \pgfsetlinewidth{0.5538bp}
    \definecolor{sc}{rgb}{0.0000,0.0000,0.0000}
    \pgfsetstrokecolor{sc}
    \pgfsetmiterjoin
    \pgfsetbuttcap
    \pgfpathqmoveto{133.3333bp}{141.6667bp}
    \pgfpathqcurveto{133.3333bp}{142.8173bp}{132.4006bp}{143.7500bp}{131.2500bp}{143.7500bp}
    \pgfpathqcurveto{130.0994bp}{143.7500bp}{129.1667bp}{142.8173bp}{129.1667bp}{141.6667bp}
    \pgfpathqcurveto{129.1667bp}{140.5161bp}{130.0994bp}{139.5833bp}{131.2500bp}{139.5833bp}
    \pgfpathqcurveto{132.4006bp}{139.5833bp}{133.3333bp}{140.5161bp}{133.3333bp}{141.6667bp}
    \pgfpathclose
    \pgfusepathqfillstroke
  \end{pgfscope}
  \begin{pgfscope}
    \definecolor{fc}{rgb}{0.0000,0.0000,0.0000}
    \pgfsetfillcolor{fc}
    \pgftransformcm{1.0000}{0.0000}{0.0000}{1.0000}{\pgfqpoint{114.5833bp}{187.5000bp}}
    \pgftransformscale{1.0417}
    \pgftext[base,left]{$g_1$}
  \end{pgfscope}
  \begin{pgfscope}
    \definecolor{fc}{rgb}{0.0000,0.0000,0.0000}
    \pgfsetfillcolor{fc}
    \pgfsetlinewidth{0.5538bp}
    \definecolor{sc}{rgb}{0.0000,0.0000,0.0000}
    \pgfsetstrokecolor{sc}
    \pgfsetmiterjoin
    \pgfsetbuttcap
    \pgfpathqmoveto{112.5000bp}{183.3333bp}
    \pgfpathqcurveto{112.5000bp}{184.4839bp}{111.5673bp}{185.4167bp}{110.4167bp}{185.4167bp}
    \pgfpathqcurveto{109.2661bp}{185.4167bp}{108.3333bp}{184.4839bp}{108.3333bp}{183.3333bp}
    \pgfpathqcurveto{108.3333bp}{182.1827bp}{109.2661bp}{181.2500bp}{110.4167bp}{181.2500bp}
    \pgfpathqcurveto{111.5673bp}{181.2500bp}{112.5000bp}{182.1827bp}{112.5000bp}{183.3333bp}
    \pgfpathclose
    \pgfusepathqfillstroke
  \end{pgfscope}
  \begin{pgfscope}
    \pgfsetlinewidth{1.0383bp}
    \definecolor{sc}{rgb}{0.0000,0.0000,0.0000}
    \pgfsetstrokecolor{sc}
    \pgfsetmiterjoin
    \pgfsetbuttcap
    \pgfpathqmoveto{110.4167bp}{183.3333bp}
    \pgfpathqlineto{131.2500bp}{141.6667bp}
    \pgfusepathqstroke
  \end{pgfscope}
  \begin{pgfscope}
    \definecolor{fc}{rgb}{0.0000,0.0000,0.0000}
    \pgfsetfillcolor{fc}
    \pgfusepathqfill
  \end{pgfscope}
  \begin{pgfscope}
    \definecolor{fc}{rgb}{0.0000,0.0000,0.0000}
    \pgfsetfillcolor{fc}
    \pgfusepathqfill
  \end{pgfscope}
  \begin{pgfscope}
    \definecolor{fc}{rgb}{0.0000,0.0000,0.0000}
    \pgfsetfillcolor{fc}
    \pgfusepathqfill
  \end{pgfscope}
  \begin{pgfscope}
    \definecolor{fc}{rgb}{0.0000,0.0000,0.0000}
    \pgfsetfillcolor{fc}
    \pgfusepathqfill
  \end{pgfscope}
  \begin{pgfscope}
    \definecolor{fc}{rgb}{0.0000,0.0000,0.0000}
    \pgfsetfillcolor{fc}
    \pgftransformcm{1.0000}{0.0000}{0.0000}{1.0000}{\pgfqpoint{72.9167bp}{20.8333bp}}
    \pgftransformscale{1.0417}
    \pgftext[base,left]{$f_2$}
  \end{pgfscope}
  \begin{pgfscope}
    \definecolor{fc}{rgb}{0.0000,0.0000,0.0000}
    \pgfsetfillcolor{fc}
    \pgfsetlinewidth{0.5538bp}
    \definecolor{sc}{rgb}{0.0000,0.0000,0.0000}
    \pgfsetstrokecolor{sc}
    \pgfsetmiterjoin
    \pgfsetbuttcap
    \pgfpathqmoveto{70.8333bp}{16.6667bp}
    \pgfpathqcurveto{70.8333bp}{17.8173bp}{69.9006bp}{18.7500bp}{68.7500bp}{18.7500bp}
    \pgfpathqcurveto{67.5994bp}{18.7500bp}{66.6667bp}{17.8173bp}{66.6667bp}{16.6667bp}
    \pgfpathqcurveto{66.6667bp}{15.5161bp}{67.5994bp}{14.5833bp}{68.7500bp}{14.5833bp}
    \pgfpathqcurveto{69.9006bp}{14.5833bp}{70.8333bp}{15.5161bp}{70.8333bp}{16.6667bp}
    \pgfpathclose
    \pgfusepathqfillstroke
  \end{pgfscope}
  \begin{pgfscope}
    \definecolor{fc}{rgb}{0.0000,0.0000,0.0000}
    \pgfsetfillcolor{fc}
    \pgftransformcm{1.0000}{0.0000}{0.0000}{1.0000}{\pgfqpoint{72.9167bp}{62.5000bp}}
    \pgftransformscale{1.0417}
    \pgftext[base,left]{$f_1$}
  \end{pgfscope}
  \begin{pgfscope}
    \definecolor{fc}{rgb}{0.0000,0.0000,0.0000}
    \pgfsetfillcolor{fc}
    \pgfsetlinewidth{0.5538bp}
    \definecolor{sc}{rgb}{0.0000,0.0000,0.0000}
    \pgfsetstrokecolor{sc}
    \pgfsetmiterjoin
    \pgfsetbuttcap
    \pgfpathqmoveto{70.8333bp}{58.3333bp}
    \pgfpathqcurveto{70.8333bp}{59.4839bp}{69.9006bp}{60.4167bp}{68.7500bp}{60.4167bp}
    \pgfpathqcurveto{67.5994bp}{60.4167bp}{66.6667bp}{59.4839bp}{66.6667bp}{58.3333bp}
    \pgfpathqcurveto{66.6667bp}{57.1827bp}{67.5994bp}{56.2500bp}{68.7500bp}{56.2500bp}
    \pgfpathqcurveto{69.9006bp}{56.2500bp}{70.8333bp}{57.1827bp}{70.8333bp}{58.3333bp}
    \pgfpathclose
    \pgfusepathqfillstroke
  \end{pgfscope}
  \begin{pgfscope}
    \pgfsetlinewidth{1.0383bp}
    \definecolor{sc}{rgb}{0.0000,0.0000,0.0000}
    \pgfsetstrokecolor{sc}
    \pgfsetmiterjoin
    \pgfsetbuttcap
    \pgfpathqmoveto{68.7500bp}{58.3333bp}
    \pgfpathqlineto{68.7500bp}{16.6667bp}
    \pgfusepathqstroke
  \end{pgfscope}
  \begin{pgfscope}
    \definecolor{fc}{rgb}{0.0000,0.0000,0.0000}
    \pgfsetfillcolor{fc}
    \pgfusepathqfill
  \end{pgfscope}
  \begin{pgfscope}
    \definecolor{fc}{rgb}{0.0000,0.0000,0.0000}
    \pgfsetfillcolor{fc}
    \pgfusepathqfill
  \end{pgfscope}
  \begin{pgfscope}
    \definecolor{fc}{rgb}{0.0000,0.0000,0.0000}
    \pgfsetfillcolor{fc}
    \pgfusepathqfill
  \end{pgfscope}
  \begin{pgfscope}
    \definecolor{fc}{rgb}{0.0000,0.0000,0.0000}
    \pgfsetfillcolor{fc}
    \pgfusepathqfill
  \end{pgfscope}
  \begin{pgfscope}
    \definecolor{fc}{rgb}{0.0000,0.0000,0.0000}
    \pgfsetfillcolor{fc}
    \pgftransformcm{1.0000}{0.0000}{0.0000}{1.0000}{\pgfqpoint{72.9167bp}{104.1667bp}}
    \pgftransformscale{1.0417}
    \pgftext[base,left]{$p_2$}
  \end{pgfscope}
  \begin{pgfscope}
    \definecolor{fc}{rgb}{0.0000,0.0000,0.0000}
    \pgfsetfillcolor{fc}
    \pgfsetlinewidth{0.5538bp}
    \definecolor{sc}{rgb}{0.0000,0.0000,0.0000}
    \pgfsetstrokecolor{sc}
    \pgfsetmiterjoin
    \pgfsetbuttcap
    \pgfpathqmoveto{70.8333bp}{100.0000bp}
    \pgfpathqcurveto{70.8333bp}{101.1506bp}{69.9006bp}{102.0833bp}{68.7500bp}{102.0833bp}
    \pgfpathqcurveto{67.5994bp}{102.0833bp}{66.6667bp}{101.1506bp}{66.6667bp}{100.0000bp}
    \pgfpathqcurveto{66.6667bp}{98.8494bp}{67.5994bp}{97.9167bp}{68.7500bp}{97.9167bp}
    \pgfpathqcurveto{69.9006bp}{97.9167bp}{70.8333bp}{98.8494bp}{70.8333bp}{100.0000bp}
    \pgfpathclose
    \pgfusepathqfillstroke
  \end{pgfscope}
  \begin{pgfscope}
    \definecolor{fc}{rgb}{0.0000,0.0000,0.0000}
    \pgfsetfillcolor{fc}
    \pgftransformcm{1.0000}{0.0000}{0.0000}{1.0000}{\pgfqpoint{72.9167bp}{145.8333bp}}
    \pgftransformscale{1.0417}
    \pgftext[base,left]{$p_1$}
  \end{pgfscope}
  \begin{pgfscope}
    \definecolor{fc}{rgb}{0.0000,0.0000,0.0000}
    \pgfsetfillcolor{fc}
    \pgfsetlinewidth{0.5538bp}
    \definecolor{sc}{rgb}{0.0000,0.0000,0.0000}
    \pgfsetstrokecolor{sc}
    \pgfsetmiterjoin
    \pgfsetbuttcap
    \pgfpathqmoveto{70.8333bp}{141.6667bp}
    \pgfpathqcurveto{70.8333bp}{142.8173bp}{69.9006bp}{143.7500bp}{68.7500bp}{143.7500bp}
    \pgfpathqcurveto{67.5994bp}{143.7500bp}{66.6667bp}{142.8173bp}{66.6667bp}{141.6667bp}
    \pgfpathqcurveto{66.6667bp}{140.5161bp}{67.5994bp}{139.5833bp}{68.7500bp}{139.5833bp}
    \pgfpathqcurveto{69.9006bp}{139.5833bp}{70.8333bp}{140.5161bp}{70.8333bp}{141.6667bp}
    \pgfpathclose
    \pgfusepathqfillstroke
  \end{pgfscope}
  \begin{pgfscope}
    \pgfsetlinewidth{1.0383bp}
    \definecolor{sc}{rgb}{0.0000,0.0000,0.0000}
    \pgfsetstrokecolor{sc}
    \pgfsetmiterjoin
    \pgfsetbuttcap
    \pgfpathqmoveto{68.7500bp}{141.6667bp}
    \pgfpathqlineto{68.7500bp}{100.0000bp}
    \pgfusepathqstroke
  \end{pgfscope}
  \begin{pgfscope}
    \definecolor{fc}{rgb}{0.0000,0.0000,0.0000}
    \pgfsetfillcolor{fc}
    \pgfusepathqfill
  \end{pgfscope}
  \begin{pgfscope}
    \definecolor{fc}{rgb}{0.0000,0.0000,0.0000}
    \pgfsetfillcolor{fc}
    \pgfusepathqfill
  \end{pgfscope}
  \begin{pgfscope}
    \definecolor{fc}{rgb}{0.0000,0.0000,0.0000}
    \pgfsetfillcolor{fc}
    \pgfusepathqfill
  \end{pgfscope}
  \begin{pgfscope}
    \definecolor{fc}{rgb}{0.0000,0.0000,0.0000}
    \pgfsetfillcolor{fc}
    \pgfusepathqfill
  \end{pgfscope}
  \begin{pgfscope}
    \definecolor{fc}{rgb}{0.0000,0.0000,0.0000}
    \pgfsetfillcolor{fc}
    \pgftransformcm{1.0000}{0.0000}{0.0000}{1.0000}{\pgfqpoint{72.9167bp}{187.5000bp}}
    \pgftransformscale{1.0417}
    \pgftext[base,left]{$q_1$}
  \end{pgfscope}
  \begin{pgfscope}
    \definecolor{fc}{rgb}{0.0000,0.0000,0.0000}
    \pgfsetfillcolor{fc}
    \pgfsetlinewidth{0.5538bp}
    \definecolor{sc}{rgb}{0.0000,0.0000,0.0000}
    \pgfsetstrokecolor{sc}
    \pgfsetmiterjoin
    \pgfsetbuttcap
    \pgfpathqmoveto{70.8333bp}{183.3333bp}
    \pgfpathqcurveto{70.8333bp}{184.4839bp}{69.9006bp}{185.4167bp}{68.7500bp}{185.4167bp}
    \pgfpathqcurveto{67.5994bp}{185.4167bp}{66.6667bp}{184.4839bp}{66.6667bp}{183.3333bp}
    \pgfpathqcurveto{66.6667bp}{182.1827bp}{67.5994bp}{181.2500bp}{68.7500bp}{181.2500bp}
    \pgfpathqcurveto{69.9006bp}{181.2500bp}{70.8333bp}{182.1827bp}{70.8333bp}{183.3333bp}
    \pgfpathclose
    \pgfusepathqfillstroke
  \end{pgfscope}
\end{pgfpicture}
  		
		\caption{\label{fig:ca:hypgraph-with-unproven} A hypergraph of the unproven preconditions for the effect $q$}
	\end{subfigure}%
	~ 
	\begin{subfigure}{.5\textwidth}
		\centering
		\begin{pgfpicture}
  \pgfpathrectangle{\pgfpointorigin}{\pgfqpoint{200.0000bp}{200.0000bp}}
  \pgfusepath{use as bounding box}
  \begin{pgfscope}
    \definecolor{fc}{rgb}{0.0000,0.0000,0.0000}
    \pgfsetfillcolor{fc}
    \pgfsetlinewidth{0.5725bp}
    \definecolor{sc}{rgb}{0.0000,0.0000,0.0000}
    \pgfsetstrokecolor{sc}
    \pgfsetmiterjoin
    \pgfsetbuttcap
    \pgfpathqmoveto{95.1220bp}{75.6098bp}
    \pgfpathqcurveto{95.1220bp}{86.3860bp}{86.3860bp}{95.1220bp}{75.6098bp}{95.1220bp}
    \pgfpathqcurveto{64.8335bp}{95.1220bp}{56.0976bp}{86.3860bp}{56.0976bp}{75.6098bp}
    \pgfpathqlineto{56.0976bp}{26.8293bp}
    \pgfpathqcurveto{56.0976bp}{16.0530bp}{64.8335bp}{7.3171bp}{75.6098bp}{7.3171bp}
    \pgfpathqcurveto{86.3860bp}{7.3171bp}{95.1220bp}{16.0530bp}{95.1220bp}{26.8293bp}
    \pgfpathqlineto{95.1220bp}{75.6098bp}
    \pgfpathqcurveto{95.1220bp}{86.3860bp}{86.3860bp}{95.1220bp}{75.6098bp}{95.1220bp}
    \pgfpathqcurveto{64.8335bp}{95.1220bp}{56.0976bp}{86.3860bp}{56.0976bp}{75.6098bp}
    \pgfpathqlineto{56.0976bp}{26.8293bp}
    \pgfpathqcurveto{56.0976bp}{16.0530bp}{64.8335bp}{7.3171bp}{75.6098bp}{7.3171bp}
    \pgfpathqcurveto{86.3860bp}{7.3171bp}{95.1220bp}{16.0530bp}{95.1220bp}{26.8293bp}
    \pgfpathqlineto{95.1220bp}{75.6098bp}
    \pgfpathclose
    \pgfusepathqfillstroke
  \end{pgfscope}
  \begin{pgfscope}
    \definecolor{fc}{rgb}{1.0000,1.0000,1.0000}
    \pgfsetfillcolor{fc}
    \pgfsetlinewidth{0.5725bp}
    \definecolor{sc}{rgb}{1.0000,1.0000,1.0000}
    \pgfsetstrokecolor{sc}
    \pgfsetmiterjoin
    \pgfsetbuttcap
    \pgfpathqmoveto{94.1463bp}{75.6098bp}
    \pgfpathqcurveto{94.1463bp}{85.8472bp}{85.8472bp}{94.1463bp}{75.6098bp}{94.1463bp}
    \pgfpathqcurveto{65.3723bp}{94.1463bp}{57.0732bp}{85.8472bp}{57.0732bp}{75.6098bp}
    \pgfpathqlineto{57.0732bp}{26.8293bp}
    \pgfpathqcurveto{57.0732bp}{16.5918bp}{65.3723bp}{8.2927bp}{75.6098bp}{8.2927bp}
    \pgfpathqcurveto{85.8472bp}{8.2927bp}{94.1463bp}{16.5918bp}{94.1463bp}{26.8293bp}
    \pgfpathqlineto{94.1463bp}{75.6098bp}
    \pgfpathqcurveto{94.1463bp}{85.8472bp}{85.8472bp}{94.1463bp}{75.6098bp}{94.1463bp}
    \pgfpathqcurveto{65.3723bp}{94.1463bp}{57.0732bp}{85.8472bp}{57.0732bp}{75.6098bp}
    \pgfpathqlineto{57.0732bp}{26.8293bp}
    \pgfpathqcurveto{57.0732bp}{16.5918bp}{65.3723bp}{8.2927bp}{75.6098bp}{8.2927bp}
    \pgfpathqcurveto{85.8472bp}{8.2927bp}{94.1463bp}{16.5918bp}{94.1463bp}{26.8293bp}
    \pgfpathqlineto{94.1463bp}{75.6098bp}
    \pgfpathclose
    \pgfusepathqfillstroke
  \end{pgfscope}
  \begin{pgfscope}
    \definecolor{fc}{rgb}{0.0000,0.0000,0.0000}
    \pgfsetfillcolor{fc}
    \pgfsetlinewidth{0.5725bp}
    \definecolor{sc}{rgb}{0.0000,0.0000,0.0000}
    \pgfsetstrokecolor{sc}
    \pgfsetmiterjoin
    \pgfsetbuttcap
    \pgfpathqmoveto{75.6098bp}{192.6829bp}
    \pgfpathqcurveto{64.8335bp}{192.6829bp}{56.0976bp}{183.9470bp}{56.0976bp}{173.1707bp}
    \pgfpathqcurveto{56.0976bp}{162.3944bp}{64.8335bp}{153.6585bp}{75.6098bp}{153.6585bp}
    \pgfpathqlineto{124.3902bp}{153.6585bp}
    \pgfpathqcurveto{135.1665bp}{153.6585bp}{143.9024bp}{162.3944bp}{143.9024bp}{173.1707bp}
    \pgfpathqcurveto{143.9024bp}{183.9470bp}{135.1665bp}{192.6829bp}{124.3902bp}{192.6829bp}
    \pgfpathqcurveto{119.2153bp}{192.6829bp}{114.2523bp}{190.6272bp}{110.5930bp}{186.9679bp}
    \pgfpathqlineto{61.8126bp}{138.1874bp}
    \pgfpathqcurveto{54.1926bp}{130.5675bp}{54.1926bp}{118.2130bp}{61.8126bp}{110.5930bp}
    \pgfpathqcurveto{69.4325bp}{102.9731bp}{81.7870bp}{102.9731bp}{89.4070bp}{110.5930bp}
    \pgfpathqcurveto{93.0662bp}{114.2523bp}{95.1220bp}{119.2153bp}{95.1220bp}{124.3902bp}
    \pgfpathqlineto{95.1220bp}{173.1707bp}
    \pgfpathqcurveto{95.1220bp}{183.9470bp}{86.3860bp}{192.6829bp}{75.6098bp}{192.6829bp}
    \pgfpathqcurveto{64.8335bp}{192.6829bp}{56.0976bp}{183.9470bp}{56.0976bp}{173.1707bp}
    \pgfpathqlineto{56.0976bp}{124.3902bp}
    \pgfpathqcurveto{56.0976bp}{113.6140bp}{64.8335bp}{104.8780bp}{75.6098bp}{104.8780bp}
    \pgfpathqcurveto{80.7847bp}{104.8780bp}{85.7477bp}{106.9338bp}{89.4070bp}{110.5930bp}
    \pgfpathqlineto{138.1874bp}{159.3735bp}
    \pgfpathqcurveto{145.8074bp}{166.9935bp}{145.8074bp}{179.3480bp}{138.1874bp}{186.9679bp}
    \pgfpathqcurveto{134.5282bp}{190.6272bp}{129.5652bp}{192.6829bp}{124.3902bp}{192.6829bp}
    \pgfpathqlineto{75.6098bp}{192.6829bp}
    \pgfpathclose
    \pgfusepathqfillstroke
  \end{pgfscope}
  \begin{pgfscope}
    \definecolor{fc}{rgb}{1.0000,1.0000,1.0000}
    \pgfsetfillcolor{fc}
    \pgfsetlinewidth{0.5725bp}
    \definecolor{sc}{rgb}{1.0000,1.0000,1.0000}
    \pgfsetstrokecolor{sc}
    \pgfsetmiterjoin
    \pgfsetbuttcap
    \pgfpathqmoveto{75.6098bp}{191.7073bp}
    \pgfpathqcurveto{65.3723bp}{191.7073bp}{57.0732bp}{183.4082bp}{57.0732bp}{173.1707bp}
    \pgfpathqcurveto{57.0732bp}{162.9333bp}{65.3723bp}{154.6341bp}{75.6098bp}{154.6341bp}
    \pgfpathqlineto{124.3902bp}{154.6341bp}
    \pgfpathqcurveto{134.6277bp}{154.6341bp}{142.9268bp}{162.9333bp}{142.9268bp}{173.1707bp}
    \pgfpathqcurveto{142.9268bp}{183.4082bp}{134.6277bp}{191.7073bp}{124.3902bp}{191.7073bp}
    \pgfpathqcurveto{119.4740bp}{191.7073bp}{114.7592bp}{189.7544bp}{111.2829bp}{186.2781bp}
    \pgfpathqlineto{62.5024bp}{137.4976bp}
    \pgfpathqcurveto{55.2634bp}{130.2586bp}{55.2634bp}{118.5219bp}{62.5024bp}{111.2829bp}
    \pgfpathqcurveto{69.7414bp}{104.0439bp}{81.4781bp}{104.0439bp}{88.7171bp}{111.2829bp}
    \pgfpathqcurveto{92.1934bp}{114.7592bp}{94.1463bp}{119.4740bp}{94.1463bp}{124.3902bp}
    \pgfpathqlineto{94.1463bp}{173.1707bp}
    \pgfpathqcurveto{94.1463bp}{183.4082bp}{85.8472bp}{191.7073bp}{75.6098bp}{191.7073bp}
    \pgfpathqcurveto{65.3723bp}{191.7073bp}{57.0732bp}{183.4082bp}{57.0732bp}{173.1707bp}
    \pgfpathqlineto{57.0732bp}{124.3902bp}
    \pgfpathqcurveto{57.0732bp}{114.1528bp}{65.3723bp}{105.8537bp}{75.6098bp}{105.8537bp}
    \pgfpathqcurveto{80.5260bp}{105.8537bp}{85.2408bp}{107.8066bp}{88.7171bp}{111.2829bp}
    \pgfpathqlineto{137.4976bp}{160.0634bp}
    \pgfpathqcurveto{144.7366bp}{167.3024bp}{144.7366bp}{179.0391bp}{137.4976bp}{186.2781bp}
    \pgfpathqcurveto{134.0213bp}{189.7544bp}{129.3065bp}{191.7073bp}{124.3902bp}{191.7073bp}
    \pgfpathqlineto{75.6098bp}{191.7073bp}
    \pgfpathclose
    \pgfusepathqfillstroke
  \end{pgfscope}
  \begin{pgfscope}
    \definecolor{fc}{rgb}{0.0000,0.0000,0.0000}
    \pgfsetfillcolor{fc}
    \pgftransformshift{\pgfqpoint{129.2683bp}{178.0488bp}}
    \pgftransformscale{1.2195}
    \pgftext[base,left]{$g_1$}
  \end{pgfscope}
  \begin{pgfscope}
    \definecolor{fc}{rgb}{0.0000,0.0000,0.0000}
    \pgfsetfillcolor{fc}
    \pgfsetlinewidth{0.5725bp}
    \definecolor{sc}{rgb}{0.0000,0.0000,0.0000}
    \pgfsetstrokecolor{sc}
    \pgfsetmiterjoin
    \pgfsetbuttcap
    \pgfpathqmoveto{126.8293bp}{173.1707bp}
    \pgfpathqcurveto{126.8293bp}{174.5178bp}{125.7373bp}{175.6098bp}{124.3902bp}{175.6098bp}
    \pgfpathqcurveto{123.0432bp}{175.6098bp}{121.9512bp}{174.5178bp}{121.9512bp}{173.1707bp}
    \pgfpathqcurveto{121.9512bp}{171.8237bp}{123.0432bp}{170.7317bp}{124.3902bp}{170.7317bp}
    \pgfpathqcurveto{125.7373bp}{170.7317bp}{126.8293bp}{171.8237bp}{126.8293bp}{173.1707bp}
    \pgfpathclose
    \pgfusepathqfillstroke
  \end{pgfscope}
  \begin{pgfscope}
    \definecolor{fc}{rgb}{0.0000,0.0000,0.0000}
    \pgfsetfillcolor{fc}
    \pgftransformshift{\pgfqpoint{80.4878bp}{31.7073bp}}
    \pgftransformscale{1.2195}
    \pgftext[base,left]{$f_1$}
  \end{pgfscope}
  \begin{pgfscope}
    \definecolor{fc}{rgb}{0.0000,0.0000,0.0000}
    \pgfsetfillcolor{fc}
    \pgfsetlinewidth{0.5725bp}
    \definecolor{sc}{rgb}{0.0000,0.0000,0.0000}
    \pgfsetstrokecolor{sc}
    \pgfsetmiterjoin
    \pgfsetbuttcap
    \pgfpathqmoveto{78.0488bp}{26.8293bp}
    \pgfpathqcurveto{78.0488bp}{28.1763bp}{76.9568bp}{29.2683bp}{75.6098bp}{29.2683bp}
    \pgfpathqcurveto{74.2627bp}{29.2683bp}{73.1707bp}{28.1763bp}{73.1707bp}{26.8293bp}
    \pgfpathqcurveto{73.1707bp}{25.4822bp}{74.2627bp}{24.3902bp}{75.6098bp}{24.3902bp}
    \pgfpathqcurveto{76.9568bp}{24.3902bp}{78.0488bp}{25.4822bp}{78.0488bp}{26.8293bp}
    \pgfpathclose
    \pgfusepathqfillstroke
  \end{pgfscope}
  \begin{pgfscope}
    \definecolor{fc}{rgb}{0.0000,0.0000,0.0000}
    \pgfsetfillcolor{fc}
    \pgftransformshift{\pgfqpoint{80.4878bp}{80.4878bp}}
    \pgftransformscale{1.2195}
    \pgftext[base,left]{$p_2$}
  \end{pgfscope}
  \begin{pgfscope}
    \definecolor{fc}{rgb}{0.0000,0.0000,0.0000}
    \pgfsetfillcolor{fc}
    \pgfsetlinewidth{0.5725bp}
    \definecolor{sc}{rgb}{0.0000,0.0000,0.0000}
    \pgfsetstrokecolor{sc}
    \pgfsetmiterjoin
    \pgfsetbuttcap
    \pgfpathqmoveto{78.0488bp}{75.6098bp}
    \pgfpathqcurveto{78.0488bp}{76.9568bp}{76.9568bp}{78.0488bp}{75.6098bp}{78.0488bp}
    \pgfpathqcurveto{74.2627bp}{78.0488bp}{73.1707bp}{76.9568bp}{73.1707bp}{75.6098bp}
    \pgfpathqcurveto{73.1707bp}{74.2627bp}{74.2627bp}{73.1707bp}{75.6098bp}{73.1707bp}
    \pgfpathqcurveto{76.9568bp}{73.1707bp}{78.0488bp}{74.2627bp}{78.0488bp}{75.6098bp}
    \pgfpathclose
    \pgfusepathqfillstroke
  \end{pgfscope}
  \begin{pgfscope}
    \definecolor{fc}{rgb}{0.0000,0.0000,0.0000}
    \pgfsetfillcolor{fc}
    \pgftransformshift{\pgfqpoint{80.4878bp}{129.2683bp}}
    \pgftransformscale{1.2195}
    \pgftext[base,left]{$p_1$}
  \end{pgfscope}
  \begin{pgfscope}
    \definecolor{fc}{rgb}{0.0000,0.0000,0.0000}
    \pgfsetfillcolor{fc}
    \pgfsetlinewidth{0.5725bp}
    \definecolor{sc}{rgb}{0.0000,0.0000,0.0000}
    \pgfsetstrokecolor{sc}
    \pgfsetmiterjoin
    \pgfsetbuttcap
    \pgfpathqmoveto{78.0488bp}{124.3902bp}
    \pgfpathqcurveto{78.0488bp}{125.7373bp}{76.9568bp}{126.8293bp}{75.6098bp}{126.8293bp}
    \pgfpathqcurveto{74.2627bp}{126.8293bp}{73.1707bp}{125.7373bp}{73.1707bp}{124.3902bp}
    \pgfpathqcurveto{73.1707bp}{123.0432bp}{74.2627bp}{121.9512bp}{75.6098bp}{121.9512bp}
    \pgfpathqcurveto{76.9568bp}{121.9512bp}{78.0488bp}{123.0432bp}{78.0488bp}{124.3902bp}
    \pgfpathclose
    \pgfusepathqfillstroke
  \end{pgfscope}
  \begin{pgfscope}
    \pgfsetlinewidth{1.0735bp}
    \definecolor{sc}{rgb}{0.0000,0.0000,0.0000}
    \pgfsetstrokecolor{sc}
    \pgfsetmiterjoin
    \pgfsetbuttcap
    \pgfpathqmoveto{75.6098bp}{124.3902bp}
    \pgfpathqlineto{75.6098bp}{75.6098bp}
    \pgfusepathqstroke
  \end{pgfscope}
  \begin{pgfscope}
    \definecolor{fc}{rgb}{0.0000,0.0000,0.0000}
    \pgfsetfillcolor{fc}
    \pgfusepathqfill
  \end{pgfscope}
  \begin{pgfscope}
    \definecolor{fc}{rgb}{0.0000,0.0000,0.0000}
    \pgfsetfillcolor{fc}
    \pgfusepathqfill
  \end{pgfscope}
  \begin{pgfscope}
    \definecolor{fc}{rgb}{0.0000,0.0000,0.0000}
    \pgfsetfillcolor{fc}
    \pgfusepathqfill
  \end{pgfscope}
  \begin{pgfscope}
    \definecolor{fc}{rgb}{0.0000,0.0000,0.0000}
    \pgfsetfillcolor{fc}
    \pgfusepathqfill
  \end{pgfscope}
  \begin{pgfscope}
    \definecolor{fc}{rgb}{0.0000,0.0000,0.0000}
    \pgfsetfillcolor{fc}
    \pgftransformshift{\pgfqpoint{80.4878bp}{178.0488bp}}
    \pgftransformscale{1.2195}
    \pgftext[base,left]{$q_1$}
  \end{pgfscope}
  \begin{pgfscope}
    \definecolor{fc}{rgb}{0.0000,0.0000,0.0000}
    \pgfsetfillcolor{fc}
    \pgfsetlinewidth{0.5725bp}
    \definecolor{sc}{rgb}{0.0000,0.0000,0.0000}
    \pgfsetstrokecolor{sc}
    \pgfsetmiterjoin
    \pgfsetbuttcap
    \pgfpathqmoveto{78.0488bp}{173.1707bp}
    \pgfpathqcurveto{78.0488bp}{174.5178bp}{76.9568bp}{175.6098bp}{75.6098bp}{175.6098bp}
    \pgfpathqcurveto{74.2627bp}{175.6098bp}{73.1707bp}{174.5178bp}{73.1707bp}{173.1707bp}
    \pgfpathqcurveto{73.1707bp}{171.8237bp}{74.2627bp}{170.7317bp}{75.6098bp}{170.7317bp}
    \pgfpathqcurveto{76.9568bp}{170.7317bp}{78.0488bp}{171.8237bp}{78.0488bp}{173.1707bp}
    \pgfpathclose
    \pgfusepathqfillstroke
  \end{pgfscope}
  \begin{pgfscope}
    \definecolor{fc}{rgb}{0.0000,0.0000,0.0000}
    \pgfsetfillcolor{fc}
    \pgfsetfillopacity{0.0000}
    \pgfsetlinewidth{0.5725bp}
    \definecolor{sc}{rgb}{1.0000,0.0000,0.0000}
    \pgfsetstrokecolor{sc}
    \pgfsetmiterjoin
    \pgfsetbuttcap
    \pgfpathqmoveto{151.2195bp}{173.1707bp}
    \pgfpathqcurveto{151.2195bp}{187.9881bp}{139.2076bp}{200.0000bp}{124.3902bp}{200.0000bp}
    \pgfpathqcurveto{109.5728bp}{200.0000bp}{97.5610bp}{187.9881bp}{97.5610bp}{173.1707bp}
    \pgfpathqcurveto{97.5610bp}{158.3533bp}{109.5728bp}{146.3415bp}{124.3902bp}{146.3415bp}
    \pgfpathqcurveto{139.2076bp}{146.3415bp}{151.2195bp}{158.3533bp}{151.2195bp}{173.1707bp}
    \pgfpathclose
    \pgfusepathqfillstroke
  \end{pgfscope}
  \begin{pgfscope}
    \definecolor{fc}{rgb}{0.0000,0.0000,0.0000}
    \pgfsetfillcolor{fc}
    \pgfsetfillopacity{0.0000}
    \pgfsetlinewidth{0.5725bp}
    \definecolor{sc}{rgb}{1.0000,0.0000,0.0000}
    \pgfsetstrokecolor{sc}
    \pgfsetmiterjoin
    \pgfsetbuttcap
    \pgfpathqmoveto{102.4390bp}{26.8293bp}
    \pgfpathqcurveto{102.4390bp}{41.6467bp}{90.4272bp}{53.6585bp}{75.6098bp}{53.6585bp}
    \pgfpathqcurveto{60.7924bp}{53.6585bp}{48.7805bp}{41.6467bp}{48.7805bp}{26.8293bp}
    \pgfpathqcurveto{48.7805bp}{12.0119bp}{60.7924bp}{-0.0000bp}{75.6098bp}{-0.0000bp}
    \pgfpathqcurveto{90.4272bp}{-0.0000bp}{102.4390bp}{12.0119bp}{102.4390bp}{26.8293bp}
    \pgfpathclose
    \pgfusepathqfillstroke
  \end{pgfscope}
\end{pgfpicture}
  
		
		
		
		\caption{\label{fig:ca:hypgraph-with-cands}The red circle indicates that it is a candidate}
	\end{subfigure}
	\caption{ A visualization of a hypergraph and a single candidate-set with two candidates. }
	\label{fig:ca:two-hypgraphs-one-cands}
\end{figure}

\subsection{Optimistic Strategy}



To overcome these problems we suggest to approach them from another angle, 
instead of trying to convert our knowledge to something that a planner can understand; 
why not instead extend the planner to understand what our knowledge means.
By this we mean: why not add functionality to the planner, that would allow us to use our hypergraphs directly when planning.
One important detail to know, is that to determine whether a candidate is satisfied in a state, we only need one simple path. 
As finding any path is solvable in polynomial time. 
This means that if we extend a planner to accept hypergraphs, that uses a search algorithm;
then we can achieve a running time that is $O(Search\texttt{-}Time \times \left| Objects \right|^k)$ where $k$ is arity of the effect. 
If we bound $k$ to a constant, as they suggest doing in \cite{Walsh2008}, then we can conclude our solution can run in polynomial time. 
The search algorithm would have to find any path in the state which is between the candidate vertex and a vertex that is in the same binding as the effect vertex, to determine what effects to add to the state; 
Furthermore, it is important to note that a lot of preprocessing can be done on the hypergraph before used for planning, thereby allowing for very efficient search algorithms. 
Additional improvement could also be achieved by filtering effects where no vertex exists that would bind it.

\begin{example} To understand planning using hypergraph let us provide the following example.
	Assume we have the candidate set shown in \Cref{fig:ca:hypgraph-with-cands}. And we have a state:
	
	\begin{equation*}
		S = \{ p(o_1,_o2), f(o_2,o_3) g(o_4,o_4), f(o_3,o_6), g(o_1,_2) \}
	\end{equation*}
	
	To find the effects we search from the candidates one at a time, mapping the bindings as we go forward.
	The first candidate $f_1$ has two nodes:
	\begin{equation*}
		f_1 = o_3 ~ or ~ f_1 = 0_2
	\end{equation*}
	 As there exists no node $p_2 = o_3$, that search path is discontinued, and continuing the other gives us:
	 \begin{equation*}
		 p(o_1,o_2) \rightarrow q(o_1)
	 \end{equation*}
	 We then do the same for $g$ but as some of the effects are already found we need not search for those, limiting us to only:
	 \begin{equation*}
		 g(o_4,o_4) \rightarrow q(o_4)
	 \end{equation*}
	 
	 As such we get the effects: $q(o_1)\land q(o_4)$, which are the correct effects.
	 It is important to note that if have we had multiple candidate sets then those would have to be intersected with each other. 
	 However even that aspect might have a more efficient solution.
	
\end{example}

\subsection{Pessimistic Strategy}

While the optimistic strategy was very hard to achieve, the pessimistic is surprisingly easy, we recall from \Cref{sec:NC:hypcon} that to make a pessimistic action schema it is only necessary to plan using only unproven preconditions. Since our hypergraph is a model of all our unproven preconditions then we need only convert it into predicates and use them in a conjunction, after which the planner should be able to handle the rest.


\begin{example} To construct a pessimistic action schema based on a hypergraph, the vertices and edges must be turned into predicate logic.
	In \Cref{fig:ca:hypgraph-with-unproven} we see a hypergraph of our unproven preconditions for a conditional effect $q$.
	Using the same approach shown in \Cref{ex:ca:hyp-paths}, we map each binding to a variable and combine the predicate edges' vertices into predicates.
	As such our pessimistic effect will be the following:
	\begin{equation*}
	\begin{split}	
	Effect&: \forall_{(x,y,z,i)} ~q(x)~ when~ g(x,z) \land p(x,y) \land f(y,i)
	\end{split}
	\end{equation*}	
	As we proved in \Cref{thm:ca:precondition-state}, this action schema will only produce the exact same effects as was produced when the agent observed them. Therefore the action schema is pessimistic.
\end{example}

\end{document}
