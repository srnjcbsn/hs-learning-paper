\documentclass[../Master.tex]{subfiles}
\providecommand{\master}{..}
\begin{document}

When the scientific learning algorithm is applied through the main program, a statistics file is produced. This file contains, in order:
\begin{itemize}
    \item A header containing the name of each action schema in the domain, along with the action's $\lits_A$.
    \item The string \verb|``-- RUNNING --''|, signifying the end of the header.
    \item For each experiment (plan) the agent conducted:
        \begin{itemize}
            \item The action which did not produce the expected outcome when applied (as explained in Section~\ref{sec:PDDLAlgo})
            \item The amount of problems the agent had solved before execution of this action.
            \item Eight integers, signifying how many  atoms the agent has proven to be positive and negative effects and preconditions for the action, and how many are neither proven nor disproven to be positive and negative effects and preconditions for the action. Lastly, the number of candidate sets for the action is listed.
        \end{itemize}
\end{itemize}

In the source code bundled with this thesis, the program \texttt{Plotting/Statistics.hs} for interpreting and visually representing statistics files are included  (along with source code for generating most diagrams used in this thesis). For each action schema mentioned in the statistics file, the program will output two line plots: one showing the effects the agent learned for the action, and one showing the corresponding preconditions.

Graphs for the six non-conditional sokoban actions (run in the setup presented in Section~\ref{sec:Impl:Results}) are presented here. The graphs should be read as follows:
The unit of the $y$-axis is number of atoms, and the marker $\lits_A$ denotes the maximum value. The $x$-axis is measured in experiments. For readability, the experiments where the relevant action was not the cause of failure are removed, as nothing is learned. The small black ticks denote a duration of 5 experiments, and the larger red ticks denote that a problem has been solved, and that a new one has started. Note that the candidates (black line) is not measured in number of predicates, but in number of candidate sets in \Cand.
\begin{figure}
    \centering
    \begin{pgfpicture}
  \pgfpathrectangle{\pgfpointorigin}{\pgfqpoint{200.0000bp}{200.0000bp}}
  \pgfusepath{use as bounding box}
  \begin{pgfscope}
    \definecolor{fc}{rgb}{0.0000,0.0000,0.0000}
    \pgfsetfillcolor{fc}
    \pgftransformshift{\pgfqpoint{28.3333bp}{25.8333bp}}
    \pgftransformscale{1.0417}
    \pgftext[base,left]{candidates}
  \end{pgfscope}
  \begin{pgfscope}
    \definecolor{fc}{rgb}{0.0000,0.0000,0.0000}
    \pgfsetfillcolor{fc}
    \pgfsetlinewidth{0.6928bp}
    \definecolor{sc}{rgb}{0.0000,0.0000,0.0000}
    \pgfsetstrokecolor{sc}
    \pgfsetmiterjoin
    \pgfsetbuttcap
    \pgfpathqmoveto{20.0000bp}{28.3333bp}
    \pgfpathqcurveto{20.0000bp}{30.1743bp}{18.5076bp}{31.6667bp}{16.6667bp}{31.6667bp}
    \pgfpathqcurveto{14.8257bp}{31.6667bp}{13.3333bp}{30.1743bp}{13.3333bp}{28.3333bp}
    \pgfpathqcurveto{13.3333bp}{26.4924bp}{14.8257bp}{25.0000bp}{16.6667bp}{25.0000bp}
    \pgfpathqcurveto{18.5076bp}{25.0000bp}{20.0000bp}{26.4924bp}{20.0000bp}{28.3333bp}
    \pgfpathclose
    \pgfusepathqfillstroke
  \end{pgfscope}
  \begin{pgfscope}
    \definecolor{fc}{rgb}{0.0000,0.0000,0.0000}
    \pgfsetfillcolor{fc}
    \pgftransformshift{\pgfqpoint{28.3333bp}{36.6667bp}}
    \pgftransformscale{1.0417}
    \pgftext[base,left]{negative unproven}
  \end{pgfscope}
  \begin{pgfscope}
    \definecolor{fc}{rgb}{1.0000,1.0000,0.0000}
    \pgfsetfillcolor{fc}
    \pgfsetlinewidth{0.6928bp}
    \definecolor{sc}{rgb}{1.0000,1.0000,0.0000}
    \pgfsetstrokecolor{sc}
    \pgfsetmiterjoin
    \pgfsetbuttcap
    \pgfpathqmoveto{20.0000bp}{39.1667bp}
    \pgfpathqcurveto{20.0000bp}{41.0076bp}{18.5076bp}{42.5000bp}{16.6667bp}{42.5000bp}
    \pgfpathqcurveto{14.8257bp}{42.5000bp}{13.3333bp}{41.0076bp}{13.3333bp}{39.1667bp}
    \pgfpathqcurveto{13.3333bp}{37.3257bp}{14.8257bp}{35.8333bp}{16.6667bp}{35.8333bp}
    \pgfpathqcurveto{18.5076bp}{35.8333bp}{20.0000bp}{37.3257bp}{20.0000bp}{39.1667bp}
    \pgfpathclose
    \pgfusepathqfillstroke
  \end{pgfscope}
  \begin{pgfscope}
    \definecolor{fc}{rgb}{0.0000,0.0000,0.0000}
    \pgfsetfillcolor{fc}
    \pgftransformshift{\pgfqpoint{28.3333bp}{47.5000bp}}
    \pgftransformscale{1.0417}
    \pgftext[base,left]{negative proven}
  \end{pgfscope}
  \begin{pgfscope}
    \definecolor{fc}{rgb}{0.0000,0.5020,0.0000}
    \pgfsetfillcolor{fc}
    \pgfsetlinewidth{0.6928bp}
    \definecolor{sc}{rgb}{0.0000,0.5020,0.0000}
    \pgfsetstrokecolor{sc}
    \pgfsetmiterjoin
    \pgfsetbuttcap
    \pgfpathqmoveto{20.0000bp}{50.0000bp}
    \pgfpathqcurveto{20.0000bp}{51.8409bp}{18.5076bp}{53.3333bp}{16.6667bp}{53.3333bp}
    \pgfpathqcurveto{14.8257bp}{53.3333bp}{13.3333bp}{51.8409bp}{13.3333bp}{50.0000bp}
    \pgfpathqcurveto{13.3333bp}{48.1591bp}{14.8257bp}{46.6667bp}{16.6667bp}{46.6667bp}
    \pgfpathqcurveto{18.5076bp}{46.6667bp}{20.0000bp}{48.1591bp}{20.0000bp}{50.0000bp}
    \pgfpathclose
    \pgfusepathqfillstroke
  \end{pgfscope}
  \begin{pgfscope}
    \definecolor{fc}{rgb}{0.0000,0.0000,0.0000}
    \pgfsetfillcolor{fc}
    \pgftransformshift{\pgfqpoint{28.3333bp}{58.3333bp}}
    \pgftransformscale{1.0417}
    \pgftext[base,left]{positive unproven}
  \end{pgfscope}
  \begin{pgfscope}
    \definecolor{fc}{rgb}{1.0000,0.0000,0.0000}
    \pgfsetfillcolor{fc}
    \pgfsetlinewidth{0.6928bp}
    \definecolor{sc}{rgb}{1.0000,0.0000,0.0000}
    \pgfsetstrokecolor{sc}
    \pgfsetmiterjoin
    \pgfsetbuttcap
    \pgfpathqmoveto{20.0000bp}{60.8333bp}
    \pgfpathqcurveto{20.0000bp}{62.6743bp}{18.5076bp}{64.1667bp}{16.6667bp}{64.1667bp}
    \pgfpathqcurveto{14.8257bp}{64.1667bp}{13.3333bp}{62.6743bp}{13.3333bp}{60.8333bp}
    \pgfpathqcurveto{13.3333bp}{58.9924bp}{14.8257bp}{57.5000bp}{16.6667bp}{57.5000bp}
    \pgfpathqcurveto{18.5076bp}{57.5000bp}{20.0000bp}{58.9924bp}{20.0000bp}{60.8333bp}
    \pgfpathclose
    \pgfusepathqfillstroke
  \end{pgfscope}
  \begin{pgfscope}
    \definecolor{fc}{rgb}{0.0000,0.0000,0.0000}
    \pgfsetfillcolor{fc}
    \pgftransformshift{\pgfqpoint{28.3333bp}{69.1667bp}}
    \pgftransformscale{1.0417}
    \pgftext[base,left]{positive proven}
  \end{pgfscope}
  \begin{pgfscope}
    \definecolor{fc}{rgb}{0.0000,0.0000,1.0000}
    \pgfsetfillcolor{fc}
    \pgfsetlinewidth{0.6928bp}
    \definecolor{sc}{rgb}{0.0000,0.0000,1.0000}
    \pgfsetstrokecolor{sc}
    \pgfsetmiterjoin
    \pgfsetbuttcap
    \pgfpathqmoveto{20.0000bp}{71.6667bp}
    \pgfpathqcurveto{20.0000bp}{73.5076bp}{18.5076bp}{75.0000bp}{16.6667bp}{75.0000bp}
    \pgfpathqcurveto{14.8257bp}{75.0000bp}{13.3333bp}{73.5076bp}{13.3333bp}{71.6667bp}
    \pgfpathqcurveto{13.3333bp}{69.8257bp}{14.8257bp}{68.3333bp}{16.6667bp}{68.3333bp}
    \pgfpathqcurveto{18.5076bp}{68.3333bp}{20.0000bp}{69.8257bp}{20.0000bp}{71.6667bp}
    \pgfpathclose
    \pgfusepathqfillstroke
  \end{pgfscope}
  \begin{pgfscope}
    \pgfsetlinewidth{0.6928bp}
    \definecolor{sc}{rgb}{1.0000,1.0000,0.0000}
    \pgfsetstrokecolor{sc}
    \pgfsetmiterjoin
    \pgfsetbuttcap
    \pgfpathqmoveto{25.0000bp}{175.0000bp}
    \pgfpathqlineto{33.3333bp}{175.0000bp}
    \pgfpathqlineto{41.6667bp}{175.0000bp}
    \pgfpathqlineto{50.0000bp}{175.0000bp}
    \pgfpathqlineto{58.3333bp}{175.0000bp}
    \pgfpathqlineto{66.6667bp}{175.0000bp}
    \pgfpathqlineto{75.0000bp}{175.0000bp}
    \pgfpathqlineto{83.3333bp}{175.0000bp}
    \pgfpathqlineto{91.6667bp}{175.0000bp}
    \pgfpathqlineto{100.0000bp}{175.0000bp}
    \pgfpathqlineto{108.3333bp}{175.0000bp}
    \pgfpathqlineto{116.6667bp}{175.0000bp}
    \pgfpathqlineto{125.0000bp}{175.0000bp}
    \pgfpathqlineto{133.3333bp}{175.0000bp}
    \pgfpathqlineto{141.6667bp}{175.0000bp}
    \pgfpathqlineto{150.0000bp}{175.0000bp}
    \pgfpathqlineto{158.3333bp}{175.0000bp}
    \pgfpathqlineto{166.6667bp}{91.6667bp}
    \pgfpathqlineto{175.0000bp}{91.6667bp}
    \pgfpathqlineto{183.3333bp}{91.6667bp}
    \pgfpathqlineto{191.6667bp}{91.6667bp}
    \pgfpathqlineto{200.0000bp}{91.6667bp}
    \pgfusepathqstroke
  \end{pgfscope}
  \begin{pgfscope}
    \pgfsetlinewidth{0.6928bp}
    \definecolor{sc}{rgb}{0.0000,0.5020,0.0000}
    \pgfsetstrokecolor{sc}
    \pgfsetmiterjoin
    \pgfsetbuttcap
    \pgfpathqmoveto{25.0000bp}{91.6667bp}
    \pgfpathqlineto{33.3333bp}{91.6667bp}
    \pgfpathqlineto{41.6667bp}{91.6667bp}
    \pgfpathqlineto{50.0000bp}{91.6667bp}
    \pgfpathqlineto{58.3333bp}{91.6667bp}
    \pgfpathqlineto{66.6667bp}{91.6667bp}
    \pgfpathqlineto{75.0000bp}{91.6667bp}
    \pgfpathqlineto{83.3333bp}{91.6667bp}
    \pgfpathqlineto{91.6667bp}{91.6667bp}
    \pgfpathqlineto{100.0000bp}{91.6667bp}
    \pgfpathqlineto{108.3333bp}{91.6667bp}
    \pgfpathqlineto{116.6667bp}{91.6667bp}
    \pgfpathqlineto{125.0000bp}{91.6667bp}
    \pgfpathqlineto{133.3333bp}{91.6667bp}
    \pgfpathqlineto{141.6667bp}{91.6667bp}
    \pgfpathqlineto{150.0000bp}{91.6667bp}
    \pgfpathqlineto{158.3333bp}{91.6667bp}
    \pgfpathqlineto{166.6667bp}{100.0000bp}
    \pgfpathqlineto{175.0000bp}{100.0000bp}
    \pgfpathqlineto{183.3333bp}{100.0000bp}
    \pgfpathqlineto{191.6667bp}{100.0000bp}
    \pgfpathqlineto{200.0000bp}{100.0000bp}
    \pgfusepathqstroke
  \end{pgfscope}
  \begin{pgfscope}
    \pgfsetlinewidth{0.6928bp}
    \definecolor{sc}{rgb}{1.0000,0.0000,0.0000}
    \pgfsetstrokecolor{sc}
    \pgfsetmiterjoin
    \pgfsetbuttcap
    \pgfpathqmoveto{25.0000bp}{175.0000bp}
    \pgfpathqlineto{33.3333bp}{175.0000bp}
    \pgfpathqlineto{41.6667bp}{175.0000bp}
    \pgfpathqlineto{50.0000bp}{175.0000bp}
    \pgfpathqlineto{58.3333bp}{175.0000bp}
    \pgfpathqlineto{66.6667bp}{175.0000bp}
    \pgfpathqlineto{75.0000bp}{175.0000bp}
    \pgfpathqlineto{83.3333bp}{175.0000bp}
    \pgfpathqlineto{91.6667bp}{175.0000bp}
    \pgfpathqlineto{100.0000bp}{175.0000bp}
    \pgfpathqlineto{108.3333bp}{175.0000bp}
    \pgfpathqlineto{116.6667bp}{175.0000bp}
    \pgfpathqlineto{125.0000bp}{175.0000bp}
    \pgfpathqlineto{133.3333bp}{175.0000bp}
    \pgfpathqlineto{141.6667bp}{175.0000bp}
    \pgfpathqlineto{150.0000bp}{175.0000bp}
    \pgfpathqlineto{158.3333bp}{175.0000bp}
    \pgfpathqlineto{166.6667bp}{91.6667bp}
    \pgfpathqlineto{175.0000bp}{91.6667bp}
    \pgfpathqlineto{183.3333bp}{91.6667bp}
    \pgfpathqlineto{191.6667bp}{91.6667bp}
    \pgfpathqlineto{200.0000bp}{91.6667bp}
    \pgfusepathqstroke
  \end{pgfscope}
  \begin{pgfscope}
    \pgfsetlinewidth{0.6928bp}
    \definecolor{sc}{rgb}{0.0000,0.0000,1.0000}
    \pgfsetstrokecolor{sc}
    \pgfsetmiterjoin
    \pgfsetbuttcap
    \pgfpathqmoveto{25.0000bp}{91.6667bp}
    \pgfpathqlineto{33.3333bp}{91.6667bp}
    \pgfpathqlineto{41.6667bp}{91.6667bp}
    \pgfpathqlineto{50.0000bp}{91.6667bp}
    \pgfpathqlineto{58.3333bp}{91.6667bp}
    \pgfpathqlineto{66.6667bp}{91.6667bp}
    \pgfpathqlineto{75.0000bp}{91.6667bp}
    \pgfpathqlineto{83.3333bp}{91.6667bp}
    \pgfpathqlineto{91.6667bp}{91.6667bp}
    \pgfpathqlineto{100.0000bp}{91.6667bp}
    \pgfpathqlineto{108.3333bp}{91.6667bp}
    \pgfpathqlineto{116.6667bp}{91.6667bp}
    \pgfpathqlineto{125.0000bp}{91.6667bp}
    \pgfpathqlineto{133.3333bp}{91.6667bp}
    \pgfpathqlineto{141.6667bp}{91.6667bp}
    \pgfpathqlineto{150.0000bp}{91.6667bp}
    \pgfpathqlineto{158.3333bp}{91.6667bp}
    \pgfpathqlineto{166.6667bp}{100.0000bp}
    \pgfpathqlineto{175.0000bp}{100.0000bp}
    \pgfpathqlineto{183.3333bp}{100.0000bp}
    \pgfpathqlineto{191.6667bp}{100.0000bp}
    \pgfpathqlineto{200.0000bp}{100.0000bp}
    \pgfusepathqstroke
  \end{pgfscope}
  \begin{pgfscope}
    \pgfsetlinewidth{0.6928bp}
    \definecolor{sc}{rgb}{1.0000,0.0000,0.0000}
    \pgfsetstrokecolor{sc}
    \pgfsetmiterjoin
    \pgfsetbuttcap
    \pgfpathqmoveto{50.0000bp}{91.6667bp}
    \pgfpathqlineto{50.0000bp}{83.3333bp}
    \pgfusepathqstroke
  \end{pgfscope}
  \begin{pgfscope}
    \pgfsetlinewidth{0.6928bp}
    \definecolor{sc}{rgb}{1.0000,0.0000,0.0000}
    \pgfsetstrokecolor{sc}
    \pgfsetmiterjoin
    \pgfsetbuttcap
    \pgfpathqmoveto{166.6667bp}{91.6667bp}
    \pgfpathqlineto{166.6667bp}{83.3333bp}
    \pgfusepathqstroke
  \end{pgfscope}
  \begin{pgfscope}
    \pgfsetlinewidth{0.6928bp}
    \definecolor{sc}{rgb}{0.0000,0.0000,0.0000}
    \pgfsetstrokecolor{sc}
    \pgfsetmiterjoin
    \pgfsetbuttcap
    \pgfpathqmoveto{183.3333bp}{91.6667bp}
    \pgfpathqlineto{183.3333bp}{87.5000bp}
    \pgfusepathqstroke
  \end{pgfscope}
  \begin{pgfscope}
    \pgfsetlinewidth{0.6928bp}
    \definecolor{sc}{rgb}{0.0000,0.0000,0.0000}
    \pgfsetstrokecolor{sc}
    \pgfsetmiterjoin
    \pgfsetbuttcap
    \pgfpathqmoveto{141.6667bp}{91.6667bp}
    \pgfpathqlineto{141.6667bp}{87.5000bp}
    \pgfusepathqstroke
  \end{pgfscope}
  \begin{pgfscope}
    \pgfsetlinewidth{0.6928bp}
    \definecolor{sc}{rgb}{0.0000,0.0000,0.0000}
    \pgfsetstrokecolor{sc}
    \pgfsetmiterjoin
    \pgfsetbuttcap
    \pgfpathqmoveto{100.0000bp}{91.6667bp}
    \pgfpathqlineto{100.0000bp}{87.5000bp}
    \pgfusepathqstroke
  \end{pgfscope}
  \begin{pgfscope}
    \pgfsetlinewidth{0.6928bp}
    \definecolor{sc}{rgb}{0.0000,0.0000,0.0000}
    \pgfsetstrokecolor{sc}
    \pgfsetmiterjoin
    \pgfsetbuttcap
    \pgfpathqmoveto{58.3333bp}{91.6667bp}
    \pgfpathqlineto{58.3333bp}{87.5000bp}
    \pgfusepathqstroke
  \end{pgfscope}
  \begin{pgfscope}
    \definecolor{fc}{rgb}{0.0000,0.0000,0.0000}
    \pgfsetfillcolor{fc}
    \pgftransformshift{\pgfqpoint{-0.0000bp}{172.5000bp}}
    \pgftransformscale{1.0417}
    \pgftext[base,left]{$\mathbb{F}_A$}
  \end{pgfscope}
  \begin{pgfscope}
    \pgfsetlinewidth{0.6928bp}
    \definecolor{sc}{rgb}{0.0000,0.0000,0.0000}
    \pgfsetstrokecolor{sc}
    \pgfsetmiterjoin
    \pgfsetbuttcap
    \pgfpathqmoveto{16.6667bp}{175.0000bp}
    \pgfpathqlineto{15.0000bp}{175.0000bp}
    \pgfusepathqstroke
  \end{pgfscope}
  \begin{pgfscope}
    \pgfsetlinewidth{0.6928bp}
    \definecolor{sc}{rgb}{0.0000,0.0000,0.0000}
    \pgfsetstrokecolor{sc}
    \pgfsetmiterjoin
    \pgfsetbuttcap
    \pgfpathqmoveto{16.6667bp}{91.6667bp}
    \pgfpathqlineto{16.6667bp}{175.0000bp}
    \pgfusepathqstroke
  \end{pgfscope}
  \begin{pgfscope}
    \pgfsetlinewidth{0.6928bp}
    \definecolor{sc}{rgb}{0.0000,0.0000,0.0000}
    \pgfsetstrokecolor{sc}
    \pgfsetmiterjoin
    \pgfsetbuttcap
    \pgfpathqmoveto{16.6667bp}{91.6667bp}
    \pgfpathqlineto{200.0000bp}{91.6667bp}
    \pgfusepathqstroke
  \end{pgfscope}
\end{pgfpicture}

    \caption{move-h effects}
\end{figure}

\begin{figure}
    \centering
    \begin{pgfpicture}
  \pgfpathrectangle{\pgfpointorigin}{\pgfqpoint{200.0000bp}{200.0000bp}}
  \pgfusepath{use as bounding box}
  \begin{pgfscope}
    \definecolor{fc}{rgb}{0.0000,0.0000,0.0000}
    \pgfsetfillcolor{fc}
    \pgftransformshift{\pgfqpoint{27.2000bp}{28.8000bp}}
    \pgftransformscale{1.0000}
    \pgftext[base,left]{candidates}
  \end{pgfscope}
  \begin{pgfscope}
    \definecolor{fc}{rgb}{0.0000,0.0000,0.0000}
    \pgfsetfillcolor{fc}
    \pgfsetlinewidth{0.6788bp}
    \definecolor{sc}{rgb}{0.0000,0.0000,0.0000}
    \pgfsetstrokecolor{sc}
    \pgfsetmiterjoin
    \pgfsetbuttcap
    \pgfpathqmoveto{19.2000bp}{31.2000bp}
    \pgfpathqcurveto{19.2000bp}{32.9673bp}{17.7673bp}{34.4000bp}{16.0000bp}{34.4000bp}
    \pgfpathqcurveto{14.2327bp}{34.4000bp}{12.8000bp}{32.9673bp}{12.8000bp}{31.2000bp}
    \pgfpathqcurveto{12.8000bp}{29.4327bp}{14.2327bp}{28.0000bp}{16.0000bp}{28.0000bp}
    \pgfpathqcurveto{17.7673bp}{28.0000bp}{19.2000bp}{29.4327bp}{19.2000bp}{31.2000bp}
    \pgfpathclose
    \pgfusepathqfillstroke
  \end{pgfscope}
  \begin{pgfscope}
    \definecolor{fc}{rgb}{0.0000,0.0000,0.0000}
    \pgfsetfillcolor{fc}
    \pgftransformshift{\pgfqpoint{27.2000bp}{39.2000bp}}
    \pgftransformscale{1.0000}
    \pgftext[base,left]{negative unproven}
  \end{pgfscope}
  \begin{pgfscope}
    \definecolor{fc}{rgb}{1.0000,1.0000,0.0000}
    \pgfsetfillcolor{fc}
    \pgfsetlinewidth{0.6788bp}
    \definecolor{sc}{rgb}{1.0000,1.0000,0.0000}
    \pgfsetstrokecolor{sc}
    \pgfsetmiterjoin
    \pgfsetbuttcap
    \pgfpathqmoveto{19.2000bp}{41.6000bp}
    \pgfpathqcurveto{19.2000bp}{43.3673bp}{17.7673bp}{44.8000bp}{16.0000bp}{44.8000bp}
    \pgfpathqcurveto{14.2327bp}{44.8000bp}{12.8000bp}{43.3673bp}{12.8000bp}{41.6000bp}
    \pgfpathqcurveto{12.8000bp}{39.8327bp}{14.2327bp}{38.4000bp}{16.0000bp}{38.4000bp}
    \pgfpathqcurveto{17.7673bp}{38.4000bp}{19.2000bp}{39.8327bp}{19.2000bp}{41.6000bp}
    \pgfpathclose
    \pgfusepathqfillstroke
  \end{pgfscope}
  \begin{pgfscope}
    \definecolor{fc}{rgb}{0.0000,0.0000,0.0000}
    \pgfsetfillcolor{fc}
    \pgftransformshift{\pgfqpoint{27.2000bp}{49.6000bp}}
    \pgftransformscale{1.0000}
    \pgftext[base,left]{negative proven}
  \end{pgfscope}
  \begin{pgfscope}
    \definecolor{fc}{rgb}{0.0000,0.5020,0.0000}
    \pgfsetfillcolor{fc}
    \pgfsetlinewidth{0.6788bp}
    \definecolor{sc}{rgb}{0.0000,0.5020,0.0000}
    \pgfsetstrokecolor{sc}
    \pgfsetmiterjoin
    \pgfsetbuttcap
    \pgfpathqmoveto{19.2000bp}{52.0000bp}
    \pgfpathqcurveto{19.2000bp}{53.7673bp}{17.7673bp}{55.2000bp}{16.0000bp}{55.2000bp}
    \pgfpathqcurveto{14.2327bp}{55.2000bp}{12.8000bp}{53.7673bp}{12.8000bp}{52.0000bp}
    \pgfpathqcurveto{12.8000bp}{50.2327bp}{14.2327bp}{48.8000bp}{16.0000bp}{48.8000bp}
    \pgfpathqcurveto{17.7673bp}{48.8000bp}{19.2000bp}{50.2327bp}{19.2000bp}{52.0000bp}
    \pgfpathclose
    \pgfusepathqfillstroke
  \end{pgfscope}
  \begin{pgfscope}
    \definecolor{fc}{rgb}{0.0000,0.0000,0.0000}
    \pgfsetfillcolor{fc}
    \pgftransformshift{\pgfqpoint{27.2000bp}{60.0000bp}}
    \pgftransformscale{1.0000}
    \pgftext[base,left]{positive unproven}
  \end{pgfscope}
  \begin{pgfscope}
    \definecolor{fc}{rgb}{1.0000,0.0000,0.0000}
    \pgfsetfillcolor{fc}
    \pgfsetlinewidth{0.6788bp}
    \definecolor{sc}{rgb}{1.0000,0.0000,0.0000}
    \pgfsetstrokecolor{sc}
    \pgfsetmiterjoin
    \pgfsetbuttcap
    \pgfpathqmoveto{19.2000bp}{62.4000bp}
    \pgfpathqcurveto{19.2000bp}{64.1673bp}{17.7673bp}{65.6000bp}{16.0000bp}{65.6000bp}
    \pgfpathqcurveto{14.2327bp}{65.6000bp}{12.8000bp}{64.1673bp}{12.8000bp}{62.4000bp}
    \pgfpathqcurveto{12.8000bp}{60.6327bp}{14.2327bp}{59.2000bp}{16.0000bp}{59.2000bp}
    \pgfpathqcurveto{17.7673bp}{59.2000bp}{19.2000bp}{60.6327bp}{19.2000bp}{62.4000bp}
    \pgfpathclose
    \pgfusepathqfillstroke
  \end{pgfscope}
  \begin{pgfscope}
    \definecolor{fc}{rgb}{0.0000,0.0000,0.0000}
    \pgfsetfillcolor{fc}
    \pgftransformshift{\pgfqpoint{27.2000bp}{70.4000bp}}
    \pgftransformscale{1.0000}
    \pgftext[base,left]{positive proven}
  \end{pgfscope}
  \begin{pgfscope}
    \definecolor{fc}{rgb}{0.0000,0.0000,1.0000}
    \pgfsetfillcolor{fc}
    \pgfsetlinewidth{0.6788bp}
    \definecolor{sc}{rgb}{0.0000,0.0000,1.0000}
    \pgfsetstrokecolor{sc}
    \pgfsetmiterjoin
    \pgfsetbuttcap
    \pgfpathqmoveto{19.2000bp}{72.8000bp}
    \pgfpathqcurveto{19.2000bp}{74.5673bp}{17.7673bp}{76.0000bp}{16.0000bp}{76.0000bp}
    \pgfpathqcurveto{14.2327bp}{76.0000bp}{12.8000bp}{74.5673bp}{12.8000bp}{72.8000bp}
    \pgfpathqcurveto{12.8000bp}{71.0327bp}{14.2327bp}{69.6000bp}{16.0000bp}{69.6000bp}
    \pgfpathqcurveto{17.7673bp}{69.6000bp}{19.2000bp}{71.0327bp}{19.2000bp}{72.8000bp}
    \pgfpathclose
    \pgfusepathqfillstroke
  \end{pgfscope}
  \begin{pgfscope}
    \pgfsetlinewidth{0.6788bp}
    \definecolor{sc}{rgb}{1.0000,1.0000,0.0000}
    \pgfsetstrokecolor{sc}
    \pgfsetmiterjoin
    \pgfsetbuttcap
    \pgfpathqmoveto{16.0000bp}{172.0000bp}
    \pgfpathqlineto{24.0000bp}{172.0000bp}
    \pgfpathqlineto{32.0000bp}{172.0000bp}
    \pgfpathqlineto{40.0000bp}{172.0000bp}
    \pgfpathqlineto{48.0000bp}{172.0000bp}
    \pgfpathqlineto{56.0000bp}{172.0000bp}
    \pgfpathqlineto{64.0000bp}{172.0000bp}
    \pgfpathqlineto{72.0000bp}{172.0000bp}
    \pgfpathqlineto{80.0000bp}{172.0000bp}
    \pgfpathqlineto{88.0000bp}{172.0000bp}
    \pgfpathqlineto{96.0000bp}{172.0000bp}
    \pgfpathqlineto{104.0000bp}{172.0000bp}
    \pgfpathqlineto{112.0000bp}{172.0000bp}
    \pgfpathqlineto{120.0000bp}{172.0000bp}
    \pgfpathqlineto{128.0000bp}{172.0000bp}
    \pgfpathqlineto{136.0000bp}{172.0000bp}
    \pgfpathqlineto{144.0000bp}{172.0000bp}
    \pgfpathqlineto{152.0000bp}{172.0000bp}
    \pgfpathqlineto{160.0000bp}{172.0000bp}
    \pgfpathqlineto{168.0000bp}{172.0000bp}
    \pgfpathqlineto{176.0000bp}{172.0000bp}
    \pgfpathqlineto{184.0000bp}{92.0000bp}
    \pgfpathqlineto{192.0000bp}{92.0000bp}
    \pgfusepathqstroke
  \end{pgfscope}
  \begin{pgfscope}
    \pgfsetlinewidth{0.6788bp}
    \definecolor{sc}{rgb}{0.0000,0.5020,0.0000}
    \pgfsetstrokecolor{sc}
    \pgfsetmiterjoin
    \pgfsetbuttcap
    \pgfpathqmoveto{16.0000bp}{92.0000bp}
    \pgfpathqlineto{24.0000bp}{92.0000bp}
    \pgfpathqlineto{32.0000bp}{92.0000bp}
    \pgfpathqlineto{40.0000bp}{92.0000bp}
    \pgfpathqlineto{48.0000bp}{92.0000bp}
    \pgfpathqlineto{56.0000bp}{92.0000bp}
    \pgfpathqlineto{64.0000bp}{92.0000bp}
    \pgfpathqlineto{72.0000bp}{92.0000bp}
    \pgfpathqlineto{80.0000bp}{92.0000bp}
    \pgfpathqlineto{88.0000bp}{92.0000bp}
    \pgfpathqlineto{96.0000bp}{92.0000bp}
    \pgfpathqlineto{104.0000bp}{92.0000bp}
    \pgfpathqlineto{112.0000bp}{92.0000bp}
    \pgfpathqlineto{120.0000bp}{92.0000bp}
    \pgfpathqlineto{128.0000bp}{92.0000bp}
    \pgfpathqlineto{136.0000bp}{92.0000bp}
    \pgfpathqlineto{144.0000bp}{92.0000bp}
    \pgfpathqlineto{152.0000bp}{92.0000bp}
    \pgfpathqlineto{160.0000bp}{92.0000bp}
    \pgfpathqlineto{168.0000bp}{92.0000bp}
    \pgfpathqlineto{176.0000bp}{92.0000bp}
    \pgfpathqlineto{184.0000bp}{100.0000bp}
    \pgfpathqlineto{192.0000bp}{100.0000bp}
    \pgfusepathqstroke
  \end{pgfscope}
  \begin{pgfscope}
    \pgfsetlinewidth{0.6788bp}
    \definecolor{sc}{rgb}{1.0000,0.0000,0.0000}
    \pgfsetstrokecolor{sc}
    \pgfsetmiterjoin
    \pgfsetbuttcap
    \pgfpathqmoveto{16.0000bp}{172.0000bp}
    \pgfpathqlineto{24.0000bp}{172.0000bp}
    \pgfpathqlineto{32.0000bp}{172.0000bp}
    \pgfpathqlineto{40.0000bp}{172.0000bp}
    \pgfpathqlineto{48.0000bp}{172.0000bp}
    \pgfpathqlineto{56.0000bp}{172.0000bp}
    \pgfpathqlineto{64.0000bp}{172.0000bp}
    \pgfpathqlineto{72.0000bp}{172.0000bp}
    \pgfpathqlineto{80.0000bp}{172.0000bp}
    \pgfpathqlineto{88.0000bp}{172.0000bp}
    \pgfpathqlineto{96.0000bp}{172.0000bp}
    \pgfpathqlineto{104.0000bp}{172.0000bp}
    \pgfpathqlineto{112.0000bp}{172.0000bp}
    \pgfpathqlineto{120.0000bp}{172.0000bp}
    \pgfpathqlineto{128.0000bp}{172.0000bp}
    \pgfpathqlineto{136.0000bp}{172.0000bp}
    \pgfpathqlineto{144.0000bp}{172.0000bp}
    \pgfpathqlineto{152.0000bp}{172.0000bp}
    \pgfpathqlineto{160.0000bp}{172.0000bp}
    \pgfpathqlineto{168.0000bp}{172.0000bp}
    \pgfpathqlineto{176.0000bp}{172.0000bp}
    \pgfpathqlineto{184.0000bp}{92.0000bp}
    \pgfpathqlineto{192.0000bp}{92.0000bp}
    \pgfusepathqstroke
  \end{pgfscope}
  \begin{pgfscope}
    \pgfsetlinewidth{0.6788bp}
    \definecolor{sc}{rgb}{0.0000,0.0000,1.0000}
    \pgfsetstrokecolor{sc}
    \pgfsetmiterjoin
    \pgfsetbuttcap
    \pgfpathqmoveto{16.0000bp}{92.0000bp}
    \pgfpathqlineto{24.0000bp}{92.0000bp}
    \pgfpathqlineto{32.0000bp}{92.0000bp}
    \pgfpathqlineto{40.0000bp}{92.0000bp}
    \pgfpathqlineto{48.0000bp}{92.0000bp}
    \pgfpathqlineto{56.0000bp}{92.0000bp}
    \pgfpathqlineto{64.0000bp}{92.0000bp}
    \pgfpathqlineto{72.0000bp}{92.0000bp}
    \pgfpathqlineto{80.0000bp}{92.0000bp}
    \pgfpathqlineto{88.0000bp}{92.0000bp}
    \pgfpathqlineto{96.0000bp}{92.0000bp}
    \pgfpathqlineto{104.0000bp}{92.0000bp}
    \pgfpathqlineto{112.0000bp}{92.0000bp}
    \pgfpathqlineto{120.0000bp}{92.0000bp}
    \pgfpathqlineto{128.0000bp}{92.0000bp}
    \pgfpathqlineto{136.0000bp}{92.0000bp}
    \pgfpathqlineto{144.0000bp}{92.0000bp}
    \pgfpathqlineto{152.0000bp}{92.0000bp}
    \pgfpathqlineto{160.0000bp}{92.0000bp}
    \pgfpathqlineto{168.0000bp}{92.0000bp}
    \pgfpathqlineto{176.0000bp}{92.0000bp}
    \pgfpathqlineto{184.0000bp}{100.0000bp}
    \pgfpathqlineto{192.0000bp}{100.0000bp}
    \pgfusepathqstroke
  \end{pgfscope}
  \begin{pgfscope}
    \pgfsetlinewidth{0.6788bp}
    \definecolor{sc}{rgb}{1.0000,0.0000,0.0000}
    \pgfsetstrokecolor{sc}
    \pgfsetmiterjoin
    \pgfsetbuttcap
    \pgfpathqmoveto{200.0000bp}{92.0000bp}
    \pgfpathqlineto{200.0000bp}{84.0000bp}
    \pgfusepathqstroke
  \end{pgfscope}
  \begin{pgfscope}
    \pgfsetlinewidth{0.6788bp}
    \definecolor{sc}{rgb}{1.0000,0.0000,0.0000}
    \pgfsetstrokecolor{sc}
    \pgfsetmiterjoin
    \pgfsetbuttcap
    \pgfpathqmoveto{192.0000bp}{92.0000bp}
    \pgfpathqlineto{192.0000bp}{84.0000bp}
    \pgfusepathqstroke
  \end{pgfscope}
  \begin{pgfscope}
    \pgfsetlinewidth{0.6788bp}
    \definecolor{sc}{rgb}{1.0000,0.0000,0.0000}
    \pgfsetstrokecolor{sc}
    \pgfsetmiterjoin
    \pgfsetbuttcap
    \pgfpathqmoveto{152.0000bp}{92.0000bp}
    \pgfpathqlineto{152.0000bp}{84.0000bp}
    \pgfusepathqstroke
  \end{pgfscope}
  \begin{pgfscope}
    \pgfsetlinewidth{0.6788bp}
    \definecolor{sc}{rgb}{1.0000,0.0000,0.0000}
    \pgfsetstrokecolor{sc}
    \pgfsetmiterjoin
    \pgfsetbuttcap
    \pgfpathqmoveto{16.0000bp}{92.0000bp}
    \pgfpathqlineto{16.0000bp}{84.0000bp}
    \pgfusepathqstroke
  \end{pgfscope}
  \begin{pgfscope}
    \pgfsetlinewidth{0.6788bp}
    \definecolor{sc}{rgb}{0.0000,0.0000,0.0000}
    \pgfsetstrokecolor{sc}
    \pgfsetmiterjoin
    \pgfsetbuttcap
    \pgfpathqmoveto{176.0000bp}{92.0000bp}
    \pgfpathqlineto{176.0000bp}{88.0000bp}
    \pgfusepathqstroke
  \end{pgfscope}
  \begin{pgfscope}
    \pgfsetlinewidth{0.6788bp}
    \definecolor{sc}{rgb}{0.0000,0.0000,0.0000}
    \pgfsetstrokecolor{sc}
    \pgfsetmiterjoin
    \pgfsetbuttcap
    \pgfpathqmoveto{136.0000bp}{92.0000bp}
    \pgfpathqlineto{136.0000bp}{88.0000bp}
    \pgfusepathqstroke
  \end{pgfscope}
  \begin{pgfscope}
    \pgfsetlinewidth{0.6788bp}
    \definecolor{sc}{rgb}{0.0000,0.0000,0.0000}
    \pgfsetstrokecolor{sc}
    \pgfsetmiterjoin
    \pgfsetbuttcap
    \pgfpathqmoveto{96.0000bp}{92.0000bp}
    \pgfpathqlineto{96.0000bp}{88.0000bp}
    \pgfusepathqstroke
  \end{pgfscope}
  \begin{pgfscope}
    \pgfsetlinewidth{0.6788bp}
    \definecolor{sc}{rgb}{0.0000,0.0000,0.0000}
    \pgfsetstrokecolor{sc}
    \pgfsetmiterjoin
    \pgfsetbuttcap
    \pgfpathqmoveto{56.0000bp}{92.0000bp}
    \pgfpathqlineto{56.0000bp}{88.0000bp}
    \pgfusepathqstroke
  \end{pgfscope}
  \begin{pgfscope}
    \definecolor{fc}{rgb}{0.0000,0.0000,0.0000}
    \pgfsetfillcolor{fc}
    \pgftransformshift{\pgfqpoint{0.0000bp}{169.6000bp}}
    \pgftransformscale{1.0000}
    \pgftext[base,left]{$\mathbb{L}_A$}
  \end{pgfscope}
  \begin{pgfscope}
    \pgfsetlinewidth{0.6788bp}
    \definecolor{sc}{rgb}{0.0000,0.0000,0.0000}
    \pgfsetstrokecolor{sc}
    \pgfsetmiterjoin
    \pgfsetbuttcap
    \pgfpathqmoveto{16.0000bp}{172.0000bp}
    \pgfpathqlineto{14.4000bp}{172.0000bp}
    \pgfusepathqstroke
  \end{pgfscope}
  \begin{pgfscope}
    \pgfsetlinewidth{0.6788bp}
    \definecolor{sc}{rgb}{0.0000,0.0000,0.0000}
    \pgfsetstrokecolor{sc}
    \pgfsetmiterjoin
    \pgfsetbuttcap
    \pgfpathqmoveto{16.0000bp}{92.0000bp}
    \pgfpathqlineto{16.0000bp}{172.0000bp}
    \pgfusepathqstroke
  \end{pgfscope}
  \begin{pgfscope}
    \pgfsetlinewidth{0.6788bp}
    \definecolor{sc}{rgb}{0.0000,0.0000,0.0000}
    \pgfsetstrokecolor{sc}
    \pgfsetmiterjoin
    \pgfsetbuttcap
    \pgfpathqmoveto{16.0000bp}{92.0000bp}
    \pgfpathqlineto{200.0000bp}{92.0000bp}
    \pgfusepathqstroke
  \end{pgfscope}
\end{pgfpicture}

        \label{fig:ex:ca:hgma:ex:move-h}
    \caption{move-v effects}\label{fig:ex:ca:hgma:ex:disconnected}
\end{figure}

\begin{figure}
    \centering
    \begin{pgfpicture}
  \pgfpathrectangle{\pgfpointorigin}{\pgfqpoint{200.0000bp}{200.0000bp}}
  \pgfusepath{use as bounding box}
  \begin{pgfscope}
    \definecolor{fc}{rgb}{0.0000,0.0000,0.0000}
    \pgfsetfillcolor{fc}
    \pgftransformcm{1.0000}{0.0000}{0.0000}{1.0000}{\pgfqpoint{5.2308bp}{69.3846bp}}
    \pgftransformscale{0.1923}
    \pgftext[base,left]{candidates}
  \end{pgfscope}
  \begin{pgfscope}
    \definecolor{fc}{rgb}{0.0000,0.0000,0.0000}
    \pgfsetfillcolor{fc}
    \pgfsetlinewidth{0.5000bp}
    \definecolor{sc}{rgb}{0.0000,0.0000,0.0000}
    \pgfsetstrokecolor{sc}
    \pgfsetmiterjoin
    \pgfsetbuttcap
    \pgfpathqmoveto{3.6923bp}{69.8462bp}
    \pgfpathqcurveto{3.6923bp}{70.1860bp}{3.4168bp}{70.4615bp}{3.0769bp}{70.4615bp}
    \pgfpathqcurveto{2.7371bp}{70.4615bp}{2.4615bp}{70.1860bp}{2.4615bp}{69.8462bp}
    \pgfpathqcurveto{2.4615bp}{69.5063bp}{2.7371bp}{69.2308bp}{3.0769bp}{69.2308bp}
    \pgfpathqcurveto{3.4168bp}{69.2308bp}{3.6923bp}{69.5063bp}{3.6923bp}{69.8462bp}
    \pgfpathclose
    \pgfusepathqfillstroke
  \end{pgfscope}
  \begin{pgfscope}
    \definecolor{fc}{rgb}{0.0000,0.0000,0.0000}
    \pgfsetfillcolor{fc}
    \pgftransformcm{1.0000}{0.0000}{0.0000}{1.0000}{\pgfqpoint{5.2308bp}{71.3846bp}}
    \pgftransformscale{0.1923}
    \pgftext[base,left]{negative unproven}
  \end{pgfscope}
  \begin{pgfscope}
    \definecolor{fc}{rgb}{1.0000,1.0000,0.0000}
    \pgfsetfillcolor{fc}
    \pgfsetlinewidth{0.5000bp}
    \definecolor{sc}{rgb}{1.0000,1.0000,0.0000}
    \pgfsetstrokecolor{sc}
    \pgfsetmiterjoin
    \pgfsetbuttcap
    \pgfpathqmoveto{3.6923bp}{71.8462bp}
    \pgfpathqcurveto{3.6923bp}{72.1860bp}{3.4168bp}{72.4615bp}{3.0769bp}{72.4615bp}
    \pgfpathqcurveto{2.7371bp}{72.4615bp}{2.4615bp}{72.1860bp}{2.4615bp}{71.8462bp}
    \pgfpathqcurveto{2.4615bp}{71.5063bp}{2.7371bp}{71.2308bp}{3.0769bp}{71.2308bp}
    \pgfpathqcurveto{3.4168bp}{71.2308bp}{3.6923bp}{71.5063bp}{3.6923bp}{71.8462bp}
    \pgfpathclose
    \pgfusepathqfillstroke
  \end{pgfscope}
  \begin{pgfscope}
    \definecolor{fc}{rgb}{0.0000,0.0000,0.0000}
    \pgfsetfillcolor{fc}
    \pgftransformcm{1.0000}{0.0000}{0.0000}{1.0000}{\pgfqpoint{5.2308bp}{73.3846bp}}
    \pgftransformscale{0.1923}
    \pgftext[base,left]{negative proven}
  \end{pgfscope}
  \begin{pgfscope}
    \definecolor{fc}{rgb}{0.0000,0.5020,0.0000}
    \pgfsetfillcolor{fc}
    \pgfsetlinewidth{0.5000bp}
    \definecolor{sc}{rgb}{0.0000,0.5020,0.0000}
    \pgfsetstrokecolor{sc}
    \pgfsetmiterjoin
    \pgfsetbuttcap
    \pgfpathqmoveto{3.6923bp}{73.8462bp}
    \pgfpathqcurveto{3.6923bp}{74.1860bp}{3.4168bp}{74.4615bp}{3.0769bp}{74.4615bp}
    \pgfpathqcurveto{2.7371bp}{74.4615bp}{2.4615bp}{74.1860bp}{2.4615bp}{73.8462bp}
    \pgfpathqcurveto{2.4615bp}{73.5063bp}{2.7371bp}{73.2308bp}{3.0769bp}{73.2308bp}
    \pgfpathqcurveto{3.4168bp}{73.2308bp}{3.6923bp}{73.5063bp}{3.6923bp}{73.8462bp}
    \pgfpathclose
    \pgfusepathqfillstroke
  \end{pgfscope}
  \begin{pgfscope}
    \definecolor{fc}{rgb}{0.0000,0.0000,0.0000}
    \pgfsetfillcolor{fc}
    \pgftransformcm{1.0000}{0.0000}{0.0000}{1.0000}{\pgfqpoint{5.2308bp}{75.3846bp}}
    \pgftransformscale{0.1923}
    \pgftext[base,left]{positive unproven}
  \end{pgfscope}
  \begin{pgfscope}
    \definecolor{fc}{rgb}{1.0000,0.0000,0.0000}
    \pgfsetfillcolor{fc}
    \pgfsetlinewidth{0.5000bp}
    \definecolor{sc}{rgb}{1.0000,0.0000,0.0000}
    \pgfsetstrokecolor{sc}
    \pgfsetmiterjoin
    \pgfsetbuttcap
    \pgfpathqmoveto{3.6923bp}{75.8462bp}
    \pgfpathqcurveto{3.6923bp}{76.1860bp}{3.4168bp}{76.4615bp}{3.0769bp}{76.4615bp}
    \pgfpathqcurveto{2.7371bp}{76.4615bp}{2.4615bp}{76.1860bp}{2.4615bp}{75.8462bp}
    \pgfpathqcurveto{2.4615bp}{75.5063bp}{2.7371bp}{75.2308bp}{3.0769bp}{75.2308bp}
    \pgfpathqcurveto{3.4168bp}{75.2308bp}{3.6923bp}{75.5063bp}{3.6923bp}{75.8462bp}
    \pgfpathclose
    \pgfusepathqfillstroke
  \end{pgfscope}
  \begin{pgfscope}
    \definecolor{fc}{rgb}{0.0000,0.0000,0.0000}
    \pgfsetfillcolor{fc}
    \pgftransformcm{1.0000}{0.0000}{0.0000}{1.0000}{\pgfqpoint{5.2308bp}{77.3846bp}}
    \pgftransformscale{0.1923}
    \pgftext[base,left]{positive proven}
  \end{pgfscope}
  \begin{pgfscope}
    \definecolor{fc}{rgb}{0.0000,0.0000,1.0000}
    \pgfsetfillcolor{fc}
    \pgfsetlinewidth{0.5000bp}
    \definecolor{sc}{rgb}{0.0000,0.0000,1.0000}
    \pgfsetstrokecolor{sc}
    \pgfsetmiterjoin
    \pgfsetbuttcap
    \pgfpathqmoveto{3.6923bp}{77.8462bp}
    \pgfpathqcurveto{3.6923bp}{78.1860bp}{3.4168bp}{78.4615bp}{3.0769bp}{78.4615bp}
    \pgfpathqcurveto{2.7371bp}{78.4615bp}{2.4615bp}{78.1860bp}{2.4615bp}{77.8462bp}
    \pgfpathqcurveto{2.4615bp}{77.5063bp}{2.7371bp}{77.2308bp}{3.0769bp}{77.2308bp}
    \pgfpathqcurveto{3.4168bp}{77.2308bp}{3.6923bp}{77.5063bp}{3.6923bp}{77.8462bp}
    \pgfpathclose
    \pgfusepathqfillstroke
  \end{pgfscope}
  \begin{pgfscope}
    \pgfsetlinewidth{0.5000bp}
    \definecolor{sc}{rgb}{1.0000,1.0000,0.0000}
    \pgfsetstrokecolor{sc}
    \pgfsetmiterjoin
    \pgfsetbuttcap
    \pgfpathqmoveto{4.6154bp}{130.7692bp}
    \pgfpathqlineto{6.1538bp}{130.7692bp}
    \pgfpathqlineto{7.6923bp}{130.7692bp}
    \pgfpathqlineto{9.2308bp}{130.7692bp}
    \pgfpathqlineto{10.7692bp}{130.7692bp}
    \pgfpathqlineto{12.3077bp}{130.7692bp}
    \pgfpathqlineto{13.8462bp}{130.7692bp}
    \pgfpathqlineto{15.3846bp}{130.7692bp}
    \pgfpathqlineto{16.9231bp}{130.7692bp}
    \pgfpathqlineto{18.4615bp}{130.7692bp}
    \pgfpathqlineto{20.0000bp}{130.7692bp}
    \pgfpathqlineto{21.5385bp}{130.7692bp}
    \pgfpathqlineto{23.0769bp}{130.7692bp}
    \pgfpathqlineto{24.6154bp}{130.7692bp}
    \pgfpathqlineto{26.1538bp}{130.7692bp}
    \pgfpathqlineto{27.6923bp}{130.7692bp}
    \pgfpathqlineto{29.2308bp}{130.7692bp}
    \pgfpathqlineto{30.7692bp}{130.7692bp}
    \pgfpathqlineto{32.3077bp}{130.7692bp}
    \pgfpathqlineto{33.8462bp}{130.7692bp}
    \pgfpathqlineto{35.3846bp}{130.7692bp}
    \pgfpathqlineto{36.9231bp}{130.7692bp}
    \pgfpathqlineto{38.4615bp}{130.7692bp}
    \pgfpathqlineto{40.0000bp}{130.7692bp}
    \pgfpathqlineto{41.5385bp}{130.7692bp}
    \pgfpathqlineto{43.0769bp}{130.7692bp}
    \pgfpathqlineto{44.6154bp}{130.7692bp}
    \pgfpathqlineto{46.1538bp}{130.7692bp}
    \pgfpathqlineto{47.6923bp}{130.7692bp}
    \pgfpathqlineto{49.2308bp}{130.7692bp}
    \pgfpathqlineto{50.7692bp}{130.7692bp}
    \pgfpathqlineto{52.3077bp}{130.7692bp}
    \pgfpathqlineto{53.8462bp}{130.7692bp}
    \pgfpathqlineto{55.3846bp}{130.7692bp}
    \pgfpathqlineto{56.9231bp}{130.7692bp}
    \pgfpathqlineto{58.4615bp}{130.7692bp}
    \pgfpathqlineto{60.0000bp}{130.7692bp}
    \pgfpathqlineto{61.5385bp}{130.7692bp}
    \pgfpathqlineto{63.0769bp}{130.7692bp}
    \pgfpathqlineto{64.6154bp}{130.7692bp}
    \pgfpathqlineto{66.1538bp}{130.7692bp}
    \pgfpathqlineto{67.6923bp}{130.7692bp}
    \pgfpathqlineto{69.2308bp}{130.7692bp}
    \pgfpathqlineto{70.7692bp}{130.7692bp}
    \pgfpathqlineto{72.3077bp}{130.7692bp}
    \pgfpathqlineto{73.8462bp}{130.7692bp}
    \pgfpathqlineto{75.3846bp}{130.7692bp}
    \pgfpathqlineto{76.9231bp}{130.7692bp}
    \pgfpathqlineto{78.4615bp}{130.7692bp}
    \pgfpathqlineto{80.0000bp}{130.7692bp}
    \pgfpathqlineto{81.5385bp}{130.7692bp}
    \pgfpathqlineto{83.0769bp}{130.7692bp}
    \pgfpathqlineto{84.6154bp}{130.7692bp}
    \pgfpathqlineto{86.1538bp}{130.7692bp}
    \pgfpathqlineto{87.6923bp}{130.7692bp}
    \pgfpathqlineto{89.2308bp}{130.7692bp}
    \pgfpathqlineto{90.7692bp}{130.7692bp}
    \pgfpathqlineto{92.3077bp}{130.7692bp}
    \pgfpathqlineto{93.8462bp}{130.7692bp}
    \pgfpathqlineto{95.3846bp}{130.7692bp}
    \pgfpathqlineto{96.9231bp}{130.7692bp}
    \pgfpathqlineto{98.4615bp}{130.7692bp}
    \pgfpathqlineto{100.0000bp}{130.7692bp}
    \pgfpathqlineto{101.5385bp}{130.7692bp}
    \pgfpathqlineto{103.0769bp}{130.7692bp}
    \pgfpathqlineto{104.6154bp}{130.7692bp}
    \pgfpathqlineto{106.1538bp}{130.7692bp}
    \pgfpathqlineto{107.6923bp}{130.7692bp}
    \pgfpathqlineto{109.2308bp}{130.7692bp}
    \pgfpathqlineto{110.7692bp}{130.7692bp}
    \pgfpathqlineto{112.3077bp}{130.7692bp}
    \pgfpathqlineto{113.8462bp}{130.7692bp}
    \pgfpathqlineto{115.3846bp}{130.7692bp}
    \pgfpathqlineto{116.9231bp}{130.7692bp}
    \pgfpathqlineto{118.4615bp}{130.7692bp}
    \pgfpathqlineto{120.0000bp}{130.7692bp}
    \pgfpathqlineto{121.5385bp}{130.7692bp}
    \pgfpathqlineto{123.0769bp}{130.7692bp}
    \pgfpathqlineto{124.6154bp}{130.7692bp}
    \pgfpathqlineto{126.1538bp}{130.7692bp}
    \pgfpathqlineto{127.6923bp}{130.7692bp}
    \pgfpathqlineto{129.2308bp}{130.7692bp}
    \pgfpathqlineto{130.7692bp}{130.7692bp}
    \pgfpathqlineto{132.3077bp}{130.7692bp}
    \pgfpathqlineto{133.8462bp}{130.7692bp}
    \pgfpathqlineto{135.3846bp}{130.7692bp}
    \pgfpathqlineto{136.9231bp}{130.7692bp}
    \pgfpathqlineto{138.4615bp}{130.7692bp}
    \pgfpathqlineto{140.0000bp}{130.7692bp}
    \pgfpathqlineto{141.5385bp}{130.7692bp}
    \pgfpathqlineto{143.0769bp}{130.7692bp}
    \pgfpathqlineto{144.6154bp}{130.7692bp}
    \pgfpathqlineto{146.1538bp}{130.7692bp}
    \pgfpathqlineto{147.6923bp}{130.7692bp}
    \pgfpathqlineto{149.2308bp}{130.7692bp}
    \pgfpathqlineto{150.7692bp}{130.7692bp}
    \pgfpathqlineto{152.3077bp}{130.7692bp}
    \pgfpathqlineto{153.8462bp}{130.7692bp}
    \pgfpathqlineto{155.3846bp}{130.7692bp}
    \pgfpathqlineto{156.9231bp}{130.7692bp}
    \pgfpathqlineto{158.4615bp}{130.7692bp}
    \pgfpathqlineto{160.0000bp}{130.7692bp}
    \pgfpathqlineto{161.5385bp}{130.7692bp}
    \pgfpathqlineto{163.0769bp}{130.7692bp}
    \pgfpathqlineto{164.6154bp}{130.7692bp}
    \pgfpathqlineto{166.1538bp}{130.7692bp}
    \pgfpathqlineto{167.6923bp}{130.7692bp}
    \pgfpathqlineto{169.2308bp}{130.7692bp}
    \pgfpathqlineto{170.7692bp}{130.7692bp}
    \pgfpathqlineto{172.3077bp}{130.7692bp}
    \pgfpathqlineto{173.8462bp}{130.7692bp}
    \pgfpathqlineto{175.3846bp}{130.7692bp}
    \pgfpathqlineto{176.9231bp}{130.7692bp}
    \pgfpathqlineto{178.4615bp}{130.7692bp}
    \pgfpathqlineto{180.0000bp}{130.7692bp}
    \pgfpathqlineto{181.5385bp}{130.7692bp}
    \pgfpathqlineto{183.0769bp}{130.7692bp}
    \pgfpathqlineto{184.6154bp}{130.7692bp}
    \pgfpathqlineto{186.1538bp}{81.5385bp}
    \pgfpathqlineto{187.6923bp}{81.5385bp}
    \pgfpathqlineto{189.2308bp}{81.5385bp}
    \pgfpathqlineto{190.7692bp}{81.5385bp}
    \pgfpathqlineto{192.3077bp}{81.5385bp}
    \pgfpathqlineto{193.8462bp}{81.5385bp}
    \pgfpathqlineto{195.3846bp}{81.5385bp}
    \pgfpathqlineto{196.9231bp}{81.5385bp}
    \pgfpathqlineto{198.4615bp}{81.5385bp}
    \pgfpathqlineto{200.0000bp}{81.5385bp}
    \pgfusepathqstroke
  \end{pgfscope}
  \begin{pgfscope}
    \pgfsetlinewidth{0.5000bp}
    \definecolor{sc}{rgb}{0.0000,0.5020,0.0000}
    \pgfsetstrokecolor{sc}
    \pgfsetmiterjoin
    \pgfsetbuttcap
    \pgfpathqmoveto{4.6154bp}{81.5385bp}
    \pgfpathqlineto{6.1538bp}{81.5385bp}
    \pgfpathqlineto{7.6923bp}{81.5385bp}
    \pgfpathqlineto{9.2308bp}{81.5385bp}
    \pgfpathqlineto{10.7692bp}{81.5385bp}
    \pgfpathqlineto{12.3077bp}{81.5385bp}
    \pgfpathqlineto{13.8462bp}{81.5385bp}
    \pgfpathqlineto{15.3846bp}{81.5385bp}
    \pgfpathqlineto{16.9231bp}{81.5385bp}
    \pgfpathqlineto{18.4615bp}{81.5385bp}
    \pgfpathqlineto{20.0000bp}{81.5385bp}
    \pgfpathqlineto{21.5385bp}{81.5385bp}
    \pgfpathqlineto{23.0769bp}{81.5385bp}
    \pgfpathqlineto{24.6154bp}{81.5385bp}
    \pgfpathqlineto{26.1538bp}{81.5385bp}
    \pgfpathqlineto{27.6923bp}{81.5385bp}
    \pgfpathqlineto{29.2308bp}{81.5385bp}
    \pgfpathqlineto{30.7692bp}{81.5385bp}
    \pgfpathqlineto{32.3077bp}{81.5385bp}
    \pgfpathqlineto{33.8462bp}{81.5385bp}
    \pgfpathqlineto{35.3846bp}{81.5385bp}
    \pgfpathqlineto{36.9231bp}{81.5385bp}
    \pgfpathqlineto{38.4615bp}{81.5385bp}
    \pgfpathqlineto{40.0000bp}{81.5385bp}
    \pgfpathqlineto{41.5385bp}{81.5385bp}
    \pgfpathqlineto{43.0769bp}{81.5385bp}
    \pgfpathqlineto{44.6154bp}{81.5385bp}
    \pgfpathqlineto{46.1538bp}{81.5385bp}
    \pgfpathqlineto{47.6923bp}{81.5385bp}
    \pgfpathqlineto{49.2308bp}{81.5385bp}
    \pgfpathqlineto{50.7692bp}{81.5385bp}
    \pgfpathqlineto{52.3077bp}{81.5385bp}
    \pgfpathqlineto{53.8462bp}{81.5385bp}
    \pgfpathqlineto{55.3846bp}{81.5385bp}
    \pgfpathqlineto{56.9231bp}{81.5385bp}
    \pgfpathqlineto{58.4615bp}{81.5385bp}
    \pgfpathqlineto{60.0000bp}{81.5385bp}
    \pgfpathqlineto{61.5385bp}{81.5385bp}
    \pgfpathqlineto{63.0769bp}{81.5385bp}
    \pgfpathqlineto{64.6154bp}{81.5385bp}
    \pgfpathqlineto{66.1538bp}{81.5385bp}
    \pgfpathqlineto{67.6923bp}{81.5385bp}
    \pgfpathqlineto{69.2308bp}{81.5385bp}
    \pgfpathqlineto{70.7692bp}{81.5385bp}
    \pgfpathqlineto{72.3077bp}{81.5385bp}
    \pgfpathqlineto{73.8462bp}{81.5385bp}
    \pgfpathqlineto{75.3846bp}{81.5385bp}
    \pgfpathqlineto{76.9231bp}{81.5385bp}
    \pgfpathqlineto{78.4615bp}{81.5385bp}
    \pgfpathqlineto{80.0000bp}{81.5385bp}
    \pgfpathqlineto{81.5385bp}{81.5385bp}
    \pgfpathqlineto{83.0769bp}{81.5385bp}
    \pgfpathqlineto{84.6154bp}{81.5385bp}
    \pgfpathqlineto{86.1538bp}{81.5385bp}
    \pgfpathqlineto{87.6923bp}{81.5385bp}
    \pgfpathqlineto{89.2308bp}{81.5385bp}
    \pgfpathqlineto{90.7692bp}{81.5385bp}
    \pgfpathqlineto{92.3077bp}{81.5385bp}
    \pgfpathqlineto{93.8462bp}{81.5385bp}
    \pgfpathqlineto{95.3846bp}{81.5385bp}
    \pgfpathqlineto{96.9231bp}{81.5385bp}
    \pgfpathqlineto{98.4615bp}{81.5385bp}
    \pgfpathqlineto{100.0000bp}{81.5385bp}
    \pgfpathqlineto{101.5385bp}{81.5385bp}
    \pgfpathqlineto{103.0769bp}{81.5385bp}
    \pgfpathqlineto{104.6154bp}{81.5385bp}
    \pgfpathqlineto{106.1538bp}{81.5385bp}
    \pgfpathqlineto{107.6923bp}{81.5385bp}
    \pgfpathqlineto{109.2308bp}{81.5385bp}
    \pgfpathqlineto{110.7692bp}{81.5385bp}
    \pgfpathqlineto{112.3077bp}{81.5385bp}
    \pgfpathqlineto{113.8462bp}{81.5385bp}
    \pgfpathqlineto{115.3846bp}{81.5385bp}
    \pgfpathqlineto{116.9231bp}{81.5385bp}
    \pgfpathqlineto{118.4615bp}{81.5385bp}
    \pgfpathqlineto{120.0000bp}{81.5385bp}
    \pgfpathqlineto{121.5385bp}{81.5385bp}
    \pgfpathqlineto{123.0769bp}{81.5385bp}
    \pgfpathqlineto{124.6154bp}{81.5385bp}
    \pgfpathqlineto{126.1538bp}{81.5385bp}
    \pgfpathqlineto{127.6923bp}{81.5385bp}
    \pgfpathqlineto{129.2308bp}{81.5385bp}
    \pgfpathqlineto{130.7692bp}{81.5385bp}
    \pgfpathqlineto{132.3077bp}{81.5385bp}
    \pgfpathqlineto{133.8462bp}{81.5385bp}
    \pgfpathqlineto{135.3846bp}{81.5385bp}
    \pgfpathqlineto{136.9231bp}{81.5385bp}
    \pgfpathqlineto{138.4615bp}{81.5385bp}
    \pgfpathqlineto{140.0000bp}{81.5385bp}
    \pgfpathqlineto{141.5385bp}{81.5385bp}
    \pgfpathqlineto{143.0769bp}{81.5385bp}
    \pgfpathqlineto{144.6154bp}{81.5385bp}
    \pgfpathqlineto{146.1538bp}{81.5385bp}
    \pgfpathqlineto{147.6923bp}{81.5385bp}
    \pgfpathqlineto{149.2308bp}{81.5385bp}
    \pgfpathqlineto{150.7692bp}{81.5385bp}
    \pgfpathqlineto{152.3077bp}{81.5385bp}
    \pgfpathqlineto{153.8462bp}{81.5385bp}
    \pgfpathqlineto{155.3846bp}{81.5385bp}
    \pgfpathqlineto{156.9231bp}{81.5385bp}
    \pgfpathqlineto{158.4615bp}{81.5385bp}
    \pgfpathqlineto{160.0000bp}{81.5385bp}
    \pgfpathqlineto{161.5385bp}{81.5385bp}
    \pgfpathqlineto{163.0769bp}{81.5385bp}
    \pgfpathqlineto{164.6154bp}{81.5385bp}
    \pgfpathqlineto{166.1538bp}{81.5385bp}
    \pgfpathqlineto{167.6923bp}{81.5385bp}
    \pgfpathqlineto{169.2308bp}{81.5385bp}
    \pgfpathqlineto{170.7692bp}{81.5385bp}
    \pgfpathqlineto{172.3077bp}{81.5385bp}
    \pgfpathqlineto{173.8462bp}{81.5385bp}
    \pgfpathqlineto{175.3846bp}{81.5385bp}
    \pgfpathqlineto{176.9231bp}{81.5385bp}
    \pgfpathqlineto{178.4615bp}{81.5385bp}
    \pgfpathqlineto{180.0000bp}{81.5385bp}
    \pgfpathqlineto{181.5385bp}{81.5385bp}
    \pgfpathqlineto{183.0769bp}{81.5385bp}
    \pgfpathqlineto{184.6154bp}{81.5385bp}
    \pgfpathqlineto{186.1538bp}{83.8462bp}
    \pgfpathqlineto{187.6923bp}{83.8462bp}
    \pgfpathqlineto{189.2308bp}{83.8462bp}
    \pgfpathqlineto{190.7692bp}{83.8462bp}
    \pgfpathqlineto{192.3077bp}{83.8462bp}
    \pgfpathqlineto{193.8462bp}{83.8462bp}
    \pgfpathqlineto{195.3846bp}{83.8462bp}
    \pgfpathqlineto{196.9231bp}{83.8462bp}
    \pgfpathqlineto{198.4615bp}{83.8462bp}
    \pgfpathqlineto{200.0000bp}{83.8462bp}
    \pgfusepathqstroke
  \end{pgfscope}
  \begin{pgfscope}
    \pgfsetlinewidth{0.5000bp}
    \definecolor{sc}{rgb}{1.0000,0.0000,0.0000}
    \pgfsetstrokecolor{sc}
    \pgfsetmiterjoin
    \pgfsetbuttcap
    \pgfpathqmoveto{4.6154bp}{130.7692bp}
    \pgfpathqlineto{6.1538bp}{130.7692bp}
    \pgfpathqlineto{7.6923bp}{130.7692bp}
    \pgfpathqlineto{9.2308bp}{130.7692bp}
    \pgfpathqlineto{10.7692bp}{130.7692bp}
    \pgfpathqlineto{12.3077bp}{130.7692bp}
    \pgfpathqlineto{13.8462bp}{130.7692bp}
    \pgfpathqlineto{15.3846bp}{130.7692bp}
    \pgfpathqlineto{16.9231bp}{130.7692bp}
    \pgfpathqlineto{18.4615bp}{130.7692bp}
    \pgfpathqlineto{20.0000bp}{130.7692bp}
    \pgfpathqlineto{21.5385bp}{130.7692bp}
    \pgfpathqlineto{23.0769bp}{130.7692bp}
    \pgfpathqlineto{24.6154bp}{130.7692bp}
    \pgfpathqlineto{26.1538bp}{130.7692bp}
    \pgfpathqlineto{27.6923bp}{130.7692bp}
    \pgfpathqlineto{29.2308bp}{130.7692bp}
    \pgfpathqlineto{30.7692bp}{130.7692bp}
    \pgfpathqlineto{32.3077bp}{130.7692bp}
    \pgfpathqlineto{33.8462bp}{130.7692bp}
    \pgfpathqlineto{35.3846bp}{130.7692bp}
    \pgfpathqlineto{36.9231bp}{130.7692bp}
    \pgfpathqlineto{38.4615bp}{130.7692bp}
    \pgfpathqlineto{40.0000bp}{130.7692bp}
    \pgfpathqlineto{41.5385bp}{130.7692bp}
    \pgfpathqlineto{43.0769bp}{130.7692bp}
    \pgfpathqlineto{44.6154bp}{130.7692bp}
    \pgfpathqlineto{46.1538bp}{130.7692bp}
    \pgfpathqlineto{47.6923bp}{130.7692bp}
    \pgfpathqlineto{49.2308bp}{130.7692bp}
    \pgfpathqlineto{50.7692bp}{130.7692bp}
    \pgfpathqlineto{52.3077bp}{130.7692bp}
    \pgfpathqlineto{53.8462bp}{130.7692bp}
    \pgfpathqlineto{55.3846bp}{130.7692bp}
    \pgfpathqlineto{56.9231bp}{130.7692bp}
    \pgfpathqlineto{58.4615bp}{130.7692bp}
    \pgfpathqlineto{60.0000bp}{130.7692bp}
    \pgfpathqlineto{61.5385bp}{130.7692bp}
    \pgfpathqlineto{63.0769bp}{130.7692bp}
    \pgfpathqlineto{64.6154bp}{130.7692bp}
    \pgfpathqlineto{66.1538bp}{130.7692bp}
    \pgfpathqlineto{67.6923bp}{130.7692bp}
    \pgfpathqlineto{69.2308bp}{130.7692bp}
    \pgfpathqlineto{70.7692bp}{130.7692bp}
    \pgfpathqlineto{72.3077bp}{130.7692bp}
    \pgfpathqlineto{73.8462bp}{130.7692bp}
    \pgfpathqlineto{75.3846bp}{130.7692bp}
    \pgfpathqlineto{76.9231bp}{130.7692bp}
    \pgfpathqlineto{78.4615bp}{130.7692bp}
    \pgfpathqlineto{80.0000bp}{130.7692bp}
    \pgfpathqlineto{81.5385bp}{130.7692bp}
    \pgfpathqlineto{83.0769bp}{130.7692bp}
    \pgfpathqlineto{84.6154bp}{130.7692bp}
    \pgfpathqlineto{86.1538bp}{130.7692bp}
    \pgfpathqlineto{87.6923bp}{130.7692bp}
    \pgfpathqlineto{89.2308bp}{130.7692bp}
    \pgfpathqlineto{90.7692bp}{130.7692bp}
    \pgfpathqlineto{92.3077bp}{130.7692bp}
    \pgfpathqlineto{93.8462bp}{130.7692bp}
    \pgfpathqlineto{95.3846bp}{130.7692bp}
    \pgfpathqlineto{96.9231bp}{130.7692bp}
    \pgfpathqlineto{98.4615bp}{130.7692bp}
    \pgfpathqlineto{100.0000bp}{130.7692bp}
    \pgfpathqlineto{101.5385bp}{130.7692bp}
    \pgfpathqlineto{103.0769bp}{130.7692bp}
    \pgfpathqlineto{104.6154bp}{130.7692bp}
    \pgfpathqlineto{106.1538bp}{130.7692bp}
    \pgfpathqlineto{107.6923bp}{130.7692bp}
    \pgfpathqlineto{109.2308bp}{130.7692bp}
    \pgfpathqlineto{110.7692bp}{130.7692bp}
    \pgfpathqlineto{112.3077bp}{130.7692bp}
    \pgfpathqlineto{113.8462bp}{130.7692bp}
    \pgfpathqlineto{115.3846bp}{130.7692bp}
    \pgfpathqlineto{116.9231bp}{130.7692bp}
    \pgfpathqlineto{118.4615bp}{130.7692bp}
    \pgfpathqlineto{120.0000bp}{130.7692bp}
    \pgfpathqlineto{121.5385bp}{130.7692bp}
    \pgfpathqlineto{123.0769bp}{130.7692bp}
    \pgfpathqlineto{124.6154bp}{130.7692bp}
    \pgfpathqlineto{126.1538bp}{130.7692bp}
    \pgfpathqlineto{127.6923bp}{130.7692bp}
    \pgfpathqlineto{129.2308bp}{130.7692bp}
    \pgfpathqlineto{130.7692bp}{130.7692bp}
    \pgfpathqlineto{132.3077bp}{130.7692bp}
    \pgfpathqlineto{133.8462bp}{130.7692bp}
    \pgfpathqlineto{135.3846bp}{130.7692bp}
    \pgfpathqlineto{136.9231bp}{130.7692bp}
    \pgfpathqlineto{138.4615bp}{130.7692bp}
    \pgfpathqlineto{140.0000bp}{130.7692bp}
    \pgfpathqlineto{141.5385bp}{130.7692bp}
    \pgfpathqlineto{143.0769bp}{130.7692bp}
    \pgfpathqlineto{144.6154bp}{130.7692bp}
    \pgfpathqlineto{146.1538bp}{130.7692bp}
    \pgfpathqlineto{147.6923bp}{130.7692bp}
    \pgfpathqlineto{149.2308bp}{130.7692bp}
    \pgfpathqlineto{150.7692bp}{130.7692bp}
    \pgfpathqlineto{152.3077bp}{130.7692bp}
    \pgfpathqlineto{153.8462bp}{130.7692bp}
    \pgfpathqlineto{155.3846bp}{130.7692bp}
    \pgfpathqlineto{156.9231bp}{130.7692bp}
    \pgfpathqlineto{158.4615bp}{130.7692bp}
    \pgfpathqlineto{160.0000bp}{130.7692bp}
    \pgfpathqlineto{161.5385bp}{130.7692bp}
    \pgfpathqlineto{163.0769bp}{130.7692bp}
    \pgfpathqlineto{164.6154bp}{130.7692bp}
    \pgfpathqlineto{166.1538bp}{130.7692bp}
    \pgfpathqlineto{167.6923bp}{130.7692bp}
    \pgfpathqlineto{169.2308bp}{130.7692bp}
    \pgfpathqlineto{170.7692bp}{130.7692bp}
    \pgfpathqlineto{172.3077bp}{130.7692bp}
    \pgfpathqlineto{173.8462bp}{130.7692bp}
    \pgfpathqlineto{175.3846bp}{130.7692bp}
    \pgfpathqlineto{176.9231bp}{130.7692bp}
    \pgfpathqlineto{178.4615bp}{130.7692bp}
    \pgfpathqlineto{180.0000bp}{130.7692bp}
    \pgfpathqlineto{181.5385bp}{130.7692bp}
    \pgfpathqlineto{183.0769bp}{130.7692bp}
    \pgfpathqlineto{184.6154bp}{130.7692bp}
    \pgfpathqlineto{186.1538bp}{81.5385bp}
    \pgfpathqlineto{187.6923bp}{81.5385bp}
    \pgfpathqlineto{189.2308bp}{81.5385bp}
    \pgfpathqlineto{190.7692bp}{81.5385bp}
    \pgfpathqlineto{192.3077bp}{81.5385bp}
    \pgfpathqlineto{193.8462bp}{81.5385bp}
    \pgfpathqlineto{195.3846bp}{81.5385bp}
    \pgfpathqlineto{196.9231bp}{81.5385bp}
    \pgfpathqlineto{198.4615bp}{81.5385bp}
    \pgfpathqlineto{200.0000bp}{81.5385bp}
    \pgfusepathqstroke
  \end{pgfscope}
  \begin{pgfscope}
    \pgfsetlinewidth{0.5000bp}
    \definecolor{sc}{rgb}{0.0000,0.0000,1.0000}
    \pgfsetstrokecolor{sc}
    \pgfsetmiterjoin
    \pgfsetbuttcap
    \pgfpathqmoveto{4.6154bp}{81.5385bp}
    \pgfpathqlineto{6.1538bp}{81.5385bp}
    \pgfpathqlineto{7.6923bp}{81.5385bp}
    \pgfpathqlineto{9.2308bp}{81.5385bp}
    \pgfpathqlineto{10.7692bp}{81.5385bp}
    \pgfpathqlineto{12.3077bp}{81.5385bp}
    \pgfpathqlineto{13.8462bp}{81.5385bp}
    \pgfpathqlineto{15.3846bp}{81.5385bp}
    \pgfpathqlineto{16.9231bp}{81.5385bp}
    \pgfpathqlineto{18.4615bp}{81.5385bp}
    \pgfpathqlineto{20.0000bp}{81.5385bp}
    \pgfpathqlineto{21.5385bp}{81.5385bp}
    \pgfpathqlineto{23.0769bp}{81.5385bp}
    \pgfpathqlineto{24.6154bp}{81.5385bp}
    \pgfpathqlineto{26.1538bp}{81.5385bp}
    \pgfpathqlineto{27.6923bp}{81.5385bp}
    \pgfpathqlineto{29.2308bp}{81.5385bp}
    \pgfpathqlineto{30.7692bp}{81.5385bp}
    \pgfpathqlineto{32.3077bp}{81.5385bp}
    \pgfpathqlineto{33.8462bp}{81.5385bp}
    \pgfpathqlineto{35.3846bp}{81.5385bp}
    \pgfpathqlineto{36.9231bp}{81.5385bp}
    \pgfpathqlineto{38.4615bp}{81.5385bp}
    \pgfpathqlineto{40.0000bp}{81.5385bp}
    \pgfpathqlineto{41.5385bp}{81.5385bp}
    \pgfpathqlineto{43.0769bp}{81.5385bp}
    \pgfpathqlineto{44.6154bp}{81.5385bp}
    \pgfpathqlineto{46.1538bp}{81.5385bp}
    \pgfpathqlineto{47.6923bp}{81.5385bp}
    \pgfpathqlineto{49.2308bp}{81.5385bp}
    \pgfpathqlineto{50.7692bp}{81.5385bp}
    \pgfpathqlineto{52.3077bp}{81.5385bp}
    \pgfpathqlineto{53.8462bp}{81.5385bp}
    \pgfpathqlineto{55.3846bp}{81.5385bp}
    \pgfpathqlineto{56.9231bp}{81.5385bp}
    \pgfpathqlineto{58.4615bp}{81.5385bp}
    \pgfpathqlineto{60.0000bp}{81.5385bp}
    \pgfpathqlineto{61.5385bp}{81.5385bp}
    \pgfpathqlineto{63.0769bp}{81.5385bp}
    \pgfpathqlineto{64.6154bp}{81.5385bp}
    \pgfpathqlineto{66.1538bp}{81.5385bp}
    \pgfpathqlineto{67.6923bp}{81.5385bp}
    \pgfpathqlineto{69.2308bp}{81.5385bp}
    \pgfpathqlineto{70.7692bp}{81.5385bp}
    \pgfpathqlineto{72.3077bp}{81.5385bp}
    \pgfpathqlineto{73.8462bp}{81.5385bp}
    \pgfpathqlineto{75.3846bp}{81.5385bp}
    \pgfpathqlineto{76.9231bp}{81.5385bp}
    \pgfpathqlineto{78.4615bp}{81.5385bp}
    \pgfpathqlineto{80.0000bp}{81.5385bp}
    \pgfpathqlineto{81.5385bp}{81.5385bp}
    \pgfpathqlineto{83.0769bp}{81.5385bp}
    \pgfpathqlineto{84.6154bp}{81.5385bp}
    \pgfpathqlineto{86.1538bp}{81.5385bp}
    \pgfpathqlineto{87.6923bp}{81.5385bp}
    \pgfpathqlineto{89.2308bp}{81.5385bp}
    \pgfpathqlineto{90.7692bp}{81.5385bp}
    \pgfpathqlineto{92.3077bp}{81.5385bp}
    \pgfpathqlineto{93.8462bp}{81.5385bp}
    \pgfpathqlineto{95.3846bp}{81.5385bp}
    \pgfpathqlineto{96.9231bp}{81.5385bp}
    \pgfpathqlineto{98.4615bp}{81.5385bp}
    \pgfpathqlineto{100.0000bp}{81.5385bp}
    \pgfpathqlineto{101.5385bp}{81.5385bp}
    \pgfpathqlineto{103.0769bp}{81.5385bp}
    \pgfpathqlineto{104.6154bp}{81.5385bp}
    \pgfpathqlineto{106.1538bp}{81.5385bp}
    \pgfpathqlineto{107.6923bp}{81.5385bp}
    \pgfpathqlineto{109.2308bp}{81.5385bp}
    \pgfpathqlineto{110.7692bp}{81.5385bp}
    \pgfpathqlineto{112.3077bp}{81.5385bp}
    \pgfpathqlineto{113.8462bp}{81.5385bp}
    \pgfpathqlineto{115.3846bp}{81.5385bp}
    \pgfpathqlineto{116.9231bp}{81.5385bp}
    \pgfpathqlineto{118.4615bp}{81.5385bp}
    \pgfpathqlineto{120.0000bp}{81.5385bp}
    \pgfpathqlineto{121.5385bp}{81.5385bp}
    \pgfpathqlineto{123.0769bp}{81.5385bp}
    \pgfpathqlineto{124.6154bp}{81.5385bp}
    \pgfpathqlineto{126.1538bp}{81.5385bp}
    \pgfpathqlineto{127.6923bp}{81.5385bp}
    \pgfpathqlineto{129.2308bp}{81.5385bp}
    \pgfpathqlineto{130.7692bp}{81.5385bp}
    \pgfpathqlineto{132.3077bp}{81.5385bp}
    \pgfpathqlineto{133.8462bp}{81.5385bp}
    \pgfpathqlineto{135.3846bp}{81.5385bp}
    \pgfpathqlineto{136.9231bp}{81.5385bp}
    \pgfpathqlineto{138.4615bp}{81.5385bp}
    \pgfpathqlineto{140.0000bp}{81.5385bp}
    \pgfpathqlineto{141.5385bp}{81.5385bp}
    \pgfpathqlineto{143.0769bp}{81.5385bp}
    \pgfpathqlineto{144.6154bp}{81.5385bp}
    \pgfpathqlineto{146.1538bp}{81.5385bp}
    \pgfpathqlineto{147.6923bp}{81.5385bp}
    \pgfpathqlineto{149.2308bp}{81.5385bp}
    \pgfpathqlineto{150.7692bp}{81.5385bp}
    \pgfpathqlineto{152.3077bp}{81.5385bp}
    \pgfpathqlineto{153.8462bp}{81.5385bp}
    \pgfpathqlineto{155.3846bp}{81.5385bp}
    \pgfpathqlineto{156.9231bp}{81.5385bp}
    \pgfpathqlineto{158.4615bp}{81.5385bp}
    \pgfpathqlineto{160.0000bp}{81.5385bp}
    \pgfpathqlineto{161.5385bp}{81.5385bp}
    \pgfpathqlineto{163.0769bp}{81.5385bp}
    \pgfpathqlineto{164.6154bp}{81.5385bp}
    \pgfpathqlineto{166.1538bp}{81.5385bp}
    \pgfpathqlineto{167.6923bp}{81.5385bp}
    \pgfpathqlineto{169.2308bp}{81.5385bp}
    \pgfpathqlineto{170.7692bp}{81.5385bp}
    \pgfpathqlineto{172.3077bp}{81.5385bp}
    \pgfpathqlineto{173.8462bp}{81.5385bp}
    \pgfpathqlineto{175.3846bp}{81.5385bp}
    \pgfpathqlineto{176.9231bp}{81.5385bp}
    \pgfpathqlineto{178.4615bp}{81.5385bp}
    \pgfpathqlineto{180.0000bp}{81.5385bp}
    \pgfpathqlineto{181.5385bp}{81.5385bp}
    \pgfpathqlineto{183.0769bp}{81.5385bp}
    \pgfpathqlineto{184.6154bp}{81.5385bp}
    \pgfpathqlineto{186.1538bp}{83.8462bp}
    \pgfpathqlineto{187.6923bp}{83.8462bp}
    \pgfpathqlineto{189.2308bp}{83.8462bp}
    \pgfpathqlineto{190.7692bp}{83.8462bp}
    \pgfpathqlineto{192.3077bp}{83.8462bp}
    \pgfpathqlineto{193.8462bp}{83.8462bp}
    \pgfpathqlineto{195.3846bp}{83.8462bp}
    \pgfpathqlineto{196.9231bp}{83.8462bp}
    \pgfpathqlineto{198.4615bp}{83.8462bp}
    \pgfpathqlineto{200.0000bp}{83.8462bp}
    \pgfusepathqstroke
  \end{pgfscope}
  \begin{pgfscope}
    \pgfsetlinewidth{0.5000bp}
    \definecolor{sc}{rgb}{1.0000,0.0000,0.0000}
    \pgfsetstrokecolor{sc}
    \pgfsetmiterjoin
    \pgfsetbuttcap
    \pgfpathqmoveto{16.9231bp}{81.5385bp}
    \pgfpathqlineto{16.9231bp}{80.0000bp}
    \pgfusepathqstroke
  \end{pgfscope}
  \begin{pgfscope}
    \pgfsetlinewidth{0.5000bp}
    \definecolor{sc}{rgb}{1.0000,0.0000,0.0000}
    \pgfsetstrokecolor{sc}
    \pgfsetmiterjoin
    \pgfsetbuttcap
    \pgfpathqmoveto{186.1538bp}{81.5385bp}
    \pgfpathqlineto{186.1538bp}{80.0000bp}
    \pgfusepathqstroke
  \end{pgfscope}
  \begin{pgfscope}
    \pgfsetlinewidth{0.5000bp}
    \definecolor{sc}{rgb}{0.0000,0.0000,0.0000}
    \pgfsetstrokecolor{sc}
    \pgfsetmiterjoin
    \pgfsetbuttcap
    \pgfpathqmoveto{195.3846bp}{81.5385bp}
    \pgfpathqlineto{195.3846bp}{80.7692bp}
    \pgfusepathqstroke
  \end{pgfscope}
  \begin{pgfscope}
    \pgfsetlinewidth{0.5000bp}
    \definecolor{sc}{rgb}{0.0000,0.0000,0.0000}
    \pgfsetstrokecolor{sc}
    \pgfsetmiterjoin
    \pgfsetbuttcap
    \pgfpathqmoveto{187.6923bp}{81.5385bp}
    \pgfpathqlineto{187.6923bp}{80.7692bp}
    \pgfusepathqstroke
  \end{pgfscope}
  \begin{pgfscope}
    \pgfsetlinewidth{0.5000bp}
    \definecolor{sc}{rgb}{0.0000,0.0000,0.0000}
    \pgfsetstrokecolor{sc}
    \pgfsetmiterjoin
    \pgfsetbuttcap
    \pgfpathqmoveto{180.0000bp}{81.5385bp}
    \pgfpathqlineto{180.0000bp}{80.7692bp}
    \pgfusepathqstroke
  \end{pgfscope}
  \begin{pgfscope}
    \pgfsetlinewidth{0.5000bp}
    \definecolor{sc}{rgb}{0.0000,0.0000,0.0000}
    \pgfsetstrokecolor{sc}
    \pgfsetmiterjoin
    \pgfsetbuttcap
    \pgfpathqmoveto{172.3077bp}{81.5385bp}
    \pgfpathqlineto{172.3077bp}{80.7692bp}
    \pgfusepathqstroke
  \end{pgfscope}
  \begin{pgfscope}
    \pgfsetlinewidth{0.5000bp}
    \definecolor{sc}{rgb}{0.0000,0.0000,0.0000}
    \pgfsetstrokecolor{sc}
    \pgfsetmiterjoin
    \pgfsetbuttcap
    \pgfpathqmoveto{164.6154bp}{81.5385bp}
    \pgfpathqlineto{164.6154bp}{80.7692bp}
    \pgfusepathqstroke
  \end{pgfscope}
  \begin{pgfscope}
    \pgfsetlinewidth{0.5000bp}
    \definecolor{sc}{rgb}{0.0000,0.0000,0.0000}
    \pgfsetstrokecolor{sc}
    \pgfsetmiterjoin
    \pgfsetbuttcap
    \pgfpathqmoveto{156.9231bp}{81.5385bp}
    \pgfpathqlineto{156.9231bp}{80.7692bp}
    \pgfusepathqstroke
  \end{pgfscope}
  \begin{pgfscope}
    \pgfsetlinewidth{0.5000bp}
    \definecolor{sc}{rgb}{0.0000,0.0000,0.0000}
    \pgfsetstrokecolor{sc}
    \pgfsetmiterjoin
    \pgfsetbuttcap
    \pgfpathqmoveto{149.2308bp}{81.5385bp}
    \pgfpathqlineto{149.2308bp}{80.7692bp}
    \pgfusepathqstroke
  \end{pgfscope}
  \begin{pgfscope}
    \pgfsetlinewidth{0.5000bp}
    \definecolor{sc}{rgb}{0.0000,0.0000,0.0000}
    \pgfsetstrokecolor{sc}
    \pgfsetmiterjoin
    \pgfsetbuttcap
    \pgfpathqmoveto{141.5385bp}{81.5385bp}
    \pgfpathqlineto{141.5385bp}{80.7692bp}
    \pgfusepathqstroke
  \end{pgfscope}
  \begin{pgfscope}
    \pgfsetlinewidth{0.5000bp}
    \definecolor{sc}{rgb}{0.0000,0.0000,0.0000}
    \pgfsetstrokecolor{sc}
    \pgfsetmiterjoin
    \pgfsetbuttcap
    \pgfpathqmoveto{133.8462bp}{81.5385bp}
    \pgfpathqlineto{133.8462bp}{80.7692bp}
    \pgfusepathqstroke
  \end{pgfscope}
  \begin{pgfscope}
    \pgfsetlinewidth{0.5000bp}
    \definecolor{sc}{rgb}{0.0000,0.0000,0.0000}
    \pgfsetstrokecolor{sc}
    \pgfsetmiterjoin
    \pgfsetbuttcap
    \pgfpathqmoveto{126.1538bp}{81.5385bp}
    \pgfpathqlineto{126.1538bp}{80.7692bp}
    \pgfusepathqstroke
  \end{pgfscope}
  \begin{pgfscope}
    \pgfsetlinewidth{0.5000bp}
    \definecolor{sc}{rgb}{0.0000,0.0000,0.0000}
    \pgfsetstrokecolor{sc}
    \pgfsetmiterjoin
    \pgfsetbuttcap
    \pgfpathqmoveto{118.4615bp}{81.5385bp}
    \pgfpathqlineto{118.4615bp}{80.7692bp}
    \pgfusepathqstroke
  \end{pgfscope}
  \begin{pgfscope}
    \pgfsetlinewidth{0.5000bp}
    \definecolor{sc}{rgb}{0.0000,0.0000,0.0000}
    \pgfsetstrokecolor{sc}
    \pgfsetmiterjoin
    \pgfsetbuttcap
    \pgfpathqmoveto{110.7692bp}{81.5385bp}
    \pgfpathqlineto{110.7692bp}{80.7692bp}
    \pgfusepathqstroke
  \end{pgfscope}
  \begin{pgfscope}
    \pgfsetlinewidth{0.5000bp}
    \definecolor{sc}{rgb}{0.0000,0.0000,0.0000}
    \pgfsetstrokecolor{sc}
    \pgfsetmiterjoin
    \pgfsetbuttcap
    \pgfpathqmoveto{103.0769bp}{81.5385bp}
    \pgfpathqlineto{103.0769bp}{80.7692bp}
    \pgfusepathqstroke
  \end{pgfscope}
  \begin{pgfscope}
    \pgfsetlinewidth{0.5000bp}
    \definecolor{sc}{rgb}{0.0000,0.0000,0.0000}
    \pgfsetstrokecolor{sc}
    \pgfsetmiterjoin
    \pgfsetbuttcap
    \pgfpathqmoveto{95.3846bp}{81.5385bp}
    \pgfpathqlineto{95.3846bp}{80.7692bp}
    \pgfusepathqstroke
  \end{pgfscope}
  \begin{pgfscope}
    \pgfsetlinewidth{0.5000bp}
    \definecolor{sc}{rgb}{0.0000,0.0000,0.0000}
    \pgfsetstrokecolor{sc}
    \pgfsetmiterjoin
    \pgfsetbuttcap
    \pgfpathqmoveto{87.6923bp}{81.5385bp}
    \pgfpathqlineto{87.6923bp}{80.7692bp}
    \pgfusepathqstroke
  \end{pgfscope}
  \begin{pgfscope}
    \pgfsetlinewidth{0.5000bp}
    \definecolor{sc}{rgb}{0.0000,0.0000,0.0000}
    \pgfsetstrokecolor{sc}
    \pgfsetmiterjoin
    \pgfsetbuttcap
    \pgfpathqmoveto{80.0000bp}{81.5385bp}
    \pgfpathqlineto{80.0000bp}{80.7692bp}
    \pgfusepathqstroke
  \end{pgfscope}
  \begin{pgfscope}
    \pgfsetlinewidth{0.5000bp}
    \definecolor{sc}{rgb}{0.0000,0.0000,0.0000}
    \pgfsetstrokecolor{sc}
    \pgfsetmiterjoin
    \pgfsetbuttcap
    \pgfpathqmoveto{72.3077bp}{81.5385bp}
    \pgfpathqlineto{72.3077bp}{80.7692bp}
    \pgfusepathqstroke
  \end{pgfscope}
  \begin{pgfscope}
    \pgfsetlinewidth{0.5000bp}
    \definecolor{sc}{rgb}{0.0000,0.0000,0.0000}
    \pgfsetstrokecolor{sc}
    \pgfsetmiterjoin
    \pgfsetbuttcap
    \pgfpathqmoveto{64.6154bp}{81.5385bp}
    \pgfpathqlineto{64.6154bp}{80.7692bp}
    \pgfusepathqstroke
  \end{pgfscope}
  \begin{pgfscope}
    \pgfsetlinewidth{0.5000bp}
    \definecolor{sc}{rgb}{0.0000,0.0000,0.0000}
    \pgfsetstrokecolor{sc}
    \pgfsetmiterjoin
    \pgfsetbuttcap
    \pgfpathqmoveto{56.9231bp}{81.5385bp}
    \pgfpathqlineto{56.9231bp}{80.7692bp}
    \pgfusepathqstroke
  \end{pgfscope}
  \begin{pgfscope}
    \pgfsetlinewidth{0.5000bp}
    \definecolor{sc}{rgb}{0.0000,0.0000,0.0000}
    \pgfsetstrokecolor{sc}
    \pgfsetmiterjoin
    \pgfsetbuttcap
    \pgfpathqmoveto{49.2308bp}{81.5385bp}
    \pgfpathqlineto{49.2308bp}{80.7692bp}
    \pgfusepathqstroke
  \end{pgfscope}
  \begin{pgfscope}
    \pgfsetlinewidth{0.5000bp}
    \definecolor{sc}{rgb}{0.0000,0.0000,0.0000}
    \pgfsetstrokecolor{sc}
    \pgfsetmiterjoin
    \pgfsetbuttcap
    \pgfpathqmoveto{41.5385bp}{81.5385bp}
    \pgfpathqlineto{41.5385bp}{80.7692bp}
    \pgfusepathqstroke
  \end{pgfscope}
  \begin{pgfscope}
    \pgfsetlinewidth{0.5000bp}
    \definecolor{sc}{rgb}{0.0000,0.0000,0.0000}
    \pgfsetstrokecolor{sc}
    \pgfsetmiterjoin
    \pgfsetbuttcap
    \pgfpathqmoveto{33.8462bp}{81.5385bp}
    \pgfpathqlineto{33.8462bp}{80.7692bp}
    \pgfusepathqstroke
  \end{pgfscope}
  \begin{pgfscope}
    \pgfsetlinewidth{0.5000bp}
    \definecolor{sc}{rgb}{0.0000,0.0000,0.0000}
    \pgfsetstrokecolor{sc}
    \pgfsetmiterjoin
    \pgfsetbuttcap
    \pgfpathqmoveto{26.1538bp}{81.5385bp}
    \pgfpathqlineto{26.1538bp}{80.7692bp}
    \pgfusepathqstroke
  \end{pgfscope}
  \begin{pgfscope}
    \pgfsetlinewidth{0.5000bp}
    \definecolor{sc}{rgb}{0.0000,0.0000,0.0000}
    \pgfsetstrokecolor{sc}
    \pgfsetmiterjoin
    \pgfsetbuttcap
    \pgfpathqmoveto{18.4615bp}{81.5385bp}
    \pgfpathqlineto{18.4615bp}{80.7692bp}
    \pgfusepathqstroke
  \end{pgfscope}
  \begin{pgfscope}
    \pgfsetlinewidth{0.5000bp}
    \definecolor{sc}{rgb}{0.0000,0.0000,0.0000}
    \pgfsetstrokecolor{sc}
    \pgfsetmiterjoin
    \pgfsetbuttcap
    \pgfpathqmoveto{10.7692bp}{81.5385bp}
    \pgfpathqlineto{10.7692bp}{80.7692bp}
    \pgfusepathqstroke
  \end{pgfscope}
  \begin{pgfscope}
    \definecolor{fc}{rgb}{0.0000,0.0000,0.0000}
    \pgfsetfillcolor{fc}
    \pgftransformcm{1.0000}{0.0000}{0.0000}{1.0000}{\pgfqpoint{-0.0000bp}{130.3077bp}}
    \pgftransformscale{0.1923}
    \pgftext[base,left]{$\mathbb{F}_A$}
  \end{pgfscope}
  \begin{pgfscope}
    \pgfsetlinewidth{0.5000bp}
    \definecolor{sc}{rgb}{0.0000,0.0000,0.0000}
    \pgfsetstrokecolor{sc}
    \pgfsetmiterjoin
    \pgfsetbuttcap
    \pgfpathqmoveto{3.0769bp}{130.7692bp}
    \pgfpathqlineto{2.7692bp}{130.7692bp}
    \pgfusepathqstroke
  \end{pgfscope}
  \begin{pgfscope}
    \pgfsetlinewidth{0.5000bp}
    \definecolor{sc}{rgb}{0.0000,0.0000,0.0000}
    \pgfsetstrokecolor{sc}
    \pgfsetmiterjoin
    \pgfsetbuttcap
    \pgfpathqmoveto{3.0769bp}{81.5385bp}
    \pgfpathqlineto{3.0769bp}{130.7692bp}
    \pgfusepathqstroke
  \end{pgfscope}
  \begin{pgfscope}
    \pgfsetlinewidth{0.5000bp}
    \definecolor{sc}{rgb}{0.0000,0.0000,0.0000}
    \pgfsetstrokecolor{sc}
    \pgfsetmiterjoin
    \pgfsetbuttcap
    \pgfpathqmoveto{3.0769bp}{81.5385bp}
    \pgfpathqlineto{200.0000bp}{81.5385bp}
    \pgfusepathqstroke
  \end{pgfscope}
\end{pgfpicture}

        \label{fig:ex:ca:hgma:ex:move-h}
    \caption{push-h effects}\label{fig:ex:ca:hgma:ex:disconnected}
\end{figure}

\begin{figure}
    \centering
    \begin{pgfpicture}
  \pgfpathrectangle{\pgfpointorigin}{\pgfqpoint{200.0000bp}{200.0000bp}}
  \pgfusepath{use as bounding box}
  \begin{pgfscope}
    \definecolor{fc}{rgb}{0.0000,0.0000,0.0000}
    \pgfsetfillcolor{fc}
    \pgftransformshift{\pgfqpoint{3.2227bp}{81.1374bp}}
    \pgftransformscale{0.1185}
    \pgftext[base,left]{candidates}
  \end{pgfscope}
  \begin{pgfscope}
    \definecolor{fc}{rgb}{0.0000,0.0000,0.0000}
    \pgfsetfillcolor{fc}
    \pgfsetlinewidth{0.5000bp}
    \definecolor{sc}{rgb}{0.0000,0.0000,0.0000}
    \pgfsetstrokecolor{sc}
    \pgfsetmiterjoin
    \pgfsetbuttcap
    \pgfpathqmoveto{2.2749bp}{81.4218bp}
    \pgfpathqcurveto{2.2749bp}{81.6312bp}{2.1051bp}{81.8009bp}{1.8957bp}{81.8009bp}
    \pgfpathqcurveto{1.6863bp}{81.8009bp}{1.5166bp}{81.6312bp}{1.5166bp}{81.4218bp}
    \pgfpathqcurveto{1.5166bp}{81.2124bp}{1.6863bp}{81.0427bp}{1.8957bp}{81.0427bp}
    \pgfpathqcurveto{2.1051bp}{81.0427bp}{2.2749bp}{81.2124bp}{2.2749bp}{81.4218bp}
    \pgfpathclose
    \pgfusepathqfillstroke
  \end{pgfscope}
  \begin{pgfscope}
    \definecolor{fc}{rgb}{0.0000,0.0000,0.0000}
    \pgfsetfillcolor{fc}
    \pgftransformshift{\pgfqpoint{3.2227bp}{82.3697bp}}
    \pgftransformscale{0.1185}
    \pgftext[base,left]{negative unproven}
  \end{pgfscope}
  \begin{pgfscope}
    \definecolor{fc}{rgb}{1.0000,1.0000,0.0000}
    \pgfsetfillcolor{fc}
    \pgfsetlinewidth{0.5000bp}
    \definecolor{sc}{rgb}{1.0000,1.0000,0.0000}
    \pgfsetstrokecolor{sc}
    \pgfsetmiterjoin
    \pgfsetbuttcap
    \pgfpathqmoveto{2.2749bp}{82.6540bp}
    \pgfpathqcurveto{2.2749bp}{82.8634bp}{2.1051bp}{83.0332bp}{1.8957bp}{83.0332bp}
    \pgfpathqcurveto{1.6863bp}{83.0332bp}{1.5166bp}{82.8634bp}{1.5166bp}{82.6540bp}
    \pgfpathqcurveto{1.5166bp}{82.4446bp}{1.6863bp}{82.2749bp}{1.8957bp}{82.2749bp}
    \pgfpathqcurveto{2.1051bp}{82.2749bp}{2.2749bp}{82.4446bp}{2.2749bp}{82.6540bp}
    \pgfpathclose
    \pgfusepathqfillstroke
  \end{pgfscope}
  \begin{pgfscope}
    \definecolor{fc}{rgb}{0.0000,0.0000,0.0000}
    \pgfsetfillcolor{fc}
    \pgftransformshift{\pgfqpoint{3.2227bp}{83.6019bp}}
    \pgftransformscale{0.1185}
    \pgftext[base,left]{negative proven}
  \end{pgfscope}
  \begin{pgfscope}
    \definecolor{fc}{rgb}{0.0000,0.5020,0.0000}
    \pgfsetfillcolor{fc}
    \pgfsetlinewidth{0.5000bp}
    \definecolor{sc}{rgb}{0.0000,0.5020,0.0000}
    \pgfsetstrokecolor{sc}
    \pgfsetmiterjoin
    \pgfsetbuttcap
    \pgfpathqmoveto{2.2749bp}{83.8863bp}
    \pgfpathqcurveto{2.2749bp}{84.0957bp}{2.1051bp}{84.2654bp}{1.8957bp}{84.2654bp}
    \pgfpathqcurveto{1.6863bp}{84.2654bp}{1.5166bp}{84.0957bp}{1.5166bp}{83.8863bp}
    \pgfpathqcurveto{1.5166bp}{83.6769bp}{1.6863bp}{83.5071bp}{1.8957bp}{83.5071bp}
    \pgfpathqcurveto{2.1051bp}{83.5071bp}{2.2749bp}{83.6769bp}{2.2749bp}{83.8863bp}
    \pgfpathclose
    \pgfusepathqfillstroke
  \end{pgfscope}
  \begin{pgfscope}
    \definecolor{fc}{rgb}{0.0000,0.0000,0.0000}
    \pgfsetfillcolor{fc}
    \pgftransformshift{\pgfqpoint{3.2227bp}{84.8341bp}}
    \pgftransformscale{0.1185}
    \pgftext[base,left]{positive unproven}
  \end{pgfscope}
  \begin{pgfscope}
    \definecolor{fc}{rgb}{1.0000,0.0000,0.0000}
    \pgfsetfillcolor{fc}
    \pgfsetlinewidth{0.5000bp}
    \definecolor{sc}{rgb}{1.0000,0.0000,0.0000}
    \pgfsetstrokecolor{sc}
    \pgfsetmiterjoin
    \pgfsetbuttcap
    \pgfpathqmoveto{2.2749bp}{85.1185bp}
    \pgfpathqcurveto{2.2749bp}{85.3279bp}{2.1051bp}{85.4976bp}{1.8957bp}{85.4976bp}
    \pgfpathqcurveto{1.6863bp}{85.4976bp}{1.5166bp}{85.3279bp}{1.5166bp}{85.1185bp}
    \pgfpathqcurveto{1.5166bp}{84.9091bp}{1.6863bp}{84.7393bp}{1.8957bp}{84.7393bp}
    \pgfpathqcurveto{2.1051bp}{84.7393bp}{2.2749bp}{84.9091bp}{2.2749bp}{85.1185bp}
    \pgfpathclose
    \pgfusepathqfillstroke
  \end{pgfscope}
  \begin{pgfscope}
    \definecolor{fc}{rgb}{0.0000,0.0000,0.0000}
    \pgfsetfillcolor{fc}
    \pgftransformshift{\pgfqpoint{3.2227bp}{86.0664bp}}
    \pgftransformscale{0.1185}
    \pgftext[base,left]{positive proven}
  \end{pgfscope}
  \begin{pgfscope}
    \definecolor{fc}{rgb}{0.0000,0.0000,1.0000}
    \pgfsetfillcolor{fc}
    \pgfsetlinewidth{0.5000bp}
    \definecolor{sc}{rgb}{0.0000,0.0000,1.0000}
    \pgfsetstrokecolor{sc}
    \pgfsetmiterjoin
    \pgfsetbuttcap
    \pgfpathqmoveto{2.2749bp}{86.3507bp}
    \pgfpathqcurveto{2.2749bp}{86.5601bp}{2.1051bp}{86.7299bp}{1.8957bp}{86.7299bp}
    \pgfpathqcurveto{1.6863bp}{86.7299bp}{1.5166bp}{86.5601bp}{1.5166bp}{86.3507bp}
    \pgfpathqcurveto{1.5166bp}{86.1413bp}{1.6863bp}{85.9716bp}{1.8957bp}{85.9716bp}
    \pgfpathqcurveto{2.1051bp}{85.9716bp}{2.2749bp}{86.1413bp}{2.2749bp}{86.3507bp}
    \pgfpathclose
    \pgfusepathqfillstroke
  \end{pgfscope}
  \begin{pgfscope}
    \pgfsetlinewidth{0.5000bp}
    \definecolor{sc}{rgb}{1.0000,1.0000,0.0000}
    \pgfsetstrokecolor{sc}
    \pgfsetmiterjoin
    \pgfsetbuttcap
    \pgfpathqmoveto{2.8436bp}{118.9573bp}
    \pgfpathqlineto{3.7915bp}{118.9573bp}
    \pgfpathqlineto{4.7393bp}{118.9573bp}
    \pgfpathqlineto{5.6872bp}{118.9573bp}
    \pgfpathqlineto{6.6351bp}{118.9573bp}
    \pgfpathqlineto{7.5829bp}{118.9573bp}
    \pgfpathqlineto{8.5308bp}{118.9573bp}
    \pgfpathqlineto{9.4787bp}{118.9573bp}
    \pgfpathqlineto{10.4265bp}{118.9573bp}
    \pgfpathqlineto{11.3744bp}{118.9573bp}
    \pgfpathqlineto{12.3223bp}{118.9573bp}
    \pgfpathqlineto{13.2701bp}{118.9573bp}
    \pgfpathqlineto{14.2180bp}{118.9573bp}
    \pgfpathqlineto{15.1659bp}{118.9573bp}
    \pgfpathqlineto{16.1137bp}{118.9573bp}
    \pgfpathqlineto{17.0616bp}{118.9573bp}
    \pgfpathqlineto{18.0095bp}{118.9573bp}
    \pgfpathqlineto{18.9573bp}{118.9573bp}
    \pgfpathqlineto{19.9052bp}{118.9573bp}
    \pgfpathqlineto{20.8531bp}{118.9573bp}
    \pgfpathqlineto{21.8009bp}{118.9573bp}
    \pgfpathqlineto{22.7488bp}{118.9573bp}
    \pgfpathqlineto{23.6967bp}{118.9573bp}
    \pgfpathqlineto{24.6445bp}{118.9573bp}
    \pgfpathqlineto{25.5924bp}{118.9573bp}
    \pgfpathqlineto{26.5403bp}{118.9573bp}
    \pgfpathqlineto{27.4882bp}{118.9573bp}
    \pgfpathqlineto{28.4360bp}{118.9573bp}
    \pgfpathqlineto{29.3839bp}{118.9573bp}
    \pgfpathqlineto{30.3318bp}{118.9573bp}
    \pgfpathqlineto{31.2796bp}{118.9573bp}
    \pgfpathqlineto{32.2275bp}{118.9573bp}
    \pgfpathqlineto{33.1754bp}{118.9573bp}
    \pgfpathqlineto{34.1232bp}{118.9573bp}
    \pgfpathqlineto{35.0711bp}{118.9573bp}
    \pgfpathqlineto{36.0190bp}{118.9573bp}
    \pgfpathqlineto{36.9668bp}{118.9573bp}
    \pgfpathqlineto{37.9147bp}{118.9573bp}
    \pgfpathqlineto{38.8626bp}{118.9573bp}
    \pgfpathqlineto{39.8104bp}{118.9573bp}
    \pgfpathqlineto{40.7583bp}{118.9573bp}
    \pgfpathqlineto{41.7062bp}{118.9573bp}
    \pgfpathqlineto{42.6540bp}{118.9573bp}
    \pgfpathqlineto{43.6019bp}{118.9573bp}
    \pgfpathqlineto{44.5498bp}{118.9573bp}
    \pgfpathqlineto{45.4976bp}{118.9573bp}
    \pgfpathqlineto{46.4455bp}{118.9573bp}
    \pgfpathqlineto{47.3934bp}{118.9573bp}
    \pgfpathqlineto{48.3412bp}{118.9573bp}
    \pgfpathqlineto{49.2891bp}{118.9573bp}
    \pgfpathqlineto{50.2370bp}{118.9573bp}
    \pgfpathqlineto{51.1848bp}{118.9573bp}
    \pgfpathqlineto{52.1327bp}{118.9573bp}
    \pgfpathqlineto{53.0806bp}{118.9573bp}
    \pgfpathqlineto{54.0284bp}{118.9573bp}
    \pgfpathqlineto{54.9763bp}{118.9573bp}
    \pgfpathqlineto{55.9242bp}{118.9573bp}
    \pgfpathqlineto{56.8720bp}{118.9573bp}
    \pgfpathqlineto{57.8199bp}{118.9573bp}
    \pgfpathqlineto{58.7678bp}{118.9573bp}
    \pgfpathqlineto{59.7156bp}{118.9573bp}
    \pgfpathqlineto{60.6635bp}{118.9573bp}
    \pgfpathqlineto{61.6114bp}{118.9573bp}
    \pgfpathqlineto{62.5592bp}{118.9573bp}
    \pgfpathqlineto{63.5071bp}{118.9573bp}
    \pgfpathqlineto{64.4550bp}{118.9573bp}
    \pgfpathqlineto{65.4028bp}{118.9573bp}
    \pgfpathqlineto{66.3507bp}{118.9573bp}
    \pgfpathqlineto{67.2986bp}{118.9573bp}
    \pgfpathqlineto{68.2464bp}{118.9573bp}
    \pgfpathqlineto{69.1943bp}{118.9573bp}
    \pgfpathqlineto{70.1422bp}{118.9573bp}
    \pgfpathqlineto{71.0900bp}{118.9573bp}
    \pgfpathqlineto{72.0379bp}{118.9573bp}
    \pgfpathqlineto{72.9858bp}{118.9573bp}
    \pgfpathqlineto{73.9336bp}{118.9573bp}
    \pgfpathqlineto{74.8815bp}{118.9573bp}
    \pgfpathqlineto{75.8294bp}{118.9573bp}
    \pgfpathqlineto{76.7773bp}{118.9573bp}
    \pgfpathqlineto{77.7251bp}{118.9573bp}
    \pgfpathqlineto{78.6730bp}{118.9573bp}
    \pgfpathqlineto{79.6209bp}{118.9573bp}
    \pgfpathqlineto{80.5687bp}{118.9573bp}
    \pgfpathqlineto{81.5166bp}{118.9573bp}
    \pgfpathqlineto{82.4645bp}{118.9573bp}
    \pgfpathqlineto{83.4123bp}{118.9573bp}
    \pgfpathqlineto{84.3602bp}{118.9573bp}
    \pgfpathqlineto{85.3081bp}{118.9573bp}
    \pgfpathqlineto{86.2559bp}{118.9573bp}
    \pgfpathqlineto{87.2038bp}{118.9573bp}
    \pgfpathqlineto{88.1517bp}{118.9573bp}
    \pgfpathqlineto{89.0995bp}{118.9573bp}
    \pgfpathqlineto{90.0474bp}{118.9573bp}
    \pgfpathqlineto{90.9953bp}{118.9573bp}
    \pgfpathqlineto{91.9431bp}{118.9573bp}
    \pgfpathqlineto{92.8910bp}{118.9573bp}
    \pgfpathqlineto{93.8389bp}{118.9573bp}
    \pgfpathqlineto{94.7867bp}{118.9573bp}
    \pgfpathqlineto{95.7346bp}{118.9573bp}
    \pgfpathqlineto{96.6825bp}{118.9573bp}
    \pgfpathqlineto{97.6303bp}{118.9573bp}
    \pgfpathqlineto{98.5782bp}{118.9573bp}
    \pgfpathqlineto{99.5261bp}{118.9573bp}
    \pgfpathqlineto{100.4739bp}{118.9573bp}
    \pgfpathqlineto{101.4218bp}{118.9573bp}
    \pgfpathqlineto{102.3697bp}{118.9573bp}
    \pgfpathqlineto{103.3175bp}{118.9573bp}
    \pgfpathqlineto{104.2654bp}{118.9573bp}
    \pgfpathqlineto{105.2133bp}{118.9573bp}
    \pgfpathqlineto{106.1611bp}{118.9573bp}
    \pgfpathqlineto{107.1090bp}{118.9573bp}
    \pgfpathqlineto{108.0569bp}{118.9573bp}
    \pgfpathqlineto{109.0047bp}{118.9573bp}
    \pgfpathqlineto{109.9526bp}{118.9573bp}
    \pgfpathqlineto{110.9005bp}{118.9573bp}
    \pgfpathqlineto{111.8483bp}{118.9573bp}
    \pgfpathqlineto{112.7962bp}{118.9573bp}
    \pgfpathqlineto{113.7441bp}{118.9573bp}
    \pgfpathqlineto{114.6919bp}{118.9573bp}
    \pgfpathqlineto{115.6398bp}{118.9573bp}
    \pgfpathqlineto{116.5877bp}{118.9573bp}
    \pgfpathqlineto{117.5355bp}{118.9573bp}
    \pgfpathqlineto{118.4834bp}{118.9573bp}
    \pgfpathqlineto{119.4313bp}{118.9573bp}
    \pgfpathqlineto{120.3791bp}{118.9573bp}
    \pgfpathqlineto{121.3270bp}{118.9573bp}
    \pgfpathqlineto{122.2749bp}{118.9573bp}
    \pgfpathqlineto{123.2227bp}{118.9573bp}
    \pgfpathqlineto{124.1706bp}{118.9573bp}
    \pgfpathqlineto{125.1185bp}{118.9573bp}
    \pgfpathqlineto{126.0664bp}{118.9573bp}
    \pgfpathqlineto{127.0142bp}{118.9573bp}
    \pgfpathqlineto{127.9621bp}{118.9573bp}
    \pgfpathqlineto{128.9100bp}{118.9573bp}
    \pgfpathqlineto{129.8578bp}{118.9573bp}
    \pgfpathqlineto{130.8057bp}{118.9573bp}
    \pgfpathqlineto{131.7536bp}{118.9573bp}
    \pgfpathqlineto{132.7014bp}{118.9573bp}
    \pgfpathqlineto{133.6493bp}{118.9573bp}
    \pgfpathqlineto{134.5972bp}{118.9573bp}
    \pgfpathqlineto{135.5450bp}{118.9573bp}
    \pgfpathqlineto{136.4929bp}{118.9573bp}
    \pgfpathqlineto{137.4408bp}{118.9573bp}
    \pgfpathqlineto{138.3886bp}{118.9573bp}
    \pgfpathqlineto{139.3365bp}{118.9573bp}
    \pgfpathqlineto{140.2844bp}{118.9573bp}
    \pgfpathqlineto{141.2322bp}{118.9573bp}
    \pgfpathqlineto{142.1801bp}{118.9573bp}
    \pgfpathqlineto{143.1280bp}{118.9573bp}
    \pgfpathqlineto{144.0758bp}{118.9573bp}
    \pgfpathqlineto{145.0237bp}{118.9573bp}
    \pgfpathqlineto{145.9716bp}{118.9573bp}
    \pgfpathqlineto{146.9194bp}{118.9573bp}
    \pgfpathqlineto{147.8673bp}{118.9573bp}
    \pgfpathqlineto{148.8152bp}{118.9573bp}
    \pgfpathqlineto{149.7630bp}{118.9573bp}
    \pgfpathqlineto{150.7109bp}{118.9573bp}
    \pgfpathqlineto{151.6588bp}{118.9573bp}
    \pgfpathqlineto{152.6066bp}{118.9573bp}
    \pgfpathqlineto{153.5545bp}{118.9573bp}
    \pgfpathqlineto{154.5024bp}{118.9573bp}
    \pgfpathqlineto{155.4502bp}{118.9573bp}
    \pgfpathqlineto{156.3981bp}{118.9573bp}
    \pgfpathqlineto{157.3460bp}{118.9573bp}
    \pgfpathqlineto{158.2938bp}{118.9573bp}
    \pgfpathqlineto{159.2417bp}{118.9573bp}
    \pgfpathqlineto{160.1896bp}{118.9573bp}
    \pgfpathqlineto{161.1374bp}{118.9573bp}
    \pgfpathqlineto{162.0853bp}{118.9573bp}
    \pgfpathqlineto{163.0332bp}{118.9573bp}
    \pgfpathqlineto{163.9810bp}{118.9573bp}
    \pgfpathqlineto{164.9289bp}{118.9573bp}
    \pgfpathqlineto{165.8768bp}{118.9573bp}
    \pgfpathqlineto{166.8246bp}{118.9573bp}
    \pgfpathqlineto{167.7725bp}{118.9573bp}
    \pgfpathqlineto{168.7204bp}{118.9573bp}
    \pgfpathqlineto{169.6682bp}{118.9573bp}
    \pgfpathqlineto{170.6161bp}{118.9573bp}
    \pgfpathqlineto{171.5640bp}{118.9573bp}
    \pgfpathqlineto{172.5118bp}{118.9573bp}
    \pgfpathqlineto{173.4597bp}{118.9573bp}
    \pgfpathqlineto{174.4076bp}{118.9573bp}
    \pgfpathqlineto{175.3555bp}{118.9573bp}
    \pgfpathqlineto{176.3033bp}{118.9573bp}
    \pgfpathqlineto{177.2512bp}{118.9573bp}
    \pgfpathqlineto{178.1991bp}{118.9573bp}
    \pgfpathqlineto{179.1469bp}{118.9573bp}
    \pgfpathqlineto{180.0948bp}{118.9573bp}
    \pgfpathqlineto{181.0427bp}{118.9573bp}
    \pgfpathqlineto{181.9905bp}{118.9573bp}
    \pgfpathqlineto{182.9384bp}{118.9573bp}
    \pgfpathqlineto{183.8863bp}{118.9573bp}
    \pgfpathqlineto{184.8341bp}{118.9573bp}
    \pgfpathqlineto{185.7820bp}{118.9573bp}
    \pgfpathqlineto{186.7299bp}{118.9573bp}
    \pgfpathqlineto{187.6777bp}{118.9573bp}
    \pgfpathqlineto{188.6256bp}{88.6256bp}
    \pgfpathqlineto{189.5735bp}{88.6256bp}
    \pgfpathqlineto{190.5213bp}{88.6256bp}
    \pgfpathqlineto{191.4692bp}{88.6256bp}
    \pgfpathqlineto{192.4171bp}{88.6256bp}
    \pgfpathqlineto{193.3649bp}{88.6256bp}
    \pgfpathqlineto{194.3128bp}{88.6256bp}
    \pgfpathqlineto{195.2607bp}{88.6256bp}
    \pgfpathqlineto{196.2085bp}{88.6256bp}
    \pgfpathqlineto{197.1564bp}{88.6256bp}
    \pgfpathqlineto{198.1043bp}{88.6256bp}
    \pgfpathqlineto{199.0521bp}{88.6256bp}
    \pgfpathqlineto{200.0000bp}{88.6256bp}
    \pgfusepathqstroke
  \end{pgfscope}
  \begin{pgfscope}
    \pgfsetlinewidth{0.5000bp}
    \definecolor{sc}{rgb}{0.0000,0.5020,0.0000}
    \pgfsetstrokecolor{sc}
    \pgfsetmiterjoin
    \pgfsetbuttcap
    \pgfpathqmoveto{2.8436bp}{88.6256bp}
    \pgfpathqlineto{3.7915bp}{88.6256bp}
    \pgfpathqlineto{4.7393bp}{88.6256bp}
    \pgfpathqlineto{5.6872bp}{88.6256bp}
    \pgfpathqlineto{6.6351bp}{88.6256bp}
    \pgfpathqlineto{7.5829bp}{88.6256bp}
    \pgfpathqlineto{8.5308bp}{88.6256bp}
    \pgfpathqlineto{9.4787bp}{88.6256bp}
    \pgfpathqlineto{10.4265bp}{88.6256bp}
    \pgfpathqlineto{11.3744bp}{88.6256bp}
    \pgfpathqlineto{12.3223bp}{88.6256bp}
    \pgfpathqlineto{13.2701bp}{88.6256bp}
    \pgfpathqlineto{14.2180bp}{88.6256bp}
    \pgfpathqlineto{15.1659bp}{88.6256bp}
    \pgfpathqlineto{16.1137bp}{88.6256bp}
    \pgfpathqlineto{17.0616bp}{88.6256bp}
    \pgfpathqlineto{18.0095bp}{88.6256bp}
    \pgfpathqlineto{18.9573bp}{88.6256bp}
    \pgfpathqlineto{19.9052bp}{88.6256bp}
    \pgfpathqlineto{20.8531bp}{88.6256bp}
    \pgfpathqlineto{21.8009bp}{88.6256bp}
    \pgfpathqlineto{22.7488bp}{88.6256bp}
    \pgfpathqlineto{23.6967bp}{88.6256bp}
    \pgfpathqlineto{24.6445bp}{88.6256bp}
    \pgfpathqlineto{25.5924bp}{88.6256bp}
    \pgfpathqlineto{26.5403bp}{88.6256bp}
    \pgfpathqlineto{27.4882bp}{88.6256bp}
    \pgfpathqlineto{28.4360bp}{88.6256bp}
    \pgfpathqlineto{29.3839bp}{88.6256bp}
    \pgfpathqlineto{30.3318bp}{88.6256bp}
    \pgfpathqlineto{31.2796bp}{88.6256bp}
    \pgfpathqlineto{32.2275bp}{88.6256bp}
    \pgfpathqlineto{33.1754bp}{88.6256bp}
    \pgfpathqlineto{34.1232bp}{88.6256bp}
    \pgfpathqlineto{35.0711bp}{88.6256bp}
    \pgfpathqlineto{36.0190bp}{88.6256bp}
    \pgfpathqlineto{36.9668bp}{88.6256bp}
    \pgfpathqlineto{37.9147bp}{88.6256bp}
    \pgfpathqlineto{38.8626bp}{88.6256bp}
    \pgfpathqlineto{39.8104bp}{88.6256bp}
    \pgfpathqlineto{40.7583bp}{88.6256bp}
    \pgfpathqlineto{41.7062bp}{88.6256bp}
    \pgfpathqlineto{42.6540bp}{88.6256bp}
    \pgfpathqlineto{43.6019bp}{88.6256bp}
    \pgfpathqlineto{44.5498bp}{88.6256bp}
    \pgfpathqlineto{45.4976bp}{88.6256bp}
    \pgfpathqlineto{46.4455bp}{88.6256bp}
    \pgfpathqlineto{47.3934bp}{88.6256bp}
    \pgfpathqlineto{48.3412bp}{88.6256bp}
    \pgfpathqlineto{49.2891bp}{88.6256bp}
    \pgfpathqlineto{50.2370bp}{88.6256bp}
    \pgfpathqlineto{51.1848bp}{88.6256bp}
    \pgfpathqlineto{52.1327bp}{88.6256bp}
    \pgfpathqlineto{53.0806bp}{88.6256bp}
    \pgfpathqlineto{54.0284bp}{88.6256bp}
    \pgfpathqlineto{54.9763bp}{88.6256bp}
    \pgfpathqlineto{55.9242bp}{88.6256bp}
    \pgfpathqlineto{56.8720bp}{88.6256bp}
    \pgfpathqlineto{57.8199bp}{88.6256bp}
    \pgfpathqlineto{58.7678bp}{88.6256bp}
    \pgfpathqlineto{59.7156bp}{88.6256bp}
    \pgfpathqlineto{60.6635bp}{88.6256bp}
    \pgfpathqlineto{61.6114bp}{88.6256bp}
    \pgfpathqlineto{62.5592bp}{88.6256bp}
    \pgfpathqlineto{63.5071bp}{88.6256bp}
    \pgfpathqlineto{64.4550bp}{88.6256bp}
    \pgfpathqlineto{65.4028bp}{88.6256bp}
    \pgfpathqlineto{66.3507bp}{88.6256bp}
    \pgfpathqlineto{67.2986bp}{88.6256bp}
    \pgfpathqlineto{68.2464bp}{88.6256bp}
    \pgfpathqlineto{69.1943bp}{88.6256bp}
    \pgfpathqlineto{70.1422bp}{88.6256bp}
    \pgfpathqlineto{71.0900bp}{88.6256bp}
    \pgfpathqlineto{72.0379bp}{88.6256bp}
    \pgfpathqlineto{72.9858bp}{88.6256bp}
    \pgfpathqlineto{73.9336bp}{88.6256bp}
    \pgfpathqlineto{74.8815bp}{88.6256bp}
    \pgfpathqlineto{75.8294bp}{88.6256bp}
    \pgfpathqlineto{76.7773bp}{88.6256bp}
    \pgfpathqlineto{77.7251bp}{88.6256bp}
    \pgfpathqlineto{78.6730bp}{88.6256bp}
    \pgfpathqlineto{79.6209bp}{88.6256bp}
    \pgfpathqlineto{80.5687bp}{88.6256bp}
    \pgfpathqlineto{81.5166bp}{88.6256bp}
    \pgfpathqlineto{82.4645bp}{88.6256bp}
    \pgfpathqlineto{83.4123bp}{88.6256bp}
    \pgfpathqlineto{84.3602bp}{88.6256bp}
    \pgfpathqlineto{85.3081bp}{88.6256bp}
    \pgfpathqlineto{86.2559bp}{88.6256bp}
    \pgfpathqlineto{87.2038bp}{88.6256bp}
    \pgfpathqlineto{88.1517bp}{88.6256bp}
    \pgfpathqlineto{89.0995bp}{88.6256bp}
    \pgfpathqlineto{90.0474bp}{88.6256bp}
    \pgfpathqlineto{90.9953bp}{88.6256bp}
    \pgfpathqlineto{91.9431bp}{88.6256bp}
    \pgfpathqlineto{92.8910bp}{88.6256bp}
    \pgfpathqlineto{93.8389bp}{88.6256bp}
    \pgfpathqlineto{94.7867bp}{88.6256bp}
    \pgfpathqlineto{95.7346bp}{88.6256bp}
    \pgfpathqlineto{96.6825bp}{88.6256bp}
    \pgfpathqlineto{97.6303bp}{88.6256bp}
    \pgfpathqlineto{98.5782bp}{88.6256bp}
    \pgfpathqlineto{99.5261bp}{88.6256bp}
    \pgfpathqlineto{100.4739bp}{88.6256bp}
    \pgfpathqlineto{101.4218bp}{88.6256bp}
    \pgfpathqlineto{102.3697bp}{88.6256bp}
    \pgfpathqlineto{103.3175bp}{88.6256bp}
    \pgfpathqlineto{104.2654bp}{88.6256bp}
    \pgfpathqlineto{105.2133bp}{88.6256bp}
    \pgfpathqlineto{106.1611bp}{88.6256bp}
    \pgfpathqlineto{107.1090bp}{88.6256bp}
    \pgfpathqlineto{108.0569bp}{88.6256bp}
    \pgfpathqlineto{109.0047bp}{88.6256bp}
    \pgfpathqlineto{109.9526bp}{88.6256bp}
    \pgfpathqlineto{110.9005bp}{88.6256bp}
    \pgfpathqlineto{111.8483bp}{88.6256bp}
    \pgfpathqlineto{112.7962bp}{88.6256bp}
    \pgfpathqlineto{113.7441bp}{88.6256bp}
    \pgfpathqlineto{114.6919bp}{88.6256bp}
    \pgfpathqlineto{115.6398bp}{88.6256bp}
    \pgfpathqlineto{116.5877bp}{88.6256bp}
    \pgfpathqlineto{117.5355bp}{88.6256bp}
    \pgfpathqlineto{118.4834bp}{88.6256bp}
    \pgfpathqlineto{119.4313bp}{88.6256bp}
    \pgfpathqlineto{120.3791bp}{88.6256bp}
    \pgfpathqlineto{121.3270bp}{88.6256bp}
    \pgfpathqlineto{122.2749bp}{88.6256bp}
    \pgfpathqlineto{123.2227bp}{88.6256bp}
    \pgfpathqlineto{124.1706bp}{88.6256bp}
    \pgfpathqlineto{125.1185bp}{88.6256bp}
    \pgfpathqlineto{126.0664bp}{88.6256bp}
    \pgfpathqlineto{127.0142bp}{88.6256bp}
    \pgfpathqlineto{127.9621bp}{88.6256bp}
    \pgfpathqlineto{128.9100bp}{88.6256bp}
    \pgfpathqlineto{129.8578bp}{88.6256bp}
    \pgfpathqlineto{130.8057bp}{88.6256bp}
    \pgfpathqlineto{131.7536bp}{88.6256bp}
    \pgfpathqlineto{132.7014bp}{88.6256bp}
    \pgfpathqlineto{133.6493bp}{88.6256bp}
    \pgfpathqlineto{134.5972bp}{88.6256bp}
    \pgfpathqlineto{135.5450bp}{88.6256bp}
    \pgfpathqlineto{136.4929bp}{88.6256bp}
    \pgfpathqlineto{137.4408bp}{88.6256bp}
    \pgfpathqlineto{138.3886bp}{88.6256bp}
    \pgfpathqlineto{139.3365bp}{88.6256bp}
    \pgfpathqlineto{140.2844bp}{88.6256bp}
    \pgfpathqlineto{141.2322bp}{88.6256bp}
    \pgfpathqlineto{142.1801bp}{88.6256bp}
    \pgfpathqlineto{143.1280bp}{88.6256bp}
    \pgfpathqlineto{144.0758bp}{88.6256bp}
    \pgfpathqlineto{145.0237bp}{88.6256bp}
    \pgfpathqlineto{145.9716bp}{88.6256bp}
    \pgfpathqlineto{146.9194bp}{88.6256bp}
    \pgfpathqlineto{147.8673bp}{88.6256bp}
    \pgfpathqlineto{148.8152bp}{88.6256bp}
    \pgfpathqlineto{149.7630bp}{88.6256bp}
    \pgfpathqlineto{150.7109bp}{88.6256bp}
    \pgfpathqlineto{151.6588bp}{88.6256bp}
    \pgfpathqlineto{152.6066bp}{88.6256bp}
    \pgfpathqlineto{153.5545bp}{88.6256bp}
    \pgfpathqlineto{154.5024bp}{88.6256bp}
    \pgfpathqlineto{155.4502bp}{88.6256bp}
    \pgfpathqlineto{156.3981bp}{88.6256bp}
    \pgfpathqlineto{157.3460bp}{88.6256bp}
    \pgfpathqlineto{158.2938bp}{88.6256bp}
    \pgfpathqlineto{159.2417bp}{88.6256bp}
    \pgfpathqlineto{160.1896bp}{88.6256bp}
    \pgfpathqlineto{161.1374bp}{88.6256bp}
    \pgfpathqlineto{162.0853bp}{88.6256bp}
    \pgfpathqlineto{163.0332bp}{88.6256bp}
    \pgfpathqlineto{163.9810bp}{88.6256bp}
    \pgfpathqlineto{164.9289bp}{88.6256bp}
    \pgfpathqlineto{165.8768bp}{88.6256bp}
    \pgfpathqlineto{166.8246bp}{88.6256bp}
    \pgfpathqlineto{167.7725bp}{88.6256bp}
    \pgfpathqlineto{168.7204bp}{88.6256bp}
    \pgfpathqlineto{169.6682bp}{88.6256bp}
    \pgfpathqlineto{170.6161bp}{88.6256bp}
    \pgfpathqlineto{171.5640bp}{88.6256bp}
    \pgfpathqlineto{172.5118bp}{88.6256bp}
    \pgfpathqlineto{173.4597bp}{88.6256bp}
    \pgfpathqlineto{174.4076bp}{88.6256bp}
    \pgfpathqlineto{175.3555bp}{88.6256bp}
    \pgfpathqlineto{176.3033bp}{88.6256bp}
    \pgfpathqlineto{177.2512bp}{88.6256bp}
    \pgfpathqlineto{178.1991bp}{88.6256bp}
    \pgfpathqlineto{179.1469bp}{88.6256bp}
    \pgfpathqlineto{180.0948bp}{88.6256bp}
    \pgfpathqlineto{181.0427bp}{88.6256bp}
    \pgfpathqlineto{181.9905bp}{88.6256bp}
    \pgfpathqlineto{182.9384bp}{88.6256bp}
    \pgfpathqlineto{183.8863bp}{88.6256bp}
    \pgfpathqlineto{184.8341bp}{88.6256bp}
    \pgfpathqlineto{185.7820bp}{88.6256bp}
    \pgfpathqlineto{186.7299bp}{88.6256bp}
    \pgfpathqlineto{187.6777bp}{88.6256bp}
    \pgfpathqlineto{188.6256bp}{90.0474bp}
    \pgfpathqlineto{189.5735bp}{90.0474bp}
    \pgfpathqlineto{190.5213bp}{90.0474bp}
    \pgfpathqlineto{191.4692bp}{90.0474bp}
    \pgfpathqlineto{192.4171bp}{90.0474bp}
    \pgfpathqlineto{193.3649bp}{90.0474bp}
    \pgfpathqlineto{194.3128bp}{90.0474bp}
    \pgfpathqlineto{195.2607bp}{90.0474bp}
    \pgfpathqlineto{196.2085bp}{90.0474bp}
    \pgfpathqlineto{197.1564bp}{90.0474bp}
    \pgfpathqlineto{198.1043bp}{90.0474bp}
    \pgfpathqlineto{199.0521bp}{90.0474bp}
    \pgfpathqlineto{200.0000bp}{90.0474bp}
    \pgfusepathqstroke
  \end{pgfscope}
  \begin{pgfscope}
    \pgfsetlinewidth{0.5000bp}
    \definecolor{sc}{rgb}{1.0000,0.0000,0.0000}
    \pgfsetstrokecolor{sc}
    \pgfsetmiterjoin
    \pgfsetbuttcap
    \pgfpathqmoveto{2.8436bp}{118.9573bp}
    \pgfpathqlineto{3.7915bp}{118.9573bp}
    \pgfpathqlineto{4.7393bp}{118.9573bp}
    \pgfpathqlineto{5.6872bp}{118.9573bp}
    \pgfpathqlineto{6.6351bp}{118.9573bp}
    \pgfpathqlineto{7.5829bp}{118.9573bp}
    \pgfpathqlineto{8.5308bp}{118.9573bp}
    \pgfpathqlineto{9.4787bp}{118.9573bp}
    \pgfpathqlineto{10.4265bp}{118.9573bp}
    \pgfpathqlineto{11.3744bp}{118.9573bp}
    \pgfpathqlineto{12.3223bp}{118.9573bp}
    \pgfpathqlineto{13.2701bp}{118.9573bp}
    \pgfpathqlineto{14.2180bp}{118.9573bp}
    \pgfpathqlineto{15.1659bp}{118.9573bp}
    \pgfpathqlineto{16.1137bp}{118.9573bp}
    \pgfpathqlineto{17.0616bp}{118.9573bp}
    \pgfpathqlineto{18.0095bp}{118.9573bp}
    \pgfpathqlineto{18.9573bp}{118.9573bp}
    \pgfpathqlineto{19.9052bp}{118.9573bp}
    \pgfpathqlineto{20.8531bp}{118.9573bp}
    \pgfpathqlineto{21.8009bp}{118.9573bp}
    \pgfpathqlineto{22.7488bp}{118.9573bp}
    \pgfpathqlineto{23.6967bp}{118.9573bp}
    \pgfpathqlineto{24.6445bp}{118.9573bp}
    \pgfpathqlineto{25.5924bp}{118.9573bp}
    \pgfpathqlineto{26.5403bp}{118.9573bp}
    \pgfpathqlineto{27.4882bp}{118.9573bp}
    \pgfpathqlineto{28.4360bp}{118.9573bp}
    \pgfpathqlineto{29.3839bp}{118.9573bp}
    \pgfpathqlineto{30.3318bp}{118.9573bp}
    \pgfpathqlineto{31.2796bp}{118.9573bp}
    \pgfpathqlineto{32.2275bp}{118.9573bp}
    \pgfpathqlineto{33.1754bp}{118.9573bp}
    \pgfpathqlineto{34.1232bp}{118.9573bp}
    \pgfpathqlineto{35.0711bp}{118.9573bp}
    \pgfpathqlineto{36.0190bp}{118.9573bp}
    \pgfpathqlineto{36.9668bp}{118.9573bp}
    \pgfpathqlineto{37.9147bp}{118.9573bp}
    \pgfpathqlineto{38.8626bp}{118.9573bp}
    \pgfpathqlineto{39.8104bp}{118.9573bp}
    \pgfpathqlineto{40.7583bp}{118.9573bp}
    \pgfpathqlineto{41.7062bp}{118.9573bp}
    \pgfpathqlineto{42.6540bp}{118.9573bp}
    \pgfpathqlineto{43.6019bp}{118.9573bp}
    \pgfpathqlineto{44.5498bp}{118.9573bp}
    \pgfpathqlineto{45.4976bp}{118.9573bp}
    \pgfpathqlineto{46.4455bp}{118.9573bp}
    \pgfpathqlineto{47.3934bp}{118.9573bp}
    \pgfpathqlineto{48.3412bp}{118.9573bp}
    \pgfpathqlineto{49.2891bp}{118.9573bp}
    \pgfpathqlineto{50.2370bp}{118.9573bp}
    \pgfpathqlineto{51.1848bp}{118.9573bp}
    \pgfpathqlineto{52.1327bp}{118.9573bp}
    \pgfpathqlineto{53.0806bp}{118.9573bp}
    \pgfpathqlineto{54.0284bp}{118.9573bp}
    \pgfpathqlineto{54.9763bp}{118.9573bp}
    \pgfpathqlineto{55.9242bp}{118.9573bp}
    \pgfpathqlineto{56.8720bp}{118.9573bp}
    \pgfpathqlineto{57.8199bp}{118.9573bp}
    \pgfpathqlineto{58.7678bp}{118.9573bp}
    \pgfpathqlineto{59.7156bp}{118.9573bp}
    \pgfpathqlineto{60.6635bp}{118.9573bp}
    \pgfpathqlineto{61.6114bp}{118.9573bp}
    \pgfpathqlineto{62.5592bp}{118.9573bp}
    \pgfpathqlineto{63.5071bp}{118.9573bp}
    \pgfpathqlineto{64.4550bp}{118.9573bp}
    \pgfpathqlineto{65.4028bp}{118.9573bp}
    \pgfpathqlineto{66.3507bp}{118.9573bp}
    \pgfpathqlineto{67.2986bp}{118.9573bp}
    \pgfpathqlineto{68.2464bp}{118.9573bp}
    \pgfpathqlineto{69.1943bp}{118.9573bp}
    \pgfpathqlineto{70.1422bp}{118.9573bp}
    \pgfpathqlineto{71.0900bp}{118.9573bp}
    \pgfpathqlineto{72.0379bp}{118.9573bp}
    \pgfpathqlineto{72.9858bp}{118.9573bp}
    \pgfpathqlineto{73.9336bp}{118.9573bp}
    \pgfpathqlineto{74.8815bp}{118.9573bp}
    \pgfpathqlineto{75.8294bp}{118.9573bp}
    \pgfpathqlineto{76.7773bp}{118.9573bp}
    \pgfpathqlineto{77.7251bp}{118.9573bp}
    \pgfpathqlineto{78.6730bp}{118.9573bp}
    \pgfpathqlineto{79.6209bp}{118.9573bp}
    \pgfpathqlineto{80.5687bp}{118.9573bp}
    \pgfpathqlineto{81.5166bp}{118.9573bp}
    \pgfpathqlineto{82.4645bp}{118.9573bp}
    \pgfpathqlineto{83.4123bp}{118.9573bp}
    \pgfpathqlineto{84.3602bp}{118.9573bp}
    \pgfpathqlineto{85.3081bp}{118.9573bp}
    \pgfpathqlineto{86.2559bp}{118.9573bp}
    \pgfpathqlineto{87.2038bp}{118.9573bp}
    \pgfpathqlineto{88.1517bp}{118.9573bp}
    \pgfpathqlineto{89.0995bp}{118.9573bp}
    \pgfpathqlineto{90.0474bp}{118.9573bp}
    \pgfpathqlineto{90.9953bp}{118.9573bp}
    \pgfpathqlineto{91.9431bp}{118.9573bp}
    \pgfpathqlineto{92.8910bp}{118.9573bp}
    \pgfpathqlineto{93.8389bp}{118.9573bp}
    \pgfpathqlineto{94.7867bp}{118.9573bp}
    \pgfpathqlineto{95.7346bp}{118.9573bp}
    \pgfpathqlineto{96.6825bp}{118.9573bp}
    \pgfpathqlineto{97.6303bp}{118.9573bp}
    \pgfpathqlineto{98.5782bp}{118.9573bp}
    \pgfpathqlineto{99.5261bp}{118.9573bp}
    \pgfpathqlineto{100.4739bp}{118.9573bp}
    \pgfpathqlineto{101.4218bp}{118.9573bp}
    \pgfpathqlineto{102.3697bp}{118.9573bp}
    \pgfpathqlineto{103.3175bp}{118.9573bp}
    \pgfpathqlineto{104.2654bp}{118.9573bp}
    \pgfpathqlineto{105.2133bp}{118.9573bp}
    \pgfpathqlineto{106.1611bp}{118.9573bp}
    \pgfpathqlineto{107.1090bp}{118.9573bp}
    \pgfpathqlineto{108.0569bp}{118.9573bp}
    \pgfpathqlineto{109.0047bp}{118.9573bp}
    \pgfpathqlineto{109.9526bp}{118.9573bp}
    \pgfpathqlineto{110.9005bp}{118.9573bp}
    \pgfpathqlineto{111.8483bp}{118.9573bp}
    \pgfpathqlineto{112.7962bp}{118.9573bp}
    \pgfpathqlineto{113.7441bp}{118.9573bp}
    \pgfpathqlineto{114.6919bp}{118.9573bp}
    \pgfpathqlineto{115.6398bp}{118.9573bp}
    \pgfpathqlineto{116.5877bp}{118.9573bp}
    \pgfpathqlineto{117.5355bp}{118.9573bp}
    \pgfpathqlineto{118.4834bp}{118.9573bp}
    \pgfpathqlineto{119.4313bp}{118.9573bp}
    \pgfpathqlineto{120.3791bp}{118.9573bp}
    \pgfpathqlineto{121.3270bp}{118.9573bp}
    \pgfpathqlineto{122.2749bp}{118.9573bp}
    \pgfpathqlineto{123.2227bp}{118.9573bp}
    \pgfpathqlineto{124.1706bp}{118.9573bp}
    \pgfpathqlineto{125.1185bp}{118.9573bp}
    \pgfpathqlineto{126.0664bp}{118.9573bp}
    \pgfpathqlineto{127.0142bp}{118.9573bp}
    \pgfpathqlineto{127.9621bp}{118.9573bp}
    \pgfpathqlineto{128.9100bp}{118.9573bp}
    \pgfpathqlineto{129.8578bp}{118.9573bp}
    \pgfpathqlineto{130.8057bp}{118.9573bp}
    \pgfpathqlineto{131.7536bp}{118.9573bp}
    \pgfpathqlineto{132.7014bp}{118.9573bp}
    \pgfpathqlineto{133.6493bp}{118.9573bp}
    \pgfpathqlineto{134.5972bp}{118.9573bp}
    \pgfpathqlineto{135.5450bp}{118.9573bp}
    \pgfpathqlineto{136.4929bp}{118.9573bp}
    \pgfpathqlineto{137.4408bp}{118.9573bp}
    \pgfpathqlineto{138.3886bp}{118.9573bp}
    \pgfpathqlineto{139.3365bp}{118.9573bp}
    \pgfpathqlineto{140.2844bp}{118.9573bp}
    \pgfpathqlineto{141.2322bp}{118.9573bp}
    \pgfpathqlineto{142.1801bp}{118.9573bp}
    \pgfpathqlineto{143.1280bp}{118.9573bp}
    \pgfpathqlineto{144.0758bp}{118.9573bp}
    \pgfpathqlineto{145.0237bp}{118.9573bp}
    \pgfpathqlineto{145.9716bp}{118.9573bp}
    \pgfpathqlineto{146.9194bp}{118.9573bp}
    \pgfpathqlineto{147.8673bp}{118.9573bp}
    \pgfpathqlineto{148.8152bp}{118.9573bp}
    \pgfpathqlineto{149.7630bp}{118.9573bp}
    \pgfpathqlineto{150.7109bp}{118.9573bp}
    \pgfpathqlineto{151.6588bp}{118.9573bp}
    \pgfpathqlineto{152.6066bp}{118.9573bp}
    \pgfpathqlineto{153.5545bp}{118.9573bp}
    \pgfpathqlineto{154.5024bp}{118.9573bp}
    \pgfpathqlineto{155.4502bp}{118.9573bp}
    \pgfpathqlineto{156.3981bp}{118.9573bp}
    \pgfpathqlineto{157.3460bp}{118.9573bp}
    \pgfpathqlineto{158.2938bp}{118.9573bp}
    \pgfpathqlineto{159.2417bp}{118.9573bp}
    \pgfpathqlineto{160.1896bp}{118.9573bp}
    \pgfpathqlineto{161.1374bp}{118.9573bp}
    \pgfpathqlineto{162.0853bp}{118.9573bp}
    \pgfpathqlineto{163.0332bp}{118.9573bp}
    \pgfpathqlineto{163.9810bp}{118.9573bp}
    \pgfpathqlineto{164.9289bp}{118.9573bp}
    \pgfpathqlineto{165.8768bp}{118.9573bp}
    \pgfpathqlineto{166.8246bp}{118.9573bp}
    \pgfpathqlineto{167.7725bp}{118.9573bp}
    \pgfpathqlineto{168.7204bp}{118.9573bp}
    \pgfpathqlineto{169.6682bp}{118.9573bp}
    \pgfpathqlineto{170.6161bp}{118.9573bp}
    \pgfpathqlineto{171.5640bp}{118.9573bp}
    \pgfpathqlineto{172.5118bp}{118.9573bp}
    \pgfpathqlineto{173.4597bp}{118.9573bp}
    \pgfpathqlineto{174.4076bp}{118.9573bp}
    \pgfpathqlineto{175.3555bp}{118.9573bp}
    \pgfpathqlineto{176.3033bp}{118.9573bp}
    \pgfpathqlineto{177.2512bp}{118.9573bp}
    \pgfpathqlineto{178.1991bp}{118.9573bp}
    \pgfpathqlineto{179.1469bp}{118.9573bp}
    \pgfpathqlineto{180.0948bp}{118.9573bp}
    \pgfpathqlineto{181.0427bp}{118.9573bp}
    \pgfpathqlineto{181.9905bp}{118.9573bp}
    \pgfpathqlineto{182.9384bp}{118.9573bp}
    \pgfpathqlineto{183.8863bp}{118.9573bp}
    \pgfpathqlineto{184.8341bp}{118.9573bp}
    \pgfpathqlineto{185.7820bp}{118.9573bp}
    \pgfpathqlineto{186.7299bp}{118.9573bp}
    \pgfpathqlineto{187.6777bp}{118.9573bp}
    \pgfpathqlineto{188.6256bp}{88.6256bp}
    \pgfpathqlineto{189.5735bp}{88.6256bp}
    \pgfpathqlineto{190.5213bp}{88.6256bp}
    \pgfpathqlineto{191.4692bp}{88.6256bp}
    \pgfpathqlineto{192.4171bp}{88.6256bp}
    \pgfpathqlineto{193.3649bp}{88.6256bp}
    \pgfpathqlineto{194.3128bp}{88.6256bp}
    \pgfpathqlineto{195.2607bp}{88.6256bp}
    \pgfpathqlineto{196.2085bp}{88.6256bp}
    \pgfpathqlineto{197.1564bp}{88.6256bp}
    \pgfpathqlineto{198.1043bp}{88.6256bp}
    \pgfpathqlineto{199.0521bp}{88.6256bp}
    \pgfpathqlineto{200.0000bp}{88.6256bp}
    \pgfusepathqstroke
  \end{pgfscope}
  \begin{pgfscope}
    \pgfsetlinewidth{0.5000bp}
    \definecolor{sc}{rgb}{0.0000,0.0000,1.0000}
    \pgfsetstrokecolor{sc}
    \pgfsetmiterjoin
    \pgfsetbuttcap
    \pgfpathqmoveto{2.8436bp}{88.6256bp}
    \pgfpathqlineto{3.7915bp}{88.6256bp}
    \pgfpathqlineto{4.7393bp}{88.6256bp}
    \pgfpathqlineto{5.6872bp}{88.6256bp}
    \pgfpathqlineto{6.6351bp}{88.6256bp}
    \pgfpathqlineto{7.5829bp}{88.6256bp}
    \pgfpathqlineto{8.5308bp}{88.6256bp}
    \pgfpathqlineto{9.4787bp}{88.6256bp}
    \pgfpathqlineto{10.4265bp}{88.6256bp}
    \pgfpathqlineto{11.3744bp}{88.6256bp}
    \pgfpathqlineto{12.3223bp}{88.6256bp}
    \pgfpathqlineto{13.2701bp}{88.6256bp}
    \pgfpathqlineto{14.2180bp}{88.6256bp}
    \pgfpathqlineto{15.1659bp}{88.6256bp}
    \pgfpathqlineto{16.1137bp}{88.6256bp}
    \pgfpathqlineto{17.0616bp}{88.6256bp}
    \pgfpathqlineto{18.0095bp}{88.6256bp}
    \pgfpathqlineto{18.9573bp}{88.6256bp}
    \pgfpathqlineto{19.9052bp}{88.6256bp}
    \pgfpathqlineto{20.8531bp}{88.6256bp}
    \pgfpathqlineto{21.8009bp}{88.6256bp}
    \pgfpathqlineto{22.7488bp}{88.6256bp}
    \pgfpathqlineto{23.6967bp}{88.6256bp}
    \pgfpathqlineto{24.6445bp}{88.6256bp}
    \pgfpathqlineto{25.5924bp}{88.6256bp}
    \pgfpathqlineto{26.5403bp}{88.6256bp}
    \pgfpathqlineto{27.4882bp}{88.6256bp}
    \pgfpathqlineto{28.4360bp}{88.6256bp}
    \pgfpathqlineto{29.3839bp}{88.6256bp}
    \pgfpathqlineto{30.3318bp}{88.6256bp}
    \pgfpathqlineto{31.2796bp}{88.6256bp}
    \pgfpathqlineto{32.2275bp}{88.6256bp}
    \pgfpathqlineto{33.1754bp}{88.6256bp}
    \pgfpathqlineto{34.1232bp}{88.6256bp}
    \pgfpathqlineto{35.0711bp}{88.6256bp}
    \pgfpathqlineto{36.0190bp}{88.6256bp}
    \pgfpathqlineto{36.9668bp}{88.6256bp}
    \pgfpathqlineto{37.9147bp}{88.6256bp}
    \pgfpathqlineto{38.8626bp}{88.6256bp}
    \pgfpathqlineto{39.8104bp}{88.6256bp}
    \pgfpathqlineto{40.7583bp}{88.6256bp}
    \pgfpathqlineto{41.7062bp}{88.6256bp}
    \pgfpathqlineto{42.6540bp}{88.6256bp}
    \pgfpathqlineto{43.6019bp}{88.6256bp}
    \pgfpathqlineto{44.5498bp}{88.6256bp}
    \pgfpathqlineto{45.4976bp}{88.6256bp}
    \pgfpathqlineto{46.4455bp}{88.6256bp}
    \pgfpathqlineto{47.3934bp}{88.6256bp}
    \pgfpathqlineto{48.3412bp}{88.6256bp}
    \pgfpathqlineto{49.2891bp}{88.6256bp}
    \pgfpathqlineto{50.2370bp}{88.6256bp}
    \pgfpathqlineto{51.1848bp}{88.6256bp}
    \pgfpathqlineto{52.1327bp}{88.6256bp}
    \pgfpathqlineto{53.0806bp}{88.6256bp}
    \pgfpathqlineto{54.0284bp}{88.6256bp}
    \pgfpathqlineto{54.9763bp}{88.6256bp}
    \pgfpathqlineto{55.9242bp}{88.6256bp}
    \pgfpathqlineto{56.8720bp}{88.6256bp}
    \pgfpathqlineto{57.8199bp}{88.6256bp}
    \pgfpathqlineto{58.7678bp}{88.6256bp}
    \pgfpathqlineto{59.7156bp}{88.6256bp}
    \pgfpathqlineto{60.6635bp}{88.6256bp}
    \pgfpathqlineto{61.6114bp}{88.6256bp}
    \pgfpathqlineto{62.5592bp}{88.6256bp}
    \pgfpathqlineto{63.5071bp}{88.6256bp}
    \pgfpathqlineto{64.4550bp}{88.6256bp}
    \pgfpathqlineto{65.4028bp}{88.6256bp}
    \pgfpathqlineto{66.3507bp}{88.6256bp}
    \pgfpathqlineto{67.2986bp}{88.6256bp}
    \pgfpathqlineto{68.2464bp}{88.6256bp}
    \pgfpathqlineto{69.1943bp}{88.6256bp}
    \pgfpathqlineto{70.1422bp}{88.6256bp}
    \pgfpathqlineto{71.0900bp}{88.6256bp}
    \pgfpathqlineto{72.0379bp}{88.6256bp}
    \pgfpathqlineto{72.9858bp}{88.6256bp}
    \pgfpathqlineto{73.9336bp}{88.6256bp}
    \pgfpathqlineto{74.8815bp}{88.6256bp}
    \pgfpathqlineto{75.8294bp}{88.6256bp}
    \pgfpathqlineto{76.7773bp}{88.6256bp}
    \pgfpathqlineto{77.7251bp}{88.6256bp}
    \pgfpathqlineto{78.6730bp}{88.6256bp}
    \pgfpathqlineto{79.6209bp}{88.6256bp}
    \pgfpathqlineto{80.5687bp}{88.6256bp}
    \pgfpathqlineto{81.5166bp}{88.6256bp}
    \pgfpathqlineto{82.4645bp}{88.6256bp}
    \pgfpathqlineto{83.4123bp}{88.6256bp}
    \pgfpathqlineto{84.3602bp}{88.6256bp}
    \pgfpathqlineto{85.3081bp}{88.6256bp}
    \pgfpathqlineto{86.2559bp}{88.6256bp}
    \pgfpathqlineto{87.2038bp}{88.6256bp}
    \pgfpathqlineto{88.1517bp}{88.6256bp}
    \pgfpathqlineto{89.0995bp}{88.6256bp}
    \pgfpathqlineto{90.0474bp}{88.6256bp}
    \pgfpathqlineto{90.9953bp}{88.6256bp}
    \pgfpathqlineto{91.9431bp}{88.6256bp}
    \pgfpathqlineto{92.8910bp}{88.6256bp}
    \pgfpathqlineto{93.8389bp}{88.6256bp}
    \pgfpathqlineto{94.7867bp}{88.6256bp}
    \pgfpathqlineto{95.7346bp}{88.6256bp}
    \pgfpathqlineto{96.6825bp}{88.6256bp}
    \pgfpathqlineto{97.6303bp}{88.6256bp}
    \pgfpathqlineto{98.5782bp}{88.6256bp}
    \pgfpathqlineto{99.5261bp}{88.6256bp}
    \pgfpathqlineto{100.4739bp}{88.6256bp}
    \pgfpathqlineto{101.4218bp}{88.6256bp}
    \pgfpathqlineto{102.3697bp}{88.6256bp}
    \pgfpathqlineto{103.3175bp}{88.6256bp}
    \pgfpathqlineto{104.2654bp}{88.6256bp}
    \pgfpathqlineto{105.2133bp}{88.6256bp}
    \pgfpathqlineto{106.1611bp}{88.6256bp}
    \pgfpathqlineto{107.1090bp}{88.6256bp}
    \pgfpathqlineto{108.0569bp}{88.6256bp}
    \pgfpathqlineto{109.0047bp}{88.6256bp}
    \pgfpathqlineto{109.9526bp}{88.6256bp}
    \pgfpathqlineto{110.9005bp}{88.6256bp}
    \pgfpathqlineto{111.8483bp}{88.6256bp}
    \pgfpathqlineto{112.7962bp}{88.6256bp}
    \pgfpathqlineto{113.7441bp}{88.6256bp}
    \pgfpathqlineto{114.6919bp}{88.6256bp}
    \pgfpathqlineto{115.6398bp}{88.6256bp}
    \pgfpathqlineto{116.5877bp}{88.6256bp}
    \pgfpathqlineto{117.5355bp}{88.6256bp}
    \pgfpathqlineto{118.4834bp}{88.6256bp}
    \pgfpathqlineto{119.4313bp}{88.6256bp}
    \pgfpathqlineto{120.3791bp}{88.6256bp}
    \pgfpathqlineto{121.3270bp}{88.6256bp}
    \pgfpathqlineto{122.2749bp}{88.6256bp}
    \pgfpathqlineto{123.2227bp}{88.6256bp}
    \pgfpathqlineto{124.1706bp}{88.6256bp}
    \pgfpathqlineto{125.1185bp}{88.6256bp}
    \pgfpathqlineto{126.0664bp}{88.6256bp}
    \pgfpathqlineto{127.0142bp}{88.6256bp}
    \pgfpathqlineto{127.9621bp}{88.6256bp}
    \pgfpathqlineto{128.9100bp}{88.6256bp}
    \pgfpathqlineto{129.8578bp}{88.6256bp}
    \pgfpathqlineto{130.8057bp}{88.6256bp}
    \pgfpathqlineto{131.7536bp}{88.6256bp}
    \pgfpathqlineto{132.7014bp}{88.6256bp}
    \pgfpathqlineto{133.6493bp}{88.6256bp}
    \pgfpathqlineto{134.5972bp}{88.6256bp}
    \pgfpathqlineto{135.5450bp}{88.6256bp}
    \pgfpathqlineto{136.4929bp}{88.6256bp}
    \pgfpathqlineto{137.4408bp}{88.6256bp}
    \pgfpathqlineto{138.3886bp}{88.6256bp}
    \pgfpathqlineto{139.3365bp}{88.6256bp}
    \pgfpathqlineto{140.2844bp}{88.6256bp}
    \pgfpathqlineto{141.2322bp}{88.6256bp}
    \pgfpathqlineto{142.1801bp}{88.6256bp}
    \pgfpathqlineto{143.1280bp}{88.6256bp}
    \pgfpathqlineto{144.0758bp}{88.6256bp}
    \pgfpathqlineto{145.0237bp}{88.6256bp}
    \pgfpathqlineto{145.9716bp}{88.6256bp}
    \pgfpathqlineto{146.9194bp}{88.6256bp}
    \pgfpathqlineto{147.8673bp}{88.6256bp}
    \pgfpathqlineto{148.8152bp}{88.6256bp}
    \pgfpathqlineto{149.7630bp}{88.6256bp}
    \pgfpathqlineto{150.7109bp}{88.6256bp}
    \pgfpathqlineto{151.6588bp}{88.6256bp}
    \pgfpathqlineto{152.6066bp}{88.6256bp}
    \pgfpathqlineto{153.5545bp}{88.6256bp}
    \pgfpathqlineto{154.5024bp}{88.6256bp}
    \pgfpathqlineto{155.4502bp}{88.6256bp}
    \pgfpathqlineto{156.3981bp}{88.6256bp}
    \pgfpathqlineto{157.3460bp}{88.6256bp}
    \pgfpathqlineto{158.2938bp}{88.6256bp}
    \pgfpathqlineto{159.2417bp}{88.6256bp}
    \pgfpathqlineto{160.1896bp}{88.6256bp}
    \pgfpathqlineto{161.1374bp}{88.6256bp}
    \pgfpathqlineto{162.0853bp}{88.6256bp}
    \pgfpathqlineto{163.0332bp}{88.6256bp}
    \pgfpathqlineto{163.9810bp}{88.6256bp}
    \pgfpathqlineto{164.9289bp}{88.6256bp}
    \pgfpathqlineto{165.8768bp}{88.6256bp}
    \pgfpathqlineto{166.8246bp}{88.6256bp}
    \pgfpathqlineto{167.7725bp}{88.6256bp}
    \pgfpathqlineto{168.7204bp}{88.6256bp}
    \pgfpathqlineto{169.6682bp}{88.6256bp}
    \pgfpathqlineto{170.6161bp}{88.6256bp}
    \pgfpathqlineto{171.5640bp}{88.6256bp}
    \pgfpathqlineto{172.5118bp}{88.6256bp}
    \pgfpathqlineto{173.4597bp}{88.6256bp}
    \pgfpathqlineto{174.4076bp}{88.6256bp}
    \pgfpathqlineto{175.3555bp}{88.6256bp}
    \pgfpathqlineto{176.3033bp}{88.6256bp}
    \pgfpathqlineto{177.2512bp}{88.6256bp}
    \pgfpathqlineto{178.1991bp}{88.6256bp}
    \pgfpathqlineto{179.1469bp}{88.6256bp}
    \pgfpathqlineto{180.0948bp}{88.6256bp}
    \pgfpathqlineto{181.0427bp}{88.6256bp}
    \pgfpathqlineto{181.9905bp}{88.6256bp}
    \pgfpathqlineto{182.9384bp}{88.6256bp}
    \pgfpathqlineto{183.8863bp}{88.6256bp}
    \pgfpathqlineto{184.8341bp}{88.6256bp}
    \pgfpathqlineto{185.7820bp}{88.6256bp}
    \pgfpathqlineto{186.7299bp}{88.6256bp}
    \pgfpathqlineto{187.6777bp}{88.6256bp}
    \pgfpathqlineto{188.6256bp}{90.0474bp}
    \pgfpathqlineto{189.5735bp}{90.0474bp}
    \pgfpathqlineto{190.5213bp}{90.0474bp}
    \pgfpathqlineto{191.4692bp}{90.0474bp}
    \pgfpathqlineto{192.4171bp}{90.0474bp}
    \pgfpathqlineto{193.3649bp}{90.0474bp}
    \pgfpathqlineto{194.3128bp}{90.0474bp}
    \pgfpathqlineto{195.2607bp}{90.0474bp}
    \pgfpathqlineto{196.2085bp}{90.0474bp}
    \pgfpathqlineto{197.1564bp}{90.0474bp}
    \pgfpathqlineto{198.1043bp}{90.0474bp}
    \pgfpathqlineto{199.0521bp}{90.0474bp}
    \pgfpathqlineto{200.0000bp}{90.0474bp}
    \pgfusepathqstroke
  \end{pgfscope}
  \begin{pgfscope}
    \pgfsetlinewidth{0.5000bp}
    \definecolor{sc}{rgb}{1.0000,0.0000,0.0000}
    \pgfsetstrokecolor{sc}
    \pgfsetmiterjoin
    \pgfsetbuttcap
    \pgfpathqmoveto{13.2701bp}{88.6256bp}
    \pgfpathqlineto{13.2701bp}{87.6777bp}
    \pgfusepathqstroke
  \end{pgfscope}
  \begin{pgfscope}
    \pgfsetlinewidth{0.5000bp}
    \definecolor{sc}{rgb}{1.0000,0.0000,0.0000}
    \pgfsetstrokecolor{sc}
    \pgfsetmiterjoin
    \pgfsetbuttcap
    \pgfpathqmoveto{71.0900bp}{88.6256bp}
    \pgfpathqlineto{71.0900bp}{87.6777bp}
    \pgfusepathqstroke
  \end{pgfscope}
  \begin{pgfscope}
    \pgfsetlinewidth{0.5000bp}
    \definecolor{sc}{rgb}{1.0000,0.0000,0.0000}
    \pgfsetstrokecolor{sc}
    \pgfsetmiterjoin
    \pgfsetbuttcap
    \pgfpathqmoveto{119.4313bp}{88.6256bp}
    \pgfpathqlineto{119.4313bp}{87.6777bp}
    \pgfusepathqstroke
  \end{pgfscope}
  \begin{pgfscope}
    \pgfsetlinewidth{0.5000bp}
    \definecolor{sc}{rgb}{0.0000,0.0000,0.0000}
    \pgfsetstrokecolor{sc}
    \pgfsetmiterjoin
    \pgfsetbuttcap
    \pgfpathqmoveto{196.2085bp}{88.6256bp}
    \pgfpathqlineto{196.2085bp}{88.1517bp}
    \pgfusepathqstroke
  \end{pgfscope}
  \begin{pgfscope}
    \pgfsetlinewidth{0.5000bp}
    \definecolor{sc}{rgb}{0.0000,0.0000,0.0000}
    \pgfsetstrokecolor{sc}
    \pgfsetmiterjoin
    \pgfsetbuttcap
    \pgfpathqmoveto{191.4692bp}{88.6256bp}
    \pgfpathqlineto{191.4692bp}{88.1517bp}
    \pgfusepathqstroke
  \end{pgfscope}
  \begin{pgfscope}
    \pgfsetlinewidth{0.5000bp}
    \definecolor{sc}{rgb}{0.0000,0.0000,0.0000}
    \pgfsetstrokecolor{sc}
    \pgfsetmiterjoin
    \pgfsetbuttcap
    \pgfpathqmoveto{186.7299bp}{88.6256bp}
    \pgfpathqlineto{186.7299bp}{88.1517bp}
    \pgfusepathqstroke
  \end{pgfscope}
  \begin{pgfscope}
    \pgfsetlinewidth{0.5000bp}
    \definecolor{sc}{rgb}{0.0000,0.0000,0.0000}
    \pgfsetstrokecolor{sc}
    \pgfsetmiterjoin
    \pgfsetbuttcap
    \pgfpathqmoveto{181.9905bp}{88.6256bp}
    \pgfpathqlineto{181.9905bp}{88.1517bp}
    \pgfusepathqstroke
  \end{pgfscope}
  \begin{pgfscope}
    \pgfsetlinewidth{0.5000bp}
    \definecolor{sc}{rgb}{0.0000,0.0000,0.0000}
    \pgfsetstrokecolor{sc}
    \pgfsetmiterjoin
    \pgfsetbuttcap
    \pgfpathqmoveto{177.2512bp}{88.6256bp}
    \pgfpathqlineto{177.2512bp}{88.1517bp}
    \pgfusepathqstroke
  \end{pgfscope}
  \begin{pgfscope}
    \pgfsetlinewidth{0.5000bp}
    \definecolor{sc}{rgb}{0.0000,0.0000,0.0000}
    \pgfsetstrokecolor{sc}
    \pgfsetmiterjoin
    \pgfsetbuttcap
    \pgfpathqmoveto{172.5118bp}{88.6256bp}
    \pgfpathqlineto{172.5118bp}{88.1517bp}
    \pgfusepathqstroke
  \end{pgfscope}
  \begin{pgfscope}
    \pgfsetlinewidth{0.5000bp}
    \definecolor{sc}{rgb}{0.0000,0.0000,0.0000}
    \pgfsetstrokecolor{sc}
    \pgfsetmiterjoin
    \pgfsetbuttcap
    \pgfpathqmoveto{167.7725bp}{88.6256bp}
    \pgfpathqlineto{167.7725bp}{88.1517bp}
    \pgfusepathqstroke
  \end{pgfscope}
  \begin{pgfscope}
    \pgfsetlinewidth{0.5000bp}
    \definecolor{sc}{rgb}{0.0000,0.0000,0.0000}
    \pgfsetstrokecolor{sc}
    \pgfsetmiterjoin
    \pgfsetbuttcap
    \pgfpathqmoveto{163.0332bp}{88.6256bp}
    \pgfpathqlineto{163.0332bp}{88.1517bp}
    \pgfusepathqstroke
  \end{pgfscope}
  \begin{pgfscope}
    \pgfsetlinewidth{0.5000bp}
    \definecolor{sc}{rgb}{0.0000,0.0000,0.0000}
    \pgfsetstrokecolor{sc}
    \pgfsetmiterjoin
    \pgfsetbuttcap
    \pgfpathqmoveto{158.2938bp}{88.6256bp}
    \pgfpathqlineto{158.2938bp}{88.1517bp}
    \pgfusepathqstroke
  \end{pgfscope}
  \begin{pgfscope}
    \pgfsetlinewidth{0.5000bp}
    \definecolor{sc}{rgb}{0.0000,0.0000,0.0000}
    \pgfsetstrokecolor{sc}
    \pgfsetmiterjoin
    \pgfsetbuttcap
    \pgfpathqmoveto{153.5545bp}{88.6256bp}
    \pgfpathqlineto{153.5545bp}{88.1517bp}
    \pgfusepathqstroke
  \end{pgfscope}
  \begin{pgfscope}
    \pgfsetlinewidth{0.5000bp}
    \definecolor{sc}{rgb}{0.0000,0.0000,0.0000}
    \pgfsetstrokecolor{sc}
    \pgfsetmiterjoin
    \pgfsetbuttcap
    \pgfpathqmoveto{148.8152bp}{88.6256bp}
    \pgfpathqlineto{148.8152bp}{88.1517bp}
    \pgfusepathqstroke
  \end{pgfscope}
  \begin{pgfscope}
    \pgfsetlinewidth{0.5000bp}
    \definecolor{sc}{rgb}{0.0000,0.0000,0.0000}
    \pgfsetstrokecolor{sc}
    \pgfsetmiterjoin
    \pgfsetbuttcap
    \pgfpathqmoveto{144.0758bp}{88.6256bp}
    \pgfpathqlineto{144.0758bp}{88.1517bp}
    \pgfusepathqstroke
  \end{pgfscope}
  \begin{pgfscope}
    \pgfsetlinewidth{0.5000bp}
    \definecolor{sc}{rgb}{0.0000,0.0000,0.0000}
    \pgfsetstrokecolor{sc}
    \pgfsetmiterjoin
    \pgfsetbuttcap
    \pgfpathqmoveto{139.3365bp}{88.6256bp}
    \pgfpathqlineto{139.3365bp}{88.1517bp}
    \pgfusepathqstroke
  \end{pgfscope}
  \begin{pgfscope}
    \pgfsetlinewidth{0.5000bp}
    \definecolor{sc}{rgb}{0.0000,0.0000,0.0000}
    \pgfsetstrokecolor{sc}
    \pgfsetmiterjoin
    \pgfsetbuttcap
    \pgfpathqmoveto{134.5972bp}{88.6256bp}
    \pgfpathqlineto{134.5972bp}{88.1517bp}
    \pgfusepathqstroke
  \end{pgfscope}
  \begin{pgfscope}
    \pgfsetlinewidth{0.5000bp}
    \definecolor{sc}{rgb}{0.0000,0.0000,0.0000}
    \pgfsetstrokecolor{sc}
    \pgfsetmiterjoin
    \pgfsetbuttcap
    \pgfpathqmoveto{129.8578bp}{88.6256bp}
    \pgfpathqlineto{129.8578bp}{88.1517bp}
    \pgfusepathqstroke
  \end{pgfscope}
  \begin{pgfscope}
    \pgfsetlinewidth{0.5000bp}
    \definecolor{sc}{rgb}{0.0000,0.0000,0.0000}
    \pgfsetstrokecolor{sc}
    \pgfsetmiterjoin
    \pgfsetbuttcap
    \pgfpathqmoveto{125.1185bp}{88.6256bp}
    \pgfpathqlineto{125.1185bp}{88.1517bp}
    \pgfusepathqstroke
  \end{pgfscope}
  \begin{pgfscope}
    \pgfsetlinewidth{0.5000bp}
    \definecolor{sc}{rgb}{0.0000,0.0000,0.0000}
    \pgfsetstrokecolor{sc}
    \pgfsetmiterjoin
    \pgfsetbuttcap
    \pgfpathqmoveto{120.3791bp}{88.6256bp}
    \pgfpathqlineto{120.3791bp}{88.1517bp}
    \pgfusepathqstroke
  \end{pgfscope}
  \begin{pgfscope}
    \pgfsetlinewidth{0.5000bp}
    \definecolor{sc}{rgb}{0.0000,0.0000,0.0000}
    \pgfsetstrokecolor{sc}
    \pgfsetmiterjoin
    \pgfsetbuttcap
    \pgfpathqmoveto{115.6398bp}{88.6256bp}
    \pgfpathqlineto{115.6398bp}{88.1517bp}
    \pgfusepathqstroke
  \end{pgfscope}
  \begin{pgfscope}
    \pgfsetlinewidth{0.5000bp}
    \definecolor{sc}{rgb}{0.0000,0.0000,0.0000}
    \pgfsetstrokecolor{sc}
    \pgfsetmiterjoin
    \pgfsetbuttcap
    \pgfpathqmoveto{110.9005bp}{88.6256bp}
    \pgfpathqlineto{110.9005bp}{88.1517bp}
    \pgfusepathqstroke
  \end{pgfscope}
  \begin{pgfscope}
    \pgfsetlinewidth{0.5000bp}
    \definecolor{sc}{rgb}{0.0000,0.0000,0.0000}
    \pgfsetstrokecolor{sc}
    \pgfsetmiterjoin
    \pgfsetbuttcap
    \pgfpathqmoveto{106.1611bp}{88.6256bp}
    \pgfpathqlineto{106.1611bp}{88.1517bp}
    \pgfusepathqstroke
  \end{pgfscope}
  \begin{pgfscope}
    \pgfsetlinewidth{0.5000bp}
    \definecolor{sc}{rgb}{0.0000,0.0000,0.0000}
    \pgfsetstrokecolor{sc}
    \pgfsetmiterjoin
    \pgfsetbuttcap
    \pgfpathqmoveto{101.4218bp}{88.6256bp}
    \pgfpathqlineto{101.4218bp}{88.1517bp}
    \pgfusepathqstroke
  \end{pgfscope}
  \begin{pgfscope}
    \pgfsetlinewidth{0.5000bp}
    \definecolor{sc}{rgb}{0.0000,0.0000,0.0000}
    \pgfsetstrokecolor{sc}
    \pgfsetmiterjoin
    \pgfsetbuttcap
    \pgfpathqmoveto{96.6825bp}{88.6256bp}
    \pgfpathqlineto{96.6825bp}{88.1517bp}
    \pgfusepathqstroke
  \end{pgfscope}
  \begin{pgfscope}
    \pgfsetlinewidth{0.5000bp}
    \definecolor{sc}{rgb}{0.0000,0.0000,0.0000}
    \pgfsetstrokecolor{sc}
    \pgfsetmiterjoin
    \pgfsetbuttcap
    \pgfpathqmoveto{91.9431bp}{88.6256bp}
    \pgfpathqlineto{91.9431bp}{88.1517bp}
    \pgfusepathqstroke
  \end{pgfscope}
  \begin{pgfscope}
    \pgfsetlinewidth{0.5000bp}
    \definecolor{sc}{rgb}{0.0000,0.0000,0.0000}
    \pgfsetstrokecolor{sc}
    \pgfsetmiterjoin
    \pgfsetbuttcap
    \pgfpathqmoveto{87.2038bp}{88.6256bp}
    \pgfpathqlineto{87.2038bp}{88.1517bp}
    \pgfusepathqstroke
  \end{pgfscope}
  \begin{pgfscope}
    \pgfsetlinewidth{0.5000bp}
    \definecolor{sc}{rgb}{0.0000,0.0000,0.0000}
    \pgfsetstrokecolor{sc}
    \pgfsetmiterjoin
    \pgfsetbuttcap
    \pgfpathqmoveto{82.4645bp}{88.6256bp}
    \pgfpathqlineto{82.4645bp}{88.1517bp}
    \pgfusepathqstroke
  \end{pgfscope}
  \begin{pgfscope}
    \pgfsetlinewidth{0.5000bp}
    \definecolor{sc}{rgb}{0.0000,0.0000,0.0000}
    \pgfsetstrokecolor{sc}
    \pgfsetmiterjoin
    \pgfsetbuttcap
    \pgfpathqmoveto{77.7251bp}{88.6256bp}
    \pgfpathqlineto{77.7251bp}{88.1517bp}
    \pgfusepathqstroke
  \end{pgfscope}
  \begin{pgfscope}
    \pgfsetlinewidth{0.5000bp}
    \definecolor{sc}{rgb}{0.0000,0.0000,0.0000}
    \pgfsetstrokecolor{sc}
    \pgfsetmiterjoin
    \pgfsetbuttcap
    \pgfpathqmoveto{72.9858bp}{88.6256bp}
    \pgfpathqlineto{72.9858bp}{88.1517bp}
    \pgfusepathqstroke
  \end{pgfscope}
  \begin{pgfscope}
    \pgfsetlinewidth{0.5000bp}
    \definecolor{sc}{rgb}{0.0000,0.0000,0.0000}
    \pgfsetstrokecolor{sc}
    \pgfsetmiterjoin
    \pgfsetbuttcap
    \pgfpathqmoveto{68.2464bp}{88.6256bp}
    \pgfpathqlineto{68.2464bp}{88.1517bp}
    \pgfusepathqstroke
  \end{pgfscope}
  \begin{pgfscope}
    \pgfsetlinewidth{0.5000bp}
    \definecolor{sc}{rgb}{0.0000,0.0000,0.0000}
    \pgfsetstrokecolor{sc}
    \pgfsetmiterjoin
    \pgfsetbuttcap
    \pgfpathqmoveto{63.5071bp}{88.6256bp}
    \pgfpathqlineto{63.5071bp}{88.1517bp}
    \pgfusepathqstroke
  \end{pgfscope}
  \begin{pgfscope}
    \pgfsetlinewidth{0.5000bp}
    \definecolor{sc}{rgb}{0.0000,0.0000,0.0000}
    \pgfsetstrokecolor{sc}
    \pgfsetmiterjoin
    \pgfsetbuttcap
    \pgfpathqmoveto{58.7678bp}{88.6256bp}
    \pgfpathqlineto{58.7678bp}{88.1517bp}
    \pgfusepathqstroke
  \end{pgfscope}
  \begin{pgfscope}
    \pgfsetlinewidth{0.5000bp}
    \definecolor{sc}{rgb}{0.0000,0.0000,0.0000}
    \pgfsetstrokecolor{sc}
    \pgfsetmiterjoin
    \pgfsetbuttcap
    \pgfpathqmoveto{54.0284bp}{88.6256bp}
    \pgfpathqlineto{54.0284bp}{88.1517bp}
    \pgfusepathqstroke
  \end{pgfscope}
  \begin{pgfscope}
    \pgfsetlinewidth{0.5000bp}
    \definecolor{sc}{rgb}{0.0000,0.0000,0.0000}
    \pgfsetstrokecolor{sc}
    \pgfsetmiterjoin
    \pgfsetbuttcap
    \pgfpathqmoveto{49.2891bp}{88.6256bp}
    \pgfpathqlineto{49.2891bp}{88.1517bp}
    \pgfusepathqstroke
  \end{pgfscope}
  \begin{pgfscope}
    \pgfsetlinewidth{0.5000bp}
    \definecolor{sc}{rgb}{0.0000,0.0000,0.0000}
    \pgfsetstrokecolor{sc}
    \pgfsetmiterjoin
    \pgfsetbuttcap
    \pgfpathqmoveto{44.5498bp}{88.6256bp}
    \pgfpathqlineto{44.5498bp}{88.1517bp}
    \pgfusepathqstroke
  \end{pgfscope}
  \begin{pgfscope}
    \pgfsetlinewidth{0.5000bp}
    \definecolor{sc}{rgb}{0.0000,0.0000,0.0000}
    \pgfsetstrokecolor{sc}
    \pgfsetmiterjoin
    \pgfsetbuttcap
    \pgfpathqmoveto{39.8104bp}{88.6256bp}
    \pgfpathqlineto{39.8104bp}{88.1517bp}
    \pgfusepathqstroke
  \end{pgfscope}
  \begin{pgfscope}
    \pgfsetlinewidth{0.5000bp}
    \definecolor{sc}{rgb}{0.0000,0.0000,0.0000}
    \pgfsetstrokecolor{sc}
    \pgfsetmiterjoin
    \pgfsetbuttcap
    \pgfpathqmoveto{35.0711bp}{88.6256bp}
    \pgfpathqlineto{35.0711bp}{88.1517bp}
    \pgfusepathqstroke
  \end{pgfscope}
  \begin{pgfscope}
    \pgfsetlinewidth{0.5000bp}
    \definecolor{sc}{rgb}{0.0000,0.0000,0.0000}
    \pgfsetstrokecolor{sc}
    \pgfsetmiterjoin
    \pgfsetbuttcap
    \pgfpathqmoveto{30.3318bp}{88.6256bp}
    \pgfpathqlineto{30.3318bp}{88.1517bp}
    \pgfusepathqstroke
  \end{pgfscope}
  \begin{pgfscope}
    \pgfsetlinewidth{0.5000bp}
    \definecolor{sc}{rgb}{0.0000,0.0000,0.0000}
    \pgfsetstrokecolor{sc}
    \pgfsetmiterjoin
    \pgfsetbuttcap
    \pgfpathqmoveto{25.5924bp}{88.6256bp}
    \pgfpathqlineto{25.5924bp}{88.1517bp}
    \pgfusepathqstroke
  \end{pgfscope}
  \begin{pgfscope}
    \pgfsetlinewidth{0.5000bp}
    \definecolor{sc}{rgb}{0.0000,0.0000,0.0000}
    \pgfsetstrokecolor{sc}
    \pgfsetmiterjoin
    \pgfsetbuttcap
    \pgfpathqmoveto{20.8531bp}{88.6256bp}
    \pgfpathqlineto{20.8531bp}{88.1517bp}
    \pgfusepathqstroke
  \end{pgfscope}
  \begin{pgfscope}
    \pgfsetlinewidth{0.5000bp}
    \definecolor{sc}{rgb}{0.0000,0.0000,0.0000}
    \pgfsetstrokecolor{sc}
    \pgfsetmiterjoin
    \pgfsetbuttcap
    \pgfpathqmoveto{16.1137bp}{88.6256bp}
    \pgfpathqlineto{16.1137bp}{88.1517bp}
    \pgfusepathqstroke
  \end{pgfscope}
  \begin{pgfscope}
    \pgfsetlinewidth{0.5000bp}
    \definecolor{sc}{rgb}{0.0000,0.0000,0.0000}
    \pgfsetstrokecolor{sc}
    \pgfsetmiterjoin
    \pgfsetbuttcap
    \pgfpathqmoveto{11.3744bp}{88.6256bp}
    \pgfpathqlineto{11.3744bp}{88.1517bp}
    \pgfusepathqstroke
  \end{pgfscope}
  \begin{pgfscope}
    \pgfsetlinewidth{0.5000bp}
    \definecolor{sc}{rgb}{0.0000,0.0000,0.0000}
    \pgfsetstrokecolor{sc}
    \pgfsetmiterjoin
    \pgfsetbuttcap
    \pgfpathqmoveto{6.6351bp}{88.6256bp}
    \pgfpathqlineto{6.6351bp}{88.1517bp}
    \pgfusepathqstroke
  \end{pgfscope}
  \begin{pgfscope}
    \definecolor{fc}{rgb}{0.0000,0.0000,0.0000}
    \pgfsetfillcolor{fc}
    \pgftransformshift{\pgfqpoint{-0.0000bp}{118.6730bp}}
    \pgftransformscale{0.1185}
    \pgftext[base,left]{$\mathbb{F}_A$}
  \end{pgfscope}
  \begin{pgfscope}
    \pgfsetlinewidth{0.5000bp}
    \definecolor{sc}{rgb}{0.0000,0.0000,0.0000}
    \pgfsetstrokecolor{sc}
    \pgfsetmiterjoin
    \pgfsetbuttcap
    \pgfpathqmoveto{1.8957bp}{118.9573bp}
    \pgfpathqlineto{1.7062bp}{118.9573bp}
    \pgfusepathqstroke
  \end{pgfscope}
  \begin{pgfscope}
    \pgfsetlinewidth{0.5000bp}
    \definecolor{sc}{rgb}{0.0000,0.0000,0.0000}
    \pgfsetstrokecolor{sc}
    \pgfsetmiterjoin
    \pgfsetbuttcap
    \pgfpathqmoveto{1.8957bp}{88.6256bp}
    \pgfpathqlineto{1.8957bp}{118.9573bp}
    \pgfusepathqstroke
  \end{pgfscope}
  \begin{pgfscope}
    \pgfsetlinewidth{0.5000bp}
    \definecolor{sc}{rgb}{0.0000,0.0000,0.0000}
    \pgfsetstrokecolor{sc}
    \pgfsetmiterjoin
    \pgfsetbuttcap
    \pgfpathqmoveto{1.8957bp}{88.6256bp}
    \pgfpathqlineto{200.0000bp}{88.6256bp}
    \pgfusepathqstroke
  \end{pgfscope}
\end{pgfpicture}

        \label{fig:ex:ca:hgma:ex:move-h}
    \caption{push-v effects}\label{fig:ex:ca:hgma:ex:disconnected}
\end{figure}

\begin{figure}
    \centering
    \begin{pgfpicture}
  \pgfpathrectangle{\pgfpointorigin}{\pgfqpoint{200.0000bp}{200.0000bp}}
  \pgfusepath{use as bounding box}
  \begin{pgfscope}
    \definecolor{fc}{rgb}{0.0000,0.0000,0.0000}
    \pgfsetfillcolor{fc}
    \pgftransformshift{\pgfqpoint{5.1515bp}{69.8485bp}}
    \pgftransformscale{0.1894}
    \pgftext[base,left]{candidates}
  \end{pgfscope}
  \begin{pgfscope}
    \definecolor{fc}{rgb}{0.0000,0.0000,0.0000}
    \pgfsetfillcolor{fc}
    \pgfsetlinewidth{0.5000bp}
    \definecolor{sc}{rgb}{0.0000,0.0000,0.0000}
    \pgfsetstrokecolor{sc}
    \pgfsetmiterjoin
    \pgfsetbuttcap
    \pgfpathqmoveto{3.6364bp}{70.3030bp}
    \pgfpathqcurveto{3.6364bp}{70.6377bp}{3.3650bp}{70.9091bp}{3.0303bp}{70.9091bp}
    \pgfpathqcurveto{2.6956bp}{70.9091bp}{2.4242bp}{70.6377bp}{2.4242bp}{70.3030bp}
    \pgfpathqcurveto{2.4242bp}{69.9683bp}{2.6956bp}{69.6970bp}{3.0303bp}{69.6970bp}
    \pgfpathqcurveto{3.3650bp}{69.6970bp}{3.6364bp}{69.9683bp}{3.6364bp}{70.3030bp}
    \pgfpathclose
    \pgfusepathqfillstroke
  \end{pgfscope}
  \begin{pgfscope}
    \definecolor{fc}{rgb}{0.0000,0.0000,0.0000}
    \pgfsetfillcolor{fc}
    \pgftransformshift{\pgfqpoint{5.1515bp}{71.8182bp}}
    \pgftransformscale{0.1894}
    \pgftext[base,left]{negative unproven}
  \end{pgfscope}
  \begin{pgfscope}
    \definecolor{fc}{rgb}{1.0000,1.0000,0.0000}
    \pgfsetfillcolor{fc}
    \pgfsetlinewidth{0.5000bp}
    \definecolor{sc}{rgb}{1.0000,1.0000,0.0000}
    \pgfsetstrokecolor{sc}
    \pgfsetmiterjoin
    \pgfsetbuttcap
    \pgfpathqmoveto{3.6364bp}{72.2727bp}
    \pgfpathqcurveto{3.6364bp}{72.6074bp}{3.3650bp}{72.8788bp}{3.0303bp}{72.8788bp}
    \pgfpathqcurveto{2.6956bp}{72.8788bp}{2.4242bp}{72.6074bp}{2.4242bp}{72.2727bp}
    \pgfpathqcurveto{2.4242bp}{71.9380bp}{2.6956bp}{71.6667bp}{3.0303bp}{71.6667bp}
    \pgfpathqcurveto{3.3650bp}{71.6667bp}{3.6364bp}{71.9380bp}{3.6364bp}{72.2727bp}
    \pgfpathclose
    \pgfusepathqfillstroke
  \end{pgfscope}
  \begin{pgfscope}
    \definecolor{fc}{rgb}{0.0000,0.0000,0.0000}
    \pgfsetfillcolor{fc}
    \pgftransformshift{\pgfqpoint{5.1515bp}{73.7879bp}}
    \pgftransformscale{0.1894}
    \pgftext[base,left]{negative proven}
  \end{pgfscope}
  \begin{pgfscope}
    \definecolor{fc}{rgb}{0.0000,0.5020,0.0000}
    \pgfsetfillcolor{fc}
    \pgfsetlinewidth{0.5000bp}
    \definecolor{sc}{rgb}{0.0000,0.5020,0.0000}
    \pgfsetstrokecolor{sc}
    \pgfsetmiterjoin
    \pgfsetbuttcap
    \pgfpathqmoveto{3.6364bp}{74.2424bp}
    \pgfpathqcurveto{3.6364bp}{74.5771bp}{3.3650bp}{74.8485bp}{3.0303bp}{74.8485bp}
    \pgfpathqcurveto{2.6956bp}{74.8485bp}{2.4242bp}{74.5771bp}{2.4242bp}{74.2424bp}
    \pgfpathqcurveto{2.4242bp}{73.9077bp}{2.6956bp}{73.6364bp}{3.0303bp}{73.6364bp}
    \pgfpathqcurveto{3.3650bp}{73.6364bp}{3.6364bp}{73.9077bp}{3.6364bp}{74.2424bp}
    \pgfpathclose
    \pgfusepathqfillstroke
  \end{pgfscope}
  \begin{pgfscope}
    \definecolor{fc}{rgb}{0.0000,0.0000,0.0000}
    \pgfsetfillcolor{fc}
    \pgftransformshift{\pgfqpoint{5.1515bp}{75.7576bp}}
    \pgftransformscale{0.1894}
    \pgftext[base,left]{positive unproven}
  \end{pgfscope}
  \begin{pgfscope}
    \definecolor{fc}{rgb}{1.0000,0.0000,0.0000}
    \pgfsetfillcolor{fc}
    \pgfsetlinewidth{0.5000bp}
    \definecolor{sc}{rgb}{1.0000,0.0000,0.0000}
    \pgfsetstrokecolor{sc}
    \pgfsetmiterjoin
    \pgfsetbuttcap
    \pgfpathqmoveto{3.6364bp}{76.2121bp}
    \pgfpathqcurveto{3.6364bp}{76.5468bp}{3.3650bp}{76.8182bp}{3.0303bp}{76.8182bp}
    \pgfpathqcurveto{2.6956bp}{76.8182bp}{2.4242bp}{76.5468bp}{2.4242bp}{76.2121bp}
    \pgfpathqcurveto{2.4242bp}{75.8774bp}{2.6956bp}{75.6061bp}{3.0303bp}{75.6061bp}
    \pgfpathqcurveto{3.3650bp}{75.6061bp}{3.6364bp}{75.8774bp}{3.6364bp}{76.2121bp}
    \pgfpathclose
    \pgfusepathqfillstroke
  \end{pgfscope}
  \begin{pgfscope}
    \definecolor{fc}{rgb}{0.0000,0.0000,0.0000}
    \pgfsetfillcolor{fc}
    \pgftransformshift{\pgfqpoint{5.1515bp}{77.7273bp}}
    \pgftransformscale{0.1894}
    \pgftext[base,left]{positive proven}
  \end{pgfscope}
  \begin{pgfscope}
    \definecolor{fc}{rgb}{0.0000,0.0000,1.0000}
    \pgfsetfillcolor{fc}
    \pgfsetlinewidth{0.5000bp}
    \definecolor{sc}{rgb}{0.0000,0.0000,1.0000}
    \pgfsetstrokecolor{sc}
    \pgfsetmiterjoin
    \pgfsetbuttcap
    \pgfpathqmoveto{3.6364bp}{78.1818bp}
    \pgfpathqcurveto{3.6364bp}{78.5165bp}{3.3650bp}{78.7879bp}{3.0303bp}{78.7879bp}
    \pgfpathqcurveto{2.6956bp}{78.7879bp}{2.4242bp}{78.5165bp}{2.4242bp}{78.1818bp}
    \pgfpathqcurveto{2.4242bp}{77.8471bp}{2.6956bp}{77.5758bp}{3.0303bp}{77.5758bp}
    \pgfpathqcurveto{3.3650bp}{77.5758bp}{3.6364bp}{77.8471bp}{3.6364bp}{78.1818bp}
    \pgfpathclose
    \pgfusepathqfillstroke
  \end{pgfscope}
  \begin{pgfscope}
    \pgfsetlinewidth{0.5000bp}
    \definecolor{sc}{rgb}{1.0000,1.0000,0.0000}
    \pgfsetstrokecolor{sc}
    \pgfsetmiterjoin
    \pgfsetbuttcap
    \pgfpathqmoveto{3.0303bp}{130.3030bp}
    \pgfpathqlineto{4.5455bp}{130.3030bp}
    \pgfpathqlineto{6.0606bp}{130.3030bp}
    \pgfpathqlineto{7.5758bp}{130.3030bp}
    \pgfpathqlineto{9.0909bp}{130.3030bp}
    \pgfpathqlineto{10.6061bp}{130.3030bp}
    \pgfpathqlineto{12.1212bp}{130.3030bp}
    \pgfpathqlineto{13.6364bp}{130.3030bp}
    \pgfpathqlineto{15.1515bp}{130.3030bp}
    \pgfpathqlineto{16.6667bp}{130.3030bp}
    \pgfpathqlineto{18.1818bp}{130.3030bp}
    \pgfpathqlineto{19.6970bp}{130.3030bp}
    \pgfpathqlineto{21.2121bp}{130.3030bp}
    \pgfpathqlineto{22.7273bp}{130.3030bp}
    \pgfpathqlineto{24.2424bp}{130.3030bp}
    \pgfpathqlineto{25.7576bp}{130.3030bp}
    \pgfpathqlineto{27.2727bp}{130.3030bp}
    \pgfpathqlineto{28.7879bp}{130.3030bp}
    \pgfpathqlineto{30.3030bp}{130.3030bp}
    \pgfpathqlineto{31.8182bp}{130.3030bp}
    \pgfpathqlineto{33.3333bp}{130.3030bp}
    \pgfpathqlineto{34.8485bp}{130.3030bp}
    \pgfpathqlineto{36.3636bp}{130.3030bp}
    \pgfpathqlineto{37.8788bp}{130.3030bp}
    \pgfpathqlineto{39.3939bp}{130.3030bp}
    \pgfpathqlineto{40.9091bp}{130.3030bp}
    \pgfpathqlineto{42.4242bp}{130.3030bp}
    \pgfpathqlineto{43.9394bp}{130.3030bp}
    \pgfpathqlineto{45.4545bp}{130.3030bp}
    \pgfpathqlineto{46.9697bp}{130.3030bp}
    \pgfpathqlineto{48.4848bp}{130.3030bp}
    \pgfpathqlineto{50.0000bp}{130.3030bp}
    \pgfpathqlineto{51.5152bp}{130.3030bp}
    \pgfpathqlineto{53.0303bp}{130.3030bp}
    \pgfpathqlineto{54.5455bp}{130.3030bp}
    \pgfpathqlineto{56.0606bp}{130.3030bp}
    \pgfpathqlineto{57.5758bp}{130.3030bp}
    \pgfpathqlineto{59.0909bp}{130.3030bp}
    \pgfpathqlineto{60.6061bp}{130.3030bp}
    \pgfpathqlineto{62.1212bp}{130.3030bp}
    \pgfpathqlineto{63.6364bp}{130.3030bp}
    \pgfpathqlineto{65.1515bp}{130.3030bp}
    \pgfpathqlineto{66.6667bp}{130.3030bp}
    \pgfpathqlineto{68.1818bp}{130.3030bp}
    \pgfpathqlineto{69.6970bp}{130.3030bp}
    \pgfpathqlineto{71.2121bp}{130.3030bp}
    \pgfpathqlineto{72.7273bp}{130.3030bp}
    \pgfpathqlineto{74.2424bp}{130.3030bp}
    \pgfpathqlineto{75.7576bp}{130.3030bp}
    \pgfpathqlineto{77.2727bp}{130.3030bp}
    \pgfpathqlineto{78.7879bp}{130.3030bp}
    \pgfpathqlineto{80.3030bp}{130.3030bp}
    \pgfpathqlineto{81.8182bp}{130.3030bp}
    \pgfpathqlineto{83.3333bp}{130.3030bp}
    \pgfpathqlineto{84.8485bp}{130.3030bp}
    \pgfpathqlineto{86.3636bp}{130.3030bp}
    \pgfpathqlineto{87.8788bp}{130.3030bp}
    \pgfpathqlineto{89.3939bp}{130.3030bp}
    \pgfpathqlineto{90.9091bp}{130.3030bp}
    \pgfpathqlineto{92.4242bp}{130.3030bp}
    \pgfpathqlineto{93.9394bp}{130.3030bp}
    \pgfpathqlineto{95.4545bp}{130.3030bp}
    \pgfpathqlineto{96.9697bp}{130.3030bp}
    \pgfpathqlineto{98.4848bp}{130.3030bp}
    \pgfpathqlineto{100.0000bp}{130.3030bp}
    \pgfpathqlineto{101.5152bp}{130.3030bp}
    \pgfpathqlineto{103.0303bp}{130.3030bp}
    \pgfpathqlineto{104.5455bp}{130.3030bp}
    \pgfpathqlineto{106.0606bp}{130.3030bp}
    \pgfpathqlineto{107.5758bp}{130.3030bp}
    \pgfpathqlineto{109.0909bp}{130.3030bp}
    \pgfpathqlineto{110.6061bp}{130.3030bp}
    \pgfpathqlineto{112.1212bp}{130.3030bp}
    \pgfpathqlineto{113.6364bp}{130.3030bp}
    \pgfpathqlineto{115.1515bp}{130.3030bp}
    \pgfpathqlineto{116.6667bp}{130.3030bp}
    \pgfpathqlineto{118.1818bp}{130.3030bp}
    \pgfpathqlineto{119.6970bp}{130.3030bp}
    \pgfpathqlineto{121.2121bp}{130.3030bp}
    \pgfpathqlineto{122.7273bp}{130.3030bp}
    \pgfpathqlineto{124.2424bp}{130.3030bp}
    \pgfpathqlineto{125.7576bp}{130.3030bp}
    \pgfpathqlineto{127.2727bp}{130.3030bp}
    \pgfpathqlineto{128.7879bp}{130.3030bp}
    \pgfpathqlineto{130.3030bp}{130.3030bp}
    \pgfpathqlineto{131.8182bp}{130.3030bp}
    \pgfpathqlineto{133.3333bp}{130.3030bp}
    \pgfpathqlineto{134.8485bp}{130.3030bp}
    \pgfpathqlineto{136.3636bp}{130.3030bp}
    \pgfpathqlineto{137.8788bp}{130.3030bp}
    \pgfpathqlineto{139.3939bp}{130.3030bp}
    \pgfpathqlineto{140.9091bp}{130.3030bp}
    \pgfpathqlineto{142.4242bp}{130.3030bp}
    \pgfpathqlineto{143.9394bp}{130.3030bp}
    \pgfpathqlineto{145.4545bp}{130.3030bp}
    \pgfpathqlineto{146.9697bp}{130.3030bp}
    \pgfpathqlineto{148.4848bp}{130.3030bp}
    \pgfpathqlineto{150.0000bp}{130.3030bp}
    \pgfpathqlineto{151.5152bp}{130.3030bp}
    \pgfpathqlineto{153.0303bp}{130.3030bp}
    \pgfpathqlineto{154.5455bp}{130.3030bp}
    \pgfpathqlineto{156.0606bp}{130.3030bp}
    \pgfpathqlineto{157.5758bp}{130.3030bp}
    \pgfpathqlineto{159.0909bp}{130.3030bp}
    \pgfpathqlineto{160.6061bp}{130.3030bp}
    \pgfpathqlineto{162.1212bp}{130.3030bp}
    \pgfpathqlineto{163.6364bp}{130.3030bp}
    \pgfpathqlineto{165.1515bp}{130.3030bp}
    \pgfpathqlineto{166.6667bp}{130.3030bp}
    \pgfpathqlineto{168.1818bp}{130.3030bp}
    \pgfpathqlineto{169.6970bp}{130.3030bp}
    \pgfpathqlineto{171.2121bp}{130.3030bp}
    \pgfpathqlineto{172.7273bp}{130.3030bp}
    \pgfpathqlineto{174.2424bp}{130.3030bp}
    \pgfpathqlineto{175.7576bp}{130.3030bp}
    \pgfpathqlineto{177.2727bp}{130.3030bp}
    \pgfpathqlineto{178.7879bp}{130.3030bp}
    \pgfpathqlineto{180.3030bp}{130.3030bp}
    \pgfpathqlineto{181.8182bp}{81.8182bp}
    \pgfpathqlineto{183.3333bp}{81.8182bp}
    \pgfpathqlineto{184.8485bp}{81.8182bp}
    \pgfpathqlineto{186.3636bp}{81.8182bp}
    \pgfpathqlineto{187.8788bp}{81.8182bp}
    \pgfpathqlineto{189.3939bp}{81.8182bp}
    \pgfpathqlineto{190.9091bp}{81.8182bp}
    \pgfpathqlineto{192.4242bp}{81.8182bp}
    \pgfpathqlineto{193.9394bp}{81.8182bp}
    \pgfpathqlineto{195.4545bp}{81.8182bp}
    \pgfpathqlineto{196.9697bp}{81.8182bp}
    \pgfpathqlineto{198.4848bp}{81.8182bp}
    \pgfusepathqstroke
  \end{pgfscope}
  \begin{pgfscope}
    \pgfsetlinewidth{0.5000bp}
    \definecolor{sc}{rgb}{0.0000,0.5020,0.0000}
    \pgfsetstrokecolor{sc}
    \pgfsetmiterjoin
    \pgfsetbuttcap
    \pgfpathqmoveto{3.0303bp}{81.8182bp}
    \pgfpathqlineto{4.5455bp}{81.8182bp}
    \pgfpathqlineto{6.0606bp}{81.8182bp}
    \pgfpathqlineto{7.5758bp}{81.8182bp}
    \pgfpathqlineto{9.0909bp}{81.8182bp}
    \pgfpathqlineto{10.6061bp}{81.8182bp}
    \pgfpathqlineto{12.1212bp}{81.8182bp}
    \pgfpathqlineto{13.6364bp}{81.8182bp}
    \pgfpathqlineto{15.1515bp}{81.8182bp}
    \pgfpathqlineto{16.6667bp}{81.8182bp}
    \pgfpathqlineto{18.1818bp}{81.8182bp}
    \pgfpathqlineto{19.6970bp}{81.8182bp}
    \pgfpathqlineto{21.2121bp}{81.8182bp}
    \pgfpathqlineto{22.7273bp}{81.8182bp}
    \pgfpathqlineto{24.2424bp}{81.8182bp}
    \pgfpathqlineto{25.7576bp}{81.8182bp}
    \pgfpathqlineto{27.2727bp}{81.8182bp}
    \pgfpathqlineto{28.7879bp}{81.8182bp}
    \pgfpathqlineto{30.3030bp}{81.8182bp}
    \pgfpathqlineto{31.8182bp}{81.8182bp}
    \pgfpathqlineto{33.3333bp}{81.8182bp}
    \pgfpathqlineto{34.8485bp}{81.8182bp}
    \pgfpathqlineto{36.3636bp}{81.8182bp}
    \pgfpathqlineto{37.8788bp}{81.8182bp}
    \pgfpathqlineto{39.3939bp}{81.8182bp}
    \pgfpathqlineto{40.9091bp}{81.8182bp}
    \pgfpathqlineto{42.4242bp}{81.8182bp}
    \pgfpathqlineto{43.9394bp}{81.8182bp}
    \pgfpathqlineto{45.4545bp}{81.8182bp}
    \pgfpathqlineto{46.9697bp}{81.8182bp}
    \pgfpathqlineto{48.4848bp}{81.8182bp}
    \pgfpathqlineto{50.0000bp}{81.8182bp}
    \pgfpathqlineto{51.5152bp}{81.8182bp}
    \pgfpathqlineto{53.0303bp}{81.8182bp}
    \pgfpathqlineto{54.5455bp}{81.8182bp}
    \pgfpathqlineto{56.0606bp}{81.8182bp}
    \pgfpathqlineto{57.5758bp}{81.8182bp}
    \pgfpathqlineto{59.0909bp}{81.8182bp}
    \pgfpathqlineto{60.6061bp}{81.8182bp}
    \pgfpathqlineto{62.1212bp}{81.8182bp}
    \pgfpathqlineto{63.6364bp}{81.8182bp}
    \pgfpathqlineto{65.1515bp}{81.8182bp}
    \pgfpathqlineto{66.6667bp}{81.8182bp}
    \pgfpathqlineto{68.1818bp}{81.8182bp}
    \pgfpathqlineto{69.6970bp}{81.8182bp}
    \pgfpathqlineto{71.2121bp}{81.8182bp}
    \pgfpathqlineto{72.7273bp}{81.8182bp}
    \pgfpathqlineto{74.2424bp}{81.8182bp}
    \pgfpathqlineto{75.7576bp}{81.8182bp}
    \pgfpathqlineto{77.2727bp}{81.8182bp}
    \pgfpathqlineto{78.7879bp}{81.8182bp}
    \pgfpathqlineto{80.3030bp}{81.8182bp}
    \pgfpathqlineto{81.8182bp}{81.8182bp}
    \pgfpathqlineto{83.3333bp}{81.8182bp}
    \pgfpathqlineto{84.8485bp}{81.8182bp}
    \pgfpathqlineto{86.3636bp}{81.8182bp}
    \pgfpathqlineto{87.8788bp}{81.8182bp}
    \pgfpathqlineto{89.3939bp}{81.8182bp}
    \pgfpathqlineto{90.9091bp}{81.8182bp}
    \pgfpathqlineto{92.4242bp}{81.8182bp}
    \pgfpathqlineto{93.9394bp}{81.8182bp}
    \pgfpathqlineto{95.4545bp}{81.8182bp}
    \pgfpathqlineto{96.9697bp}{81.8182bp}
    \pgfpathqlineto{98.4848bp}{81.8182bp}
    \pgfpathqlineto{100.0000bp}{81.8182bp}
    \pgfpathqlineto{101.5152bp}{81.8182bp}
    \pgfpathqlineto{103.0303bp}{81.8182bp}
    \pgfpathqlineto{104.5455bp}{81.8182bp}
    \pgfpathqlineto{106.0606bp}{81.8182bp}
    \pgfpathqlineto{107.5758bp}{81.8182bp}
    \pgfpathqlineto{109.0909bp}{81.8182bp}
    \pgfpathqlineto{110.6061bp}{81.8182bp}
    \pgfpathqlineto{112.1212bp}{81.8182bp}
    \pgfpathqlineto{113.6364bp}{81.8182bp}
    \pgfpathqlineto{115.1515bp}{81.8182bp}
    \pgfpathqlineto{116.6667bp}{81.8182bp}
    \pgfpathqlineto{118.1818bp}{81.8182bp}
    \pgfpathqlineto{119.6970bp}{81.8182bp}
    \pgfpathqlineto{121.2121bp}{81.8182bp}
    \pgfpathqlineto{122.7273bp}{81.8182bp}
    \pgfpathqlineto{124.2424bp}{81.8182bp}
    \pgfpathqlineto{125.7576bp}{81.8182bp}
    \pgfpathqlineto{127.2727bp}{81.8182bp}
    \pgfpathqlineto{128.7879bp}{81.8182bp}
    \pgfpathqlineto{130.3030bp}{81.8182bp}
    \pgfpathqlineto{131.8182bp}{81.8182bp}
    \pgfpathqlineto{133.3333bp}{81.8182bp}
    \pgfpathqlineto{134.8485bp}{81.8182bp}
    \pgfpathqlineto{136.3636bp}{81.8182bp}
    \pgfpathqlineto{137.8788bp}{81.8182bp}
    \pgfpathqlineto{139.3939bp}{81.8182bp}
    \pgfpathqlineto{140.9091bp}{81.8182bp}
    \pgfpathqlineto{142.4242bp}{81.8182bp}
    \pgfpathqlineto{143.9394bp}{81.8182bp}
    \pgfpathqlineto{145.4545bp}{81.8182bp}
    \pgfpathqlineto{146.9697bp}{81.8182bp}
    \pgfpathqlineto{148.4848bp}{81.8182bp}
    \pgfpathqlineto{150.0000bp}{81.8182bp}
    \pgfpathqlineto{151.5152bp}{81.8182bp}
    \pgfpathqlineto{153.0303bp}{81.8182bp}
    \pgfpathqlineto{154.5455bp}{81.8182bp}
    \pgfpathqlineto{156.0606bp}{81.8182bp}
    \pgfpathqlineto{157.5758bp}{81.8182bp}
    \pgfpathqlineto{159.0909bp}{81.8182bp}
    \pgfpathqlineto{160.6061bp}{81.8182bp}
    \pgfpathqlineto{162.1212bp}{81.8182bp}
    \pgfpathqlineto{163.6364bp}{81.8182bp}
    \pgfpathqlineto{165.1515bp}{81.8182bp}
    \pgfpathqlineto{166.6667bp}{81.8182bp}
    \pgfpathqlineto{168.1818bp}{81.8182bp}
    \pgfpathqlineto{169.6970bp}{81.8182bp}
    \pgfpathqlineto{171.2121bp}{81.8182bp}
    \pgfpathqlineto{172.7273bp}{81.8182bp}
    \pgfpathqlineto{174.2424bp}{81.8182bp}
    \pgfpathqlineto{175.7576bp}{81.8182bp}
    \pgfpathqlineto{177.2727bp}{81.8182bp}
    \pgfpathqlineto{178.7879bp}{81.8182bp}
    \pgfpathqlineto{180.3030bp}{81.8182bp}
    \pgfpathqlineto{181.8182bp}{84.0909bp}
    \pgfpathqlineto{183.3333bp}{84.0909bp}
    \pgfpathqlineto{184.8485bp}{84.0909bp}
    \pgfpathqlineto{186.3636bp}{84.0909bp}
    \pgfpathqlineto{187.8788bp}{84.0909bp}
    \pgfpathqlineto{189.3939bp}{84.0909bp}
    \pgfpathqlineto{190.9091bp}{84.0909bp}
    \pgfpathqlineto{192.4242bp}{84.0909bp}
    \pgfpathqlineto{193.9394bp}{84.0909bp}
    \pgfpathqlineto{195.4545bp}{84.0909bp}
    \pgfpathqlineto{196.9697bp}{84.0909bp}
    \pgfpathqlineto{198.4848bp}{84.0909bp}
    \pgfusepathqstroke
  \end{pgfscope}
  \begin{pgfscope}
    \pgfsetlinewidth{0.5000bp}
    \definecolor{sc}{rgb}{1.0000,0.0000,0.0000}
    \pgfsetstrokecolor{sc}
    \pgfsetmiterjoin
    \pgfsetbuttcap
    \pgfpathqmoveto{3.0303bp}{130.3030bp}
    \pgfpathqlineto{4.5455bp}{130.3030bp}
    \pgfpathqlineto{6.0606bp}{130.3030bp}
    \pgfpathqlineto{7.5758bp}{130.3030bp}
    \pgfpathqlineto{9.0909bp}{130.3030bp}
    \pgfpathqlineto{10.6061bp}{130.3030bp}
    \pgfpathqlineto{12.1212bp}{130.3030bp}
    \pgfpathqlineto{13.6364bp}{130.3030bp}
    \pgfpathqlineto{15.1515bp}{130.3030bp}
    \pgfpathqlineto{16.6667bp}{130.3030bp}
    \pgfpathqlineto{18.1818bp}{130.3030bp}
    \pgfpathqlineto{19.6970bp}{130.3030bp}
    \pgfpathqlineto{21.2121bp}{130.3030bp}
    \pgfpathqlineto{22.7273bp}{130.3030bp}
    \pgfpathqlineto{24.2424bp}{130.3030bp}
    \pgfpathqlineto{25.7576bp}{130.3030bp}
    \pgfpathqlineto{27.2727bp}{130.3030bp}
    \pgfpathqlineto{28.7879bp}{130.3030bp}
    \pgfpathqlineto{30.3030bp}{130.3030bp}
    \pgfpathqlineto{31.8182bp}{130.3030bp}
    \pgfpathqlineto{33.3333bp}{130.3030bp}
    \pgfpathqlineto{34.8485bp}{130.3030bp}
    \pgfpathqlineto{36.3636bp}{130.3030bp}
    \pgfpathqlineto{37.8788bp}{130.3030bp}
    \pgfpathqlineto{39.3939bp}{130.3030bp}
    \pgfpathqlineto{40.9091bp}{130.3030bp}
    \pgfpathqlineto{42.4242bp}{130.3030bp}
    \pgfpathqlineto{43.9394bp}{130.3030bp}
    \pgfpathqlineto{45.4545bp}{130.3030bp}
    \pgfpathqlineto{46.9697bp}{130.3030bp}
    \pgfpathqlineto{48.4848bp}{130.3030bp}
    \pgfpathqlineto{50.0000bp}{130.3030bp}
    \pgfpathqlineto{51.5152bp}{130.3030bp}
    \pgfpathqlineto{53.0303bp}{130.3030bp}
    \pgfpathqlineto{54.5455bp}{130.3030bp}
    \pgfpathqlineto{56.0606bp}{130.3030bp}
    \pgfpathqlineto{57.5758bp}{130.3030bp}
    \pgfpathqlineto{59.0909bp}{130.3030bp}
    \pgfpathqlineto{60.6061bp}{130.3030bp}
    \pgfpathqlineto{62.1212bp}{130.3030bp}
    \pgfpathqlineto{63.6364bp}{130.3030bp}
    \pgfpathqlineto{65.1515bp}{130.3030bp}
    \pgfpathqlineto{66.6667bp}{130.3030bp}
    \pgfpathqlineto{68.1818bp}{130.3030bp}
    \pgfpathqlineto{69.6970bp}{130.3030bp}
    \pgfpathqlineto{71.2121bp}{130.3030bp}
    \pgfpathqlineto{72.7273bp}{130.3030bp}
    \pgfpathqlineto{74.2424bp}{130.3030bp}
    \pgfpathqlineto{75.7576bp}{130.3030bp}
    \pgfpathqlineto{77.2727bp}{130.3030bp}
    \pgfpathqlineto{78.7879bp}{130.3030bp}
    \pgfpathqlineto{80.3030bp}{130.3030bp}
    \pgfpathqlineto{81.8182bp}{130.3030bp}
    \pgfpathqlineto{83.3333bp}{130.3030bp}
    \pgfpathqlineto{84.8485bp}{130.3030bp}
    \pgfpathqlineto{86.3636bp}{130.3030bp}
    \pgfpathqlineto{87.8788bp}{130.3030bp}
    \pgfpathqlineto{89.3939bp}{130.3030bp}
    \pgfpathqlineto{90.9091bp}{130.3030bp}
    \pgfpathqlineto{92.4242bp}{130.3030bp}
    \pgfpathqlineto{93.9394bp}{130.3030bp}
    \pgfpathqlineto{95.4545bp}{130.3030bp}
    \pgfpathqlineto{96.9697bp}{130.3030bp}
    \pgfpathqlineto{98.4848bp}{130.3030bp}
    \pgfpathqlineto{100.0000bp}{130.3030bp}
    \pgfpathqlineto{101.5152bp}{130.3030bp}
    \pgfpathqlineto{103.0303bp}{130.3030bp}
    \pgfpathqlineto{104.5455bp}{130.3030bp}
    \pgfpathqlineto{106.0606bp}{130.3030bp}
    \pgfpathqlineto{107.5758bp}{130.3030bp}
    \pgfpathqlineto{109.0909bp}{130.3030bp}
    \pgfpathqlineto{110.6061bp}{130.3030bp}
    \pgfpathqlineto{112.1212bp}{130.3030bp}
    \pgfpathqlineto{113.6364bp}{130.3030bp}
    \pgfpathqlineto{115.1515bp}{130.3030bp}
    \pgfpathqlineto{116.6667bp}{130.3030bp}
    \pgfpathqlineto{118.1818bp}{130.3030bp}
    \pgfpathqlineto{119.6970bp}{130.3030bp}
    \pgfpathqlineto{121.2121bp}{130.3030bp}
    \pgfpathqlineto{122.7273bp}{130.3030bp}
    \pgfpathqlineto{124.2424bp}{130.3030bp}
    \pgfpathqlineto{125.7576bp}{130.3030bp}
    \pgfpathqlineto{127.2727bp}{130.3030bp}
    \pgfpathqlineto{128.7879bp}{130.3030bp}
    \pgfpathqlineto{130.3030bp}{130.3030bp}
    \pgfpathqlineto{131.8182bp}{130.3030bp}
    \pgfpathqlineto{133.3333bp}{130.3030bp}
    \pgfpathqlineto{134.8485bp}{130.3030bp}
    \pgfpathqlineto{136.3636bp}{130.3030bp}
    \pgfpathqlineto{137.8788bp}{130.3030bp}
    \pgfpathqlineto{139.3939bp}{130.3030bp}
    \pgfpathqlineto{140.9091bp}{130.3030bp}
    \pgfpathqlineto{142.4242bp}{130.3030bp}
    \pgfpathqlineto{143.9394bp}{130.3030bp}
    \pgfpathqlineto{145.4545bp}{130.3030bp}
    \pgfpathqlineto{146.9697bp}{130.3030bp}
    \pgfpathqlineto{148.4848bp}{130.3030bp}
    \pgfpathqlineto{150.0000bp}{130.3030bp}
    \pgfpathqlineto{151.5152bp}{130.3030bp}
    \pgfpathqlineto{153.0303bp}{130.3030bp}
    \pgfpathqlineto{154.5455bp}{130.3030bp}
    \pgfpathqlineto{156.0606bp}{130.3030bp}
    \pgfpathqlineto{157.5758bp}{130.3030bp}
    \pgfpathqlineto{159.0909bp}{130.3030bp}
    \pgfpathqlineto{160.6061bp}{130.3030bp}
    \pgfpathqlineto{162.1212bp}{130.3030bp}
    \pgfpathqlineto{163.6364bp}{130.3030bp}
    \pgfpathqlineto{165.1515bp}{130.3030bp}
    \pgfpathqlineto{166.6667bp}{130.3030bp}
    \pgfpathqlineto{168.1818bp}{130.3030bp}
    \pgfpathqlineto{169.6970bp}{130.3030bp}
    \pgfpathqlineto{171.2121bp}{130.3030bp}
    \pgfpathqlineto{172.7273bp}{130.3030bp}
    \pgfpathqlineto{174.2424bp}{130.3030bp}
    \pgfpathqlineto{175.7576bp}{130.3030bp}
    \pgfpathqlineto{177.2727bp}{130.3030bp}
    \pgfpathqlineto{178.7879bp}{130.3030bp}
    \pgfpathqlineto{180.3030bp}{130.3030bp}
    \pgfpathqlineto{181.8182bp}{81.8182bp}
    \pgfpathqlineto{183.3333bp}{81.8182bp}
    \pgfpathqlineto{184.8485bp}{81.8182bp}
    \pgfpathqlineto{186.3636bp}{81.8182bp}
    \pgfpathqlineto{187.8788bp}{81.8182bp}
    \pgfpathqlineto{189.3939bp}{81.8182bp}
    \pgfpathqlineto{190.9091bp}{81.8182bp}
    \pgfpathqlineto{192.4242bp}{81.8182bp}
    \pgfpathqlineto{193.9394bp}{81.8182bp}
    \pgfpathqlineto{195.4545bp}{81.8182bp}
    \pgfpathqlineto{196.9697bp}{81.8182bp}
    \pgfpathqlineto{198.4848bp}{81.8182bp}
    \pgfusepathqstroke
  \end{pgfscope}
  \begin{pgfscope}
    \pgfsetlinewidth{0.5000bp}
    \definecolor{sc}{rgb}{0.0000,0.0000,1.0000}
    \pgfsetstrokecolor{sc}
    \pgfsetmiterjoin
    \pgfsetbuttcap
    \pgfpathqmoveto{3.0303bp}{81.8182bp}
    \pgfpathqlineto{4.5455bp}{81.8182bp}
    \pgfpathqlineto{6.0606bp}{81.8182bp}
    \pgfpathqlineto{7.5758bp}{81.8182bp}
    \pgfpathqlineto{9.0909bp}{81.8182bp}
    \pgfpathqlineto{10.6061bp}{81.8182bp}
    \pgfpathqlineto{12.1212bp}{81.8182bp}
    \pgfpathqlineto{13.6364bp}{81.8182bp}
    \pgfpathqlineto{15.1515bp}{81.8182bp}
    \pgfpathqlineto{16.6667bp}{81.8182bp}
    \pgfpathqlineto{18.1818bp}{81.8182bp}
    \pgfpathqlineto{19.6970bp}{81.8182bp}
    \pgfpathqlineto{21.2121bp}{81.8182bp}
    \pgfpathqlineto{22.7273bp}{81.8182bp}
    \pgfpathqlineto{24.2424bp}{81.8182bp}
    \pgfpathqlineto{25.7576bp}{81.8182bp}
    \pgfpathqlineto{27.2727bp}{81.8182bp}
    \pgfpathqlineto{28.7879bp}{81.8182bp}
    \pgfpathqlineto{30.3030bp}{81.8182bp}
    \pgfpathqlineto{31.8182bp}{81.8182bp}
    \pgfpathqlineto{33.3333bp}{81.8182bp}
    \pgfpathqlineto{34.8485bp}{81.8182bp}
    \pgfpathqlineto{36.3636bp}{81.8182bp}
    \pgfpathqlineto{37.8788bp}{81.8182bp}
    \pgfpathqlineto{39.3939bp}{81.8182bp}
    \pgfpathqlineto{40.9091bp}{81.8182bp}
    \pgfpathqlineto{42.4242bp}{81.8182bp}
    \pgfpathqlineto{43.9394bp}{81.8182bp}
    \pgfpathqlineto{45.4545bp}{81.8182bp}
    \pgfpathqlineto{46.9697bp}{81.8182bp}
    \pgfpathqlineto{48.4848bp}{81.8182bp}
    \pgfpathqlineto{50.0000bp}{81.8182bp}
    \pgfpathqlineto{51.5152bp}{81.8182bp}
    \pgfpathqlineto{53.0303bp}{81.8182bp}
    \pgfpathqlineto{54.5455bp}{81.8182bp}
    \pgfpathqlineto{56.0606bp}{81.8182bp}
    \pgfpathqlineto{57.5758bp}{81.8182bp}
    \pgfpathqlineto{59.0909bp}{81.8182bp}
    \pgfpathqlineto{60.6061bp}{81.8182bp}
    \pgfpathqlineto{62.1212bp}{81.8182bp}
    \pgfpathqlineto{63.6364bp}{81.8182bp}
    \pgfpathqlineto{65.1515bp}{81.8182bp}
    \pgfpathqlineto{66.6667bp}{81.8182bp}
    \pgfpathqlineto{68.1818bp}{81.8182bp}
    \pgfpathqlineto{69.6970bp}{81.8182bp}
    \pgfpathqlineto{71.2121bp}{81.8182bp}
    \pgfpathqlineto{72.7273bp}{81.8182bp}
    \pgfpathqlineto{74.2424bp}{81.8182bp}
    \pgfpathqlineto{75.7576bp}{81.8182bp}
    \pgfpathqlineto{77.2727bp}{81.8182bp}
    \pgfpathqlineto{78.7879bp}{81.8182bp}
    \pgfpathqlineto{80.3030bp}{81.8182bp}
    \pgfpathqlineto{81.8182bp}{81.8182bp}
    \pgfpathqlineto{83.3333bp}{81.8182bp}
    \pgfpathqlineto{84.8485bp}{81.8182bp}
    \pgfpathqlineto{86.3636bp}{81.8182bp}
    \pgfpathqlineto{87.8788bp}{81.8182bp}
    \pgfpathqlineto{89.3939bp}{81.8182bp}
    \pgfpathqlineto{90.9091bp}{81.8182bp}
    \pgfpathqlineto{92.4242bp}{81.8182bp}
    \pgfpathqlineto{93.9394bp}{81.8182bp}
    \pgfpathqlineto{95.4545bp}{81.8182bp}
    \pgfpathqlineto{96.9697bp}{81.8182bp}
    \pgfpathqlineto{98.4848bp}{81.8182bp}
    \pgfpathqlineto{100.0000bp}{81.8182bp}
    \pgfpathqlineto{101.5152bp}{81.8182bp}
    \pgfpathqlineto{103.0303bp}{81.8182bp}
    \pgfpathqlineto{104.5455bp}{81.8182bp}
    \pgfpathqlineto{106.0606bp}{81.8182bp}
    \pgfpathqlineto{107.5758bp}{81.8182bp}
    \pgfpathqlineto{109.0909bp}{81.8182bp}
    \pgfpathqlineto{110.6061bp}{81.8182bp}
    \pgfpathqlineto{112.1212bp}{81.8182bp}
    \pgfpathqlineto{113.6364bp}{81.8182bp}
    \pgfpathqlineto{115.1515bp}{81.8182bp}
    \pgfpathqlineto{116.6667bp}{81.8182bp}
    \pgfpathqlineto{118.1818bp}{81.8182bp}
    \pgfpathqlineto{119.6970bp}{81.8182bp}
    \pgfpathqlineto{121.2121bp}{81.8182bp}
    \pgfpathqlineto{122.7273bp}{81.8182bp}
    \pgfpathqlineto{124.2424bp}{81.8182bp}
    \pgfpathqlineto{125.7576bp}{81.8182bp}
    \pgfpathqlineto{127.2727bp}{81.8182bp}
    \pgfpathqlineto{128.7879bp}{81.8182bp}
    \pgfpathqlineto{130.3030bp}{81.8182bp}
    \pgfpathqlineto{131.8182bp}{81.8182bp}
    \pgfpathqlineto{133.3333bp}{81.8182bp}
    \pgfpathqlineto{134.8485bp}{81.8182bp}
    \pgfpathqlineto{136.3636bp}{81.8182bp}
    \pgfpathqlineto{137.8788bp}{81.8182bp}
    \pgfpathqlineto{139.3939bp}{81.8182bp}
    \pgfpathqlineto{140.9091bp}{81.8182bp}
    \pgfpathqlineto{142.4242bp}{81.8182bp}
    \pgfpathqlineto{143.9394bp}{81.8182bp}
    \pgfpathqlineto{145.4545bp}{81.8182bp}
    \pgfpathqlineto{146.9697bp}{81.8182bp}
    \pgfpathqlineto{148.4848bp}{81.8182bp}
    \pgfpathqlineto{150.0000bp}{81.8182bp}
    \pgfpathqlineto{151.5152bp}{81.8182bp}
    \pgfpathqlineto{153.0303bp}{81.8182bp}
    \pgfpathqlineto{154.5455bp}{81.8182bp}
    \pgfpathqlineto{156.0606bp}{81.8182bp}
    \pgfpathqlineto{157.5758bp}{81.8182bp}
    \pgfpathqlineto{159.0909bp}{81.8182bp}
    \pgfpathqlineto{160.6061bp}{81.8182bp}
    \pgfpathqlineto{162.1212bp}{81.8182bp}
    \pgfpathqlineto{163.6364bp}{81.8182bp}
    \pgfpathqlineto{165.1515bp}{81.8182bp}
    \pgfpathqlineto{166.6667bp}{81.8182bp}
    \pgfpathqlineto{168.1818bp}{81.8182bp}
    \pgfpathqlineto{169.6970bp}{81.8182bp}
    \pgfpathqlineto{171.2121bp}{81.8182bp}
    \pgfpathqlineto{172.7273bp}{81.8182bp}
    \pgfpathqlineto{174.2424bp}{81.8182bp}
    \pgfpathqlineto{175.7576bp}{81.8182bp}
    \pgfpathqlineto{177.2727bp}{81.8182bp}
    \pgfpathqlineto{178.7879bp}{81.8182bp}
    \pgfpathqlineto{180.3030bp}{81.8182bp}
    \pgfpathqlineto{181.8182bp}{84.8485bp}
    \pgfpathqlineto{183.3333bp}{84.8485bp}
    \pgfpathqlineto{184.8485bp}{84.8485bp}
    \pgfpathqlineto{186.3636bp}{84.8485bp}
    \pgfpathqlineto{187.8788bp}{84.8485bp}
    \pgfpathqlineto{189.3939bp}{84.8485bp}
    \pgfpathqlineto{190.9091bp}{84.8485bp}
    \pgfpathqlineto{192.4242bp}{84.8485bp}
    \pgfpathqlineto{193.9394bp}{84.8485bp}
    \pgfpathqlineto{195.4545bp}{84.8485bp}
    \pgfpathqlineto{196.9697bp}{84.8485bp}
    \pgfpathqlineto{198.4848bp}{84.8485bp}
    \pgfusepathqstroke
  \end{pgfscope}
  \begin{pgfscope}
    \pgfsetlinewidth{0.5000bp}
    \definecolor{sc}{rgb}{1.0000,0.0000,0.0000}
    \pgfsetstrokecolor{sc}
    \pgfsetmiterjoin
    \pgfsetbuttcap
    \pgfpathqmoveto{200.0000bp}{81.8182bp}
    \pgfpathqlineto{200.0000bp}{80.3030bp}
    \pgfusepathqstroke
  \end{pgfscope}
  \begin{pgfscope}
    \pgfsetlinewidth{0.5000bp}
    \definecolor{sc}{rgb}{1.0000,0.0000,0.0000}
    \pgfsetstrokecolor{sc}
    \pgfsetmiterjoin
    \pgfsetbuttcap
    \pgfpathqmoveto{183.3333bp}{81.8182bp}
    \pgfpathqlineto{183.3333bp}{80.3030bp}
    \pgfusepathqstroke
  \end{pgfscope}
  \begin{pgfscope}
    \pgfsetlinewidth{0.5000bp}
    \definecolor{sc}{rgb}{1.0000,0.0000,0.0000}
    \pgfsetstrokecolor{sc}
    \pgfsetmiterjoin
    \pgfsetbuttcap
    \pgfpathqmoveto{3.0303bp}{81.8182bp}
    \pgfpathqlineto{3.0303bp}{80.3030bp}
    \pgfusepathqstroke
  \end{pgfscope}
  \begin{pgfscope}
    \pgfsetlinewidth{0.5000bp}
    \definecolor{sc}{rgb}{0.0000,0.0000,0.0000}
    \pgfsetstrokecolor{sc}
    \pgfsetmiterjoin
    \pgfsetbuttcap
    \pgfpathqmoveto{200.0000bp}{81.8182bp}
    \pgfpathqlineto{200.0000bp}{81.0606bp}
    \pgfusepathqstroke
  \end{pgfscope}
  \begin{pgfscope}
    \pgfsetlinewidth{0.5000bp}
    \definecolor{sc}{rgb}{0.0000,0.0000,0.0000}
    \pgfsetstrokecolor{sc}
    \pgfsetmiterjoin
    \pgfsetbuttcap
    \pgfpathqmoveto{192.4242bp}{81.8182bp}
    \pgfpathqlineto{192.4242bp}{81.0606bp}
    \pgfusepathqstroke
  \end{pgfscope}
  \begin{pgfscope}
    \pgfsetlinewidth{0.5000bp}
    \definecolor{sc}{rgb}{0.0000,0.0000,0.0000}
    \pgfsetstrokecolor{sc}
    \pgfsetmiterjoin
    \pgfsetbuttcap
    \pgfpathqmoveto{184.8485bp}{81.8182bp}
    \pgfpathqlineto{184.8485bp}{81.0606bp}
    \pgfusepathqstroke
  \end{pgfscope}
  \begin{pgfscope}
    \pgfsetlinewidth{0.5000bp}
    \definecolor{sc}{rgb}{0.0000,0.0000,0.0000}
    \pgfsetstrokecolor{sc}
    \pgfsetmiterjoin
    \pgfsetbuttcap
    \pgfpathqmoveto{177.2727bp}{81.8182bp}
    \pgfpathqlineto{177.2727bp}{81.0606bp}
    \pgfusepathqstroke
  \end{pgfscope}
  \begin{pgfscope}
    \pgfsetlinewidth{0.5000bp}
    \definecolor{sc}{rgb}{0.0000,0.0000,0.0000}
    \pgfsetstrokecolor{sc}
    \pgfsetmiterjoin
    \pgfsetbuttcap
    \pgfpathqmoveto{169.6970bp}{81.8182bp}
    \pgfpathqlineto{169.6970bp}{81.0606bp}
    \pgfusepathqstroke
  \end{pgfscope}
  \begin{pgfscope}
    \pgfsetlinewidth{0.5000bp}
    \definecolor{sc}{rgb}{0.0000,0.0000,0.0000}
    \pgfsetstrokecolor{sc}
    \pgfsetmiterjoin
    \pgfsetbuttcap
    \pgfpathqmoveto{162.1212bp}{81.8182bp}
    \pgfpathqlineto{162.1212bp}{81.0606bp}
    \pgfusepathqstroke
  \end{pgfscope}
  \begin{pgfscope}
    \pgfsetlinewidth{0.5000bp}
    \definecolor{sc}{rgb}{0.0000,0.0000,0.0000}
    \pgfsetstrokecolor{sc}
    \pgfsetmiterjoin
    \pgfsetbuttcap
    \pgfpathqmoveto{154.5455bp}{81.8182bp}
    \pgfpathqlineto{154.5455bp}{81.0606bp}
    \pgfusepathqstroke
  \end{pgfscope}
  \begin{pgfscope}
    \pgfsetlinewidth{0.5000bp}
    \definecolor{sc}{rgb}{0.0000,0.0000,0.0000}
    \pgfsetstrokecolor{sc}
    \pgfsetmiterjoin
    \pgfsetbuttcap
    \pgfpathqmoveto{146.9697bp}{81.8182bp}
    \pgfpathqlineto{146.9697bp}{81.0606bp}
    \pgfusepathqstroke
  \end{pgfscope}
  \begin{pgfscope}
    \pgfsetlinewidth{0.5000bp}
    \definecolor{sc}{rgb}{0.0000,0.0000,0.0000}
    \pgfsetstrokecolor{sc}
    \pgfsetmiterjoin
    \pgfsetbuttcap
    \pgfpathqmoveto{139.3939bp}{81.8182bp}
    \pgfpathqlineto{139.3939bp}{81.0606bp}
    \pgfusepathqstroke
  \end{pgfscope}
  \begin{pgfscope}
    \pgfsetlinewidth{0.5000bp}
    \definecolor{sc}{rgb}{0.0000,0.0000,0.0000}
    \pgfsetstrokecolor{sc}
    \pgfsetmiterjoin
    \pgfsetbuttcap
    \pgfpathqmoveto{131.8182bp}{81.8182bp}
    \pgfpathqlineto{131.8182bp}{81.0606bp}
    \pgfusepathqstroke
  \end{pgfscope}
  \begin{pgfscope}
    \pgfsetlinewidth{0.5000bp}
    \definecolor{sc}{rgb}{0.0000,0.0000,0.0000}
    \pgfsetstrokecolor{sc}
    \pgfsetmiterjoin
    \pgfsetbuttcap
    \pgfpathqmoveto{124.2424bp}{81.8182bp}
    \pgfpathqlineto{124.2424bp}{81.0606bp}
    \pgfusepathqstroke
  \end{pgfscope}
  \begin{pgfscope}
    \pgfsetlinewidth{0.5000bp}
    \definecolor{sc}{rgb}{0.0000,0.0000,0.0000}
    \pgfsetstrokecolor{sc}
    \pgfsetmiterjoin
    \pgfsetbuttcap
    \pgfpathqmoveto{116.6667bp}{81.8182bp}
    \pgfpathqlineto{116.6667bp}{81.0606bp}
    \pgfusepathqstroke
  \end{pgfscope}
  \begin{pgfscope}
    \pgfsetlinewidth{0.5000bp}
    \definecolor{sc}{rgb}{0.0000,0.0000,0.0000}
    \pgfsetstrokecolor{sc}
    \pgfsetmiterjoin
    \pgfsetbuttcap
    \pgfpathqmoveto{109.0909bp}{81.8182bp}
    \pgfpathqlineto{109.0909bp}{81.0606bp}
    \pgfusepathqstroke
  \end{pgfscope}
  \begin{pgfscope}
    \pgfsetlinewidth{0.5000bp}
    \definecolor{sc}{rgb}{0.0000,0.0000,0.0000}
    \pgfsetstrokecolor{sc}
    \pgfsetmiterjoin
    \pgfsetbuttcap
    \pgfpathqmoveto{101.5152bp}{81.8182bp}
    \pgfpathqlineto{101.5152bp}{81.0606bp}
    \pgfusepathqstroke
  \end{pgfscope}
  \begin{pgfscope}
    \pgfsetlinewidth{0.5000bp}
    \definecolor{sc}{rgb}{0.0000,0.0000,0.0000}
    \pgfsetstrokecolor{sc}
    \pgfsetmiterjoin
    \pgfsetbuttcap
    \pgfpathqmoveto{93.9394bp}{81.8182bp}
    \pgfpathqlineto{93.9394bp}{81.0606bp}
    \pgfusepathqstroke
  \end{pgfscope}
  \begin{pgfscope}
    \pgfsetlinewidth{0.5000bp}
    \definecolor{sc}{rgb}{0.0000,0.0000,0.0000}
    \pgfsetstrokecolor{sc}
    \pgfsetmiterjoin
    \pgfsetbuttcap
    \pgfpathqmoveto{86.3636bp}{81.8182bp}
    \pgfpathqlineto{86.3636bp}{81.0606bp}
    \pgfusepathqstroke
  \end{pgfscope}
  \begin{pgfscope}
    \pgfsetlinewidth{0.5000bp}
    \definecolor{sc}{rgb}{0.0000,0.0000,0.0000}
    \pgfsetstrokecolor{sc}
    \pgfsetmiterjoin
    \pgfsetbuttcap
    \pgfpathqmoveto{78.7879bp}{81.8182bp}
    \pgfpathqlineto{78.7879bp}{81.0606bp}
    \pgfusepathqstroke
  \end{pgfscope}
  \begin{pgfscope}
    \pgfsetlinewidth{0.5000bp}
    \definecolor{sc}{rgb}{0.0000,0.0000,0.0000}
    \pgfsetstrokecolor{sc}
    \pgfsetmiterjoin
    \pgfsetbuttcap
    \pgfpathqmoveto{71.2121bp}{81.8182bp}
    \pgfpathqlineto{71.2121bp}{81.0606bp}
    \pgfusepathqstroke
  \end{pgfscope}
  \begin{pgfscope}
    \pgfsetlinewidth{0.5000bp}
    \definecolor{sc}{rgb}{0.0000,0.0000,0.0000}
    \pgfsetstrokecolor{sc}
    \pgfsetmiterjoin
    \pgfsetbuttcap
    \pgfpathqmoveto{63.6364bp}{81.8182bp}
    \pgfpathqlineto{63.6364bp}{81.0606bp}
    \pgfusepathqstroke
  \end{pgfscope}
  \begin{pgfscope}
    \pgfsetlinewidth{0.5000bp}
    \definecolor{sc}{rgb}{0.0000,0.0000,0.0000}
    \pgfsetstrokecolor{sc}
    \pgfsetmiterjoin
    \pgfsetbuttcap
    \pgfpathqmoveto{56.0606bp}{81.8182bp}
    \pgfpathqlineto{56.0606bp}{81.0606bp}
    \pgfusepathqstroke
  \end{pgfscope}
  \begin{pgfscope}
    \pgfsetlinewidth{0.5000bp}
    \definecolor{sc}{rgb}{0.0000,0.0000,0.0000}
    \pgfsetstrokecolor{sc}
    \pgfsetmiterjoin
    \pgfsetbuttcap
    \pgfpathqmoveto{48.4848bp}{81.8182bp}
    \pgfpathqlineto{48.4848bp}{81.0606bp}
    \pgfusepathqstroke
  \end{pgfscope}
  \begin{pgfscope}
    \pgfsetlinewidth{0.5000bp}
    \definecolor{sc}{rgb}{0.0000,0.0000,0.0000}
    \pgfsetstrokecolor{sc}
    \pgfsetmiterjoin
    \pgfsetbuttcap
    \pgfpathqmoveto{40.9091bp}{81.8182bp}
    \pgfpathqlineto{40.9091bp}{81.0606bp}
    \pgfusepathqstroke
  \end{pgfscope}
  \begin{pgfscope}
    \pgfsetlinewidth{0.5000bp}
    \definecolor{sc}{rgb}{0.0000,0.0000,0.0000}
    \pgfsetstrokecolor{sc}
    \pgfsetmiterjoin
    \pgfsetbuttcap
    \pgfpathqmoveto{33.3333bp}{81.8182bp}
    \pgfpathqlineto{33.3333bp}{81.0606bp}
    \pgfusepathqstroke
  \end{pgfscope}
  \begin{pgfscope}
    \pgfsetlinewidth{0.5000bp}
    \definecolor{sc}{rgb}{0.0000,0.0000,0.0000}
    \pgfsetstrokecolor{sc}
    \pgfsetmiterjoin
    \pgfsetbuttcap
    \pgfpathqmoveto{25.7576bp}{81.8182bp}
    \pgfpathqlineto{25.7576bp}{81.0606bp}
    \pgfusepathqstroke
  \end{pgfscope}
  \begin{pgfscope}
    \pgfsetlinewidth{0.5000bp}
    \definecolor{sc}{rgb}{0.0000,0.0000,0.0000}
    \pgfsetstrokecolor{sc}
    \pgfsetmiterjoin
    \pgfsetbuttcap
    \pgfpathqmoveto{18.1818bp}{81.8182bp}
    \pgfpathqlineto{18.1818bp}{81.0606bp}
    \pgfusepathqstroke
  \end{pgfscope}
  \begin{pgfscope}
    \pgfsetlinewidth{0.5000bp}
    \definecolor{sc}{rgb}{0.0000,0.0000,0.0000}
    \pgfsetstrokecolor{sc}
    \pgfsetmiterjoin
    \pgfsetbuttcap
    \pgfpathqmoveto{10.6061bp}{81.8182bp}
    \pgfpathqlineto{10.6061bp}{81.0606bp}
    \pgfusepathqstroke
  \end{pgfscope}
  \begin{pgfscope}
    \definecolor{fc}{rgb}{0.0000,0.0000,0.0000}
    \pgfsetfillcolor{fc}
    \pgftransformshift{\pgfqpoint{0.0000bp}{129.8485bp}}
    \pgftransformscale{0.1894}
    \pgftext[base,left]{$\mathbb{F}_A$}
  \end{pgfscope}
  \begin{pgfscope}
    \pgfsetlinewidth{0.5000bp}
    \definecolor{sc}{rgb}{0.0000,0.0000,0.0000}
    \pgfsetstrokecolor{sc}
    \pgfsetmiterjoin
    \pgfsetbuttcap
    \pgfpathqmoveto{3.0303bp}{130.3030bp}
    \pgfpathqlineto{2.7273bp}{130.3030bp}
    \pgfusepathqstroke
  \end{pgfscope}
  \begin{pgfscope}
    \pgfsetlinewidth{0.5000bp}
    \definecolor{sc}{rgb}{0.0000,0.0000,0.0000}
    \pgfsetstrokecolor{sc}
    \pgfsetmiterjoin
    \pgfsetbuttcap
    \pgfpathqmoveto{3.0303bp}{81.8182bp}
    \pgfpathqlineto{3.0303bp}{130.3030bp}
    \pgfusepathqstroke
  \end{pgfscope}
  \begin{pgfscope}
    \pgfsetlinewidth{0.5000bp}
    \definecolor{sc}{rgb}{0.0000,0.0000,0.0000}
    \pgfsetstrokecolor{sc}
    \pgfsetmiterjoin
    \pgfsetbuttcap
    \pgfpathqmoveto{3.0303bp}{81.8182bp}
    \pgfpathqlineto{200.0000bp}{81.8182bp}
    \pgfusepathqstroke
  \end{pgfscope}
\end{pgfpicture}

        \label{fig:ex:ca:hgma:ex:move-h}
    \caption{push-h-goal effects}\label{fig:ex:ca:hgma:ex:disconnected}
\end{figure}

\begin{figure}
    \centering
    \begin{pgfpicture}
  \pgfpathrectangle{\pgfpointorigin}{\pgfqpoint{200.0000bp}{200.0000bp}}
  \pgfusepath{use as bounding box}
  \begin{pgfscope}
    \definecolor{fc}{rgb}{0.0000,0.0000,0.0000}
    \pgfsetfillcolor{fc}
    \pgftransformshift{\pgfqpoint{3.4000bp}{80.1000bp}}
    \pgftransformscale{0.1250}
    \pgftext[base,left]{candidates}
  \end{pgfscope}
  \begin{pgfscope}
    \definecolor{fc}{rgb}{0.0000,0.0000,0.0000}
    \pgfsetfillcolor{fc}
    \pgfsetlinewidth{0.5000bp}
    \definecolor{sc}{rgb}{0.0000,0.0000,0.0000}
    \pgfsetstrokecolor{sc}
    \pgfsetmiterjoin
    \pgfsetbuttcap
    \pgfpathqmoveto{2.4000bp}{80.4000bp}
    \pgfpathqcurveto{2.4000bp}{80.6209bp}{2.2209bp}{80.8000bp}{2.0000bp}{80.8000bp}
    \pgfpathqcurveto{1.7791bp}{80.8000bp}{1.6000bp}{80.6209bp}{1.6000bp}{80.4000bp}
    \pgfpathqcurveto{1.6000bp}{80.1791bp}{1.7791bp}{80.0000bp}{2.0000bp}{80.0000bp}
    \pgfpathqcurveto{2.2209bp}{80.0000bp}{2.4000bp}{80.1791bp}{2.4000bp}{80.4000bp}
    \pgfpathclose
    \pgfusepathqfillstroke
  \end{pgfscope}
  \begin{pgfscope}
    \definecolor{fc}{rgb}{0.0000,0.0000,0.0000}
    \pgfsetfillcolor{fc}
    \pgftransformshift{\pgfqpoint{3.4000bp}{81.4000bp}}
    \pgftransformscale{0.1250}
    \pgftext[base,left]{negative unproven}
  \end{pgfscope}
  \begin{pgfscope}
    \definecolor{fc}{rgb}{1.0000,1.0000,0.0000}
    \pgfsetfillcolor{fc}
    \pgfsetlinewidth{0.5000bp}
    \definecolor{sc}{rgb}{1.0000,1.0000,0.0000}
    \pgfsetstrokecolor{sc}
    \pgfsetmiterjoin
    \pgfsetbuttcap
    \pgfpathqmoveto{2.4000bp}{81.7000bp}
    \pgfpathqcurveto{2.4000bp}{81.9209bp}{2.2209bp}{82.1000bp}{2.0000bp}{82.1000bp}
    \pgfpathqcurveto{1.7791bp}{82.1000bp}{1.6000bp}{81.9209bp}{1.6000bp}{81.7000bp}
    \pgfpathqcurveto{1.6000bp}{81.4791bp}{1.7791bp}{81.3000bp}{2.0000bp}{81.3000bp}
    \pgfpathqcurveto{2.2209bp}{81.3000bp}{2.4000bp}{81.4791bp}{2.4000bp}{81.7000bp}
    \pgfpathclose
    \pgfusepathqfillstroke
  \end{pgfscope}
  \begin{pgfscope}
    \definecolor{fc}{rgb}{0.0000,0.0000,0.0000}
    \pgfsetfillcolor{fc}
    \pgftransformshift{\pgfqpoint{3.4000bp}{82.7000bp}}
    \pgftransformscale{0.1250}
    \pgftext[base,left]{negative proven}
  \end{pgfscope}
  \begin{pgfscope}
    \definecolor{fc}{rgb}{0.0000,0.5020,0.0000}
    \pgfsetfillcolor{fc}
    \pgfsetlinewidth{0.5000bp}
    \definecolor{sc}{rgb}{0.0000,0.5020,0.0000}
    \pgfsetstrokecolor{sc}
    \pgfsetmiterjoin
    \pgfsetbuttcap
    \pgfpathqmoveto{2.4000bp}{83.0000bp}
    \pgfpathqcurveto{2.4000bp}{83.2209bp}{2.2209bp}{83.4000bp}{2.0000bp}{83.4000bp}
    \pgfpathqcurveto{1.7791bp}{83.4000bp}{1.6000bp}{83.2209bp}{1.6000bp}{83.0000bp}
    \pgfpathqcurveto{1.6000bp}{82.7791bp}{1.7791bp}{82.6000bp}{2.0000bp}{82.6000bp}
    \pgfpathqcurveto{2.2209bp}{82.6000bp}{2.4000bp}{82.7791bp}{2.4000bp}{83.0000bp}
    \pgfpathclose
    \pgfusepathqfillstroke
  \end{pgfscope}
  \begin{pgfscope}
    \definecolor{fc}{rgb}{0.0000,0.0000,0.0000}
    \pgfsetfillcolor{fc}
    \pgftransformshift{\pgfqpoint{3.4000bp}{84.0000bp}}
    \pgftransformscale{0.1250}
    \pgftext[base,left]{positive unproven}
  \end{pgfscope}
  \begin{pgfscope}
    \definecolor{fc}{rgb}{1.0000,0.0000,0.0000}
    \pgfsetfillcolor{fc}
    \pgfsetlinewidth{0.5000bp}
    \definecolor{sc}{rgb}{1.0000,0.0000,0.0000}
    \pgfsetstrokecolor{sc}
    \pgfsetmiterjoin
    \pgfsetbuttcap
    \pgfpathqmoveto{2.4000bp}{84.3000bp}
    \pgfpathqcurveto{2.4000bp}{84.5209bp}{2.2209bp}{84.7000bp}{2.0000bp}{84.7000bp}
    \pgfpathqcurveto{1.7791bp}{84.7000bp}{1.6000bp}{84.5209bp}{1.6000bp}{84.3000bp}
    \pgfpathqcurveto{1.6000bp}{84.0791bp}{1.7791bp}{83.9000bp}{2.0000bp}{83.9000bp}
    \pgfpathqcurveto{2.2209bp}{83.9000bp}{2.4000bp}{84.0791bp}{2.4000bp}{84.3000bp}
    \pgfpathclose
    \pgfusepathqfillstroke
  \end{pgfscope}
  \begin{pgfscope}
    \definecolor{fc}{rgb}{0.0000,0.0000,0.0000}
    \pgfsetfillcolor{fc}
    \pgftransformshift{\pgfqpoint{3.4000bp}{85.3000bp}}
    \pgftransformscale{0.1250}
    \pgftext[base,left]{positive proven}
  \end{pgfscope}
  \begin{pgfscope}
    \definecolor{fc}{rgb}{0.0000,0.0000,1.0000}
    \pgfsetfillcolor{fc}
    \pgfsetlinewidth{0.5000bp}
    \definecolor{sc}{rgb}{0.0000,0.0000,1.0000}
    \pgfsetstrokecolor{sc}
    \pgfsetmiterjoin
    \pgfsetbuttcap
    \pgfpathqmoveto{2.4000bp}{85.6000bp}
    \pgfpathqcurveto{2.4000bp}{85.8209bp}{2.2209bp}{86.0000bp}{2.0000bp}{86.0000bp}
    \pgfpathqcurveto{1.7791bp}{86.0000bp}{1.6000bp}{85.8209bp}{1.6000bp}{85.6000bp}
    \pgfpathqcurveto{1.6000bp}{85.3791bp}{1.7791bp}{85.2000bp}{2.0000bp}{85.2000bp}
    \pgfpathqcurveto{2.2209bp}{85.2000bp}{2.4000bp}{85.3791bp}{2.4000bp}{85.6000bp}
    \pgfpathclose
    \pgfusepathqfillstroke
  \end{pgfscope}
  \begin{pgfscope}
    \pgfsetlinewidth{0.5000bp}
    \definecolor{sc}{rgb}{1.0000,1.0000,0.0000}
    \pgfsetstrokecolor{sc}
    \pgfsetmiterjoin
    \pgfsetbuttcap
    \pgfpathqmoveto{2.0000bp}{120.0000bp}
    \pgfpathqlineto{3.0000bp}{120.0000bp}
    \pgfpathqlineto{4.0000bp}{120.0000bp}
    \pgfpathqlineto{5.0000bp}{120.0000bp}
    \pgfpathqlineto{6.0000bp}{120.0000bp}
    \pgfpathqlineto{7.0000bp}{120.0000bp}
    \pgfpathqlineto{8.0000bp}{120.0000bp}
    \pgfpathqlineto{9.0000bp}{120.0000bp}
    \pgfpathqlineto{10.0000bp}{120.0000bp}
    \pgfpathqlineto{11.0000bp}{120.0000bp}
    \pgfpathqlineto{12.0000bp}{120.0000bp}
    \pgfpathqlineto{13.0000bp}{120.0000bp}
    \pgfpathqlineto{14.0000bp}{120.0000bp}
    \pgfpathqlineto{15.0000bp}{120.0000bp}
    \pgfpathqlineto{16.0000bp}{120.0000bp}
    \pgfpathqlineto{17.0000bp}{120.0000bp}
    \pgfpathqlineto{18.0000bp}{120.0000bp}
    \pgfpathqlineto{19.0000bp}{120.0000bp}
    \pgfpathqlineto{20.0000bp}{120.0000bp}
    \pgfpathqlineto{21.0000bp}{120.0000bp}
    \pgfpathqlineto{22.0000bp}{120.0000bp}
    \pgfpathqlineto{23.0000bp}{120.0000bp}
    \pgfpathqlineto{24.0000bp}{120.0000bp}
    \pgfpathqlineto{25.0000bp}{120.0000bp}
    \pgfpathqlineto{26.0000bp}{120.0000bp}
    \pgfpathqlineto{27.0000bp}{120.0000bp}
    \pgfpathqlineto{28.0000bp}{120.0000bp}
    \pgfpathqlineto{29.0000bp}{120.0000bp}
    \pgfpathqlineto{30.0000bp}{120.0000bp}
    \pgfpathqlineto{31.0000bp}{120.0000bp}
    \pgfpathqlineto{32.0000bp}{120.0000bp}
    \pgfpathqlineto{33.0000bp}{120.0000bp}
    \pgfpathqlineto{34.0000bp}{120.0000bp}
    \pgfpathqlineto{35.0000bp}{120.0000bp}
    \pgfpathqlineto{36.0000bp}{120.0000bp}
    \pgfpathqlineto{37.0000bp}{120.0000bp}
    \pgfpathqlineto{38.0000bp}{120.0000bp}
    \pgfpathqlineto{39.0000bp}{120.0000bp}
    \pgfpathqlineto{40.0000bp}{120.0000bp}
    \pgfpathqlineto{41.0000bp}{120.0000bp}
    \pgfpathqlineto{42.0000bp}{120.0000bp}
    \pgfpathqlineto{43.0000bp}{120.0000bp}
    \pgfpathqlineto{44.0000bp}{120.0000bp}
    \pgfpathqlineto{45.0000bp}{120.0000bp}
    \pgfpathqlineto{46.0000bp}{120.0000bp}
    \pgfpathqlineto{47.0000bp}{120.0000bp}
    \pgfpathqlineto{48.0000bp}{120.0000bp}
    \pgfpathqlineto{49.0000bp}{120.0000bp}
    \pgfpathqlineto{50.0000bp}{120.0000bp}
    \pgfpathqlineto{51.0000bp}{120.0000bp}
    \pgfpathqlineto{52.0000bp}{120.0000bp}
    \pgfpathqlineto{53.0000bp}{120.0000bp}
    \pgfpathqlineto{54.0000bp}{120.0000bp}
    \pgfpathqlineto{55.0000bp}{120.0000bp}
    \pgfpathqlineto{56.0000bp}{120.0000bp}
    \pgfpathqlineto{57.0000bp}{120.0000bp}
    \pgfpathqlineto{58.0000bp}{120.0000bp}
    \pgfpathqlineto{59.0000bp}{120.0000bp}
    \pgfpathqlineto{60.0000bp}{120.0000bp}
    \pgfpathqlineto{61.0000bp}{120.0000bp}
    \pgfpathqlineto{62.0000bp}{120.0000bp}
    \pgfpathqlineto{63.0000bp}{120.0000bp}
    \pgfpathqlineto{64.0000bp}{120.0000bp}
    \pgfpathqlineto{65.0000bp}{120.0000bp}
    \pgfpathqlineto{66.0000bp}{120.0000bp}
    \pgfpathqlineto{67.0000bp}{120.0000bp}
    \pgfpathqlineto{68.0000bp}{120.0000bp}
    \pgfpathqlineto{69.0000bp}{120.0000bp}
    \pgfpathqlineto{70.0000bp}{120.0000bp}
    \pgfpathqlineto{71.0000bp}{120.0000bp}
    \pgfpathqlineto{72.0000bp}{120.0000bp}
    \pgfpathqlineto{73.0000bp}{120.0000bp}
    \pgfpathqlineto{74.0000bp}{120.0000bp}
    \pgfpathqlineto{75.0000bp}{120.0000bp}
    \pgfpathqlineto{76.0000bp}{120.0000bp}
    \pgfpathqlineto{77.0000bp}{120.0000bp}
    \pgfpathqlineto{78.0000bp}{120.0000bp}
    \pgfpathqlineto{79.0000bp}{120.0000bp}
    \pgfpathqlineto{80.0000bp}{120.0000bp}
    \pgfpathqlineto{81.0000bp}{120.0000bp}
    \pgfpathqlineto{82.0000bp}{120.0000bp}
    \pgfpathqlineto{83.0000bp}{120.0000bp}
    \pgfpathqlineto{84.0000bp}{120.0000bp}
    \pgfpathqlineto{85.0000bp}{120.0000bp}
    \pgfpathqlineto{86.0000bp}{120.0000bp}
    \pgfpathqlineto{87.0000bp}{120.0000bp}
    \pgfpathqlineto{88.0000bp}{120.0000bp}
    \pgfpathqlineto{89.0000bp}{120.0000bp}
    \pgfpathqlineto{90.0000bp}{120.0000bp}
    \pgfpathqlineto{91.0000bp}{120.0000bp}
    \pgfpathqlineto{92.0000bp}{120.0000bp}
    \pgfpathqlineto{93.0000bp}{120.0000bp}
    \pgfpathqlineto{94.0000bp}{120.0000bp}
    \pgfpathqlineto{95.0000bp}{120.0000bp}
    \pgfpathqlineto{96.0000bp}{120.0000bp}
    \pgfpathqlineto{97.0000bp}{120.0000bp}
    \pgfpathqlineto{98.0000bp}{120.0000bp}
    \pgfpathqlineto{99.0000bp}{120.0000bp}
    \pgfpathqlineto{100.0000bp}{120.0000bp}
    \pgfpathqlineto{101.0000bp}{120.0000bp}
    \pgfpathqlineto{102.0000bp}{120.0000bp}
    \pgfpathqlineto{103.0000bp}{120.0000bp}
    \pgfpathqlineto{104.0000bp}{120.0000bp}
    \pgfpathqlineto{105.0000bp}{120.0000bp}
    \pgfpathqlineto{106.0000bp}{120.0000bp}
    \pgfpathqlineto{107.0000bp}{120.0000bp}
    \pgfpathqlineto{108.0000bp}{120.0000bp}
    \pgfpathqlineto{109.0000bp}{120.0000bp}
    \pgfpathqlineto{110.0000bp}{120.0000bp}
    \pgfpathqlineto{111.0000bp}{120.0000bp}
    \pgfpathqlineto{112.0000bp}{120.0000bp}
    \pgfpathqlineto{113.0000bp}{120.0000bp}
    \pgfpathqlineto{114.0000bp}{120.0000bp}
    \pgfpathqlineto{115.0000bp}{120.0000bp}
    \pgfpathqlineto{116.0000bp}{120.0000bp}
    \pgfpathqlineto{117.0000bp}{120.0000bp}
    \pgfpathqlineto{118.0000bp}{120.0000bp}
    \pgfpathqlineto{119.0000bp}{120.0000bp}
    \pgfpathqlineto{120.0000bp}{120.0000bp}
    \pgfpathqlineto{121.0000bp}{120.0000bp}
    \pgfpathqlineto{122.0000bp}{120.0000bp}
    \pgfpathqlineto{123.0000bp}{120.0000bp}
    \pgfpathqlineto{124.0000bp}{120.0000bp}
    \pgfpathqlineto{125.0000bp}{120.0000bp}
    \pgfpathqlineto{126.0000bp}{120.0000bp}
    \pgfpathqlineto{127.0000bp}{120.0000bp}
    \pgfpathqlineto{128.0000bp}{120.0000bp}
    \pgfpathqlineto{129.0000bp}{120.0000bp}
    \pgfpathqlineto{130.0000bp}{120.0000bp}
    \pgfpathqlineto{131.0000bp}{120.0000bp}
    \pgfpathqlineto{132.0000bp}{120.0000bp}
    \pgfpathqlineto{133.0000bp}{120.0000bp}
    \pgfpathqlineto{134.0000bp}{120.0000bp}
    \pgfpathqlineto{135.0000bp}{120.0000bp}
    \pgfpathqlineto{136.0000bp}{120.0000bp}
    \pgfpathqlineto{137.0000bp}{120.0000bp}
    \pgfpathqlineto{138.0000bp}{120.0000bp}
    \pgfpathqlineto{139.0000bp}{120.0000bp}
    \pgfpathqlineto{140.0000bp}{120.0000bp}
    \pgfpathqlineto{141.0000bp}{120.0000bp}
    \pgfpathqlineto{142.0000bp}{120.0000bp}
    \pgfpathqlineto{143.0000bp}{120.0000bp}
    \pgfpathqlineto{144.0000bp}{120.0000bp}
    \pgfpathqlineto{145.0000bp}{120.0000bp}
    \pgfpathqlineto{146.0000bp}{120.0000bp}
    \pgfpathqlineto{147.0000bp}{120.0000bp}
    \pgfpathqlineto{148.0000bp}{120.0000bp}
    \pgfpathqlineto{149.0000bp}{120.0000bp}
    \pgfpathqlineto{150.0000bp}{120.0000bp}
    \pgfpathqlineto{151.0000bp}{120.0000bp}
    \pgfpathqlineto{152.0000bp}{120.0000bp}
    \pgfpathqlineto{153.0000bp}{120.0000bp}
    \pgfpathqlineto{154.0000bp}{120.0000bp}
    \pgfpathqlineto{155.0000bp}{120.0000bp}
    \pgfpathqlineto{156.0000bp}{120.0000bp}
    \pgfpathqlineto{157.0000bp}{120.0000bp}
    \pgfpathqlineto{158.0000bp}{120.0000bp}
    \pgfpathqlineto{159.0000bp}{120.0000bp}
    \pgfpathqlineto{160.0000bp}{120.0000bp}
    \pgfpathqlineto{161.0000bp}{120.0000bp}
    \pgfpathqlineto{162.0000bp}{120.0000bp}
    \pgfpathqlineto{163.0000bp}{120.0000bp}
    \pgfpathqlineto{164.0000bp}{120.0000bp}
    \pgfpathqlineto{165.0000bp}{120.0000bp}
    \pgfpathqlineto{166.0000bp}{120.0000bp}
    \pgfpathqlineto{167.0000bp}{120.0000bp}
    \pgfpathqlineto{168.0000bp}{120.0000bp}
    \pgfpathqlineto{169.0000bp}{120.0000bp}
    \pgfpathqlineto{170.0000bp}{120.0000bp}
    \pgfpathqlineto{171.0000bp}{120.0000bp}
    \pgfpathqlineto{172.0000bp}{120.0000bp}
    \pgfpathqlineto{173.0000bp}{120.0000bp}
    \pgfpathqlineto{174.0000bp}{120.0000bp}
    \pgfpathqlineto{175.0000bp}{120.0000bp}
    \pgfpathqlineto{176.0000bp}{120.0000bp}
    \pgfpathqlineto{177.0000bp}{120.0000bp}
    \pgfpathqlineto{178.0000bp}{120.0000bp}
    \pgfpathqlineto{179.0000bp}{120.0000bp}
    \pgfpathqlineto{180.0000bp}{120.0000bp}
    \pgfpathqlineto{181.0000bp}{120.0000bp}
    \pgfpathqlineto{182.0000bp}{120.0000bp}
    \pgfpathqlineto{183.0000bp}{120.0000bp}
    \pgfpathqlineto{184.0000bp}{120.0000bp}
    \pgfpathqlineto{185.0000bp}{120.0000bp}
    \pgfpathqlineto{186.0000bp}{120.0000bp}
    \pgfpathqlineto{187.0000bp}{120.0000bp}
    \pgfpathqlineto{188.0000bp}{120.0000bp}
    \pgfpathqlineto{189.0000bp}{120.0000bp}
    \pgfpathqlineto{190.0000bp}{120.0000bp}
    \pgfpathqlineto{191.0000bp}{120.0000bp}
    \pgfpathqlineto{192.0000bp}{88.0000bp}
    \pgfpathqlineto{193.0000bp}{88.0000bp}
    \pgfpathqlineto{194.0000bp}{88.0000bp}
    \pgfpathqlineto{195.0000bp}{88.0000bp}
    \pgfpathqlineto{196.0000bp}{88.0000bp}
    \pgfpathqlineto{197.0000bp}{88.0000bp}
    \pgfpathqlineto{198.0000bp}{88.0000bp}
    \pgfpathqlineto{199.0000bp}{88.0000bp}
    \pgfusepathqstroke
  \end{pgfscope}
  \begin{pgfscope}
    \pgfsetlinewidth{0.5000bp}
    \definecolor{sc}{rgb}{0.0000,0.5020,0.0000}
    \pgfsetstrokecolor{sc}
    \pgfsetmiterjoin
    \pgfsetbuttcap
    \pgfpathqmoveto{2.0000bp}{88.0000bp}
    \pgfpathqlineto{3.0000bp}{88.0000bp}
    \pgfpathqlineto{4.0000bp}{88.0000bp}
    \pgfpathqlineto{5.0000bp}{88.0000bp}
    \pgfpathqlineto{6.0000bp}{88.0000bp}
    \pgfpathqlineto{7.0000bp}{88.0000bp}
    \pgfpathqlineto{8.0000bp}{88.0000bp}
    \pgfpathqlineto{9.0000bp}{88.0000bp}
    \pgfpathqlineto{10.0000bp}{88.0000bp}
    \pgfpathqlineto{11.0000bp}{88.0000bp}
    \pgfpathqlineto{12.0000bp}{88.0000bp}
    \pgfpathqlineto{13.0000bp}{88.0000bp}
    \pgfpathqlineto{14.0000bp}{88.0000bp}
    \pgfpathqlineto{15.0000bp}{88.0000bp}
    \pgfpathqlineto{16.0000bp}{88.0000bp}
    \pgfpathqlineto{17.0000bp}{88.0000bp}
    \pgfpathqlineto{18.0000bp}{88.0000bp}
    \pgfpathqlineto{19.0000bp}{88.0000bp}
    \pgfpathqlineto{20.0000bp}{88.0000bp}
    \pgfpathqlineto{21.0000bp}{88.0000bp}
    \pgfpathqlineto{22.0000bp}{88.0000bp}
    \pgfpathqlineto{23.0000bp}{88.0000bp}
    \pgfpathqlineto{24.0000bp}{88.0000bp}
    \pgfpathqlineto{25.0000bp}{88.0000bp}
    \pgfpathqlineto{26.0000bp}{88.0000bp}
    \pgfpathqlineto{27.0000bp}{88.0000bp}
    \pgfpathqlineto{28.0000bp}{88.0000bp}
    \pgfpathqlineto{29.0000bp}{88.0000bp}
    \pgfpathqlineto{30.0000bp}{88.0000bp}
    \pgfpathqlineto{31.0000bp}{88.0000bp}
    \pgfpathqlineto{32.0000bp}{88.0000bp}
    \pgfpathqlineto{33.0000bp}{88.0000bp}
    \pgfpathqlineto{34.0000bp}{88.0000bp}
    \pgfpathqlineto{35.0000bp}{88.0000bp}
    \pgfpathqlineto{36.0000bp}{88.0000bp}
    \pgfpathqlineto{37.0000bp}{88.0000bp}
    \pgfpathqlineto{38.0000bp}{88.0000bp}
    \pgfpathqlineto{39.0000bp}{88.0000bp}
    \pgfpathqlineto{40.0000bp}{88.0000bp}
    \pgfpathqlineto{41.0000bp}{88.0000bp}
    \pgfpathqlineto{42.0000bp}{88.0000bp}
    \pgfpathqlineto{43.0000bp}{88.0000bp}
    \pgfpathqlineto{44.0000bp}{88.0000bp}
    \pgfpathqlineto{45.0000bp}{88.0000bp}
    \pgfpathqlineto{46.0000bp}{88.0000bp}
    \pgfpathqlineto{47.0000bp}{88.0000bp}
    \pgfpathqlineto{48.0000bp}{88.0000bp}
    \pgfpathqlineto{49.0000bp}{88.0000bp}
    \pgfpathqlineto{50.0000bp}{88.0000bp}
    \pgfpathqlineto{51.0000bp}{88.0000bp}
    \pgfpathqlineto{52.0000bp}{88.0000bp}
    \pgfpathqlineto{53.0000bp}{88.0000bp}
    \pgfpathqlineto{54.0000bp}{88.0000bp}
    \pgfpathqlineto{55.0000bp}{88.0000bp}
    \pgfpathqlineto{56.0000bp}{88.0000bp}
    \pgfpathqlineto{57.0000bp}{88.0000bp}
    \pgfpathqlineto{58.0000bp}{88.0000bp}
    \pgfpathqlineto{59.0000bp}{88.0000bp}
    \pgfpathqlineto{60.0000bp}{88.0000bp}
    \pgfpathqlineto{61.0000bp}{88.0000bp}
    \pgfpathqlineto{62.0000bp}{88.0000bp}
    \pgfpathqlineto{63.0000bp}{88.0000bp}
    \pgfpathqlineto{64.0000bp}{88.0000bp}
    \pgfpathqlineto{65.0000bp}{88.0000bp}
    \pgfpathqlineto{66.0000bp}{88.0000bp}
    \pgfpathqlineto{67.0000bp}{88.0000bp}
    \pgfpathqlineto{68.0000bp}{88.0000bp}
    \pgfpathqlineto{69.0000bp}{88.0000bp}
    \pgfpathqlineto{70.0000bp}{88.0000bp}
    \pgfpathqlineto{71.0000bp}{88.0000bp}
    \pgfpathqlineto{72.0000bp}{88.0000bp}
    \pgfpathqlineto{73.0000bp}{88.0000bp}
    \pgfpathqlineto{74.0000bp}{88.0000bp}
    \pgfpathqlineto{75.0000bp}{88.0000bp}
    \pgfpathqlineto{76.0000bp}{88.0000bp}
    \pgfpathqlineto{77.0000bp}{88.0000bp}
    \pgfpathqlineto{78.0000bp}{88.0000bp}
    \pgfpathqlineto{79.0000bp}{88.0000bp}
    \pgfpathqlineto{80.0000bp}{88.0000bp}
    \pgfpathqlineto{81.0000bp}{88.0000bp}
    \pgfpathqlineto{82.0000bp}{88.0000bp}
    \pgfpathqlineto{83.0000bp}{88.0000bp}
    \pgfpathqlineto{84.0000bp}{88.0000bp}
    \pgfpathqlineto{85.0000bp}{88.0000bp}
    \pgfpathqlineto{86.0000bp}{88.0000bp}
    \pgfpathqlineto{87.0000bp}{88.0000bp}
    \pgfpathqlineto{88.0000bp}{88.0000bp}
    \pgfpathqlineto{89.0000bp}{88.0000bp}
    \pgfpathqlineto{90.0000bp}{88.0000bp}
    \pgfpathqlineto{91.0000bp}{88.0000bp}
    \pgfpathqlineto{92.0000bp}{88.0000bp}
    \pgfpathqlineto{93.0000bp}{88.0000bp}
    \pgfpathqlineto{94.0000bp}{88.0000bp}
    \pgfpathqlineto{95.0000bp}{88.0000bp}
    \pgfpathqlineto{96.0000bp}{88.0000bp}
    \pgfpathqlineto{97.0000bp}{88.0000bp}
    \pgfpathqlineto{98.0000bp}{88.0000bp}
    \pgfpathqlineto{99.0000bp}{88.0000bp}
    \pgfpathqlineto{100.0000bp}{88.0000bp}
    \pgfpathqlineto{101.0000bp}{88.0000bp}
    \pgfpathqlineto{102.0000bp}{88.0000bp}
    \pgfpathqlineto{103.0000bp}{88.0000bp}
    \pgfpathqlineto{104.0000bp}{88.0000bp}
    \pgfpathqlineto{105.0000bp}{88.0000bp}
    \pgfpathqlineto{106.0000bp}{88.0000bp}
    \pgfpathqlineto{107.0000bp}{88.0000bp}
    \pgfpathqlineto{108.0000bp}{88.0000bp}
    \pgfpathqlineto{109.0000bp}{88.0000bp}
    \pgfpathqlineto{110.0000bp}{88.0000bp}
    \pgfpathqlineto{111.0000bp}{88.0000bp}
    \pgfpathqlineto{112.0000bp}{88.0000bp}
    \pgfpathqlineto{113.0000bp}{88.0000bp}
    \pgfpathqlineto{114.0000bp}{88.0000bp}
    \pgfpathqlineto{115.0000bp}{88.0000bp}
    \pgfpathqlineto{116.0000bp}{88.0000bp}
    \pgfpathqlineto{117.0000bp}{88.0000bp}
    \pgfpathqlineto{118.0000bp}{88.0000bp}
    \pgfpathqlineto{119.0000bp}{88.0000bp}
    \pgfpathqlineto{120.0000bp}{88.0000bp}
    \pgfpathqlineto{121.0000bp}{88.0000bp}
    \pgfpathqlineto{122.0000bp}{88.0000bp}
    \pgfpathqlineto{123.0000bp}{88.0000bp}
    \pgfpathqlineto{124.0000bp}{88.0000bp}
    \pgfpathqlineto{125.0000bp}{88.0000bp}
    \pgfpathqlineto{126.0000bp}{88.0000bp}
    \pgfpathqlineto{127.0000bp}{88.0000bp}
    \pgfpathqlineto{128.0000bp}{88.0000bp}
    \pgfpathqlineto{129.0000bp}{88.0000bp}
    \pgfpathqlineto{130.0000bp}{88.0000bp}
    \pgfpathqlineto{131.0000bp}{88.0000bp}
    \pgfpathqlineto{132.0000bp}{88.0000bp}
    \pgfpathqlineto{133.0000bp}{88.0000bp}
    \pgfpathqlineto{134.0000bp}{88.0000bp}
    \pgfpathqlineto{135.0000bp}{88.0000bp}
    \pgfpathqlineto{136.0000bp}{88.0000bp}
    \pgfpathqlineto{137.0000bp}{88.0000bp}
    \pgfpathqlineto{138.0000bp}{88.0000bp}
    \pgfpathqlineto{139.0000bp}{88.0000bp}
    \pgfpathqlineto{140.0000bp}{88.0000bp}
    \pgfpathqlineto{141.0000bp}{88.0000bp}
    \pgfpathqlineto{142.0000bp}{88.0000bp}
    \pgfpathqlineto{143.0000bp}{88.0000bp}
    \pgfpathqlineto{144.0000bp}{88.0000bp}
    \pgfpathqlineto{145.0000bp}{88.0000bp}
    \pgfpathqlineto{146.0000bp}{88.0000bp}
    \pgfpathqlineto{147.0000bp}{88.0000bp}
    \pgfpathqlineto{148.0000bp}{88.0000bp}
    \pgfpathqlineto{149.0000bp}{88.0000bp}
    \pgfpathqlineto{150.0000bp}{88.0000bp}
    \pgfpathqlineto{151.0000bp}{88.0000bp}
    \pgfpathqlineto{152.0000bp}{88.0000bp}
    \pgfpathqlineto{153.0000bp}{88.0000bp}
    \pgfpathqlineto{154.0000bp}{88.0000bp}
    \pgfpathqlineto{155.0000bp}{88.0000bp}
    \pgfpathqlineto{156.0000bp}{88.0000bp}
    \pgfpathqlineto{157.0000bp}{88.0000bp}
    \pgfpathqlineto{158.0000bp}{88.0000bp}
    \pgfpathqlineto{159.0000bp}{88.0000bp}
    \pgfpathqlineto{160.0000bp}{88.0000bp}
    \pgfpathqlineto{161.0000bp}{88.0000bp}
    \pgfpathqlineto{162.0000bp}{88.0000bp}
    \pgfpathqlineto{163.0000bp}{88.0000bp}
    \pgfpathqlineto{164.0000bp}{88.0000bp}
    \pgfpathqlineto{165.0000bp}{88.0000bp}
    \pgfpathqlineto{166.0000bp}{88.0000bp}
    \pgfpathqlineto{167.0000bp}{88.0000bp}
    \pgfpathqlineto{168.0000bp}{88.0000bp}
    \pgfpathqlineto{169.0000bp}{88.0000bp}
    \pgfpathqlineto{170.0000bp}{88.0000bp}
    \pgfpathqlineto{171.0000bp}{88.0000bp}
    \pgfpathqlineto{172.0000bp}{88.0000bp}
    \pgfpathqlineto{173.0000bp}{88.0000bp}
    \pgfpathqlineto{174.0000bp}{88.0000bp}
    \pgfpathqlineto{175.0000bp}{88.0000bp}
    \pgfpathqlineto{176.0000bp}{88.0000bp}
    \pgfpathqlineto{177.0000bp}{88.0000bp}
    \pgfpathqlineto{178.0000bp}{88.0000bp}
    \pgfpathqlineto{179.0000bp}{88.0000bp}
    \pgfpathqlineto{180.0000bp}{88.0000bp}
    \pgfpathqlineto{181.0000bp}{88.0000bp}
    \pgfpathqlineto{182.0000bp}{88.0000bp}
    \pgfpathqlineto{183.0000bp}{88.0000bp}
    \pgfpathqlineto{184.0000bp}{88.0000bp}
    \pgfpathqlineto{185.0000bp}{88.0000bp}
    \pgfpathqlineto{186.0000bp}{88.0000bp}
    \pgfpathqlineto{187.0000bp}{88.0000bp}
    \pgfpathqlineto{188.0000bp}{88.0000bp}
    \pgfpathqlineto{189.0000bp}{88.0000bp}
    \pgfpathqlineto{190.0000bp}{88.0000bp}
    \pgfpathqlineto{191.0000bp}{88.0000bp}
    \pgfpathqlineto{192.0000bp}{89.5000bp}
    \pgfpathqlineto{193.0000bp}{89.5000bp}
    \pgfpathqlineto{194.0000bp}{89.5000bp}
    \pgfpathqlineto{195.0000bp}{89.5000bp}
    \pgfpathqlineto{196.0000bp}{89.5000bp}
    \pgfpathqlineto{197.0000bp}{89.5000bp}
    \pgfpathqlineto{198.0000bp}{89.5000bp}
    \pgfpathqlineto{199.0000bp}{89.5000bp}
    \pgfusepathqstroke
  \end{pgfscope}
  \begin{pgfscope}
    \pgfsetlinewidth{0.5000bp}
    \definecolor{sc}{rgb}{1.0000,0.0000,0.0000}
    \pgfsetstrokecolor{sc}
    \pgfsetmiterjoin
    \pgfsetbuttcap
    \pgfpathqmoveto{2.0000bp}{120.0000bp}
    \pgfpathqlineto{3.0000bp}{120.0000bp}
    \pgfpathqlineto{4.0000bp}{120.0000bp}
    \pgfpathqlineto{5.0000bp}{120.0000bp}
    \pgfpathqlineto{6.0000bp}{120.0000bp}
    \pgfpathqlineto{7.0000bp}{120.0000bp}
    \pgfpathqlineto{8.0000bp}{120.0000bp}
    \pgfpathqlineto{9.0000bp}{120.0000bp}
    \pgfpathqlineto{10.0000bp}{120.0000bp}
    \pgfpathqlineto{11.0000bp}{120.0000bp}
    \pgfpathqlineto{12.0000bp}{120.0000bp}
    \pgfpathqlineto{13.0000bp}{120.0000bp}
    \pgfpathqlineto{14.0000bp}{120.0000bp}
    \pgfpathqlineto{15.0000bp}{120.0000bp}
    \pgfpathqlineto{16.0000bp}{120.0000bp}
    \pgfpathqlineto{17.0000bp}{120.0000bp}
    \pgfpathqlineto{18.0000bp}{120.0000bp}
    \pgfpathqlineto{19.0000bp}{120.0000bp}
    \pgfpathqlineto{20.0000bp}{120.0000bp}
    \pgfpathqlineto{21.0000bp}{120.0000bp}
    \pgfpathqlineto{22.0000bp}{120.0000bp}
    \pgfpathqlineto{23.0000bp}{120.0000bp}
    \pgfpathqlineto{24.0000bp}{120.0000bp}
    \pgfpathqlineto{25.0000bp}{120.0000bp}
    \pgfpathqlineto{26.0000bp}{120.0000bp}
    \pgfpathqlineto{27.0000bp}{120.0000bp}
    \pgfpathqlineto{28.0000bp}{120.0000bp}
    \pgfpathqlineto{29.0000bp}{120.0000bp}
    \pgfpathqlineto{30.0000bp}{120.0000bp}
    \pgfpathqlineto{31.0000bp}{120.0000bp}
    \pgfpathqlineto{32.0000bp}{120.0000bp}
    \pgfpathqlineto{33.0000bp}{120.0000bp}
    \pgfpathqlineto{34.0000bp}{120.0000bp}
    \pgfpathqlineto{35.0000bp}{120.0000bp}
    \pgfpathqlineto{36.0000bp}{120.0000bp}
    \pgfpathqlineto{37.0000bp}{120.0000bp}
    \pgfpathqlineto{38.0000bp}{120.0000bp}
    \pgfpathqlineto{39.0000bp}{120.0000bp}
    \pgfpathqlineto{40.0000bp}{120.0000bp}
    \pgfpathqlineto{41.0000bp}{120.0000bp}
    \pgfpathqlineto{42.0000bp}{120.0000bp}
    \pgfpathqlineto{43.0000bp}{120.0000bp}
    \pgfpathqlineto{44.0000bp}{120.0000bp}
    \pgfpathqlineto{45.0000bp}{120.0000bp}
    \pgfpathqlineto{46.0000bp}{120.0000bp}
    \pgfpathqlineto{47.0000bp}{120.0000bp}
    \pgfpathqlineto{48.0000bp}{120.0000bp}
    \pgfpathqlineto{49.0000bp}{120.0000bp}
    \pgfpathqlineto{50.0000bp}{120.0000bp}
    \pgfpathqlineto{51.0000bp}{120.0000bp}
    \pgfpathqlineto{52.0000bp}{120.0000bp}
    \pgfpathqlineto{53.0000bp}{120.0000bp}
    \pgfpathqlineto{54.0000bp}{120.0000bp}
    \pgfpathqlineto{55.0000bp}{120.0000bp}
    \pgfpathqlineto{56.0000bp}{120.0000bp}
    \pgfpathqlineto{57.0000bp}{120.0000bp}
    \pgfpathqlineto{58.0000bp}{120.0000bp}
    \pgfpathqlineto{59.0000bp}{120.0000bp}
    \pgfpathqlineto{60.0000bp}{120.0000bp}
    \pgfpathqlineto{61.0000bp}{120.0000bp}
    \pgfpathqlineto{62.0000bp}{120.0000bp}
    \pgfpathqlineto{63.0000bp}{120.0000bp}
    \pgfpathqlineto{64.0000bp}{120.0000bp}
    \pgfpathqlineto{65.0000bp}{120.0000bp}
    \pgfpathqlineto{66.0000bp}{120.0000bp}
    \pgfpathqlineto{67.0000bp}{120.0000bp}
    \pgfpathqlineto{68.0000bp}{120.0000bp}
    \pgfpathqlineto{69.0000bp}{120.0000bp}
    \pgfpathqlineto{70.0000bp}{120.0000bp}
    \pgfpathqlineto{71.0000bp}{120.0000bp}
    \pgfpathqlineto{72.0000bp}{120.0000bp}
    \pgfpathqlineto{73.0000bp}{120.0000bp}
    \pgfpathqlineto{74.0000bp}{120.0000bp}
    \pgfpathqlineto{75.0000bp}{120.0000bp}
    \pgfpathqlineto{76.0000bp}{120.0000bp}
    \pgfpathqlineto{77.0000bp}{120.0000bp}
    \pgfpathqlineto{78.0000bp}{120.0000bp}
    \pgfpathqlineto{79.0000bp}{120.0000bp}
    \pgfpathqlineto{80.0000bp}{120.0000bp}
    \pgfpathqlineto{81.0000bp}{120.0000bp}
    \pgfpathqlineto{82.0000bp}{120.0000bp}
    \pgfpathqlineto{83.0000bp}{120.0000bp}
    \pgfpathqlineto{84.0000bp}{120.0000bp}
    \pgfpathqlineto{85.0000bp}{120.0000bp}
    \pgfpathqlineto{86.0000bp}{120.0000bp}
    \pgfpathqlineto{87.0000bp}{120.0000bp}
    \pgfpathqlineto{88.0000bp}{120.0000bp}
    \pgfpathqlineto{89.0000bp}{120.0000bp}
    \pgfpathqlineto{90.0000bp}{120.0000bp}
    \pgfpathqlineto{91.0000bp}{120.0000bp}
    \pgfpathqlineto{92.0000bp}{120.0000bp}
    \pgfpathqlineto{93.0000bp}{120.0000bp}
    \pgfpathqlineto{94.0000bp}{120.0000bp}
    \pgfpathqlineto{95.0000bp}{120.0000bp}
    \pgfpathqlineto{96.0000bp}{120.0000bp}
    \pgfpathqlineto{97.0000bp}{120.0000bp}
    \pgfpathqlineto{98.0000bp}{120.0000bp}
    \pgfpathqlineto{99.0000bp}{120.0000bp}
    \pgfpathqlineto{100.0000bp}{120.0000bp}
    \pgfpathqlineto{101.0000bp}{120.0000bp}
    \pgfpathqlineto{102.0000bp}{120.0000bp}
    \pgfpathqlineto{103.0000bp}{120.0000bp}
    \pgfpathqlineto{104.0000bp}{120.0000bp}
    \pgfpathqlineto{105.0000bp}{120.0000bp}
    \pgfpathqlineto{106.0000bp}{120.0000bp}
    \pgfpathqlineto{107.0000bp}{120.0000bp}
    \pgfpathqlineto{108.0000bp}{120.0000bp}
    \pgfpathqlineto{109.0000bp}{120.0000bp}
    \pgfpathqlineto{110.0000bp}{120.0000bp}
    \pgfpathqlineto{111.0000bp}{120.0000bp}
    \pgfpathqlineto{112.0000bp}{120.0000bp}
    \pgfpathqlineto{113.0000bp}{120.0000bp}
    \pgfpathqlineto{114.0000bp}{120.0000bp}
    \pgfpathqlineto{115.0000bp}{120.0000bp}
    \pgfpathqlineto{116.0000bp}{120.0000bp}
    \pgfpathqlineto{117.0000bp}{120.0000bp}
    \pgfpathqlineto{118.0000bp}{120.0000bp}
    \pgfpathqlineto{119.0000bp}{120.0000bp}
    \pgfpathqlineto{120.0000bp}{120.0000bp}
    \pgfpathqlineto{121.0000bp}{120.0000bp}
    \pgfpathqlineto{122.0000bp}{120.0000bp}
    \pgfpathqlineto{123.0000bp}{120.0000bp}
    \pgfpathqlineto{124.0000bp}{120.0000bp}
    \pgfpathqlineto{125.0000bp}{120.0000bp}
    \pgfpathqlineto{126.0000bp}{120.0000bp}
    \pgfpathqlineto{127.0000bp}{120.0000bp}
    \pgfpathqlineto{128.0000bp}{120.0000bp}
    \pgfpathqlineto{129.0000bp}{120.0000bp}
    \pgfpathqlineto{130.0000bp}{120.0000bp}
    \pgfpathqlineto{131.0000bp}{120.0000bp}
    \pgfpathqlineto{132.0000bp}{120.0000bp}
    \pgfpathqlineto{133.0000bp}{120.0000bp}
    \pgfpathqlineto{134.0000bp}{120.0000bp}
    \pgfpathqlineto{135.0000bp}{120.0000bp}
    \pgfpathqlineto{136.0000bp}{120.0000bp}
    \pgfpathqlineto{137.0000bp}{120.0000bp}
    \pgfpathqlineto{138.0000bp}{120.0000bp}
    \pgfpathqlineto{139.0000bp}{120.0000bp}
    \pgfpathqlineto{140.0000bp}{120.0000bp}
    \pgfpathqlineto{141.0000bp}{120.0000bp}
    \pgfpathqlineto{142.0000bp}{120.0000bp}
    \pgfpathqlineto{143.0000bp}{120.0000bp}
    \pgfpathqlineto{144.0000bp}{120.0000bp}
    \pgfpathqlineto{145.0000bp}{120.0000bp}
    \pgfpathqlineto{146.0000bp}{120.0000bp}
    \pgfpathqlineto{147.0000bp}{120.0000bp}
    \pgfpathqlineto{148.0000bp}{120.0000bp}
    \pgfpathqlineto{149.0000bp}{120.0000bp}
    \pgfpathqlineto{150.0000bp}{120.0000bp}
    \pgfpathqlineto{151.0000bp}{120.0000bp}
    \pgfpathqlineto{152.0000bp}{120.0000bp}
    \pgfpathqlineto{153.0000bp}{120.0000bp}
    \pgfpathqlineto{154.0000bp}{120.0000bp}
    \pgfpathqlineto{155.0000bp}{120.0000bp}
    \pgfpathqlineto{156.0000bp}{120.0000bp}
    \pgfpathqlineto{157.0000bp}{120.0000bp}
    \pgfpathqlineto{158.0000bp}{120.0000bp}
    \pgfpathqlineto{159.0000bp}{120.0000bp}
    \pgfpathqlineto{160.0000bp}{120.0000bp}
    \pgfpathqlineto{161.0000bp}{120.0000bp}
    \pgfpathqlineto{162.0000bp}{120.0000bp}
    \pgfpathqlineto{163.0000bp}{120.0000bp}
    \pgfpathqlineto{164.0000bp}{120.0000bp}
    \pgfpathqlineto{165.0000bp}{120.0000bp}
    \pgfpathqlineto{166.0000bp}{120.0000bp}
    \pgfpathqlineto{167.0000bp}{120.0000bp}
    \pgfpathqlineto{168.0000bp}{120.0000bp}
    \pgfpathqlineto{169.0000bp}{120.0000bp}
    \pgfpathqlineto{170.0000bp}{120.0000bp}
    \pgfpathqlineto{171.0000bp}{120.0000bp}
    \pgfpathqlineto{172.0000bp}{120.0000bp}
    \pgfpathqlineto{173.0000bp}{120.0000bp}
    \pgfpathqlineto{174.0000bp}{120.0000bp}
    \pgfpathqlineto{175.0000bp}{120.0000bp}
    \pgfpathqlineto{176.0000bp}{120.0000bp}
    \pgfpathqlineto{177.0000bp}{120.0000bp}
    \pgfpathqlineto{178.0000bp}{120.0000bp}
    \pgfpathqlineto{179.0000bp}{120.0000bp}
    \pgfpathqlineto{180.0000bp}{120.0000bp}
    \pgfpathqlineto{181.0000bp}{120.0000bp}
    \pgfpathqlineto{182.0000bp}{120.0000bp}
    \pgfpathqlineto{183.0000bp}{120.0000bp}
    \pgfpathqlineto{184.0000bp}{120.0000bp}
    \pgfpathqlineto{185.0000bp}{120.0000bp}
    \pgfpathqlineto{186.0000bp}{120.0000bp}
    \pgfpathqlineto{187.0000bp}{120.0000bp}
    \pgfpathqlineto{188.0000bp}{120.0000bp}
    \pgfpathqlineto{189.0000bp}{120.0000bp}
    \pgfpathqlineto{190.0000bp}{120.0000bp}
    \pgfpathqlineto{191.0000bp}{120.0000bp}
    \pgfpathqlineto{192.0000bp}{88.0000bp}
    \pgfpathqlineto{193.0000bp}{88.0000bp}
    \pgfpathqlineto{194.0000bp}{88.0000bp}
    \pgfpathqlineto{195.0000bp}{88.0000bp}
    \pgfpathqlineto{196.0000bp}{88.0000bp}
    \pgfpathqlineto{197.0000bp}{88.0000bp}
    \pgfpathqlineto{198.0000bp}{88.0000bp}
    \pgfpathqlineto{199.0000bp}{88.0000bp}
    \pgfusepathqstroke
  \end{pgfscope}
  \begin{pgfscope}
    \pgfsetlinewidth{0.5000bp}
    \definecolor{sc}{rgb}{0.0000,0.0000,1.0000}
    \pgfsetstrokecolor{sc}
    \pgfsetmiterjoin
    \pgfsetbuttcap
    \pgfpathqmoveto{2.0000bp}{88.0000bp}
    \pgfpathqlineto{3.0000bp}{88.0000bp}
    \pgfpathqlineto{4.0000bp}{88.0000bp}
    \pgfpathqlineto{5.0000bp}{88.0000bp}
    \pgfpathqlineto{6.0000bp}{88.0000bp}
    \pgfpathqlineto{7.0000bp}{88.0000bp}
    \pgfpathqlineto{8.0000bp}{88.0000bp}
    \pgfpathqlineto{9.0000bp}{88.0000bp}
    \pgfpathqlineto{10.0000bp}{88.0000bp}
    \pgfpathqlineto{11.0000bp}{88.0000bp}
    \pgfpathqlineto{12.0000bp}{88.0000bp}
    \pgfpathqlineto{13.0000bp}{88.0000bp}
    \pgfpathqlineto{14.0000bp}{88.0000bp}
    \pgfpathqlineto{15.0000bp}{88.0000bp}
    \pgfpathqlineto{16.0000bp}{88.0000bp}
    \pgfpathqlineto{17.0000bp}{88.0000bp}
    \pgfpathqlineto{18.0000bp}{88.0000bp}
    \pgfpathqlineto{19.0000bp}{88.0000bp}
    \pgfpathqlineto{20.0000bp}{88.0000bp}
    \pgfpathqlineto{21.0000bp}{88.0000bp}
    \pgfpathqlineto{22.0000bp}{88.0000bp}
    \pgfpathqlineto{23.0000bp}{88.0000bp}
    \pgfpathqlineto{24.0000bp}{88.0000bp}
    \pgfpathqlineto{25.0000bp}{88.0000bp}
    \pgfpathqlineto{26.0000bp}{88.0000bp}
    \pgfpathqlineto{27.0000bp}{88.0000bp}
    \pgfpathqlineto{28.0000bp}{88.0000bp}
    \pgfpathqlineto{29.0000bp}{88.0000bp}
    \pgfpathqlineto{30.0000bp}{88.0000bp}
    \pgfpathqlineto{31.0000bp}{88.0000bp}
    \pgfpathqlineto{32.0000bp}{88.0000bp}
    \pgfpathqlineto{33.0000bp}{88.0000bp}
    \pgfpathqlineto{34.0000bp}{88.0000bp}
    \pgfpathqlineto{35.0000bp}{88.0000bp}
    \pgfpathqlineto{36.0000bp}{88.0000bp}
    \pgfpathqlineto{37.0000bp}{88.0000bp}
    \pgfpathqlineto{38.0000bp}{88.0000bp}
    \pgfpathqlineto{39.0000bp}{88.0000bp}
    \pgfpathqlineto{40.0000bp}{88.0000bp}
    \pgfpathqlineto{41.0000bp}{88.0000bp}
    \pgfpathqlineto{42.0000bp}{88.0000bp}
    \pgfpathqlineto{43.0000bp}{88.0000bp}
    \pgfpathqlineto{44.0000bp}{88.0000bp}
    \pgfpathqlineto{45.0000bp}{88.0000bp}
    \pgfpathqlineto{46.0000bp}{88.0000bp}
    \pgfpathqlineto{47.0000bp}{88.0000bp}
    \pgfpathqlineto{48.0000bp}{88.0000bp}
    \pgfpathqlineto{49.0000bp}{88.0000bp}
    \pgfpathqlineto{50.0000bp}{88.0000bp}
    \pgfpathqlineto{51.0000bp}{88.0000bp}
    \pgfpathqlineto{52.0000bp}{88.0000bp}
    \pgfpathqlineto{53.0000bp}{88.0000bp}
    \pgfpathqlineto{54.0000bp}{88.0000bp}
    \pgfpathqlineto{55.0000bp}{88.0000bp}
    \pgfpathqlineto{56.0000bp}{88.0000bp}
    \pgfpathqlineto{57.0000bp}{88.0000bp}
    \pgfpathqlineto{58.0000bp}{88.0000bp}
    \pgfpathqlineto{59.0000bp}{88.0000bp}
    \pgfpathqlineto{60.0000bp}{88.0000bp}
    \pgfpathqlineto{61.0000bp}{88.0000bp}
    \pgfpathqlineto{62.0000bp}{88.0000bp}
    \pgfpathqlineto{63.0000bp}{88.0000bp}
    \pgfpathqlineto{64.0000bp}{88.0000bp}
    \pgfpathqlineto{65.0000bp}{88.0000bp}
    \pgfpathqlineto{66.0000bp}{88.0000bp}
    \pgfpathqlineto{67.0000bp}{88.0000bp}
    \pgfpathqlineto{68.0000bp}{88.0000bp}
    \pgfpathqlineto{69.0000bp}{88.0000bp}
    \pgfpathqlineto{70.0000bp}{88.0000bp}
    \pgfpathqlineto{71.0000bp}{88.0000bp}
    \pgfpathqlineto{72.0000bp}{88.0000bp}
    \pgfpathqlineto{73.0000bp}{88.0000bp}
    \pgfpathqlineto{74.0000bp}{88.0000bp}
    \pgfpathqlineto{75.0000bp}{88.0000bp}
    \pgfpathqlineto{76.0000bp}{88.0000bp}
    \pgfpathqlineto{77.0000bp}{88.0000bp}
    \pgfpathqlineto{78.0000bp}{88.0000bp}
    \pgfpathqlineto{79.0000bp}{88.0000bp}
    \pgfpathqlineto{80.0000bp}{88.0000bp}
    \pgfpathqlineto{81.0000bp}{88.0000bp}
    \pgfpathqlineto{82.0000bp}{88.0000bp}
    \pgfpathqlineto{83.0000bp}{88.0000bp}
    \pgfpathqlineto{84.0000bp}{88.0000bp}
    \pgfpathqlineto{85.0000bp}{88.0000bp}
    \pgfpathqlineto{86.0000bp}{88.0000bp}
    \pgfpathqlineto{87.0000bp}{88.0000bp}
    \pgfpathqlineto{88.0000bp}{88.0000bp}
    \pgfpathqlineto{89.0000bp}{88.0000bp}
    \pgfpathqlineto{90.0000bp}{88.0000bp}
    \pgfpathqlineto{91.0000bp}{88.0000bp}
    \pgfpathqlineto{92.0000bp}{88.0000bp}
    \pgfpathqlineto{93.0000bp}{88.0000bp}
    \pgfpathqlineto{94.0000bp}{88.0000bp}
    \pgfpathqlineto{95.0000bp}{88.0000bp}
    \pgfpathqlineto{96.0000bp}{88.0000bp}
    \pgfpathqlineto{97.0000bp}{88.0000bp}
    \pgfpathqlineto{98.0000bp}{88.0000bp}
    \pgfpathqlineto{99.0000bp}{88.0000bp}
    \pgfpathqlineto{100.0000bp}{88.0000bp}
    \pgfpathqlineto{101.0000bp}{88.0000bp}
    \pgfpathqlineto{102.0000bp}{88.0000bp}
    \pgfpathqlineto{103.0000bp}{88.0000bp}
    \pgfpathqlineto{104.0000bp}{88.0000bp}
    \pgfpathqlineto{105.0000bp}{88.0000bp}
    \pgfpathqlineto{106.0000bp}{88.0000bp}
    \pgfpathqlineto{107.0000bp}{88.0000bp}
    \pgfpathqlineto{108.0000bp}{88.0000bp}
    \pgfpathqlineto{109.0000bp}{88.0000bp}
    \pgfpathqlineto{110.0000bp}{88.0000bp}
    \pgfpathqlineto{111.0000bp}{88.0000bp}
    \pgfpathqlineto{112.0000bp}{88.0000bp}
    \pgfpathqlineto{113.0000bp}{88.0000bp}
    \pgfpathqlineto{114.0000bp}{88.0000bp}
    \pgfpathqlineto{115.0000bp}{88.0000bp}
    \pgfpathqlineto{116.0000bp}{88.0000bp}
    \pgfpathqlineto{117.0000bp}{88.0000bp}
    \pgfpathqlineto{118.0000bp}{88.0000bp}
    \pgfpathqlineto{119.0000bp}{88.0000bp}
    \pgfpathqlineto{120.0000bp}{88.0000bp}
    \pgfpathqlineto{121.0000bp}{88.0000bp}
    \pgfpathqlineto{122.0000bp}{88.0000bp}
    \pgfpathqlineto{123.0000bp}{88.0000bp}
    \pgfpathqlineto{124.0000bp}{88.0000bp}
    \pgfpathqlineto{125.0000bp}{88.0000bp}
    \pgfpathqlineto{126.0000bp}{88.0000bp}
    \pgfpathqlineto{127.0000bp}{88.0000bp}
    \pgfpathqlineto{128.0000bp}{88.0000bp}
    \pgfpathqlineto{129.0000bp}{88.0000bp}
    \pgfpathqlineto{130.0000bp}{88.0000bp}
    \pgfpathqlineto{131.0000bp}{88.0000bp}
    \pgfpathqlineto{132.0000bp}{88.0000bp}
    \pgfpathqlineto{133.0000bp}{88.0000bp}
    \pgfpathqlineto{134.0000bp}{88.0000bp}
    \pgfpathqlineto{135.0000bp}{88.0000bp}
    \pgfpathqlineto{136.0000bp}{88.0000bp}
    \pgfpathqlineto{137.0000bp}{88.0000bp}
    \pgfpathqlineto{138.0000bp}{88.0000bp}
    \pgfpathqlineto{139.0000bp}{88.0000bp}
    \pgfpathqlineto{140.0000bp}{88.0000bp}
    \pgfpathqlineto{141.0000bp}{88.0000bp}
    \pgfpathqlineto{142.0000bp}{88.0000bp}
    \pgfpathqlineto{143.0000bp}{88.0000bp}
    \pgfpathqlineto{144.0000bp}{88.0000bp}
    \pgfpathqlineto{145.0000bp}{88.0000bp}
    \pgfpathqlineto{146.0000bp}{88.0000bp}
    \pgfpathqlineto{147.0000bp}{88.0000bp}
    \pgfpathqlineto{148.0000bp}{88.0000bp}
    \pgfpathqlineto{149.0000bp}{88.0000bp}
    \pgfpathqlineto{150.0000bp}{88.0000bp}
    \pgfpathqlineto{151.0000bp}{88.0000bp}
    \pgfpathqlineto{152.0000bp}{88.0000bp}
    \pgfpathqlineto{153.0000bp}{88.0000bp}
    \pgfpathqlineto{154.0000bp}{88.0000bp}
    \pgfpathqlineto{155.0000bp}{88.0000bp}
    \pgfpathqlineto{156.0000bp}{88.0000bp}
    \pgfpathqlineto{157.0000bp}{88.0000bp}
    \pgfpathqlineto{158.0000bp}{88.0000bp}
    \pgfpathqlineto{159.0000bp}{88.0000bp}
    \pgfpathqlineto{160.0000bp}{88.0000bp}
    \pgfpathqlineto{161.0000bp}{88.0000bp}
    \pgfpathqlineto{162.0000bp}{88.0000bp}
    \pgfpathqlineto{163.0000bp}{88.0000bp}
    \pgfpathqlineto{164.0000bp}{88.0000bp}
    \pgfpathqlineto{165.0000bp}{88.0000bp}
    \pgfpathqlineto{166.0000bp}{88.0000bp}
    \pgfpathqlineto{167.0000bp}{88.0000bp}
    \pgfpathqlineto{168.0000bp}{88.0000bp}
    \pgfpathqlineto{169.0000bp}{88.0000bp}
    \pgfpathqlineto{170.0000bp}{88.0000bp}
    \pgfpathqlineto{171.0000bp}{88.0000bp}
    \pgfpathqlineto{172.0000bp}{88.0000bp}
    \pgfpathqlineto{173.0000bp}{88.0000bp}
    \pgfpathqlineto{174.0000bp}{88.0000bp}
    \pgfpathqlineto{175.0000bp}{88.0000bp}
    \pgfpathqlineto{176.0000bp}{88.0000bp}
    \pgfpathqlineto{177.0000bp}{88.0000bp}
    \pgfpathqlineto{178.0000bp}{88.0000bp}
    \pgfpathqlineto{179.0000bp}{88.0000bp}
    \pgfpathqlineto{180.0000bp}{88.0000bp}
    \pgfpathqlineto{181.0000bp}{88.0000bp}
    \pgfpathqlineto{182.0000bp}{88.0000bp}
    \pgfpathqlineto{183.0000bp}{88.0000bp}
    \pgfpathqlineto{184.0000bp}{88.0000bp}
    \pgfpathqlineto{185.0000bp}{88.0000bp}
    \pgfpathqlineto{186.0000bp}{88.0000bp}
    \pgfpathqlineto{187.0000bp}{88.0000bp}
    \pgfpathqlineto{188.0000bp}{88.0000bp}
    \pgfpathqlineto{189.0000bp}{88.0000bp}
    \pgfpathqlineto{190.0000bp}{88.0000bp}
    \pgfpathqlineto{191.0000bp}{88.0000bp}
    \pgfpathqlineto{192.0000bp}{90.0000bp}
    \pgfpathqlineto{193.0000bp}{90.0000bp}
    \pgfpathqlineto{194.0000bp}{90.0000bp}
    \pgfpathqlineto{195.0000bp}{90.0000bp}
    \pgfpathqlineto{196.0000bp}{90.0000bp}
    \pgfpathqlineto{197.0000bp}{90.0000bp}
    \pgfpathqlineto{198.0000bp}{90.0000bp}
    \pgfpathqlineto{199.0000bp}{90.0000bp}
    \pgfusepathqstroke
  \end{pgfscope}
  \begin{pgfscope}
    \pgfsetlinewidth{0.5000bp}
    \definecolor{sc}{rgb}{1.0000,0.0000,0.0000}
    \pgfsetstrokecolor{sc}
    \pgfsetmiterjoin
    \pgfsetbuttcap
    \pgfpathqmoveto{200.0000bp}{88.0000bp}
    \pgfpathqlineto{200.0000bp}{87.0000bp}
    \pgfusepathqstroke
  \end{pgfscope}
  \begin{pgfscope}
    \pgfsetlinewidth{0.5000bp}
    \definecolor{sc}{rgb}{1.0000,0.0000,0.0000}
    \pgfsetstrokecolor{sc}
    \pgfsetmiterjoin
    \pgfsetbuttcap
    \pgfpathqmoveto{193.0000bp}{88.0000bp}
    \pgfpathqlineto{193.0000bp}{87.0000bp}
    \pgfusepathqstroke
  \end{pgfscope}
  \begin{pgfscope}
    \pgfsetlinewidth{0.5000bp}
    \definecolor{sc}{rgb}{1.0000,0.0000,0.0000}
    \pgfsetstrokecolor{sc}
    \pgfsetmiterjoin
    \pgfsetbuttcap
    \pgfpathqmoveto{120.0000bp}{88.0000bp}
    \pgfpathqlineto{120.0000bp}{87.0000bp}
    \pgfusepathqstroke
  \end{pgfscope}
  \begin{pgfscope}
    \pgfsetlinewidth{0.5000bp}
    \definecolor{sc}{rgb}{1.0000,0.0000,0.0000}
    \pgfsetstrokecolor{sc}
    \pgfsetmiterjoin
    \pgfsetbuttcap
    \pgfpathqmoveto{2.0000bp}{88.0000bp}
    \pgfpathqlineto{2.0000bp}{87.0000bp}
    \pgfusepathqstroke
  \end{pgfscope}
  \begin{pgfscope}
    \pgfsetlinewidth{0.5000bp}
    \definecolor{sc}{rgb}{0.0000,0.0000,0.0000}
    \pgfsetstrokecolor{sc}
    \pgfsetmiterjoin
    \pgfsetbuttcap
    \pgfpathqmoveto{197.0000bp}{88.0000bp}
    \pgfpathqlineto{197.0000bp}{87.5000bp}
    \pgfusepathqstroke
  \end{pgfscope}
  \begin{pgfscope}
    \pgfsetlinewidth{0.5000bp}
    \definecolor{sc}{rgb}{0.0000,0.0000,0.0000}
    \pgfsetstrokecolor{sc}
    \pgfsetmiterjoin
    \pgfsetbuttcap
    \pgfpathqmoveto{192.0000bp}{88.0000bp}
    \pgfpathqlineto{192.0000bp}{87.5000bp}
    \pgfusepathqstroke
  \end{pgfscope}
  \begin{pgfscope}
    \pgfsetlinewidth{0.5000bp}
    \definecolor{sc}{rgb}{0.0000,0.0000,0.0000}
    \pgfsetstrokecolor{sc}
    \pgfsetmiterjoin
    \pgfsetbuttcap
    \pgfpathqmoveto{187.0000bp}{88.0000bp}
    \pgfpathqlineto{187.0000bp}{87.5000bp}
    \pgfusepathqstroke
  \end{pgfscope}
  \begin{pgfscope}
    \pgfsetlinewidth{0.5000bp}
    \definecolor{sc}{rgb}{0.0000,0.0000,0.0000}
    \pgfsetstrokecolor{sc}
    \pgfsetmiterjoin
    \pgfsetbuttcap
    \pgfpathqmoveto{182.0000bp}{88.0000bp}
    \pgfpathqlineto{182.0000bp}{87.5000bp}
    \pgfusepathqstroke
  \end{pgfscope}
  \begin{pgfscope}
    \pgfsetlinewidth{0.5000bp}
    \definecolor{sc}{rgb}{0.0000,0.0000,0.0000}
    \pgfsetstrokecolor{sc}
    \pgfsetmiterjoin
    \pgfsetbuttcap
    \pgfpathqmoveto{177.0000bp}{88.0000bp}
    \pgfpathqlineto{177.0000bp}{87.5000bp}
    \pgfusepathqstroke
  \end{pgfscope}
  \begin{pgfscope}
    \pgfsetlinewidth{0.5000bp}
    \definecolor{sc}{rgb}{0.0000,0.0000,0.0000}
    \pgfsetstrokecolor{sc}
    \pgfsetmiterjoin
    \pgfsetbuttcap
    \pgfpathqmoveto{172.0000bp}{88.0000bp}
    \pgfpathqlineto{172.0000bp}{87.5000bp}
    \pgfusepathqstroke
  \end{pgfscope}
  \begin{pgfscope}
    \pgfsetlinewidth{0.5000bp}
    \definecolor{sc}{rgb}{0.0000,0.0000,0.0000}
    \pgfsetstrokecolor{sc}
    \pgfsetmiterjoin
    \pgfsetbuttcap
    \pgfpathqmoveto{167.0000bp}{88.0000bp}
    \pgfpathqlineto{167.0000bp}{87.5000bp}
    \pgfusepathqstroke
  \end{pgfscope}
  \begin{pgfscope}
    \pgfsetlinewidth{0.5000bp}
    \definecolor{sc}{rgb}{0.0000,0.0000,0.0000}
    \pgfsetstrokecolor{sc}
    \pgfsetmiterjoin
    \pgfsetbuttcap
    \pgfpathqmoveto{162.0000bp}{88.0000bp}
    \pgfpathqlineto{162.0000bp}{87.5000bp}
    \pgfusepathqstroke
  \end{pgfscope}
  \begin{pgfscope}
    \pgfsetlinewidth{0.5000bp}
    \definecolor{sc}{rgb}{0.0000,0.0000,0.0000}
    \pgfsetstrokecolor{sc}
    \pgfsetmiterjoin
    \pgfsetbuttcap
    \pgfpathqmoveto{157.0000bp}{88.0000bp}
    \pgfpathqlineto{157.0000bp}{87.5000bp}
    \pgfusepathqstroke
  \end{pgfscope}
  \begin{pgfscope}
    \pgfsetlinewidth{0.5000bp}
    \definecolor{sc}{rgb}{0.0000,0.0000,0.0000}
    \pgfsetstrokecolor{sc}
    \pgfsetmiterjoin
    \pgfsetbuttcap
    \pgfpathqmoveto{152.0000bp}{88.0000bp}
    \pgfpathqlineto{152.0000bp}{87.5000bp}
    \pgfusepathqstroke
  \end{pgfscope}
  \begin{pgfscope}
    \pgfsetlinewidth{0.5000bp}
    \definecolor{sc}{rgb}{0.0000,0.0000,0.0000}
    \pgfsetstrokecolor{sc}
    \pgfsetmiterjoin
    \pgfsetbuttcap
    \pgfpathqmoveto{147.0000bp}{88.0000bp}
    \pgfpathqlineto{147.0000bp}{87.5000bp}
    \pgfusepathqstroke
  \end{pgfscope}
  \begin{pgfscope}
    \pgfsetlinewidth{0.5000bp}
    \definecolor{sc}{rgb}{0.0000,0.0000,0.0000}
    \pgfsetstrokecolor{sc}
    \pgfsetmiterjoin
    \pgfsetbuttcap
    \pgfpathqmoveto{142.0000bp}{88.0000bp}
    \pgfpathqlineto{142.0000bp}{87.5000bp}
    \pgfusepathqstroke
  \end{pgfscope}
  \begin{pgfscope}
    \pgfsetlinewidth{0.5000bp}
    \definecolor{sc}{rgb}{0.0000,0.0000,0.0000}
    \pgfsetstrokecolor{sc}
    \pgfsetmiterjoin
    \pgfsetbuttcap
    \pgfpathqmoveto{137.0000bp}{88.0000bp}
    \pgfpathqlineto{137.0000bp}{87.5000bp}
    \pgfusepathqstroke
  \end{pgfscope}
  \begin{pgfscope}
    \pgfsetlinewidth{0.5000bp}
    \definecolor{sc}{rgb}{0.0000,0.0000,0.0000}
    \pgfsetstrokecolor{sc}
    \pgfsetmiterjoin
    \pgfsetbuttcap
    \pgfpathqmoveto{132.0000bp}{88.0000bp}
    \pgfpathqlineto{132.0000bp}{87.5000bp}
    \pgfusepathqstroke
  \end{pgfscope}
  \begin{pgfscope}
    \pgfsetlinewidth{0.5000bp}
    \definecolor{sc}{rgb}{0.0000,0.0000,0.0000}
    \pgfsetstrokecolor{sc}
    \pgfsetmiterjoin
    \pgfsetbuttcap
    \pgfpathqmoveto{127.0000bp}{88.0000bp}
    \pgfpathqlineto{127.0000bp}{87.5000bp}
    \pgfusepathqstroke
  \end{pgfscope}
  \begin{pgfscope}
    \pgfsetlinewidth{0.5000bp}
    \definecolor{sc}{rgb}{0.0000,0.0000,0.0000}
    \pgfsetstrokecolor{sc}
    \pgfsetmiterjoin
    \pgfsetbuttcap
    \pgfpathqmoveto{122.0000bp}{88.0000bp}
    \pgfpathqlineto{122.0000bp}{87.5000bp}
    \pgfusepathqstroke
  \end{pgfscope}
  \begin{pgfscope}
    \pgfsetlinewidth{0.5000bp}
    \definecolor{sc}{rgb}{0.0000,0.0000,0.0000}
    \pgfsetstrokecolor{sc}
    \pgfsetmiterjoin
    \pgfsetbuttcap
    \pgfpathqmoveto{117.0000bp}{88.0000bp}
    \pgfpathqlineto{117.0000bp}{87.5000bp}
    \pgfusepathqstroke
  \end{pgfscope}
  \begin{pgfscope}
    \pgfsetlinewidth{0.5000bp}
    \definecolor{sc}{rgb}{0.0000,0.0000,0.0000}
    \pgfsetstrokecolor{sc}
    \pgfsetmiterjoin
    \pgfsetbuttcap
    \pgfpathqmoveto{112.0000bp}{88.0000bp}
    \pgfpathqlineto{112.0000bp}{87.5000bp}
    \pgfusepathqstroke
  \end{pgfscope}
  \begin{pgfscope}
    \pgfsetlinewidth{0.5000bp}
    \definecolor{sc}{rgb}{0.0000,0.0000,0.0000}
    \pgfsetstrokecolor{sc}
    \pgfsetmiterjoin
    \pgfsetbuttcap
    \pgfpathqmoveto{107.0000bp}{88.0000bp}
    \pgfpathqlineto{107.0000bp}{87.5000bp}
    \pgfusepathqstroke
  \end{pgfscope}
  \begin{pgfscope}
    \pgfsetlinewidth{0.5000bp}
    \definecolor{sc}{rgb}{0.0000,0.0000,0.0000}
    \pgfsetstrokecolor{sc}
    \pgfsetmiterjoin
    \pgfsetbuttcap
    \pgfpathqmoveto{102.0000bp}{88.0000bp}
    \pgfpathqlineto{102.0000bp}{87.5000bp}
    \pgfusepathqstroke
  \end{pgfscope}
  \begin{pgfscope}
    \pgfsetlinewidth{0.5000bp}
    \definecolor{sc}{rgb}{0.0000,0.0000,0.0000}
    \pgfsetstrokecolor{sc}
    \pgfsetmiterjoin
    \pgfsetbuttcap
    \pgfpathqmoveto{97.0000bp}{88.0000bp}
    \pgfpathqlineto{97.0000bp}{87.5000bp}
    \pgfusepathqstroke
  \end{pgfscope}
  \begin{pgfscope}
    \pgfsetlinewidth{0.5000bp}
    \definecolor{sc}{rgb}{0.0000,0.0000,0.0000}
    \pgfsetstrokecolor{sc}
    \pgfsetmiterjoin
    \pgfsetbuttcap
    \pgfpathqmoveto{92.0000bp}{88.0000bp}
    \pgfpathqlineto{92.0000bp}{87.5000bp}
    \pgfusepathqstroke
  \end{pgfscope}
  \begin{pgfscope}
    \pgfsetlinewidth{0.5000bp}
    \definecolor{sc}{rgb}{0.0000,0.0000,0.0000}
    \pgfsetstrokecolor{sc}
    \pgfsetmiterjoin
    \pgfsetbuttcap
    \pgfpathqmoveto{87.0000bp}{88.0000bp}
    \pgfpathqlineto{87.0000bp}{87.5000bp}
    \pgfusepathqstroke
  \end{pgfscope}
  \begin{pgfscope}
    \pgfsetlinewidth{0.5000bp}
    \definecolor{sc}{rgb}{0.0000,0.0000,0.0000}
    \pgfsetstrokecolor{sc}
    \pgfsetmiterjoin
    \pgfsetbuttcap
    \pgfpathqmoveto{82.0000bp}{88.0000bp}
    \pgfpathqlineto{82.0000bp}{87.5000bp}
    \pgfusepathqstroke
  \end{pgfscope}
  \begin{pgfscope}
    \pgfsetlinewidth{0.5000bp}
    \definecolor{sc}{rgb}{0.0000,0.0000,0.0000}
    \pgfsetstrokecolor{sc}
    \pgfsetmiterjoin
    \pgfsetbuttcap
    \pgfpathqmoveto{77.0000bp}{88.0000bp}
    \pgfpathqlineto{77.0000bp}{87.5000bp}
    \pgfusepathqstroke
  \end{pgfscope}
  \begin{pgfscope}
    \pgfsetlinewidth{0.5000bp}
    \definecolor{sc}{rgb}{0.0000,0.0000,0.0000}
    \pgfsetstrokecolor{sc}
    \pgfsetmiterjoin
    \pgfsetbuttcap
    \pgfpathqmoveto{72.0000bp}{88.0000bp}
    \pgfpathqlineto{72.0000bp}{87.5000bp}
    \pgfusepathqstroke
  \end{pgfscope}
  \begin{pgfscope}
    \pgfsetlinewidth{0.5000bp}
    \definecolor{sc}{rgb}{0.0000,0.0000,0.0000}
    \pgfsetstrokecolor{sc}
    \pgfsetmiterjoin
    \pgfsetbuttcap
    \pgfpathqmoveto{67.0000bp}{88.0000bp}
    \pgfpathqlineto{67.0000bp}{87.5000bp}
    \pgfusepathqstroke
  \end{pgfscope}
  \begin{pgfscope}
    \pgfsetlinewidth{0.5000bp}
    \definecolor{sc}{rgb}{0.0000,0.0000,0.0000}
    \pgfsetstrokecolor{sc}
    \pgfsetmiterjoin
    \pgfsetbuttcap
    \pgfpathqmoveto{62.0000bp}{88.0000bp}
    \pgfpathqlineto{62.0000bp}{87.5000bp}
    \pgfusepathqstroke
  \end{pgfscope}
  \begin{pgfscope}
    \pgfsetlinewidth{0.5000bp}
    \definecolor{sc}{rgb}{0.0000,0.0000,0.0000}
    \pgfsetstrokecolor{sc}
    \pgfsetmiterjoin
    \pgfsetbuttcap
    \pgfpathqmoveto{57.0000bp}{88.0000bp}
    \pgfpathqlineto{57.0000bp}{87.5000bp}
    \pgfusepathqstroke
  \end{pgfscope}
  \begin{pgfscope}
    \pgfsetlinewidth{0.5000bp}
    \definecolor{sc}{rgb}{0.0000,0.0000,0.0000}
    \pgfsetstrokecolor{sc}
    \pgfsetmiterjoin
    \pgfsetbuttcap
    \pgfpathqmoveto{52.0000bp}{88.0000bp}
    \pgfpathqlineto{52.0000bp}{87.5000bp}
    \pgfusepathqstroke
  \end{pgfscope}
  \begin{pgfscope}
    \pgfsetlinewidth{0.5000bp}
    \definecolor{sc}{rgb}{0.0000,0.0000,0.0000}
    \pgfsetstrokecolor{sc}
    \pgfsetmiterjoin
    \pgfsetbuttcap
    \pgfpathqmoveto{47.0000bp}{88.0000bp}
    \pgfpathqlineto{47.0000bp}{87.5000bp}
    \pgfusepathqstroke
  \end{pgfscope}
  \begin{pgfscope}
    \pgfsetlinewidth{0.5000bp}
    \definecolor{sc}{rgb}{0.0000,0.0000,0.0000}
    \pgfsetstrokecolor{sc}
    \pgfsetmiterjoin
    \pgfsetbuttcap
    \pgfpathqmoveto{42.0000bp}{88.0000bp}
    \pgfpathqlineto{42.0000bp}{87.5000bp}
    \pgfusepathqstroke
  \end{pgfscope}
  \begin{pgfscope}
    \pgfsetlinewidth{0.5000bp}
    \definecolor{sc}{rgb}{0.0000,0.0000,0.0000}
    \pgfsetstrokecolor{sc}
    \pgfsetmiterjoin
    \pgfsetbuttcap
    \pgfpathqmoveto{37.0000bp}{88.0000bp}
    \pgfpathqlineto{37.0000bp}{87.5000bp}
    \pgfusepathqstroke
  \end{pgfscope}
  \begin{pgfscope}
    \pgfsetlinewidth{0.5000bp}
    \definecolor{sc}{rgb}{0.0000,0.0000,0.0000}
    \pgfsetstrokecolor{sc}
    \pgfsetmiterjoin
    \pgfsetbuttcap
    \pgfpathqmoveto{32.0000bp}{88.0000bp}
    \pgfpathqlineto{32.0000bp}{87.5000bp}
    \pgfusepathqstroke
  \end{pgfscope}
  \begin{pgfscope}
    \pgfsetlinewidth{0.5000bp}
    \definecolor{sc}{rgb}{0.0000,0.0000,0.0000}
    \pgfsetstrokecolor{sc}
    \pgfsetmiterjoin
    \pgfsetbuttcap
    \pgfpathqmoveto{27.0000bp}{88.0000bp}
    \pgfpathqlineto{27.0000bp}{87.5000bp}
    \pgfusepathqstroke
  \end{pgfscope}
  \begin{pgfscope}
    \pgfsetlinewidth{0.5000bp}
    \definecolor{sc}{rgb}{0.0000,0.0000,0.0000}
    \pgfsetstrokecolor{sc}
    \pgfsetmiterjoin
    \pgfsetbuttcap
    \pgfpathqmoveto{22.0000bp}{88.0000bp}
    \pgfpathqlineto{22.0000bp}{87.5000bp}
    \pgfusepathqstroke
  \end{pgfscope}
  \begin{pgfscope}
    \pgfsetlinewidth{0.5000bp}
    \definecolor{sc}{rgb}{0.0000,0.0000,0.0000}
    \pgfsetstrokecolor{sc}
    \pgfsetmiterjoin
    \pgfsetbuttcap
    \pgfpathqmoveto{17.0000bp}{88.0000bp}
    \pgfpathqlineto{17.0000bp}{87.5000bp}
    \pgfusepathqstroke
  \end{pgfscope}
  \begin{pgfscope}
    \pgfsetlinewidth{0.5000bp}
    \definecolor{sc}{rgb}{0.0000,0.0000,0.0000}
    \pgfsetstrokecolor{sc}
    \pgfsetmiterjoin
    \pgfsetbuttcap
    \pgfpathqmoveto{12.0000bp}{88.0000bp}
    \pgfpathqlineto{12.0000bp}{87.5000bp}
    \pgfusepathqstroke
  \end{pgfscope}
  \begin{pgfscope}
    \pgfsetlinewidth{0.5000bp}
    \definecolor{sc}{rgb}{0.0000,0.0000,0.0000}
    \pgfsetstrokecolor{sc}
    \pgfsetmiterjoin
    \pgfsetbuttcap
    \pgfpathqmoveto{7.0000bp}{88.0000bp}
    \pgfpathqlineto{7.0000bp}{87.5000bp}
    \pgfusepathqstroke
  \end{pgfscope}
  \begin{pgfscope}
    \definecolor{fc}{rgb}{0.0000,0.0000,0.0000}
    \pgfsetfillcolor{fc}
    \pgftransformshift{\pgfqpoint{0.0000bp}{119.7000bp}}
    \pgftransformscale{0.1250}
    \pgftext[base,left]{$\mathbb{L}_A$}
  \end{pgfscope}
  \begin{pgfscope}
    \pgfsetlinewidth{0.5000bp}
    \definecolor{sc}{rgb}{0.0000,0.0000,0.0000}
    \pgfsetstrokecolor{sc}
    \pgfsetmiterjoin
    \pgfsetbuttcap
    \pgfpathqmoveto{2.0000bp}{120.0000bp}
    \pgfpathqlineto{1.8000bp}{120.0000bp}
    \pgfusepathqstroke
  \end{pgfscope}
  \begin{pgfscope}
    \pgfsetlinewidth{0.5000bp}
    \definecolor{sc}{rgb}{0.0000,0.0000,0.0000}
    \pgfsetstrokecolor{sc}
    \pgfsetmiterjoin
    \pgfsetbuttcap
    \pgfpathqmoveto{2.0000bp}{88.0000bp}
    \pgfpathqlineto{2.0000bp}{120.0000bp}
    \pgfusepathqstroke
  \end{pgfscope}
  \begin{pgfscope}
    \pgfsetlinewidth{0.5000bp}
    \definecolor{sc}{rgb}{0.0000,0.0000,0.0000}
    \pgfsetstrokecolor{sc}
    \pgfsetmiterjoin
    \pgfsetbuttcap
    \pgfpathqmoveto{2.0000bp}{88.0000bp}
    \pgfpathqlineto{200.0000bp}{88.0000bp}
    \pgfusepathqstroke
  \end{pgfscope}
\end{pgfpicture}

        \label{fig:ex:ca:hgma:ex:move-h}
    \caption{push-v-goal effects}\label{fig:ex:ca:hgma:ex:disconnected}
\end{figure}

\begin{figure}
    \centering
    \begin{pgfpicture}
  \pgfpathrectangle{\pgfpointorigin}{\pgfqpoint{200.0000bp}{200.0000bp}}
  \pgfusepath{use as bounding box}
  \begin{pgfscope}
    \definecolor{fc}{rgb}{0.0000,0.0000,0.0000}
    \pgfsetfillcolor{fc}
    \pgftransformshift{\pgfqpoint{28.3333bp}{27.9167bp}}
    \pgftransformscale{1.0417}
    \pgftext[base,left]{candidates}
  \end{pgfscope}
  \begin{pgfscope}
    \definecolor{fc}{rgb}{0.0000,0.0000,0.0000}
    \pgfsetfillcolor{fc}
    \pgfsetlinewidth{0.6831bp}
    \definecolor{sc}{rgb}{0.0000,0.0000,0.0000}
    \pgfsetstrokecolor{sc}
    \pgfsetmiterjoin
    \pgfsetbuttcap
    \pgfpathqmoveto{20.0000bp}{30.4167bp}
    \pgfpathqcurveto{20.0000bp}{32.2576bp}{18.5076bp}{33.7500bp}{16.6667bp}{33.7500bp}
    \pgfpathqcurveto{14.8257bp}{33.7500bp}{13.3333bp}{32.2576bp}{13.3333bp}{30.4167bp}
    \pgfpathqcurveto{13.3333bp}{28.5757bp}{14.8257bp}{27.0833bp}{16.6667bp}{27.0833bp}
    \pgfpathqcurveto{18.5076bp}{27.0833bp}{20.0000bp}{28.5757bp}{20.0000bp}{30.4167bp}
    \pgfpathclose
    \pgfusepathqfillstroke
  \end{pgfscope}
  \begin{pgfscope}
    \definecolor{fc}{rgb}{0.0000,0.0000,0.0000}
    \pgfsetfillcolor{fc}
    \pgftransformshift{\pgfqpoint{28.3333bp}{38.7500bp}}
    \pgftransformscale{1.0417}
    \pgftext[base,left]{negative unproven}
  \end{pgfscope}
  \begin{pgfscope}
    \definecolor{fc}{rgb}{1.0000,1.0000,0.0000}
    \pgfsetfillcolor{fc}
    \pgfsetlinewidth{0.6831bp}
    \definecolor{sc}{rgb}{1.0000,1.0000,0.0000}
    \pgfsetstrokecolor{sc}
    \pgfsetmiterjoin
    \pgfsetbuttcap
    \pgfpathqmoveto{20.0000bp}{41.2500bp}
    \pgfpathqcurveto{20.0000bp}{43.0909bp}{18.5076bp}{44.5833bp}{16.6667bp}{44.5833bp}
    \pgfpathqcurveto{14.8257bp}{44.5833bp}{13.3333bp}{43.0909bp}{13.3333bp}{41.2500bp}
    \pgfpathqcurveto{13.3333bp}{39.4091bp}{14.8257bp}{37.9167bp}{16.6667bp}{37.9167bp}
    \pgfpathqcurveto{18.5076bp}{37.9167bp}{20.0000bp}{39.4091bp}{20.0000bp}{41.2500bp}
    \pgfpathclose
    \pgfusepathqfillstroke
  \end{pgfscope}
  \begin{pgfscope}
    \definecolor{fc}{rgb}{0.0000,0.0000,0.0000}
    \pgfsetfillcolor{fc}
    \pgftransformshift{\pgfqpoint{28.3333bp}{49.5833bp}}
    \pgftransformscale{1.0417}
    \pgftext[base,left]{negative proven}
  \end{pgfscope}
  \begin{pgfscope}
    \definecolor{fc}{rgb}{0.0000,0.5020,0.0000}
    \pgfsetfillcolor{fc}
    \pgfsetlinewidth{0.6831bp}
    \definecolor{sc}{rgb}{0.0000,0.5020,0.0000}
    \pgfsetstrokecolor{sc}
    \pgfsetmiterjoin
    \pgfsetbuttcap
    \pgfpathqmoveto{20.0000bp}{52.0833bp}
    \pgfpathqcurveto{20.0000bp}{53.9243bp}{18.5076bp}{55.4167bp}{16.6667bp}{55.4167bp}
    \pgfpathqcurveto{14.8257bp}{55.4167bp}{13.3333bp}{53.9243bp}{13.3333bp}{52.0833bp}
    \pgfpathqcurveto{13.3333bp}{50.2424bp}{14.8257bp}{48.7500bp}{16.6667bp}{48.7500bp}
    \pgfpathqcurveto{18.5076bp}{48.7500bp}{20.0000bp}{50.2424bp}{20.0000bp}{52.0833bp}
    \pgfpathclose
    \pgfusepathqfillstroke
  \end{pgfscope}
  \begin{pgfscope}
    \definecolor{fc}{rgb}{0.0000,0.0000,0.0000}
    \pgfsetfillcolor{fc}
    \pgftransformshift{\pgfqpoint{28.3333bp}{60.4167bp}}
    \pgftransformscale{1.0417}
    \pgftext[base,left]{positive unproven}
  \end{pgfscope}
  \begin{pgfscope}
    \definecolor{fc}{rgb}{1.0000,0.0000,0.0000}
    \pgfsetfillcolor{fc}
    \pgfsetlinewidth{0.6831bp}
    \definecolor{sc}{rgb}{1.0000,0.0000,0.0000}
    \pgfsetstrokecolor{sc}
    \pgfsetmiterjoin
    \pgfsetbuttcap
    \pgfpathqmoveto{20.0000bp}{62.9167bp}
    \pgfpathqcurveto{20.0000bp}{64.7576bp}{18.5076bp}{66.2500bp}{16.6667bp}{66.2500bp}
    \pgfpathqcurveto{14.8257bp}{66.2500bp}{13.3333bp}{64.7576bp}{13.3333bp}{62.9167bp}
    \pgfpathqcurveto{13.3333bp}{61.0757bp}{14.8257bp}{59.5833bp}{16.6667bp}{59.5833bp}
    \pgfpathqcurveto{18.5076bp}{59.5833bp}{20.0000bp}{61.0757bp}{20.0000bp}{62.9167bp}
    \pgfpathclose
    \pgfusepathqfillstroke
  \end{pgfscope}
  \begin{pgfscope}
    \definecolor{fc}{rgb}{0.0000,0.0000,0.0000}
    \pgfsetfillcolor{fc}
    \pgftransformshift{\pgfqpoint{28.3333bp}{71.2500bp}}
    \pgftransformscale{1.0417}
    \pgftext[base,left]{positive proven}
  \end{pgfscope}
  \begin{pgfscope}
    \definecolor{fc}{rgb}{0.0000,0.0000,1.0000}
    \pgfsetfillcolor{fc}
    \pgfsetlinewidth{0.6831bp}
    \definecolor{sc}{rgb}{0.0000,0.0000,1.0000}
    \pgfsetstrokecolor{sc}
    \pgfsetmiterjoin
    \pgfsetbuttcap
    \pgfpathqmoveto{20.0000bp}{73.7500bp}
    \pgfpathqcurveto{20.0000bp}{75.5909bp}{18.5076bp}{77.0833bp}{16.6667bp}{77.0833bp}
    \pgfpathqcurveto{14.8257bp}{77.0833bp}{13.3333bp}{75.5909bp}{13.3333bp}{73.7500bp}
    \pgfpathqcurveto{13.3333bp}{71.9091bp}{14.8257bp}{70.4167bp}{16.6667bp}{70.4167bp}
    \pgfpathqcurveto{18.5076bp}{70.4167bp}{20.0000bp}{71.9091bp}{20.0000bp}{73.7500bp}
    \pgfpathclose
    \pgfusepathqfillstroke
  \end{pgfscope}
  \begin{pgfscope}
    \pgfsetlinewidth{0.6831bp}
    \definecolor{sc}{rgb}{0.0000,0.0000,0.0000}
    \pgfsetstrokecolor{sc}
    \pgfsetmiterjoin
    \pgfsetbuttcap
    \pgfpathqmoveto{25.0000bp}{93.7500bp}
    \pgfpathqlineto{33.3333bp}{97.9167bp}
    \pgfpathqlineto{41.6667bp}{102.0833bp}
    \pgfpathqlineto{50.0000bp}{106.2500bp}
    \pgfpathqlineto{58.3333bp}{110.4167bp}
    \pgfpathqlineto{66.6667bp}{114.5833bp}
    \pgfpathqlineto{75.0000bp}{118.7500bp}
    \pgfpathqlineto{83.3333bp}{122.9167bp}
    \pgfpathqlineto{91.6667bp}{127.0833bp}
    \pgfpathqlineto{100.0000bp}{131.2500bp}
    \pgfpathqlineto{108.3333bp}{135.4167bp}
    \pgfpathqlineto{116.6667bp}{139.5833bp}
    \pgfpathqlineto{125.0000bp}{143.7500bp}
    \pgfpathqlineto{133.3333bp}{147.9167bp}
    \pgfpathqlineto{141.6667bp}{152.0833bp}
    \pgfpathqlineto{150.0000bp}{156.2500bp}
    \pgfpathqlineto{158.3333bp}{160.4167bp}
    \pgfpathqlineto{166.6667bp}{102.0833bp}
    \pgfpathqlineto{175.0000bp}{97.9167bp}
    \pgfpathqlineto{183.3333bp}{97.9167bp}
    \pgfpathqlineto{191.6667bp}{102.0833bp}
    \pgfpathqlineto{200.0000bp}{102.0833bp}
    \pgfusepathqstroke
  \end{pgfscope}
  \begin{pgfscope}
    \pgfsetlinewidth{0.6831bp}
    \definecolor{sc}{rgb}{1.0000,1.0000,0.0000}
    \pgfsetstrokecolor{sc}
    \pgfsetmiterjoin
    \pgfsetbuttcap
    \pgfpathqmoveto{25.0000bp}{172.9167bp}
    \pgfpathqlineto{33.3333bp}{172.9167bp}
    \pgfpathqlineto{41.6667bp}{172.9167bp}
    \pgfpathqlineto{50.0000bp}{172.9167bp}
    \pgfpathqlineto{58.3333bp}{172.9167bp}
    \pgfpathqlineto{66.6667bp}{172.9167bp}
    \pgfpathqlineto{75.0000bp}{172.9167bp}
    \pgfpathqlineto{83.3333bp}{172.9167bp}
    \pgfpathqlineto{91.6667bp}{172.9167bp}
    \pgfpathqlineto{100.0000bp}{172.9167bp}
    \pgfpathqlineto{108.3333bp}{172.9167bp}
    \pgfpathqlineto{116.6667bp}{172.9167bp}
    \pgfpathqlineto{125.0000bp}{172.9167bp}
    \pgfpathqlineto{133.3333bp}{172.9167bp}
    \pgfpathqlineto{141.6667bp}{172.9167bp}
    \pgfpathqlineto{150.0000bp}{172.9167bp}
    \pgfpathqlineto{158.3333bp}{172.9167bp}
    \pgfpathqlineto{166.6667bp}{156.2500bp}
    \pgfpathqlineto{175.0000bp}{156.2500bp}
    \pgfpathqlineto{183.3333bp}{156.2500bp}
    \pgfpathqlineto{191.6667bp}{156.2500bp}
    \pgfpathqlineto{200.0000bp}{156.2500bp}
    \pgfusepathqstroke
  \end{pgfscope}
  \begin{pgfscope}
    \pgfsetlinewidth{0.6831bp}
    \definecolor{sc}{rgb}{0.0000,0.5020,0.0000}
    \pgfsetstrokecolor{sc}
    \pgfsetmiterjoin
    \pgfsetbuttcap
    \pgfpathqmoveto{25.0000bp}{89.5833bp}
    \pgfpathqlineto{33.3333bp}{89.5833bp}
    \pgfpathqlineto{41.6667bp}{89.5833bp}
    \pgfpathqlineto{50.0000bp}{89.5833bp}
    \pgfpathqlineto{58.3333bp}{89.5833bp}
    \pgfpathqlineto{66.6667bp}{89.5833bp}
    \pgfpathqlineto{75.0000bp}{89.5833bp}
    \pgfpathqlineto{83.3333bp}{89.5833bp}
    \pgfpathqlineto{91.6667bp}{89.5833bp}
    \pgfpathqlineto{100.0000bp}{89.5833bp}
    \pgfpathqlineto{108.3333bp}{89.5833bp}
    \pgfpathqlineto{116.6667bp}{89.5833bp}
    \pgfpathqlineto{125.0000bp}{89.5833bp}
    \pgfpathqlineto{133.3333bp}{89.5833bp}
    \pgfpathqlineto{141.6667bp}{89.5833bp}
    \pgfpathqlineto{150.0000bp}{89.5833bp}
    \pgfpathqlineto{158.3333bp}{89.5833bp}
    \pgfpathqlineto{166.6667bp}{89.5833bp}
    \pgfpathqlineto{175.0000bp}{89.5833bp}
    \pgfpathqlineto{183.3333bp}{89.5833bp}
    \pgfpathqlineto{191.6667bp}{89.5833bp}
    \pgfpathqlineto{200.0000bp}{89.5833bp}
    \pgfusepathqstroke
  \end{pgfscope}
  \begin{pgfscope}
    \pgfsetlinewidth{0.6831bp}
    \definecolor{sc}{rgb}{1.0000,0.0000,0.0000}
    \pgfsetstrokecolor{sc}
    \pgfsetmiterjoin
    \pgfsetbuttcap
    \pgfpathqmoveto{25.0000bp}{172.9167bp}
    \pgfpathqlineto{33.3333bp}{172.9167bp}
    \pgfpathqlineto{41.6667bp}{172.9167bp}
    \pgfpathqlineto{50.0000bp}{172.9167bp}
    \pgfpathqlineto{58.3333bp}{172.9167bp}
    \pgfpathqlineto{66.6667bp}{172.9167bp}
    \pgfpathqlineto{75.0000bp}{172.9167bp}
    \pgfpathqlineto{83.3333bp}{172.9167bp}
    \pgfpathqlineto{91.6667bp}{172.9167bp}
    \pgfpathqlineto{100.0000bp}{172.9167bp}
    \pgfpathqlineto{108.3333bp}{172.9167bp}
    \pgfpathqlineto{116.6667bp}{172.9167bp}
    \pgfpathqlineto{125.0000bp}{172.9167bp}
    \pgfpathqlineto{133.3333bp}{172.9167bp}
    \pgfpathqlineto{141.6667bp}{172.9167bp}
    \pgfpathqlineto{150.0000bp}{172.9167bp}
    \pgfpathqlineto{158.3333bp}{172.9167bp}
    \pgfpathqlineto{166.6667bp}{102.0833bp}
    \pgfpathqlineto{175.0000bp}{97.9167bp}
    \pgfpathqlineto{183.3333bp}{97.9167bp}
    \pgfpathqlineto{191.6667bp}{97.9167bp}
    \pgfpathqlineto{200.0000bp}{97.9167bp}
    \pgfusepathqstroke
  \end{pgfscope}
  \begin{pgfscope}
    \pgfsetlinewidth{0.6831bp}
    \definecolor{sc}{rgb}{0.0000,0.0000,1.0000}
    \pgfsetstrokecolor{sc}
    \pgfsetmiterjoin
    \pgfsetbuttcap
    \pgfpathqmoveto{25.0000bp}{89.5833bp}
    \pgfpathqlineto{33.3333bp}{89.5833bp}
    \pgfpathqlineto{41.6667bp}{89.5833bp}
    \pgfpathqlineto{50.0000bp}{89.5833bp}
    \pgfpathqlineto{58.3333bp}{89.5833bp}
    \pgfpathqlineto{66.6667bp}{89.5833bp}
    \pgfpathqlineto{75.0000bp}{89.5833bp}
    \pgfpathqlineto{83.3333bp}{89.5833bp}
    \pgfpathqlineto{91.6667bp}{89.5833bp}
    \pgfpathqlineto{100.0000bp}{89.5833bp}
    \pgfpathqlineto{108.3333bp}{89.5833bp}
    \pgfpathqlineto{116.6667bp}{89.5833bp}
    \pgfpathqlineto{125.0000bp}{89.5833bp}
    \pgfpathqlineto{133.3333bp}{89.5833bp}
    \pgfpathqlineto{141.6667bp}{89.5833bp}
    \pgfpathqlineto{150.0000bp}{89.5833bp}
    \pgfpathqlineto{158.3333bp}{89.5833bp}
    \pgfpathqlineto{166.6667bp}{93.7500bp}
    \pgfpathqlineto{175.0000bp}{97.9167bp}
    \pgfpathqlineto{183.3333bp}{97.9167bp}
    \pgfpathqlineto{191.6667bp}{97.9167bp}
    \pgfpathqlineto{200.0000bp}{97.9167bp}
    \pgfusepathqstroke
  \end{pgfscope}
  \begin{pgfscope}
    \pgfsetlinewidth{0.6831bp}
    \definecolor{sc}{rgb}{1.0000,0.0000,0.0000}
    \pgfsetstrokecolor{sc}
    \pgfsetmiterjoin
    \pgfsetbuttcap
    \pgfpathqmoveto{50.0000bp}{89.5833bp}
    \pgfpathqlineto{50.0000bp}{85.4167bp}
    \pgfusepathqstroke
  \end{pgfscope}
  \begin{pgfscope}
    \pgfsetlinewidth{0.6831bp}
    \definecolor{sc}{rgb}{1.0000,0.0000,0.0000}
    \pgfsetstrokecolor{sc}
    \pgfsetmiterjoin
    \pgfsetbuttcap
    \pgfpathqmoveto{166.6667bp}{89.5833bp}
    \pgfpathqlineto{166.6667bp}{85.4167bp}
    \pgfusepathqstroke
  \end{pgfscope}
  \begin{pgfscope}
    \pgfsetlinewidth{0.6831bp}
    \definecolor{sc}{rgb}{0.0000,0.0000,0.0000}
    \pgfsetstrokecolor{sc}
    \pgfsetmiterjoin
    \pgfsetbuttcap
    \pgfpathqmoveto{183.3333bp}{89.5833bp}
    \pgfpathqlineto{183.3333bp}{85.4167bp}
    \pgfusepathqstroke
  \end{pgfscope}
  \begin{pgfscope}
    \pgfsetlinewidth{0.6831bp}
    \definecolor{sc}{rgb}{0.0000,0.0000,0.0000}
    \pgfsetstrokecolor{sc}
    \pgfsetmiterjoin
    \pgfsetbuttcap
    \pgfpathqmoveto{141.6667bp}{89.5833bp}
    \pgfpathqlineto{141.6667bp}{85.4167bp}
    \pgfusepathqstroke
  \end{pgfscope}
  \begin{pgfscope}
    \pgfsetlinewidth{0.6831bp}
    \definecolor{sc}{rgb}{0.0000,0.0000,0.0000}
    \pgfsetstrokecolor{sc}
    \pgfsetmiterjoin
    \pgfsetbuttcap
    \pgfpathqmoveto{100.0000bp}{89.5833bp}
    \pgfpathqlineto{100.0000bp}{85.4167bp}
    \pgfusepathqstroke
  \end{pgfscope}
  \begin{pgfscope}
    \pgfsetlinewidth{0.6831bp}
    \definecolor{sc}{rgb}{0.0000,0.0000,0.0000}
    \pgfsetstrokecolor{sc}
    \pgfsetmiterjoin
    \pgfsetbuttcap
    \pgfpathqmoveto{58.3333bp}{89.5833bp}
    \pgfpathqlineto{58.3333bp}{85.4167bp}
    \pgfusepathqstroke
  \end{pgfscope}
  \begin{pgfscope}
    \definecolor{fc}{rgb}{0.0000,0.0000,0.0000}
    \pgfsetfillcolor{fc}
    \pgftransformshift{\pgfqpoint{-0.0000bp}{170.4167bp}}
    \pgftransformscale{1.0417}
    \pgftext[base,left]{$\mathbb{F}_A$}
  \end{pgfscope}
  \begin{pgfscope}
    \pgfsetlinewidth{0.6831bp}
    \definecolor{sc}{rgb}{0.0000,0.0000,0.0000}
    \pgfsetstrokecolor{sc}
    \pgfsetmiterjoin
    \pgfsetbuttcap
    \pgfpathqmoveto{16.6667bp}{172.9167bp}
    \pgfpathqlineto{15.0000bp}{172.9167bp}
    \pgfusepathqstroke
  \end{pgfscope}
  \begin{pgfscope}
    \pgfsetlinewidth{0.6831bp}
    \definecolor{sc}{rgb}{0.0000,0.0000,0.0000}
    \pgfsetstrokecolor{sc}
    \pgfsetmiterjoin
    \pgfsetbuttcap
    \pgfpathqmoveto{16.6667bp}{89.5833bp}
    \pgfpathqlineto{16.6667bp}{172.9167bp}
    \pgfusepathqstroke
  \end{pgfscope}
  \begin{pgfscope}
    \pgfsetlinewidth{0.6831bp}
    \definecolor{sc}{rgb}{0.0000,0.0000,0.0000}
    \pgfsetstrokecolor{sc}
    \pgfsetmiterjoin
    \pgfsetbuttcap
    \pgfpathqmoveto{16.6667bp}{89.5833bp}
    \pgfpathqlineto{200.0000bp}{89.5833bp}
    \pgfusepathqstroke
  \end{pgfscope}
\end{pgfpicture}

    \caption{move-h preconditions}
\end{figure}

\begin{figure}
    \centering
    \begin{pgfpicture}
  \pgfpathrectangle{\pgfpointorigin}{\pgfqpoint{200.0000bp}{200.0000bp}}
  \pgfusepath{use as bounding box}
  \begin{pgfscope}
    \definecolor{fc}{rgb}{0.0000,0.0000,0.0000}
    \pgfsetfillcolor{fc}
    \pgftransformshift{\pgfqpoint{27.2000bp}{28.8000bp}}
    \pgftransformscale{1.0000}
    \pgftext[base,left]{candidates}
  \end{pgfscope}
  \begin{pgfscope}
    \definecolor{fc}{rgb}{0.0000,0.0000,0.0000}
    \pgfsetfillcolor{fc}
    \pgfsetlinewidth{0.6788bp}
    \definecolor{sc}{rgb}{0.0000,0.0000,0.0000}
    \pgfsetstrokecolor{sc}
    \pgfsetmiterjoin
    \pgfsetbuttcap
    \pgfpathqmoveto{19.2000bp}{31.2000bp}
    \pgfpathqcurveto{19.2000bp}{32.9673bp}{17.7673bp}{34.4000bp}{16.0000bp}{34.4000bp}
    \pgfpathqcurveto{14.2327bp}{34.4000bp}{12.8000bp}{32.9673bp}{12.8000bp}{31.2000bp}
    \pgfpathqcurveto{12.8000bp}{29.4327bp}{14.2327bp}{28.0000bp}{16.0000bp}{28.0000bp}
    \pgfpathqcurveto{17.7673bp}{28.0000bp}{19.2000bp}{29.4327bp}{19.2000bp}{31.2000bp}
    \pgfpathclose
    \pgfusepathqfillstroke
  \end{pgfscope}
  \begin{pgfscope}
    \definecolor{fc}{rgb}{0.0000,0.0000,0.0000}
    \pgfsetfillcolor{fc}
    \pgftransformshift{\pgfqpoint{27.2000bp}{39.2000bp}}
    \pgftransformscale{1.0000}
    \pgftext[base,left]{negative unproven}
  \end{pgfscope}
  \begin{pgfscope}
    \definecolor{fc}{rgb}{1.0000,1.0000,0.0000}
    \pgfsetfillcolor{fc}
    \pgfsetlinewidth{0.6788bp}
    \definecolor{sc}{rgb}{1.0000,1.0000,0.0000}
    \pgfsetstrokecolor{sc}
    \pgfsetmiterjoin
    \pgfsetbuttcap
    \pgfpathqmoveto{19.2000bp}{41.6000bp}
    \pgfpathqcurveto{19.2000bp}{43.3673bp}{17.7673bp}{44.8000bp}{16.0000bp}{44.8000bp}
    \pgfpathqcurveto{14.2327bp}{44.8000bp}{12.8000bp}{43.3673bp}{12.8000bp}{41.6000bp}
    \pgfpathqcurveto{12.8000bp}{39.8327bp}{14.2327bp}{38.4000bp}{16.0000bp}{38.4000bp}
    \pgfpathqcurveto{17.7673bp}{38.4000bp}{19.2000bp}{39.8327bp}{19.2000bp}{41.6000bp}
    \pgfpathclose
    \pgfusepathqfillstroke
  \end{pgfscope}
  \begin{pgfscope}
    \definecolor{fc}{rgb}{0.0000,0.0000,0.0000}
    \pgfsetfillcolor{fc}
    \pgftransformshift{\pgfqpoint{27.2000bp}{49.6000bp}}
    \pgftransformscale{1.0000}
    \pgftext[base,left]{negative proven}
  \end{pgfscope}
  \begin{pgfscope}
    \definecolor{fc}{rgb}{0.0000,0.5020,0.0000}
    \pgfsetfillcolor{fc}
    \pgfsetlinewidth{0.6788bp}
    \definecolor{sc}{rgb}{0.0000,0.5020,0.0000}
    \pgfsetstrokecolor{sc}
    \pgfsetmiterjoin
    \pgfsetbuttcap
    \pgfpathqmoveto{19.2000bp}{52.0000bp}
    \pgfpathqcurveto{19.2000bp}{53.7673bp}{17.7673bp}{55.2000bp}{16.0000bp}{55.2000bp}
    \pgfpathqcurveto{14.2327bp}{55.2000bp}{12.8000bp}{53.7673bp}{12.8000bp}{52.0000bp}
    \pgfpathqcurveto{12.8000bp}{50.2327bp}{14.2327bp}{48.8000bp}{16.0000bp}{48.8000bp}
    \pgfpathqcurveto{17.7673bp}{48.8000bp}{19.2000bp}{50.2327bp}{19.2000bp}{52.0000bp}
    \pgfpathclose
    \pgfusepathqfillstroke
  \end{pgfscope}
  \begin{pgfscope}
    \definecolor{fc}{rgb}{0.0000,0.0000,0.0000}
    \pgfsetfillcolor{fc}
    \pgftransformshift{\pgfqpoint{27.2000bp}{60.0000bp}}
    \pgftransformscale{1.0000}
    \pgftext[base,left]{positive unproven}
  \end{pgfscope}
  \begin{pgfscope}
    \definecolor{fc}{rgb}{1.0000,0.0000,0.0000}
    \pgfsetfillcolor{fc}
    \pgfsetlinewidth{0.6788bp}
    \definecolor{sc}{rgb}{1.0000,0.0000,0.0000}
    \pgfsetstrokecolor{sc}
    \pgfsetmiterjoin
    \pgfsetbuttcap
    \pgfpathqmoveto{19.2000bp}{62.4000bp}
    \pgfpathqcurveto{19.2000bp}{64.1673bp}{17.7673bp}{65.6000bp}{16.0000bp}{65.6000bp}
    \pgfpathqcurveto{14.2327bp}{65.6000bp}{12.8000bp}{64.1673bp}{12.8000bp}{62.4000bp}
    \pgfpathqcurveto{12.8000bp}{60.6327bp}{14.2327bp}{59.2000bp}{16.0000bp}{59.2000bp}
    \pgfpathqcurveto{17.7673bp}{59.2000bp}{19.2000bp}{60.6327bp}{19.2000bp}{62.4000bp}
    \pgfpathclose
    \pgfusepathqfillstroke
  \end{pgfscope}
  \begin{pgfscope}
    \definecolor{fc}{rgb}{0.0000,0.0000,0.0000}
    \pgfsetfillcolor{fc}
    \pgftransformshift{\pgfqpoint{27.2000bp}{70.4000bp}}
    \pgftransformscale{1.0000}
    \pgftext[base,left]{positive proven}
  \end{pgfscope}
  \begin{pgfscope}
    \definecolor{fc}{rgb}{0.0000,0.0000,1.0000}
    \pgfsetfillcolor{fc}
    \pgfsetlinewidth{0.6788bp}
    \definecolor{sc}{rgb}{0.0000,0.0000,1.0000}
    \pgfsetstrokecolor{sc}
    \pgfsetmiterjoin
    \pgfsetbuttcap
    \pgfpathqmoveto{19.2000bp}{72.8000bp}
    \pgfpathqcurveto{19.2000bp}{74.5673bp}{17.7673bp}{76.0000bp}{16.0000bp}{76.0000bp}
    \pgfpathqcurveto{14.2327bp}{76.0000bp}{12.8000bp}{74.5673bp}{12.8000bp}{72.8000bp}
    \pgfpathqcurveto{12.8000bp}{71.0327bp}{14.2327bp}{69.6000bp}{16.0000bp}{69.6000bp}
    \pgfpathqcurveto{17.7673bp}{69.6000bp}{19.2000bp}{71.0327bp}{19.2000bp}{72.8000bp}
    \pgfpathclose
    \pgfusepathqfillstroke
  \end{pgfscope}
  \begin{pgfscope}
    \pgfsetlinewidth{0.6788bp}
    \definecolor{sc}{rgb}{0.0000,0.0000,0.0000}
    \pgfsetstrokecolor{sc}
    \pgfsetmiterjoin
    \pgfsetbuttcap
    \pgfpathqmoveto{16.0000bp}{92.0000bp}
    \pgfpathqlineto{24.0000bp}{96.0000bp}
    \pgfpathqlineto{32.0000bp}{100.0000bp}
    \pgfpathqlineto{40.0000bp}{104.0000bp}
    \pgfpathqlineto{48.0000bp}{108.0000bp}
    \pgfpathqlineto{56.0000bp}{112.0000bp}
    \pgfpathqlineto{64.0000bp}{116.0000bp}
    \pgfpathqlineto{72.0000bp}{120.0000bp}
    \pgfpathqlineto{80.0000bp}{124.0000bp}
    \pgfpathqlineto{88.0000bp}{128.0000bp}
    \pgfpathqlineto{96.0000bp}{132.0000bp}
    \pgfpathqlineto{104.0000bp}{136.0000bp}
    \pgfpathqlineto{112.0000bp}{140.0000bp}
    \pgfpathqlineto{120.0000bp}{144.0000bp}
    \pgfpathqlineto{128.0000bp}{148.0000bp}
    \pgfpathqlineto{136.0000bp}{152.0000bp}
    \pgfpathqlineto{144.0000bp}{156.0000bp}
    \pgfpathqlineto{152.0000bp}{160.0000bp}
    \pgfpathqlineto{160.0000bp}{164.0000bp}
    \pgfpathqlineto{168.0000bp}{168.0000bp}
    \pgfpathqlineto{176.0000bp}{172.0000bp}
    \pgfpathqlineto{184.0000bp}{100.0000bp}
    \pgfpathqlineto{192.0000bp}{96.0000bp}
    \pgfusepathqstroke
  \end{pgfscope}
  \begin{pgfscope}
    \pgfsetlinewidth{0.6788bp}
    \definecolor{sc}{rgb}{1.0000,1.0000,0.0000}
    \pgfsetstrokecolor{sc}
    \pgfsetmiterjoin
    \pgfsetbuttcap
    \pgfpathqmoveto{16.0000bp}{168.0000bp}
    \pgfpathqlineto{24.0000bp}{168.0000bp}
    \pgfpathqlineto{32.0000bp}{168.0000bp}
    \pgfpathqlineto{40.0000bp}{168.0000bp}
    \pgfpathqlineto{48.0000bp}{168.0000bp}
    \pgfpathqlineto{56.0000bp}{168.0000bp}
    \pgfpathqlineto{64.0000bp}{168.0000bp}
    \pgfpathqlineto{72.0000bp}{168.0000bp}
    \pgfpathqlineto{80.0000bp}{168.0000bp}
    \pgfpathqlineto{88.0000bp}{168.0000bp}
    \pgfpathqlineto{96.0000bp}{168.0000bp}
    \pgfpathqlineto{104.0000bp}{168.0000bp}
    \pgfpathqlineto{112.0000bp}{168.0000bp}
    \pgfpathqlineto{120.0000bp}{168.0000bp}
    \pgfpathqlineto{128.0000bp}{168.0000bp}
    \pgfpathqlineto{136.0000bp}{168.0000bp}
    \pgfpathqlineto{144.0000bp}{168.0000bp}
    \pgfpathqlineto{152.0000bp}{168.0000bp}
    \pgfpathqlineto{160.0000bp}{168.0000bp}
    \pgfpathqlineto{168.0000bp}{168.0000bp}
    \pgfpathqlineto{176.0000bp}{168.0000bp}
    \pgfpathqlineto{184.0000bp}{152.0000bp}
    \pgfpathqlineto{192.0000bp}{152.0000bp}
    \pgfusepathqstroke
  \end{pgfscope}
  \begin{pgfscope}
    \pgfsetlinewidth{0.6788bp}
    \definecolor{sc}{rgb}{0.0000,0.5020,0.0000}
    \pgfsetstrokecolor{sc}
    \pgfsetmiterjoin
    \pgfsetbuttcap
    \pgfpathqmoveto{16.0000bp}{88.0000bp}
    \pgfpathqlineto{24.0000bp}{88.0000bp}
    \pgfpathqlineto{32.0000bp}{88.0000bp}
    \pgfpathqlineto{40.0000bp}{88.0000bp}
    \pgfpathqlineto{48.0000bp}{88.0000bp}
    \pgfpathqlineto{56.0000bp}{88.0000bp}
    \pgfpathqlineto{64.0000bp}{88.0000bp}
    \pgfpathqlineto{72.0000bp}{88.0000bp}
    \pgfpathqlineto{80.0000bp}{88.0000bp}
    \pgfpathqlineto{88.0000bp}{88.0000bp}
    \pgfpathqlineto{96.0000bp}{88.0000bp}
    \pgfpathqlineto{104.0000bp}{88.0000bp}
    \pgfpathqlineto{112.0000bp}{88.0000bp}
    \pgfpathqlineto{120.0000bp}{88.0000bp}
    \pgfpathqlineto{128.0000bp}{88.0000bp}
    \pgfpathqlineto{136.0000bp}{88.0000bp}
    \pgfpathqlineto{144.0000bp}{88.0000bp}
    \pgfpathqlineto{152.0000bp}{88.0000bp}
    \pgfpathqlineto{160.0000bp}{88.0000bp}
    \pgfpathqlineto{168.0000bp}{88.0000bp}
    \pgfpathqlineto{176.0000bp}{88.0000bp}
    \pgfpathqlineto{184.0000bp}{88.0000bp}
    \pgfpathqlineto{192.0000bp}{88.0000bp}
    \pgfusepathqstroke
  \end{pgfscope}
  \begin{pgfscope}
    \pgfsetlinewidth{0.6788bp}
    \definecolor{sc}{rgb}{1.0000,0.0000,0.0000}
    \pgfsetstrokecolor{sc}
    \pgfsetmiterjoin
    \pgfsetbuttcap
    \pgfpathqmoveto{16.0000bp}{168.0000bp}
    \pgfpathqlineto{24.0000bp}{168.0000bp}
    \pgfpathqlineto{32.0000bp}{168.0000bp}
    \pgfpathqlineto{40.0000bp}{168.0000bp}
    \pgfpathqlineto{48.0000bp}{168.0000bp}
    \pgfpathqlineto{56.0000bp}{168.0000bp}
    \pgfpathqlineto{64.0000bp}{168.0000bp}
    \pgfpathqlineto{72.0000bp}{168.0000bp}
    \pgfpathqlineto{80.0000bp}{168.0000bp}
    \pgfpathqlineto{88.0000bp}{168.0000bp}
    \pgfpathqlineto{96.0000bp}{168.0000bp}
    \pgfpathqlineto{104.0000bp}{168.0000bp}
    \pgfpathqlineto{112.0000bp}{168.0000bp}
    \pgfpathqlineto{120.0000bp}{168.0000bp}
    \pgfpathqlineto{128.0000bp}{168.0000bp}
    \pgfpathqlineto{136.0000bp}{168.0000bp}
    \pgfpathqlineto{144.0000bp}{168.0000bp}
    \pgfpathqlineto{152.0000bp}{168.0000bp}
    \pgfpathqlineto{160.0000bp}{168.0000bp}
    \pgfpathqlineto{168.0000bp}{168.0000bp}
    \pgfpathqlineto{176.0000bp}{168.0000bp}
    \pgfpathqlineto{184.0000bp}{100.0000bp}
    \pgfpathqlineto{192.0000bp}{96.0000bp}
    \pgfusepathqstroke
  \end{pgfscope}
  \begin{pgfscope}
    \pgfsetlinewidth{0.6788bp}
    \definecolor{sc}{rgb}{0.0000,0.0000,1.0000}
    \pgfsetstrokecolor{sc}
    \pgfsetmiterjoin
    \pgfsetbuttcap
    \pgfpathqmoveto{16.0000bp}{88.0000bp}
    \pgfpathqlineto{24.0000bp}{88.0000bp}
    \pgfpathqlineto{32.0000bp}{88.0000bp}
    \pgfpathqlineto{40.0000bp}{88.0000bp}
    \pgfpathqlineto{48.0000bp}{88.0000bp}
    \pgfpathqlineto{56.0000bp}{88.0000bp}
    \pgfpathqlineto{64.0000bp}{88.0000bp}
    \pgfpathqlineto{72.0000bp}{88.0000bp}
    \pgfpathqlineto{80.0000bp}{88.0000bp}
    \pgfpathqlineto{88.0000bp}{88.0000bp}
    \pgfpathqlineto{96.0000bp}{88.0000bp}
    \pgfpathqlineto{104.0000bp}{88.0000bp}
    \pgfpathqlineto{112.0000bp}{88.0000bp}
    \pgfpathqlineto{120.0000bp}{88.0000bp}
    \pgfpathqlineto{128.0000bp}{88.0000bp}
    \pgfpathqlineto{136.0000bp}{88.0000bp}
    \pgfpathqlineto{144.0000bp}{88.0000bp}
    \pgfpathqlineto{152.0000bp}{88.0000bp}
    \pgfpathqlineto{160.0000bp}{88.0000bp}
    \pgfpathqlineto{168.0000bp}{88.0000bp}
    \pgfpathqlineto{176.0000bp}{88.0000bp}
    \pgfpathqlineto{184.0000bp}{92.0000bp}
    \pgfpathqlineto{192.0000bp}{96.0000bp}
    \pgfusepathqstroke
  \end{pgfscope}
  \begin{pgfscope}
    \pgfsetlinewidth{0.6788bp}
    \definecolor{sc}{rgb}{1.0000,0.0000,0.0000}
    \pgfsetstrokecolor{sc}
    \pgfsetmiterjoin
    \pgfsetbuttcap
    \pgfpathqmoveto{200.0000bp}{88.0000bp}
    \pgfpathqlineto{200.0000bp}{84.0000bp}
    \pgfusepathqstroke
  \end{pgfscope}
  \begin{pgfscope}
    \pgfsetlinewidth{0.6788bp}
    \definecolor{sc}{rgb}{1.0000,0.0000,0.0000}
    \pgfsetstrokecolor{sc}
    \pgfsetmiterjoin
    \pgfsetbuttcap
    \pgfpathqmoveto{192.0000bp}{88.0000bp}
    \pgfpathqlineto{192.0000bp}{84.0000bp}
    \pgfusepathqstroke
  \end{pgfscope}
  \begin{pgfscope}
    \pgfsetlinewidth{0.6788bp}
    \definecolor{sc}{rgb}{1.0000,0.0000,0.0000}
    \pgfsetstrokecolor{sc}
    \pgfsetmiterjoin
    \pgfsetbuttcap
    \pgfpathqmoveto{152.0000bp}{88.0000bp}
    \pgfpathqlineto{152.0000bp}{84.0000bp}
    \pgfusepathqstroke
  \end{pgfscope}
  \begin{pgfscope}
    \pgfsetlinewidth{0.6788bp}
    \definecolor{sc}{rgb}{1.0000,0.0000,0.0000}
    \pgfsetstrokecolor{sc}
    \pgfsetmiterjoin
    \pgfsetbuttcap
    \pgfpathqmoveto{16.0000bp}{88.0000bp}
    \pgfpathqlineto{16.0000bp}{84.0000bp}
    \pgfusepathqstroke
  \end{pgfscope}
  \begin{pgfscope}
    \pgfsetlinewidth{0.6788bp}
    \definecolor{sc}{rgb}{0.0000,0.0000,0.0000}
    \pgfsetstrokecolor{sc}
    \pgfsetmiterjoin
    \pgfsetbuttcap
    \pgfpathqmoveto{176.0000bp}{88.0000bp}
    \pgfpathqlineto{176.0000bp}{84.0000bp}
    \pgfusepathqstroke
  \end{pgfscope}
  \begin{pgfscope}
    \pgfsetlinewidth{0.6788bp}
    \definecolor{sc}{rgb}{0.0000,0.0000,0.0000}
    \pgfsetstrokecolor{sc}
    \pgfsetmiterjoin
    \pgfsetbuttcap
    \pgfpathqmoveto{136.0000bp}{88.0000bp}
    \pgfpathqlineto{136.0000bp}{84.0000bp}
    \pgfusepathqstroke
  \end{pgfscope}
  \begin{pgfscope}
    \pgfsetlinewidth{0.6788bp}
    \definecolor{sc}{rgb}{0.0000,0.0000,0.0000}
    \pgfsetstrokecolor{sc}
    \pgfsetmiterjoin
    \pgfsetbuttcap
    \pgfpathqmoveto{96.0000bp}{88.0000bp}
    \pgfpathqlineto{96.0000bp}{84.0000bp}
    \pgfusepathqstroke
  \end{pgfscope}
  \begin{pgfscope}
    \pgfsetlinewidth{0.6788bp}
    \definecolor{sc}{rgb}{0.0000,0.0000,0.0000}
    \pgfsetstrokecolor{sc}
    \pgfsetmiterjoin
    \pgfsetbuttcap
    \pgfpathqmoveto{56.0000bp}{88.0000bp}
    \pgfpathqlineto{56.0000bp}{84.0000bp}
    \pgfusepathqstroke
  \end{pgfscope}
  \begin{pgfscope}
    \definecolor{fc}{rgb}{0.0000,0.0000,0.0000}
    \pgfsetfillcolor{fc}
    \pgftransformshift{\pgfqpoint{0.0000bp}{165.6000bp}}
    \pgftransformscale{1.0000}
    \pgftext[base,left]{$\mathbb{L}_A$}
  \end{pgfscope}
  \begin{pgfscope}
    \pgfsetlinewidth{0.6788bp}
    \definecolor{sc}{rgb}{0.0000,0.0000,0.0000}
    \pgfsetstrokecolor{sc}
    \pgfsetmiterjoin
    \pgfsetbuttcap
    \pgfpathqmoveto{16.0000bp}{168.0000bp}
    \pgfpathqlineto{14.4000bp}{168.0000bp}
    \pgfusepathqstroke
  \end{pgfscope}
  \begin{pgfscope}
    \pgfsetlinewidth{0.6788bp}
    \definecolor{sc}{rgb}{0.0000,0.0000,0.0000}
    \pgfsetstrokecolor{sc}
    \pgfsetmiterjoin
    \pgfsetbuttcap
    \pgfpathqmoveto{16.0000bp}{88.0000bp}
    \pgfpathqlineto{16.0000bp}{172.0000bp}
    \pgfusepathqstroke
  \end{pgfscope}
  \begin{pgfscope}
    \pgfsetlinewidth{0.6788bp}
    \definecolor{sc}{rgb}{0.0000,0.0000,0.0000}
    \pgfsetstrokecolor{sc}
    \pgfsetmiterjoin
    \pgfsetbuttcap
    \pgfpathqmoveto{16.0000bp}{88.0000bp}
    \pgfpathqlineto{200.0000bp}{88.0000bp}
    \pgfusepathqstroke
  \end{pgfscope}
\end{pgfpicture}

        \label{fig:ex:ca:hgma:ex:move-h}
    \caption{move-v preconditions}\label{fig:ex:ca:hgma:ex:disconnected}
\end{figure}

\begin{figure}
    \centering
    \begin{pgfpicture}
  \pgfpathrectangle{\pgfpointorigin}{\pgfqpoint{200.0000bp}{200.0000bp}}
  \pgfusepath{use as bounding box}
  \begin{pgfscope}
    \definecolor{fc}{rgb}{0.0000,0.0000,0.0000}
    \pgfsetfillcolor{fc}
    \pgftransformcm{1.0000}{0.0000}{0.0000}{1.0000}{\pgfqpoint{5.2308bp}{49.0000bp}}
    \pgftransformscale{0.1923}
    \pgftext[base,left]{candidates}
  \end{pgfscope}
  \begin{pgfscope}
    \definecolor{fc}{rgb}{0.0000,0.0000,0.0000}
    \pgfsetfillcolor{fc}
    \pgfsetlinewidth{0.5722bp}
    \definecolor{sc}{rgb}{0.0000,0.0000,0.0000}
    \pgfsetstrokecolor{sc}
    \pgfsetmiterjoin
    \pgfsetbuttcap
    \pgfpathqmoveto{3.6923bp}{49.4615bp}
    \pgfpathqcurveto{3.6923bp}{49.8014bp}{3.4168bp}{50.0769bp}{3.0769bp}{50.0769bp}
    \pgfpathqcurveto{2.7371bp}{50.0769bp}{2.4615bp}{49.8014bp}{2.4615bp}{49.4615bp}
    \pgfpathqcurveto{2.4615bp}{49.1217bp}{2.7371bp}{48.8462bp}{3.0769bp}{48.8462bp}
    \pgfpathqcurveto{3.4168bp}{48.8462bp}{3.6923bp}{49.1217bp}{3.6923bp}{49.4615bp}
    \pgfpathclose
    \pgfusepathqfillstroke
  \end{pgfscope}
  \begin{pgfscope}
    \definecolor{fc}{rgb}{0.0000,0.0000,0.0000}
    \pgfsetfillcolor{fc}
    \pgftransformcm{1.0000}{0.0000}{0.0000}{1.0000}{\pgfqpoint{5.2308bp}{51.0000bp}}
    \pgftransformscale{0.1923}
    \pgftext[base,left]{negative unproven}
  \end{pgfscope}
  \begin{pgfscope}
    \definecolor{fc}{rgb}{1.0000,1.0000,0.0000}
    \pgfsetfillcolor{fc}
    \pgfsetlinewidth{0.5722bp}
    \definecolor{sc}{rgb}{1.0000,1.0000,0.0000}
    \pgfsetstrokecolor{sc}
    \pgfsetmiterjoin
    \pgfsetbuttcap
    \pgfpathqmoveto{3.6923bp}{51.4615bp}
    \pgfpathqcurveto{3.6923bp}{51.8014bp}{3.4168bp}{52.0769bp}{3.0769bp}{52.0769bp}
    \pgfpathqcurveto{2.7371bp}{52.0769bp}{2.4615bp}{51.8014bp}{2.4615bp}{51.4615bp}
    \pgfpathqcurveto{2.4615bp}{51.1217bp}{2.7371bp}{50.8462bp}{3.0769bp}{50.8462bp}
    \pgfpathqcurveto{3.4168bp}{50.8462bp}{3.6923bp}{51.1217bp}{3.6923bp}{51.4615bp}
    \pgfpathclose
    \pgfusepathqfillstroke
  \end{pgfscope}
  \begin{pgfscope}
    \definecolor{fc}{rgb}{0.0000,0.0000,0.0000}
    \pgfsetfillcolor{fc}
    \pgftransformcm{1.0000}{0.0000}{0.0000}{1.0000}{\pgfqpoint{5.2308bp}{53.0000bp}}
    \pgftransformscale{0.1923}
    \pgftext[base,left]{negative proven}
  \end{pgfscope}
  \begin{pgfscope}
    \definecolor{fc}{rgb}{0.0000,0.5020,0.0000}
    \pgfsetfillcolor{fc}
    \pgfsetlinewidth{0.5722bp}
    \definecolor{sc}{rgb}{0.0000,0.5020,0.0000}
    \pgfsetstrokecolor{sc}
    \pgfsetmiterjoin
    \pgfsetbuttcap
    \pgfpathqmoveto{3.6923bp}{53.4615bp}
    \pgfpathqcurveto{3.6923bp}{53.8014bp}{3.4168bp}{54.0769bp}{3.0769bp}{54.0769bp}
    \pgfpathqcurveto{2.7371bp}{54.0769bp}{2.4615bp}{53.8014bp}{2.4615bp}{53.4615bp}
    \pgfpathqcurveto{2.4615bp}{53.1217bp}{2.7371bp}{52.8462bp}{3.0769bp}{52.8462bp}
    \pgfpathqcurveto{3.4168bp}{52.8462bp}{3.6923bp}{53.1217bp}{3.6923bp}{53.4615bp}
    \pgfpathclose
    \pgfusepathqfillstroke
  \end{pgfscope}
  \begin{pgfscope}
    \definecolor{fc}{rgb}{0.0000,0.0000,0.0000}
    \pgfsetfillcolor{fc}
    \pgftransformcm{1.0000}{0.0000}{0.0000}{1.0000}{\pgfqpoint{5.2308bp}{55.0000bp}}
    \pgftransformscale{0.1923}
    \pgftext[base,left]{positive unproven}
  \end{pgfscope}
  \begin{pgfscope}
    \definecolor{fc}{rgb}{1.0000,0.0000,0.0000}
    \pgfsetfillcolor{fc}
    \pgfsetlinewidth{0.5722bp}
    \definecolor{sc}{rgb}{1.0000,0.0000,0.0000}
    \pgfsetstrokecolor{sc}
    \pgfsetmiterjoin
    \pgfsetbuttcap
    \pgfpathqmoveto{3.6923bp}{55.4615bp}
    \pgfpathqcurveto{3.6923bp}{55.8014bp}{3.4168bp}{56.0769bp}{3.0769bp}{56.0769bp}
    \pgfpathqcurveto{2.7371bp}{56.0769bp}{2.4615bp}{55.8014bp}{2.4615bp}{55.4615bp}
    \pgfpathqcurveto{2.4615bp}{55.1217bp}{2.7371bp}{54.8462bp}{3.0769bp}{54.8462bp}
    \pgfpathqcurveto{3.4168bp}{54.8462bp}{3.6923bp}{55.1217bp}{3.6923bp}{55.4615bp}
    \pgfpathclose
    \pgfusepathqfillstroke
  \end{pgfscope}
  \begin{pgfscope}
    \definecolor{fc}{rgb}{0.0000,0.0000,0.0000}
    \pgfsetfillcolor{fc}
    \pgftransformcm{1.0000}{0.0000}{0.0000}{1.0000}{\pgfqpoint{5.2308bp}{57.0000bp}}
    \pgftransformscale{0.1923}
    \pgftext[base,left]{positive proven}
  \end{pgfscope}
  \begin{pgfscope}
    \definecolor{fc}{rgb}{0.0000,0.0000,1.0000}
    \pgfsetfillcolor{fc}
    \pgfsetlinewidth{0.5722bp}
    \definecolor{sc}{rgb}{0.0000,0.0000,1.0000}
    \pgfsetstrokecolor{sc}
    \pgfsetmiterjoin
    \pgfsetbuttcap
    \pgfpathqmoveto{3.6923bp}{57.4615bp}
    \pgfpathqcurveto{3.6923bp}{57.8014bp}{3.4168bp}{58.0769bp}{3.0769bp}{58.0769bp}
    \pgfpathqcurveto{2.7371bp}{58.0769bp}{2.4615bp}{57.8014bp}{2.4615bp}{57.4615bp}
    \pgfpathqcurveto{2.4615bp}{57.1217bp}{2.7371bp}{56.8462bp}{3.0769bp}{56.8462bp}
    \pgfpathqcurveto{3.4168bp}{56.8462bp}{3.6923bp}{57.1217bp}{3.6923bp}{57.4615bp}
    \pgfpathclose
    \pgfusepathqfillstroke
  \end{pgfscope}
  \begin{pgfscope}
    \pgfsetlinewidth{0.5722bp}
    \definecolor{sc}{rgb}{0.0000,0.0000,0.0000}
    \pgfsetstrokecolor{sc}
    \pgfsetmiterjoin
    \pgfsetbuttcap
    \pgfpathqmoveto{3.0769bp}{61.1538bp}
    \pgfpathqlineto{4.6154bp}{61.9231bp}
    \pgfpathqlineto{6.1538bp}{62.6923bp}
    \pgfpathqlineto{7.6923bp}{63.4615bp}
    \pgfpathqlineto{9.2308bp}{64.2308bp}
    \pgfpathqlineto{10.7692bp}{65.0000bp}
    \pgfpathqlineto{12.3077bp}{65.7692bp}
    \pgfpathqlineto{13.8462bp}{66.5385bp}
    \pgfpathqlineto{15.3846bp}{67.3077bp}
    \pgfpathqlineto{16.9231bp}{68.0769bp}
    \pgfpathqlineto{18.4615bp}{68.8462bp}
    \pgfpathqlineto{20.0000bp}{69.6154bp}
    \pgfpathqlineto{21.5385bp}{70.3846bp}
    \pgfpathqlineto{23.0769bp}{71.1538bp}
    \pgfpathqlineto{24.6154bp}{71.9231bp}
    \pgfpathqlineto{26.1538bp}{72.6923bp}
    \pgfpathqlineto{27.6923bp}{73.4615bp}
    \pgfpathqlineto{29.2308bp}{74.2308bp}
    \pgfpathqlineto{30.7692bp}{75.0000bp}
    \pgfpathqlineto{32.3077bp}{75.7692bp}
    \pgfpathqlineto{33.8462bp}{76.5385bp}
    \pgfpathqlineto{35.3846bp}{77.3077bp}
    \pgfpathqlineto{36.9231bp}{78.0769bp}
    \pgfpathqlineto{38.4615bp}{78.8462bp}
    \pgfpathqlineto{40.0000bp}{79.6154bp}
    \pgfpathqlineto{41.5385bp}{80.3846bp}
    \pgfpathqlineto{43.0769bp}{81.1538bp}
    \pgfpathqlineto{44.6154bp}{81.9231bp}
    \pgfpathqlineto{46.1538bp}{82.6923bp}
    \pgfpathqlineto{47.6923bp}{83.4615bp}
    \pgfpathqlineto{49.2308bp}{84.2308bp}
    \pgfpathqlineto{50.7692bp}{85.0000bp}
    \pgfpathqlineto{52.3077bp}{85.7692bp}
    \pgfpathqlineto{53.8462bp}{86.5385bp}
    \pgfpathqlineto{55.3846bp}{87.3077bp}
    \pgfpathqlineto{56.9231bp}{88.0769bp}
    \pgfpathqlineto{58.4615bp}{88.8462bp}
    \pgfpathqlineto{60.0000bp}{89.6154bp}
    \pgfpathqlineto{61.5385bp}{90.3846bp}
    \pgfpathqlineto{63.0769bp}{91.1538bp}
    \pgfpathqlineto{64.6154bp}{91.9231bp}
    \pgfpathqlineto{66.1538bp}{92.6923bp}
    \pgfpathqlineto{67.6923bp}{93.4615bp}
    \pgfpathqlineto{69.2308bp}{94.2308bp}
    \pgfpathqlineto{70.7692bp}{95.0000bp}
    \pgfpathqlineto{72.3077bp}{95.7692bp}
    \pgfpathqlineto{73.8462bp}{96.5385bp}
    \pgfpathqlineto{75.3846bp}{97.3077bp}
    \pgfpathqlineto{76.9231bp}{98.0769bp}
    \pgfpathqlineto{78.4615bp}{98.8462bp}
    \pgfpathqlineto{80.0000bp}{99.6154bp}
    \pgfpathqlineto{81.5385bp}{100.3846bp}
    \pgfpathqlineto{83.0769bp}{101.1538bp}
    \pgfpathqlineto{84.6154bp}{101.9231bp}
    \pgfpathqlineto{86.1538bp}{102.6923bp}
    \pgfpathqlineto{87.6923bp}{103.4615bp}
    \pgfpathqlineto{89.2308bp}{104.2308bp}
    \pgfpathqlineto{90.7692bp}{105.0000bp}
    \pgfpathqlineto{92.3077bp}{105.7692bp}
    \pgfpathqlineto{93.8462bp}{106.5385bp}
    \pgfpathqlineto{95.3846bp}{107.3077bp}
    \pgfpathqlineto{96.9231bp}{108.0769bp}
    \pgfpathqlineto{98.4615bp}{108.8462bp}
    \pgfpathqlineto{100.0000bp}{109.6154bp}
    \pgfpathqlineto{101.5385bp}{110.3846bp}
    \pgfpathqlineto{103.0769bp}{111.1538bp}
    \pgfpathqlineto{104.6154bp}{111.9231bp}
    \pgfpathqlineto{106.1538bp}{112.6923bp}
    \pgfpathqlineto{107.6923bp}{113.4615bp}
    \pgfpathqlineto{109.2308bp}{114.2308bp}
    \pgfpathqlineto{110.7692bp}{115.0000bp}
    \pgfpathqlineto{112.3077bp}{115.7692bp}
    \pgfpathqlineto{113.8462bp}{116.5385bp}
    \pgfpathqlineto{115.3846bp}{117.3077bp}
    \pgfpathqlineto{116.9231bp}{118.0769bp}
    \pgfpathqlineto{118.4615bp}{118.8462bp}
    \pgfpathqlineto{120.0000bp}{119.6154bp}
    \pgfpathqlineto{121.5385bp}{120.3846bp}
    \pgfpathqlineto{123.0769bp}{121.1538bp}
    \pgfpathqlineto{124.6154bp}{121.9231bp}
    \pgfpathqlineto{126.1538bp}{122.6923bp}
    \pgfpathqlineto{127.6923bp}{123.4615bp}
    \pgfpathqlineto{129.2308bp}{124.2308bp}
    \pgfpathqlineto{130.7692bp}{125.0000bp}
    \pgfpathqlineto{132.3077bp}{125.7692bp}
    \pgfpathqlineto{133.8462bp}{126.5385bp}
    \pgfpathqlineto{135.3846bp}{127.3077bp}
    \pgfpathqlineto{136.9231bp}{128.0769bp}
    \pgfpathqlineto{138.4615bp}{128.8462bp}
    \pgfpathqlineto{140.0000bp}{129.6154bp}
    \pgfpathqlineto{141.5385bp}{130.3846bp}
    \pgfpathqlineto{143.0769bp}{131.1538bp}
    \pgfpathqlineto{144.6154bp}{131.9231bp}
    \pgfpathqlineto{146.1538bp}{132.6923bp}
    \pgfpathqlineto{147.6923bp}{133.4615bp}
    \pgfpathqlineto{149.2308bp}{134.2308bp}
    \pgfpathqlineto{150.7692bp}{135.0000bp}
    \pgfpathqlineto{152.3077bp}{135.7692bp}
    \pgfpathqlineto{153.8462bp}{136.5385bp}
    \pgfpathqlineto{155.3846bp}{137.3077bp}
    \pgfpathqlineto{156.9231bp}{138.0769bp}
    \pgfpathqlineto{158.4615bp}{138.8462bp}
    \pgfpathqlineto{160.0000bp}{139.6154bp}
    \pgfpathqlineto{161.5385bp}{140.3846bp}
    \pgfpathqlineto{163.0769bp}{141.1538bp}
    \pgfpathqlineto{164.6154bp}{141.9231bp}
    \pgfpathqlineto{166.1538bp}{142.6923bp}
    \pgfpathqlineto{167.6923bp}{143.4615bp}
    \pgfpathqlineto{169.2308bp}{144.2308bp}
    \pgfpathqlineto{170.7692bp}{145.0000bp}
    \pgfpathqlineto{172.3077bp}{145.7692bp}
    \pgfpathqlineto{173.8462bp}{146.5385bp}
    \pgfpathqlineto{175.3846bp}{147.3077bp}
    \pgfpathqlineto{176.9231bp}{148.0769bp}
    \pgfpathqlineto{178.4615bp}{148.8462bp}
    \pgfpathqlineto{180.0000bp}{149.6154bp}
    \pgfpathqlineto{181.5385bp}{150.3846bp}
    \pgfpathqlineto{183.0769bp}{151.1538bp}
    \pgfpathqlineto{184.6154bp}{151.1538bp}
    \pgfpathqlineto{186.1538bp}{135.7692bp}
    \pgfpathqlineto{187.6923bp}{97.3077bp}
    \pgfpathqlineto{189.2308bp}{95.0000bp}
    \pgfpathqlineto{190.7692bp}{95.0000bp}
    \pgfpathqlineto{192.3077bp}{78.8462bp}
    \pgfpathqlineto{193.8462bp}{75.0000bp}
    \pgfpathqlineto{195.3846bp}{75.7692bp}
    \pgfpathqlineto{196.9231bp}{66.5385bp}
    \pgfpathqlineto{198.4615bp}{66.5385bp}
    \pgfusepathqstroke
  \end{pgfscope}
  \begin{pgfscope}
    \pgfsetlinewidth{0.5722bp}
    \definecolor{sc}{rgb}{1.0000,1.0000,0.0000}
    \pgfsetstrokecolor{sc}
    \pgfsetmiterjoin
    \pgfsetbuttcap
    \pgfpathqmoveto{3.0769bp}{109.6154bp}
    \pgfpathqlineto{4.6154bp}{109.6154bp}
    \pgfpathqlineto{6.1538bp}{109.6154bp}
    \pgfpathqlineto{7.6923bp}{109.6154bp}
    \pgfpathqlineto{9.2308bp}{109.6154bp}
    \pgfpathqlineto{10.7692bp}{109.6154bp}
    \pgfpathqlineto{12.3077bp}{109.6154bp}
    \pgfpathqlineto{13.8462bp}{109.6154bp}
    \pgfpathqlineto{15.3846bp}{109.6154bp}
    \pgfpathqlineto{16.9231bp}{109.6154bp}
    \pgfpathqlineto{18.4615bp}{109.6154bp}
    \pgfpathqlineto{20.0000bp}{109.6154bp}
    \pgfpathqlineto{21.5385bp}{109.6154bp}
    \pgfpathqlineto{23.0769bp}{109.6154bp}
    \pgfpathqlineto{24.6154bp}{109.6154bp}
    \pgfpathqlineto{26.1538bp}{109.6154bp}
    \pgfpathqlineto{27.6923bp}{109.6154bp}
    \pgfpathqlineto{29.2308bp}{109.6154bp}
    \pgfpathqlineto{30.7692bp}{109.6154bp}
    \pgfpathqlineto{32.3077bp}{109.6154bp}
    \pgfpathqlineto{33.8462bp}{109.6154bp}
    \pgfpathqlineto{35.3846bp}{109.6154bp}
    \pgfpathqlineto{36.9231bp}{109.6154bp}
    \pgfpathqlineto{38.4615bp}{109.6154bp}
    \pgfpathqlineto{40.0000bp}{109.6154bp}
    \pgfpathqlineto{41.5385bp}{109.6154bp}
    \pgfpathqlineto{43.0769bp}{109.6154bp}
    \pgfpathqlineto{44.6154bp}{109.6154bp}
    \pgfpathqlineto{46.1538bp}{109.6154bp}
    \pgfpathqlineto{47.6923bp}{109.6154bp}
    \pgfpathqlineto{49.2308bp}{109.6154bp}
    \pgfpathqlineto{50.7692bp}{109.6154bp}
    \pgfpathqlineto{52.3077bp}{109.6154bp}
    \pgfpathqlineto{53.8462bp}{109.6154bp}
    \pgfpathqlineto{55.3846bp}{109.6154bp}
    \pgfpathqlineto{56.9231bp}{109.6154bp}
    \pgfpathqlineto{58.4615bp}{109.6154bp}
    \pgfpathqlineto{60.0000bp}{109.6154bp}
    \pgfpathqlineto{61.5385bp}{109.6154bp}
    \pgfpathqlineto{63.0769bp}{109.6154bp}
    \pgfpathqlineto{64.6154bp}{109.6154bp}
    \pgfpathqlineto{66.1538bp}{109.6154bp}
    \pgfpathqlineto{67.6923bp}{109.6154bp}
    \pgfpathqlineto{69.2308bp}{109.6154bp}
    \pgfpathqlineto{70.7692bp}{109.6154bp}
    \pgfpathqlineto{72.3077bp}{109.6154bp}
    \pgfpathqlineto{73.8462bp}{109.6154bp}
    \pgfpathqlineto{75.3846bp}{109.6154bp}
    \pgfpathqlineto{76.9231bp}{109.6154bp}
    \pgfpathqlineto{78.4615bp}{109.6154bp}
    \pgfpathqlineto{80.0000bp}{109.6154bp}
    \pgfpathqlineto{81.5385bp}{109.6154bp}
    \pgfpathqlineto{83.0769bp}{109.6154bp}
    \pgfpathqlineto{84.6154bp}{109.6154bp}
    \pgfpathqlineto{86.1538bp}{109.6154bp}
    \pgfpathqlineto{87.6923bp}{109.6154bp}
    \pgfpathqlineto{89.2308bp}{109.6154bp}
    \pgfpathqlineto{90.7692bp}{109.6154bp}
    \pgfpathqlineto{92.3077bp}{109.6154bp}
    \pgfpathqlineto{93.8462bp}{109.6154bp}
    \pgfpathqlineto{95.3846bp}{109.6154bp}
    \pgfpathqlineto{96.9231bp}{109.6154bp}
    \pgfpathqlineto{98.4615bp}{109.6154bp}
    \pgfpathqlineto{100.0000bp}{109.6154bp}
    \pgfpathqlineto{101.5385bp}{109.6154bp}
    \pgfpathqlineto{103.0769bp}{109.6154bp}
    \pgfpathqlineto{104.6154bp}{109.6154bp}
    \pgfpathqlineto{106.1538bp}{109.6154bp}
    \pgfpathqlineto{107.6923bp}{109.6154bp}
    \pgfpathqlineto{109.2308bp}{109.6154bp}
    \pgfpathqlineto{110.7692bp}{109.6154bp}
    \pgfpathqlineto{112.3077bp}{109.6154bp}
    \pgfpathqlineto{113.8462bp}{109.6154bp}
    \pgfpathqlineto{115.3846bp}{109.6154bp}
    \pgfpathqlineto{116.9231bp}{109.6154bp}
    \pgfpathqlineto{118.4615bp}{109.6154bp}
    \pgfpathqlineto{120.0000bp}{109.6154bp}
    \pgfpathqlineto{121.5385bp}{109.6154bp}
    \pgfpathqlineto{123.0769bp}{109.6154bp}
    \pgfpathqlineto{124.6154bp}{109.6154bp}
    \pgfpathqlineto{126.1538bp}{109.6154bp}
    \pgfpathqlineto{127.6923bp}{109.6154bp}
    \pgfpathqlineto{129.2308bp}{109.6154bp}
    \pgfpathqlineto{130.7692bp}{109.6154bp}
    \pgfpathqlineto{132.3077bp}{109.6154bp}
    \pgfpathqlineto{133.8462bp}{109.6154bp}
    \pgfpathqlineto{135.3846bp}{109.6154bp}
    \pgfpathqlineto{136.9231bp}{109.6154bp}
    \pgfpathqlineto{138.4615bp}{109.6154bp}
    \pgfpathqlineto{140.0000bp}{109.6154bp}
    \pgfpathqlineto{141.5385bp}{109.6154bp}
    \pgfpathqlineto{143.0769bp}{109.6154bp}
    \pgfpathqlineto{144.6154bp}{109.6154bp}
    \pgfpathqlineto{146.1538bp}{109.6154bp}
    \pgfpathqlineto{147.6923bp}{109.6154bp}
    \pgfpathqlineto{149.2308bp}{109.6154bp}
    \pgfpathqlineto{150.7692bp}{109.6154bp}
    \pgfpathqlineto{152.3077bp}{109.6154bp}
    \pgfpathqlineto{153.8462bp}{109.6154bp}
    \pgfpathqlineto{155.3846bp}{109.6154bp}
    \pgfpathqlineto{156.9231bp}{109.6154bp}
    \pgfpathqlineto{158.4615bp}{109.6154bp}
    \pgfpathqlineto{160.0000bp}{109.6154bp}
    \pgfpathqlineto{161.5385bp}{109.6154bp}
    \pgfpathqlineto{163.0769bp}{109.6154bp}
    \pgfpathqlineto{164.6154bp}{109.6154bp}
    \pgfpathqlineto{166.1538bp}{109.6154bp}
    \pgfpathqlineto{167.6923bp}{109.6154bp}
    \pgfpathqlineto{169.2308bp}{109.6154bp}
    \pgfpathqlineto{170.7692bp}{109.6154bp}
    \pgfpathqlineto{172.3077bp}{109.6154bp}
    \pgfpathqlineto{173.8462bp}{109.6154bp}
    \pgfpathqlineto{175.3846bp}{109.6154bp}
    \pgfpathqlineto{176.9231bp}{109.6154bp}
    \pgfpathqlineto{178.4615bp}{109.6154bp}
    \pgfpathqlineto{180.0000bp}{109.6154bp}
    \pgfpathqlineto{181.5385bp}{109.6154bp}
    \pgfpathqlineto{183.0769bp}{109.6154bp}
    \pgfpathqlineto{184.6154bp}{104.2308bp}
    \pgfpathqlineto{186.1538bp}{104.2308bp}
    \pgfpathqlineto{187.6923bp}{104.2308bp}
    \pgfpathqlineto{189.2308bp}{104.2308bp}
    \pgfpathqlineto{190.7692bp}{104.2308bp}
    \pgfpathqlineto{192.3077bp}{104.2308bp}
    \pgfpathqlineto{193.8462bp}{104.2308bp}
    \pgfpathqlineto{195.3846bp}{104.2308bp}
    \pgfpathqlineto{196.9231bp}{104.2308bp}
    \pgfpathqlineto{198.4615bp}{104.2308bp}
    \pgfusepathqstroke
  \end{pgfscope}
  \begin{pgfscope}
    \pgfsetlinewidth{0.5722bp}
    \definecolor{sc}{rgb}{0.0000,0.5020,0.0000}
    \pgfsetstrokecolor{sc}
    \pgfsetmiterjoin
    \pgfsetbuttcap
    \pgfpathqmoveto{3.0769bp}{60.3846bp}
    \pgfpathqlineto{4.6154bp}{60.3846bp}
    \pgfpathqlineto{6.1538bp}{60.3846bp}
    \pgfpathqlineto{7.6923bp}{60.3846bp}
    \pgfpathqlineto{9.2308bp}{60.3846bp}
    \pgfpathqlineto{10.7692bp}{60.3846bp}
    \pgfpathqlineto{12.3077bp}{60.3846bp}
    \pgfpathqlineto{13.8462bp}{60.3846bp}
    \pgfpathqlineto{15.3846bp}{60.3846bp}
    \pgfpathqlineto{16.9231bp}{60.3846bp}
    \pgfpathqlineto{18.4615bp}{60.3846bp}
    \pgfpathqlineto{20.0000bp}{60.3846bp}
    \pgfpathqlineto{21.5385bp}{60.3846bp}
    \pgfpathqlineto{23.0769bp}{60.3846bp}
    \pgfpathqlineto{24.6154bp}{60.3846bp}
    \pgfpathqlineto{26.1538bp}{60.3846bp}
    \pgfpathqlineto{27.6923bp}{60.3846bp}
    \pgfpathqlineto{29.2308bp}{60.3846bp}
    \pgfpathqlineto{30.7692bp}{60.3846bp}
    \pgfpathqlineto{32.3077bp}{60.3846bp}
    \pgfpathqlineto{33.8462bp}{60.3846bp}
    \pgfpathqlineto{35.3846bp}{60.3846bp}
    \pgfpathqlineto{36.9231bp}{60.3846bp}
    \pgfpathqlineto{38.4615bp}{60.3846bp}
    \pgfpathqlineto{40.0000bp}{60.3846bp}
    \pgfpathqlineto{41.5385bp}{60.3846bp}
    \pgfpathqlineto{43.0769bp}{60.3846bp}
    \pgfpathqlineto{44.6154bp}{60.3846bp}
    \pgfpathqlineto{46.1538bp}{60.3846bp}
    \pgfpathqlineto{47.6923bp}{60.3846bp}
    \pgfpathqlineto{49.2308bp}{60.3846bp}
    \pgfpathqlineto{50.7692bp}{60.3846bp}
    \pgfpathqlineto{52.3077bp}{60.3846bp}
    \pgfpathqlineto{53.8462bp}{60.3846bp}
    \pgfpathqlineto{55.3846bp}{60.3846bp}
    \pgfpathqlineto{56.9231bp}{60.3846bp}
    \pgfpathqlineto{58.4615bp}{60.3846bp}
    \pgfpathqlineto{60.0000bp}{60.3846bp}
    \pgfpathqlineto{61.5385bp}{60.3846bp}
    \pgfpathqlineto{63.0769bp}{60.3846bp}
    \pgfpathqlineto{64.6154bp}{60.3846bp}
    \pgfpathqlineto{66.1538bp}{60.3846bp}
    \pgfpathqlineto{67.6923bp}{60.3846bp}
    \pgfpathqlineto{69.2308bp}{60.3846bp}
    \pgfpathqlineto{70.7692bp}{60.3846bp}
    \pgfpathqlineto{72.3077bp}{60.3846bp}
    \pgfpathqlineto{73.8462bp}{60.3846bp}
    \pgfpathqlineto{75.3846bp}{60.3846bp}
    \pgfpathqlineto{76.9231bp}{60.3846bp}
    \pgfpathqlineto{78.4615bp}{60.3846bp}
    \pgfpathqlineto{80.0000bp}{60.3846bp}
    \pgfpathqlineto{81.5385bp}{60.3846bp}
    \pgfpathqlineto{83.0769bp}{60.3846bp}
    \pgfpathqlineto{84.6154bp}{60.3846bp}
    \pgfpathqlineto{86.1538bp}{60.3846bp}
    \pgfpathqlineto{87.6923bp}{60.3846bp}
    \pgfpathqlineto{89.2308bp}{60.3846bp}
    \pgfpathqlineto{90.7692bp}{60.3846bp}
    \pgfpathqlineto{92.3077bp}{60.3846bp}
    \pgfpathqlineto{93.8462bp}{60.3846bp}
    \pgfpathqlineto{95.3846bp}{60.3846bp}
    \pgfpathqlineto{96.9231bp}{60.3846bp}
    \pgfpathqlineto{98.4615bp}{60.3846bp}
    \pgfpathqlineto{100.0000bp}{60.3846bp}
    \pgfpathqlineto{101.5385bp}{60.3846bp}
    \pgfpathqlineto{103.0769bp}{60.3846bp}
    \pgfpathqlineto{104.6154bp}{60.3846bp}
    \pgfpathqlineto{106.1538bp}{60.3846bp}
    \pgfpathqlineto{107.6923bp}{60.3846bp}
    \pgfpathqlineto{109.2308bp}{60.3846bp}
    \pgfpathqlineto{110.7692bp}{60.3846bp}
    \pgfpathqlineto{112.3077bp}{60.3846bp}
    \pgfpathqlineto{113.8462bp}{60.3846bp}
    \pgfpathqlineto{115.3846bp}{60.3846bp}
    \pgfpathqlineto{116.9231bp}{60.3846bp}
    \pgfpathqlineto{118.4615bp}{60.3846bp}
    \pgfpathqlineto{120.0000bp}{60.3846bp}
    \pgfpathqlineto{121.5385bp}{60.3846bp}
    \pgfpathqlineto{123.0769bp}{60.3846bp}
    \pgfpathqlineto{124.6154bp}{60.3846bp}
    \pgfpathqlineto{126.1538bp}{60.3846bp}
    \pgfpathqlineto{127.6923bp}{60.3846bp}
    \pgfpathqlineto{129.2308bp}{60.3846bp}
    \pgfpathqlineto{130.7692bp}{60.3846bp}
    \pgfpathqlineto{132.3077bp}{60.3846bp}
    \pgfpathqlineto{133.8462bp}{60.3846bp}
    \pgfpathqlineto{135.3846bp}{60.3846bp}
    \pgfpathqlineto{136.9231bp}{60.3846bp}
    \pgfpathqlineto{138.4615bp}{60.3846bp}
    \pgfpathqlineto{140.0000bp}{60.3846bp}
    \pgfpathqlineto{141.5385bp}{60.3846bp}
    \pgfpathqlineto{143.0769bp}{60.3846bp}
    \pgfpathqlineto{144.6154bp}{60.3846bp}
    \pgfpathqlineto{146.1538bp}{60.3846bp}
    \pgfpathqlineto{147.6923bp}{60.3846bp}
    \pgfpathqlineto{149.2308bp}{60.3846bp}
    \pgfpathqlineto{150.7692bp}{60.3846bp}
    \pgfpathqlineto{152.3077bp}{60.3846bp}
    \pgfpathqlineto{153.8462bp}{60.3846bp}
    \pgfpathqlineto{155.3846bp}{60.3846bp}
    \pgfpathqlineto{156.9231bp}{60.3846bp}
    \pgfpathqlineto{158.4615bp}{60.3846bp}
    \pgfpathqlineto{160.0000bp}{60.3846bp}
    \pgfpathqlineto{161.5385bp}{60.3846bp}
    \pgfpathqlineto{163.0769bp}{60.3846bp}
    \pgfpathqlineto{164.6154bp}{60.3846bp}
    \pgfpathqlineto{166.1538bp}{60.3846bp}
    \pgfpathqlineto{167.6923bp}{60.3846bp}
    \pgfpathqlineto{169.2308bp}{60.3846bp}
    \pgfpathqlineto{170.7692bp}{60.3846bp}
    \pgfpathqlineto{172.3077bp}{60.3846bp}
    \pgfpathqlineto{173.8462bp}{60.3846bp}
    \pgfpathqlineto{175.3846bp}{60.3846bp}
    \pgfpathqlineto{176.9231bp}{60.3846bp}
    \pgfpathqlineto{178.4615bp}{60.3846bp}
    \pgfpathqlineto{180.0000bp}{60.3846bp}
    \pgfpathqlineto{181.5385bp}{60.3846bp}
    \pgfpathqlineto{183.0769bp}{60.3846bp}
    \pgfpathqlineto{184.6154bp}{60.3846bp}
    \pgfpathqlineto{186.1538bp}{60.3846bp}
    \pgfpathqlineto{187.6923bp}{60.3846bp}
    \pgfpathqlineto{189.2308bp}{60.3846bp}
    \pgfpathqlineto{190.7692bp}{60.3846bp}
    \pgfpathqlineto{192.3077bp}{60.3846bp}
    \pgfpathqlineto{193.8462bp}{60.3846bp}
    \pgfpathqlineto{195.3846bp}{60.3846bp}
    \pgfpathqlineto{196.9231bp}{60.3846bp}
    \pgfpathqlineto{198.4615bp}{60.3846bp}
    \pgfusepathqstroke
  \end{pgfscope}
  \begin{pgfscope}
    \pgfsetlinewidth{0.5722bp}
    \definecolor{sc}{rgb}{1.0000,0.0000,0.0000}
    \pgfsetstrokecolor{sc}
    \pgfsetmiterjoin
    \pgfsetbuttcap
    \pgfpathqmoveto{3.0769bp}{109.6154bp}
    \pgfpathqlineto{4.6154bp}{109.6154bp}
    \pgfpathqlineto{6.1538bp}{109.6154bp}
    \pgfpathqlineto{7.6923bp}{109.6154bp}
    \pgfpathqlineto{9.2308bp}{109.6154bp}
    \pgfpathqlineto{10.7692bp}{109.6154bp}
    \pgfpathqlineto{12.3077bp}{109.6154bp}
    \pgfpathqlineto{13.8462bp}{109.6154bp}
    \pgfpathqlineto{15.3846bp}{109.6154bp}
    \pgfpathqlineto{16.9231bp}{109.6154bp}
    \pgfpathqlineto{18.4615bp}{109.6154bp}
    \pgfpathqlineto{20.0000bp}{109.6154bp}
    \pgfpathqlineto{21.5385bp}{109.6154bp}
    \pgfpathqlineto{23.0769bp}{109.6154bp}
    \pgfpathqlineto{24.6154bp}{109.6154bp}
    \pgfpathqlineto{26.1538bp}{109.6154bp}
    \pgfpathqlineto{27.6923bp}{109.6154bp}
    \pgfpathqlineto{29.2308bp}{109.6154bp}
    \pgfpathqlineto{30.7692bp}{109.6154bp}
    \pgfpathqlineto{32.3077bp}{109.6154bp}
    \pgfpathqlineto{33.8462bp}{109.6154bp}
    \pgfpathqlineto{35.3846bp}{109.6154bp}
    \pgfpathqlineto{36.9231bp}{109.6154bp}
    \pgfpathqlineto{38.4615bp}{109.6154bp}
    \pgfpathqlineto{40.0000bp}{109.6154bp}
    \pgfpathqlineto{41.5385bp}{109.6154bp}
    \pgfpathqlineto{43.0769bp}{109.6154bp}
    \pgfpathqlineto{44.6154bp}{109.6154bp}
    \pgfpathqlineto{46.1538bp}{109.6154bp}
    \pgfpathqlineto{47.6923bp}{109.6154bp}
    \pgfpathqlineto{49.2308bp}{109.6154bp}
    \pgfpathqlineto{50.7692bp}{109.6154bp}
    \pgfpathqlineto{52.3077bp}{109.6154bp}
    \pgfpathqlineto{53.8462bp}{109.6154bp}
    \pgfpathqlineto{55.3846bp}{109.6154bp}
    \pgfpathqlineto{56.9231bp}{109.6154bp}
    \pgfpathqlineto{58.4615bp}{109.6154bp}
    \pgfpathqlineto{60.0000bp}{109.6154bp}
    \pgfpathqlineto{61.5385bp}{109.6154bp}
    \pgfpathqlineto{63.0769bp}{109.6154bp}
    \pgfpathqlineto{64.6154bp}{109.6154bp}
    \pgfpathqlineto{66.1538bp}{109.6154bp}
    \pgfpathqlineto{67.6923bp}{109.6154bp}
    \pgfpathqlineto{69.2308bp}{109.6154bp}
    \pgfpathqlineto{70.7692bp}{109.6154bp}
    \pgfpathqlineto{72.3077bp}{109.6154bp}
    \pgfpathqlineto{73.8462bp}{109.6154bp}
    \pgfpathqlineto{75.3846bp}{109.6154bp}
    \pgfpathqlineto{76.9231bp}{109.6154bp}
    \pgfpathqlineto{78.4615bp}{109.6154bp}
    \pgfpathqlineto{80.0000bp}{109.6154bp}
    \pgfpathqlineto{81.5385bp}{109.6154bp}
    \pgfpathqlineto{83.0769bp}{109.6154bp}
    \pgfpathqlineto{84.6154bp}{109.6154bp}
    \pgfpathqlineto{86.1538bp}{109.6154bp}
    \pgfpathqlineto{87.6923bp}{109.6154bp}
    \pgfpathqlineto{89.2308bp}{109.6154bp}
    \pgfpathqlineto{90.7692bp}{109.6154bp}
    \pgfpathqlineto{92.3077bp}{109.6154bp}
    \pgfpathqlineto{93.8462bp}{109.6154bp}
    \pgfpathqlineto{95.3846bp}{109.6154bp}
    \pgfpathqlineto{96.9231bp}{109.6154bp}
    \pgfpathqlineto{98.4615bp}{109.6154bp}
    \pgfpathqlineto{100.0000bp}{109.6154bp}
    \pgfpathqlineto{101.5385bp}{109.6154bp}
    \pgfpathqlineto{103.0769bp}{109.6154bp}
    \pgfpathqlineto{104.6154bp}{109.6154bp}
    \pgfpathqlineto{106.1538bp}{109.6154bp}
    \pgfpathqlineto{107.6923bp}{109.6154bp}
    \pgfpathqlineto{109.2308bp}{109.6154bp}
    \pgfpathqlineto{110.7692bp}{109.6154bp}
    \pgfpathqlineto{112.3077bp}{109.6154bp}
    \pgfpathqlineto{113.8462bp}{109.6154bp}
    \pgfpathqlineto{115.3846bp}{109.6154bp}
    \pgfpathqlineto{116.9231bp}{109.6154bp}
    \pgfpathqlineto{118.4615bp}{109.6154bp}
    \pgfpathqlineto{120.0000bp}{109.6154bp}
    \pgfpathqlineto{121.5385bp}{109.6154bp}
    \pgfpathqlineto{123.0769bp}{109.6154bp}
    \pgfpathqlineto{124.6154bp}{109.6154bp}
    \pgfpathqlineto{126.1538bp}{109.6154bp}
    \pgfpathqlineto{127.6923bp}{109.6154bp}
    \pgfpathqlineto{129.2308bp}{109.6154bp}
    \pgfpathqlineto{130.7692bp}{109.6154bp}
    \pgfpathqlineto{132.3077bp}{109.6154bp}
    \pgfpathqlineto{133.8462bp}{109.6154bp}
    \pgfpathqlineto{135.3846bp}{109.6154bp}
    \pgfpathqlineto{136.9231bp}{109.6154bp}
    \pgfpathqlineto{138.4615bp}{109.6154bp}
    \pgfpathqlineto{140.0000bp}{109.6154bp}
    \pgfpathqlineto{141.5385bp}{109.6154bp}
    \pgfpathqlineto{143.0769bp}{109.6154bp}
    \pgfpathqlineto{144.6154bp}{109.6154bp}
    \pgfpathqlineto{146.1538bp}{109.6154bp}
    \pgfpathqlineto{147.6923bp}{109.6154bp}
    \pgfpathqlineto{149.2308bp}{109.6154bp}
    \pgfpathqlineto{150.7692bp}{109.6154bp}
    \pgfpathqlineto{152.3077bp}{109.6154bp}
    \pgfpathqlineto{153.8462bp}{109.6154bp}
    \pgfpathqlineto{155.3846bp}{109.6154bp}
    \pgfpathqlineto{156.9231bp}{109.6154bp}
    \pgfpathqlineto{158.4615bp}{109.6154bp}
    \pgfpathqlineto{160.0000bp}{109.6154bp}
    \pgfpathqlineto{161.5385bp}{109.6154bp}
    \pgfpathqlineto{163.0769bp}{109.6154bp}
    \pgfpathqlineto{164.6154bp}{109.6154bp}
    \pgfpathqlineto{166.1538bp}{109.6154bp}
    \pgfpathqlineto{167.6923bp}{109.6154bp}
    \pgfpathqlineto{169.2308bp}{109.6154bp}
    \pgfpathqlineto{170.7692bp}{109.6154bp}
    \pgfpathqlineto{172.3077bp}{109.6154bp}
    \pgfpathqlineto{173.8462bp}{109.6154bp}
    \pgfpathqlineto{175.3846bp}{109.6154bp}
    \pgfpathqlineto{176.9231bp}{109.6154bp}
    \pgfpathqlineto{178.4615bp}{109.6154bp}
    \pgfpathqlineto{180.0000bp}{109.6154bp}
    \pgfpathqlineto{181.5385bp}{109.6154bp}
    \pgfpathqlineto{183.0769bp}{109.6154bp}
    \pgfpathqlineto{184.6154bp}{65.7692bp}
    \pgfpathqlineto{186.1538bp}{65.7692bp}
    \pgfpathqlineto{187.6923bp}{65.7692bp}
    \pgfpathqlineto{189.2308bp}{65.7692bp}
    \pgfpathqlineto{190.7692bp}{65.7692bp}
    \pgfpathqlineto{192.3077bp}{65.0000bp}
    \pgfpathqlineto{193.8462bp}{65.0000bp}
    \pgfpathqlineto{195.3846bp}{65.0000bp}
    \pgfpathqlineto{196.9231bp}{65.0000bp}
    \pgfpathqlineto{198.4615bp}{65.0000bp}
    \pgfusepathqstroke
  \end{pgfscope}
  \begin{pgfscope}
    \pgfsetlinewidth{0.5722bp}
    \definecolor{sc}{rgb}{0.0000,0.0000,1.0000}
    \pgfsetstrokecolor{sc}
    \pgfsetmiterjoin
    \pgfsetbuttcap
    \pgfpathqmoveto{3.0769bp}{60.3846bp}
    \pgfpathqlineto{4.6154bp}{60.3846bp}
    \pgfpathqlineto{6.1538bp}{60.3846bp}
    \pgfpathqlineto{7.6923bp}{60.3846bp}
    \pgfpathqlineto{9.2308bp}{60.3846bp}
    \pgfpathqlineto{10.7692bp}{60.3846bp}
    \pgfpathqlineto{12.3077bp}{60.3846bp}
    \pgfpathqlineto{13.8462bp}{60.3846bp}
    \pgfpathqlineto{15.3846bp}{60.3846bp}
    \pgfpathqlineto{16.9231bp}{60.3846bp}
    \pgfpathqlineto{18.4615bp}{60.3846bp}
    \pgfpathqlineto{20.0000bp}{60.3846bp}
    \pgfpathqlineto{21.5385bp}{60.3846bp}
    \pgfpathqlineto{23.0769bp}{60.3846bp}
    \pgfpathqlineto{24.6154bp}{60.3846bp}
    \pgfpathqlineto{26.1538bp}{60.3846bp}
    \pgfpathqlineto{27.6923bp}{60.3846bp}
    \pgfpathqlineto{29.2308bp}{60.3846bp}
    \pgfpathqlineto{30.7692bp}{60.3846bp}
    \pgfpathqlineto{32.3077bp}{60.3846bp}
    \pgfpathqlineto{33.8462bp}{60.3846bp}
    \pgfpathqlineto{35.3846bp}{60.3846bp}
    \pgfpathqlineto{36.9231bp}{60.3846bp}
    \pgfpathqlineto{38.4615bp}{60.3846bp}
    \pgfpathqlineto{40.0000bp}{60.3846bp}
    \pgfpathqlineto{41.5385bp}{60.3846bp}
    \pgfpathqlineto{43.0769bp}{60.3846bp}
    \pgfpathqlineto{44.6154bp}{60.3846bp}
    \pgfpathqlineto{46.1538bp}{60.3846bp}
    \pgfpathqlineto{47.6923bp}{60.3846bp}
    \pgfpathqlineto{49.2308bp}{60.3846bp}
    \pgfpathqlineto{50.7692bp}{60.3846bp}
    \pgfpathqlineto{52.3077bp}{60.3846bp}
    \pgfpathqlineto{53.8462bp}{60.3846bp}
    \pgfpathqlineto{55.3846bp}{60.3846bp}
    \pgfpathqlineto{56.9231bp}{60.3846bp}
    \pgfpathqlineto{58.4615bp}{60.3846bp}
    \pgfpathqlineto{60.0000bp}{60.3846bp}
    \pgfpathqlineto{61.5385bp}{60.3846bp}
    \pgfpathqlineto{63.0769bp}{60.3846bp}
    \pgfpathqlineto{64.6154bp}{60.3846bp}
    \pgfpathqlineto{66.1538bp}{60.3846bp}
    \pgfpathqlineto{67.6923bp}{60.3846bp}
    \pgfpathqlineto{69.2308bp}{60.3846bp}
    \pgfpathqlineto{70.7692bp}{60.3846bp}
    \pgfpathqlineto{72.3077bp}{60.3846bp}
    \pgfpathqlineto{73.8462bp}{60.3846bp}
    \pgfpathqlineto{75.3846bp}{60.3846bp}
    \pgfpathqlineto{76.9231bp}{60.3846bp}
    \pgfpathqlineto{78.4615bp}{60.3846bp}
    \pgfpathqlineto{80.0000bp}{60.3846bp}
    \pgfpathqlineto{81.5385bp}{60.3846bp}
    \pgfpathqlineto{83.0769bp}{60.3846bp}
    \pgfpathqlineto{84.6154bp}{60.3846bp}
    \pgfpathqlineto{86.1538bp}{60.3846bp}
    \pgfpathqlineto{87.6923bp}{60.3846bp}
    \pgfpathqlineto{89.2308bp}{60.3846bp}
    \pgfpathqlineto{90.7692bp}{60.3846bp}
    \pgfpathqlineto{92.3077bp}{60.3846bp}
    \pgfpathqlineto{93.8462bp}{60.3846bp}
    \pgfpathqlineto{95.3846bp}{60.3846bp}
    \pgfpathqlineto{96.9231bp}{60.3846bp}
    \pgfpathqlineto{98.4615bp}{60.3846bp}
    \pgfpathqlineto{100.0000bp}{60.3846bp}
    \pgfpathqlineto{101.5385bp}{60.3846bp}
    \pgfpathqlineto{103.0769bp}{60.3846bp}
    \pgfpathqlineto{104.6154bp}{60.3846bp}
    \pgfpathqlineto{106.1538bp}{60.3846bp}
    \pgfpathqlineto{107.6923bp}{60.3846bp}
    \pgfpathqlineto{109.2308bp}{60.3846bp}
    \pgfpathqlineto{110.7692bp}{60.3846bp}
    \pgfpathqlineto{112.3077bp}{60.3846bp}
    \pgfpathqlineto{113.8462bp}{60.3846bp}
    \pgfpathqlineto{115.3846bp}{60.3846bp}
    \pgfpathqlineto{116.9231bp}{60.3846bp}
    \pgfpathqlineto{118.4615bp}{60.3846bp}
    \pgfpathqlineto{120.0000bp}{60.3846bp}
    \pgfpathqlineto{121.5385bp}{60.3846bp}
    \pgfpathqlineto{123.0769bp}{60.3846bp}
    \pgfpathqlineto{124.6154bp}{60.3846bp}
    \pgfpathqlineto{126.1538bp}{60.3846bp}
    \pgfpathqlineto{127.6923bp}{60.3846bp}
    \pgfpathqlineto{129.2308bp}{60.3846bp}
    \pgfpathqlineto{130.7692bp}{60.3846bp}
    \pgfpathqlineto{132.3077bp}{60.3846bp}
    \pgfpathqlineto{133.8462bp}{60.3846bp}
    \pgfpathqlineto{135.3846bp}{60.3846bp}
    \pgfpathqlineto{136.9231bp}{60.3846bp}
    \pgfpathqlineto{138.4615bp}{60.3846bp}
    \pgfpathqlineto{140.0000bp}{60.3846bp}
    \pgfpathqlineto{141.5385bp}{60.3846bp}
    \pgfpathqlineto{143.0769bp}{60.3846bp}
    \pgfpathqlineto{144.6154bp}{60.3846bp}
    \pgfpathqlineto{146.1538bp}{60.3846bp}
    \pgfpathqlineto{147.6923bp}{60.3846bp}
    \pgfpathqlineto{149.2308bp}{60.3846bp}
    \pgfpathqlineto{150.7692bp}{60.3846bp}
    \pgfpathqlineto{152.3077bp}{60.3846bp}
    \pgfpathqlineto{153.8462bp}{60.3846bp}
    \pgfpathqlineto{155.3846bp}{60.3846bp}
    \pgfpathqlineto{156.9231bp}{60.3846bp}
    \pgfpathqlineto{158.4615bp}{60.3846bp}
    \pgfpathqlineto{160.0000bp}{60.3846bp}
    \pgfpathqlineto{161.5385bp}{60.3846bp}
    \pgfpathqlineto{163.0769bp}{60.3846bp}
    \pgfpathqlineto{164.6154bp}{60.3846bp}
    \pgfpathqlineto{166.1538bp}{60.3846bp}
    \pgfpathqlineto{167.6923bp}{60.3846bp}
    \pgfpathqlineto{169.2308bp}{60.3846bp}
    \pgfpathqlineto{170.7692bp}{60.3846bp}
    \pgfpathqlineto{172.3077bp}{60.3846bp}
    \pgfpathqlineto{173.8462bp}{60.3846bp}
    \pgfpathqlineto{175.3846bp}{60.3846bp}
    \pgfpathqlineto{176.9231bp}{60.3846bp}
    \pgfpathqlineto{178.4615bp}{60.3846bp}
    \pgfpathqlineto{180.0000bp}{60.3846bp}
    \pgfpathqlineto{181.5385bp}{60.3846bp}
    \pgfpathqlineto{183.0769bp}{60.3846bp}
    \pgfpathqlineto{184.6154bp}{60.3846bp}
    \pgfpathqlineto{186.1538bp}{60.3846bp}
    \pgfpathqlineto{187.6923bp}{60.3846bp}
    \pgfpathqlineto{189.2308bp}{60.3846bp}
    \pgfpathqlineto{190.7692bp}{60.3846bp}
    \pgfpathqlineto{192.3077bp}{61.1538bp}
    \pgfpathqlineto{193.8462bp}{61.1538bp}
    \pgfpathqlineto{195.3846bp}{61.1538bp}
    \pgfpathqlineto{196.9231bp}{61.1538bp}
    \pgfpathqlineto{198.4615bp}{61.1538bp}
    \pgfusepathqstroke
  \end{pgfscope}
  \begin{pgfscope}
    \pgfsetlinewidth{0.5722bp}
    \definecolor{sc}{rgb}{1.0000,0.0000,0.0000}
    \pgfsetstrokecolor{sc}
    \pgfsetmiterjoin
    \pgfsetbuttcap
    \pgfpathqmoveto{200.0000bp}{60.3846bp}
    \pgfpathqlineto{200.0000bp}{59.6154bp}
    \pgfusepathqstroke
  \end{pgfscope}
  \begin{pgfscope}
    \pgfsetlinewidth{0.5722bp}
    \definecolor{sc}{rgb}{1.0000,0.0000,0.0000}
    \pgfsetstrokecolor{sc}
    \pgfsetmiterjoin
    \pgfsetbuttcap
    \pgfpathqmoveto{186.1538bp}{60.3846bp}
    \pgfpathqlineto{186.1538bp}{59.6154bp}
    \pgfusepathqstroke
  \end{pgfscope}
  \begin{pgfscope}
    \pgfsetlinewidth{0.5722bp}
    \definecolor{sc}{rgb}{1.0000,0.0000,0.0000}
    \pgfsetstrokecolor{sc}
    \pgfsetmiterjoin
    \pgfsetbuttcap
    \pgfpathqmoveto{3.0769bp}{60.3846bp}
    \pgfpathqlineto{3.0769bp}{59.6154bp}
    \pgfusepathqstroke
  \end{pgfscope}
  \begin{pgfscope}
    \pgfsetlinewidth{0.5722bp}
    \definecolor{sc}{rgb}{0.0000,0.0000,0.0000}
    \pgfsetstrokecolor{sc}
    \pgfsetmiterjoin
    \pgfsetbuttcap
    \pgfpathqmoveto{195.3846bp}{60.3846bp}
    \pgfpathqlineto{195.3846bp}{59.6154bp}
    \pgfusepathqstroke
  \end{pgfscope}
  \begin{pgfscope}
    \pgfsetlinewidth{0.5722bp}
    \definecolor{sc}{rgb}{0.0000,0.0000,0.0000}
    \pgfsetstrokecolor{sc}
    \pgfsetmiterjoin
    \pgfsetbuttcap
    \pgfpathqmoveto{187.6923bp}{60.3846bp}
    \pgfpathqlineto{187.6923bp}{59.6154bp}
    \pgfusepathqstroke
  \end{pgfscope}
  \begin{pgfscope}
    \pgfsetlinewidth{0.5722bp}
    \definecolor{sc}{rgb}{0.0000,0.0000,0.0000}
    \pgfsetstrokecolor{sc}
    \pgfsetmiterjoin
    \pgfsetbuttcap
    \pgfpathqmoveto{180.0000bp}{60.3846bp}
    \pgfpathqlineto{180.0000bp}{59.6154bp}
    \pgfusepathqstroke
  \end{pgfscope}
  \begin{pgfscope}
    \pgfsetlinewidth{0.5722bp}
    \definecolor{sc}{rgb}{0.0000,0.0000,0.0000}
    \pgfsetstrokecolor{sc}
    \pgfsetmiterjoin
    \pgfsetbuttcap
    \pgfpathqmoveto{172.3077bp}{60.3846bp}
    \pgfpathqlineto{172.3077bp}{59.6154bp}
    \pgfusepathqstroke
  \end{pgfscope}
  \begin{pgfscope}
    \pgfsetlinewidth{0.5722bp}
    \definecolor{sc}{rgb}{0.0000,0.0000,0.0000}
    \pgfsetstrokecolor{sc}
    \pgfsetmiterjoin
    \pgfsetbuttcap
    \pgfpathqmoveto{164.6154bp}{60.3846bp}
    \pgfpathqlineto{164.6154bp}{59.6154bp}
    \pgfusepathqstroke
  \end{pgfscope}
  \begin{pgfscope}
    \pgfsetlinewidth{0.5722bp}
    \definecolor{sc}{rgb}{0.0000,0.0000,0.0000}
    \pgfsetstrokecolor{sc}
    \pgfsetmiterjoin
    \pgfsetbuttcap
    \pgfpathqmoveto{156.9231bp}{60.3846bp}
    \pgfpathqlineto{156.9231bp}{59.6154bp}
    \pgfusepathqstroke
  \end{pgfscope}
  \begin{pgfscope}
    \pgfsetlinewidth{0.5722bp}
    \definecolor{sc}{rgb}{0.0000,0.0000,0.0000}
    \pgfsetstrokecolor{sc}
    \pgfsetmiterjoin
    \pgfsetbuttcap
    \pgfpathqmoveto{149.2308bp}{60.3846bp}
    \pgfpathqlineto{149.2308bp}{59.6154bp}
    \pgfusepathqstroke
  \end{pgfscope}
  \begin{pgfscope}
    \pgfsetlinewidth{0.5722bp}
    \definecolor{sc}{rgb}{0.0000,0.0000,0.0000}
    \pgfsetstrokecolor{sc}
    \pgfsetmiterjoin
    \pgfsetbuttcap
    \pgfpathqmoveto{141.5385bp}{60.3846bp}
    \pgfpathqlineto{141.5385bp}{59.6154bp}
    \pgfusepathqstroke
  \end{pgfscope}
  \begin{pgfscope}
    \pgfsetlinewidth{0.5722bp}
    \definecolor{sc}{rgb}{0.0000,0.0000,0.0000}
    \pgfsetstrokecolor{sc}
    \pgfsetmiterjoin
    \pgfsetbuttcap
    \pgfpathqmoveto{133.8462bp}{60.3846bp}
    \pgfpathqlineto{133.8462bp}{59.6154bp}
    \pgfusepathqstroke
  \end{pgfscope}
  \begin{pgfscope}
    \pgfsetlinewidth{0.5722bp}
    \definecolor{sc}{rgb}{0.0000,0.0000,0.0000}
    \pgfsetstrokecolor{sc}
    \pgfsetmiterjoin
    \pgfsetbuttcap
    \pgfpathqmoveto{126.1538bp}{60.3846bp}
    \pgfpathqlineto{126.1538bp}{59.6154bp}
    \pgfusepathqstroke
  \end{pgfscope}
  \begin{pgfscope}
    \pgfsetlinewidth{0.5722bp}
    \definecolor{sc}{rgb}{0.0000,0.0000,0.0000}
    \pgfsetstrokecolor{sc}
    \pgfsetmiterjoin
    \pgfsetbuttcap
    \pgfpathqmoveto{118.4615bp}{60.3846bp}
    \pgfpathqlineto{118.4615bp}{59.6154bp}
    \pgfusepathqstroke
  \end{pgfscope}
  \begin{pgfscope}
    \pgfsetlinewidth{0.5722bp}
    \definecolor{sc}{rgb}{0.0000,0.0000,0.0000}
    \pgfsetstrokecolor{sc}
    \pgfsetmiterjoin
    \pgfsetbuttcap
    \pgfpathqmoveto{110.7692bp}{60.3846bp}
    \pgfpathqlineto{110.7692bp}{59.6154bp}
    \pgfusepathqstroke
  \end{pgfscope}
  \begin{pgfscope}
    \pgfsetlinewidth{0.5722bp}
    \definecolor{sc}{rgb}{0.0000,0.0000,0.0000}
    \pgfsetstrokecolor{sc}
    \pgfsetmiterjoin
    \pgfsetbuttcap
    \pgfpathqmoveto{103.0769bp}{60.3846bp}
    \pgfpathqlineto{103.0769bp}{59.6154bp}
    \pgfusepathqstroke
  \end{pgfscope}
  \begin{pgfscope}
    \pgfsetlinewidth{0.5722bp}
    \definecolor{sc}{rgb}{0.0000,0.0000,0.0000}
    \pgfsetstrokecolor{sc}
    \pgfsetmiterjoin
    \pgfsetbuttcap
    \pgfpathqmoveto{95.3846bp}{60.3846bp}
    \pgfpathqlineto{95.3846bp}{59.6154bp}
    \pgfusepathqstroke
  \end{pgfscope}
  \begin{pgfscope}
    \pgfsetlinewidth{0.5722bp}
    \definecolor{sc}{rgb}{0.0000,0.0000,0.0000}
    \pgfsetstrokecolor{sc}
    \pgfsetmiterjoin
    \pgfsetbuttcap
    \pgfpathqmoveto{87.6923bp}{60.3846bp}
    \pgfpathqlineto{87.6923bp}{59.6154bp}
    \pgfusepathqstroke
  \end{pgfscope}
  \begin{pgfscope}
    \pgfsetlinewidth{0.5722bp}
    \definecolor{sc}{rgb}{0.0000,0.0000,0.0000}
    \pgfsetstrokecolor{sc}
    \pgfsetmiterjoin
    \pgfsetbuttcap
    \pgfpathqmoveto{80.0000bp}{60.3846bp}
    \pgfpathqlineto{80.0000bp}{59.6154bp}
    \pgfusepathqstroke
  \end{pgfscope}
  \begin{pgfscope}
    \pgfsetlinewidth{0.5722bp}
    \definecolor{sc}{rgb}{0.0000,0.0000,0.0000}
    \pgfsetstrokecolor{sc}
    \pgfsetmiterjoin
    \pgfsetbuttcap
    \pgfpathqmoveto{72.3077bp}{60.3846bp}
    \pgfpathqlineto{72.3077bp}{59.6154bp}
    \pgfusepathqstroke
  \end{pgfscope}
  \begin{pgfscope}
    \pgfsetlinewidth{0.5722bp}
    \definecolor{sc}{rgb}{0.0000,0.0000,0.0000}
    \pgfsetstrokecolor{sc}
    \pgfsetmiterjoin
    \pgfsetbuttcap
    \pgfpathqmoveto{64.6154bp}{60.3846bp}
    \pgfpathqlineto{64.6154bp}{59.6154bp}
    \pgfusepathqstroke
  \end{pgfscope}
  \begin{pgfscope}
    \pgfsetlinewidth{0.5722bp}
    \definecolor{sc}{rgb}{0.0000,0.0000,0.0000}
    \pgfsetstrokecolor{sc}
    \pgfsetmiterjoin
    \pgfsetbuttcap
    \pgfpathqmoveto{56.9231bp}{60.3846bp}
    \pgfpathqlineto{56.9231bp}{59.6154bp}
    \pgfusepathqstroke
  \end{pgfscope}
  \begin{pgfscope}
    \pgfsetlinewidth{0.5722bp}
    \definecolor{sc}{rgb}{0.0000,0.0000,0.0000}
    \pgfsetstrokecolor{sc}
    \pgfsetmiterjoin
    \pgfsetbuttcap
    \pgfpathqmoveto{49.2308bp}{60.3846bp}
    \pgfpathqlineto{49.2308bp}{59.6154bp}
    \pgfusepathqstroke
  \end{pgfscope}
  \begin{pgfscope}
    \pgfsetlinewidth{0.5722bp}
    \definecolor{sc}{rgb}{0.0000,0.0000,0.0000}
    \pgfsetstrokecolor{sc}
    \pgfsetmiterjoin
    \pgfsetbuttcap
    \pgfpathqmoveto{41.5385bp}{60.3846bp}
    \pgfpathqlineto{41.5385bp}{59.6154bp}
    \pgfusepathqstroke
  \end{pgfscope}
  \begin{pgfscope}
    \pgfsetlinewidth{0.5722bp}
    \definecolor{sc}{rgb}{0.0000,0.0000,0.0000}
    \pgfsetstrokecolor{sc}
    \pgfsetmiterjoin
    \pgfsetbuttcap
    \pgfpathqmoveto{33.8462bp}{60.3846bp}
    \pgfpathqlineto{33.8462bp}{59.6154bp}
    \pgfusepathqstroke
  \end{pgfscope}
  \begin{pgfscope}
    \pgfsetlinewidth{0.5722bp}
    \definecolor{sc}{rgb}{0.0000,0.0000,0.0000}
    \pgfsetstrokecolor{sc}
    \pgfsetmiterjoin
    \pgfsetbuttcap
    \pgfpathqmoveto{26.1538bp}{60.3846bp}
    \pgfpathqlineto{26.1538bp}{59.6154bp}
    \pgfusepathqstroke
  \end{pgfscope}
  \begin{pgfscope}
    \pgfsetlinewidth{0.5722bp}
    \definecolor{sc}{rgb}{0.0000,0.0000,0.0000}
    \pgfsetstrokecolor{sc}
    \pgfsetmiterjoin
    \pgfsetbuttcap
    \pgfpathqmoveto{18.4615bp}{60.3846bp}
    \pgfpathqlineto{18.4615bp}{59.6154bp}
    \pgfusepathqstroke
  \end{pgfscope}
  \begin{pgfscope}
    \pgfsetlinewidth{0.5722bp}
    \definecolor{sc}{rgb}{0.0000,0.0000,0.0000}
    \pgfsetstrokecolor{sc}
    \pgfsetmiterjoin
    \pgfsetbuttcap
    \pgfpathqmoveto{10.7692bp}{60.3846bp}
    \pgfpathqlineto{10.7692bp}{59.6154bp}
    \pgfusepathqstroke
  \end{pgfscope}
  \begin{pgfscope}
    \definecolor{fc}{rgb}{0.0000,0.0000,0.0000}
    \pgfsetfillcolor{fc}
    \pgftransformcm{1.0000}{0.0000}{0.0000}{1.0000}{\pgfqpoint{-0.0000bp}{109.1538bp}}
    \pgftransformscale{0.1923}
    \pgftext[base,left]{$\mathbb{L}_A$}
  \end{pgfscope}
  \begin{pgfscope}
    \pgfsetlinewidth{0.5722bp}
    \definecolor{sc}{rgb}{0.0000,0.0000,0.0000}
    \pgfsetstrokecolor{sc}
    \pgfsetmiterjoin
    \pgfsetbuttcap
    \pgfpathqmoveto{3.0769bp}{109.6154bp}
    \pgfpathqlineto{2.7692bp}{109.6154bp}
    \pgfusepathqstroke
  \end{pgfscope}
  \begin{pgfscope}
    \pgfsetlinewidth{0.5722bp}
    \definecolor{sc}{rgb}{0.0000,0.0000,0.0000}
    \pgfsetstrokecolor{sc}
    \pgfsetmiterjoin
    \pgfsetbuttcap
    \pgfpathqmoveto{3.0769bp}{60.3846bp}
    \pgfpathqlineto{3.0769bp}{151.1538bp}
    \pgfusepathqstroke
  \end{pgfscope}
  \begin{pgfscope}
    \pgfsetlinewidth{0.5722bp}
    \definecolor{sc}{rgb}{0.0000,0.0000,0.0000}
    \pgfsetstrokecolor{sc}
    \pgfsetmiterjoin
    \pgfsetbuttcap
    \pgfpathqmoveto{3.0769bp}{60.3846bp}
    \pgfpathqlineto{200.0000bp}{60.3846bp}
    \pgfusepathqstroke
  \end{pgfscope}
\end{pgfpicture}

        \label{fig:ex:ca:hgma:ex:move-h}
    \caption{push-h preconditions}\label{fig:ex:ca:hgma:ex:disconnected}
\end{figure}

\begin{figure}
    \centering
    \begin{pgfpicture}
  \pgfpathrectangle{\pgfpointorigin}{\pgfqpoint{200.0000bp}{200.0000bp}}
  \pgfusepath{use as bounding box}
  \begin{pgfscope}
    \definecolor{fc}{rgb}{0.0000,0.0000,0.0000}
    \pgfsetfillcolor{fc}
    \pgftransformshift{\pgfqpoint{3.2227bp}{49.8578bp}}
    \pgftransformscale{0.1185}
    \pgftext[base,left]{candidates}
  \end{pgfscope}
  \begin{pgfscope}
    \definecolor{fc}{rgb}{0.0000,0.0000,0.0000}
    \pgfsetfillcolor{fc}
    \pgfsetlinewidth{0.5670bp}
    \definecolor{sc}{rgb}{0.0000,0.0000,0.0000}
    \pgfsetstrokecolor{sc}
    \pgfsetmiterjoin
    \pgfsetbuttcap
    \pgfpathqmoveto{2.2749bp}{50.1422bp}
    \pgfpathqcurveto{2.2749bp}{50.3516bp}{2.1051bp}{50.5213bp}{1.8957bp}{50.5213bp}
    \pgfpathqcurveto{1.6863bp}{50.5213bp}{1.5166bp}{50.3516bp}{1.5166bp}{50.1422bp}
    \pgfpathqcurveto{1.5166bp}{49.9328bp}{1.6863bp}{49.7630bp}{1.8957bp}{49.7630bp}
    \pgfpathqcurveto{2.1051bp}{49.7630bp}{2.2749bp}{49.9328bp}{2.2749bp}{50.1422bp}
    \pgfpathclose
    \pgfusepathqfillstroke
  \end{pgfscope}
  \begin{pgfscope}
    \definecolor{fc}{rgb}{0.0000,0.0000,0.0000}
    \pgfsetfillcolor{fc}
    \pgftransformshift{\pgfqpoint{3.2227bp}{51.0900bp}}
    \pgftransformscale{0.1185}
    \pgftext[base,left]{negative unproven}
  \end{pgfscope}
  \begin{pgfscope}
    \definecolor{fc}{rgb}{1.0000,1.0000,0.0000}
    \pgfsetfillcolor{fc}
    \pgfsetlinewidth{0.5670bp}
    \definecolor{sc}{rgb}{1.0000,1.0000,0.0000}
    \pgfsetstrokecolor{sc}
    \pgfsetmiterjoin
    \pgfsetbuttcap
    \pgfpathqmoveto{2.2749bp}{51.3744bp}
    \pgfpathqcurveto{2.2749bp}{51.5838bp}{2.1051bp}{51.7536bp}{1.8957bp}{51.7536bp}
    \pgfpathqcurveto{1.6863bp}{51.7536bp}{1.5166bp}{51.5838bp}{1.5166bp}{51.3744bp}
    \pgfpathqcurveto{1.5166bp}{51.1650bp}{1.6863bp}{50.9953bp}{1.8957bp}{50.9953bp}
    \pgfpathqcurveto{2.1051bp}{50.9953bp}{2.2749bp}{51.1650bp}{2.2749bp}{51.3744bp}
    \pgfpathclose
    \pgfusepathqfillstroke
  \end{pgfscope}
  \begin{pgfscope}
    \definecolor{fc}{rgb}{0.0000,0.0000,0.0000}
    \pgfsetfillcolor{fc}
    \pgftransformshift{\pgfqpoint{3.2227bp}{52.3223bp}}
    \pgftransformscale{0.1185}
    \pgftext[base,left]{negative proven}
  \end{pgfscope}
  \begin{pgfscope}
    \definecolor{fc}{rgb}{0.0000,0.5020,0.0000}
    \pgfsetfillcolor{fc}
    \pgfsetlinewidth{0.5670bp}
    \definecolor{sc}{rgb}{0.0000,0.5020,0.0000}
    \pgfsetstrokecolor{sc}
    \pgfsetmiterjoin
    \pgfsetbuttcap
    \pgfpathqmoveto{2.2749bp}{52.6066bp}
    \pgfpathqcurveto{2.2749bp}{52.8160bp}{2.1051bp}{52.9858bp}{1.8957bp}{52.9858bp}
    \pgfpathqcurveto{1.6863bp}{52.9858bp}{1.5166bp}{52.8160bp}{1.5166bp}{52.6066bp}
    \pgfpathqcurveto{1.5166bp}{52.3972bp}{1.6863bp}{52.2275bp}{1.8957bp}{52.2275bp}
    \pgfpathqcurveto{2.1051bp}{52.2275bp}{2.2749bp}{52.3972bp}{2.2749bp}{52.6066bp}
    \pgfpathclose
    \pgfusepathqfillstroke
  \end{pgfscope}
  \begin{pgfscope}
    \definecolor{fc}{rgb}{0.0000,0.0000,0.0000}
    \pgfsetfillcolor{fc}
    \pgftransformshift{\pgfqpoint{3.2227bp}{53.5545bp}}
    \pgftransformscale{0.1185}
    \pgftext[base,left]{positive unproven}
  \end{pgfscope}
  \begin{pgfscope}
    \definecolor{fc}{rgb}{1.0000,0.0000,0.0000}
    \pgfsetfillcolor{fc}
    \pgfsetlinewidth{0.5670bp}
    \definecolor{sc}{rgb}{1.0000,0.0000,0.0000}
    \pgfsetstrokecolor{sc}
    \pgfsetmiterjoin
    \pgfsetbuttcap
    \pgfpathqmoveto{2.2749bp}{53.8389bp}
    \pgfpathqcurveto{2.2749bp}{54.0483bp}{2.1051bp}{54.2180bp}{1.8957bp}{54.2180bp}
    \pgfpathqcurveto{1.6863bp}{54.2180bp}{1.5166bp}{54.0483bp}{1.5166bp}{53.8389bp}
    \pgfpathqcurveto{1.5166bp}{53.6295bp}{1.6863bp}{53.4597bp}{1.8957bp}{53.4597bp}
    \pgfpathqcurveto{2.1051bp}{53.4597bp}{2.2749bp}{53.6295bp}{2.2749bp}{53.8389bp}
    \pgfpathclose
    \pgfusepathqfillstroke
  \end{pgfscope}
  \begin{pgfscope}
    \definecolor{fc}{rgb}{0.0000,0.0000,0.0000}
    \pgfsetfillcolor{fc}
    \pgftransformshift{\pgfqpoint{3.2227bp}{54.7867bp}}
    \pgftransformscale{0.1185}
    \pgftext[base,left]{positive proven}
  \end{pgfscope}
  \begin{pgfscope}
    \definecolor{fc}{rgb}{0.0000,0.0000,1.0000}
    \pgfsetfillcolor{fc}
    \pgfsetlinewidth{0.5670bp}
    \definecolor{sc}{rgb}{0.0000,0.0000,1.0000}
    \pgfsetstrokecolor{sc}
    \pgfsetmiterjoin
    \pgfsetbuttcap
    \pgfpathqmoveto{2.2749bp}{55.0711bp}
    \pgfpathqcurveto{2.2749bp}{55.2805bp}{2.1051bp}{55.4502bp}{1.8957bp}{55.4502bp}
    \pgfpathqcurveto{1.6863bp}{55.4502bp}{1.5166bp}{55.2805bp}{1.5166bp}{55.0711bp}
    \pgfpathqcurveto{1.5166bp}{54.8617bp}{1.6863bp}{54.6919bp}{1.8957bp}{54.6919bp}
    \pgfpathqcurveto{2.1051bp}{54.6919bp}{2.2749bp}{54.8617bp}{2.2749bp}{55.0711bp}
    \pgfpathclose
    \pgfusepathqfillstroke
  \end{pgfscope}
  \begin{pgfscope}
    \pgfsetlinewidth{0.5670bp}
    \definecolor{sc}{rgb}{0.0000,0.0000,0.0000}
    \pgfsetstrokecolor{sc}
    \pgfsetmiterjoin
    \pgfsetbuttcap
    \pgfpathqmoveto{1.8957bp}{57.3460bp}
    \pgfpathqlineto{2.8436bp}{57.8199bp}
    \pgfpathqlineto{3.7915bp}{58.2938bp}
    \pgfpathqlineto{4.7393bp}{58.7678bp}
    \pgfpathqlineto{5.6872bp}{59.2417bp}
    \pgfpathqlineto{6.6351bp}{59.7156bp}
    \pgfpathqlineto{7.5829bp}{60.1896bp}
    \pgfpathqlineto{8.5308bp}{60.6635bp}
    \pgfpathqlineto{9.4787bp}{61.1374bp}
    \pgfpathqlineto{10.4265bp}{61.6114bp}
    \pgfpathqlineto{11.3744bp}{62.0853bp}
    \pgfpathqlineto{12.3223bp}{62.5592bp}
    \pgfpathqlineto{13.2701bp}{63.0332bp}
    \pgfpathqlineto{14.2180bp}{63.5071bp}
    \pgfpathqlineto{15.1659bp}{63.9810bp}
    \pgfpathqlineto{16.1137bp}{64.4550bp}
    \pgfpathqlineto{17.0616bp}{64.9289bp}
    \pgfpathqlineto{18.0095bp}{65.4028bp}
    \pgfpathqlineto{18.9573bp}{65.8768bp}
    \pgfpathqlineto{19.9052bp}{66.3507bp}
    \pgfpathqlineto{20.8531bp}{66.8246bp}
    \pgfpathqlineto{21.8009bp}{67.2986bp}
    \pgfpathqlineto{22.7488bp}{67.7725bp}
    \pgfpathqlineto{23.6967bp}{68.2464bp}
    \pgfpathqlineto{24.6445bp}{68.7204bp}
    \pgfpathqlineto{25.5924bp}{69.1943bp}
    \pgfpathqlineto{26.5403bp}{69.6682bp}
    \pgfpathqlineto{27.4882bp}{70.1422bp}
    \pgfpathqlineto{28.4360bp}{70.6161bp}
    \pgfpathqlineto{29.3839bp}{71.0900bp}
    \pgfpathqlineto{30.3318bp}{71.5640bp}
    \pgfpathqlineto{31.2796bp}{72.0379bp}
    \pgfpathqlineto{32.2275bp}{72.5118bp}
    \pgfpathqlineto{33.1754bp}{72.9858bp}
    \pgfpathqlineto{34.1232bp}{73.4597bp}
    \pgfpathqlineto{35.0711bp}{73.9336bp}
    \pgfpathqlineto{36.0190bp}{74.4076bp}
    \pgfpathqlineto{36.9668bp}{74.8815bp}
    \pgfpathqlineto{37.9147bp}{75.3555bp}
    \pgfpathqlineto{38.8626bp}{75.8294bp}
    \pgfpathqlineto{39.8104bp}{76.3033bp}
    \pgfpathqlineto{40.7583bp}{76.7773bp}
    \pgfpathqlineto{41.7062bp}{77.2512bp}
    \pgfpathqlineto{42.6540bp}{77.7251bp}
    \pgfpathqlineto{43.6019bp}{78.1991bp}
    \pgfpathqlineto{44.5498bp}{78.6730bp}
    \pgfpathqlineto{45.4976bp}{79.1469bp}
    \pgfpathqlineto{46.4455bp}{79.6209bp}
    \pgfpathqlineto{47.3934bp}{80.0948bp}
    \pgfpathqlineto{48.3412bp}{80.5687bp}
    \pgfpathqlineto{49.2891bp}{81.0427bp}
    \pgfpathqlineto{50.2370bp}{81.5166bp}
    \pgfpathqlineto{51.1848bp}{81.9905bp}
    \pgfpathqlineto{52.1327bp}{82.4645bp}
    \pgfpathqlineto{53.0806bp}{82.9384bp}
    \pgfpathqlineto{54.0284bp}{83.4123bp}
    \pgfpathqlineto{54.9763bp}{83.8863bp}
    \pgfpathqlineto{55.9242bp}{84.3602bp}
    \pgfpathqlineto{56.8720bp}{84.8341bp}
    \pgfpathqlineto{57.8199bp}{85.3081bp}
    \pgfpathqlineto{58.7678bp}{85.7820bp}
    \pgfpathqlineto{59.7156bp}{86.2559bp}
    \pgfpathqlineto{60.6635bp}{86.7299bp}
    \pgfpathqlineto{61.6114bp}{87.2038bp}
    \pgfpathqlineto{62.5592bp}{87.6777bp}
    \pgfpathqlineto{63.5071bp}{88.1517bp}
    \pgfpathqlineto{64.4550bp}{88.6256bp}
    \pgfpathqlineto{65.4028bp}{89.0995bp}
    \pgfpathqlineto{66.3507bp}{89.5735bp}
    \pgfpathqlineto{67.2986bp}{90.0474bp}
    \pgfpathqlineto{68.2464bp}{90.5213bp}
    \pgfpathqlineto{69.1943bp}{90.9953bp}
    \pgfpathqlineto{70.1422bp}{91.4692bp}
    \pgfpathqlineto{71.0900bp}{91.9431bp}
    \pgfpathqlineto{72.0379bp}{92.4171bp}
    \pgfpathqlineto{72.9858bp}{92.8910bp}
    \pgfpathqlineto{73.9336bp}{93.3649bp}
    \pgfpathqlineto{74.8815bp}{93.8389bp}
    \pgfpathqlineto{75.8294bp}{94.3128bp}
    \pgfpathqlineto{76.7773bp}{94.7867bp}
    \pgfpathqlineto{77.7251bp}{95.2607bp}
    \pgfpathqlineto{78.6730bp}{95.7346bp}
    \pgfpathqlineto{79.6209bp}{96.2085bp}
    \pgfpathqlineto{80.5687bp}{96.6825bp}
    \pgfpathqlineto{81.5166bp}{97.1564bp}
    \pgfpathqlineto{82.4645bp}{97.6303bp}
    \pgfpathqlineto{83.4123bp}{98.1043bp}
    \pgfpathqlineto{84.3602bp}{98.5782bp}
    \pgfpathqlineto{85.3081bp}{99.0521bp}
    \pgfpathqlineto{86.2559bp}{99.5261bp}
    \pgfpathqlineto{87.2038bp}{100.0000bp}
    \pgfpathqlineto{88.1517bp}{100.4739bp}
    \pgfpathqlineto{89.0995bp}{100.9479bp}
    \pgfpathqlineto{90.0474bp}{101.4218bp}
    \pgfpathqlineto{90.9953bp}{101.8957bp}
    \pgfpathqlineto{91.9431bp}{102.3697bp}
    \pgfpathqlineto{92.8910bp}{102.8436bp}
    \pgfpathqlineto{93.8389bp}{103.3175bp}
    \pgfpathqlineto{94.7867bp}{103.7915bp}
    \pgfpathqlineto{95.7346bp}{104.2654bp}
    \pgfpathqlineto{96.6825bp}{104.7393bp}
    \pgfpathqlineto{97.6303bp}{105.2133bp}
    \pgfpathqlineto{98.5782bp}{105.6872bp}
    \pgfpathqlineto{99.5261bp}{106.1611bp}
    \pgfpathqlineto{100.4739bp}{106.6351bp}
    \pgfpathqlineto{101.4218bp}{107.1090bp}
    \pgfpathqlineto{102.3697bp}{107.5829bp}
    \pgfpathqlineto{103.3175bp}{108.0569bp}
    \pgfpathqlineto{104.2654bp}{108.5308bp}
    \pgfpathqlineto{105.2133bp}{109.0047bp}
    \pgfpathqlineto{106.1611bp}{109.4787bp}
    \pgfpathqlineto{107.1090bp}{109.9526bp}
    \pgfpathqlineto{108.0569bp}{110.4265bp}
    \pgfpathqlineto{109.0047bp}{110.9005bp}
    \pgfpathqlineto{109.9526bp}{111.3744bp}
    \pgfpathqlineto{110.9005bp}{111.8483bp}
    \pgfpathqlineto{111.8483bp}{112.3223bp}
    \pgfpathqlineto{112.7962bp}{112.7962bp}
    \pgfpathqlineto{113.7441bp}{113.2701bp}
    \pgfpathqlineto{114.6919bp}{113.7441bp}
    \pgfpathqlineto{115.6398bp}{114.2180bp}
    \pgfpathqlineto{116.5877bp}{114.6919bp}
    \pgfpathqlineto{117.5355bp}{115.1659bp}
    \pgfpathqlineto{118.4834bp}{115.6398bp}
    \pgfpathqlineto{119.4313bp}{116.1137bp}
    \pgfpathqlineto{120.3791bp}{116.5877bp}
    \pgfpathqlineto{121.3270bp}{117.0616bp}
    \pgfpathqlineto{122.2749bp}{117.5355bp}
    \pgfpathqlineto{123.2227bp}{118.0095bp}
    \pgfpathqlineto{124.1706bp}{118.4834bp}
    \pgfpathqlineto{125.1185bp}{118.9573bp}
    \pgfpathqlineto{126.0664bp}{119.4313bp}
    \pgfpathqlineto{127.0142bp}{119.9052bp}
    \pgfpathqlineto{127.9621bp}{120.3791bp}
    \pgfpathqlineto{128.9100bp}{120.8531bp}
    \pgfpathqlineto{129.8578bp}{121.3270bp}
    \pgfpathqlineto{130.8057bp}{121.8009bp}
    \pgfpathqlineto{131.7536bp}{122.2749bp}
    \pgfpathqlineto{132.7014bp}{122.7488bp}
    \pgfpathqlineto{133.6493bp}{123.2227bp}
    \pgfpathqlineto{134.5972bp}{123.6967bp}
    \pgfpathqlineto{135.5450bp}{124.1706bp}
    \pgfpathqlineto{136.4929bp}{124.6445bp}
    \pgfpathqlineto{137.4408bp}{125.1185bp}
    \pgfpathqlineto{138.3886bp}{125.5924bp}
    \pgfpathqlineto{139.3365bp}{126.0664bp}
    \pgfpathqlineto{140.2844bp}{126.5403bp}
    \pgfpathqlineto{141.2322bp}{127.0142bp}
    \pgfpathqlineto{142.1801bp}{127.4882bp}
    \pgfpathqlineto{143.1280bp}{127.9621bp}
    \pgfpathqlineto{144.0758bp}{128.4360bp}
    \pgfpathqlineto{145.0237bp}{128.9100bp}
    \pgfpathqlineto{145.9716bp}{129.3839bp}
    \pgfpathqlineto{146.9194bp}{129.8578bp}
    \pgfpathqlineto{147.8673bp}{130.3318bp}
    \pgfpathqlineto{148.8152bp}{130.8057bp}
    \pgfpathqlineto{149.7630bp}{131.2796bp}
    \pgfpathqlineto{150.7109bp}{131.7536bp}
    \pgfpathqlineto{151.6588bp}{132.2275bp}
    \pgfpathqlineto{152.6066bp}{132.7014bp}
    \pgfpathqlineto{153.5545bp}{133.1754bp}
    \pgfpathqlineto{154.5024bp}{133.6493bp}
    \pgfpathqlineto{155.4502bp}{134.1232bp}
    \pgfpathqlineto{156.3981bp}{134.5972bp}
    \pgfpathqlineto{157.3460bp}{135.0711bp}
    \pgfpathqlineto{158.2938bp}{135.5450bp}
    \pgfpathqlineto{159.2417bp}{136.0190bp}
    \pgfpathqlineto{160.1896bp}{136.4929bp}
    \pgfpathqlineto{161.1374bp}{136.9668bp}
    \pgfpathqlineto{162.0853bp}{137.4408bp}
    \pgfpathqlineto{163.0332bp}{137.9147bp}
    \pgfpathqlineto{163.9810bp}{138.3886bp}
    \pgfpathqlineto{164.9289bp}{138.8626bp}
    \pgfpathqlineto{165.8768bp}{139.3365bp}
    \pgfpathqlineto{166.8246bp}{139.8104bp}
    \pgfpathqlineto{167.7725bp}{140.2844bp}
    \pgfpathqlineto{168.7204bp}{140.7583bp}
    \pgfpathqlineto{169.6682bp}{141.2322bp}
    \pgfpathqlineto{170.6161bp}{141.7062bp}
    \pgfpathqlineto{171.5640bp}{142.1801bp}
    \pgfpathqlineto{172.5118bp}{142.6540bp}
    \pgfpathqlineto{173.4597bp}{143.1280bp}
    \pgfpathqlineto{174.4076bp}{143.6019bp}
    \pgfpathqlineto{175.3555bp}{144.0758bp}
    \pgfpathqlineto{176.3033bp}{144.5498bp}
    \pgfpathqlineto{177.2512bp}{145.0237bp}
    \pgfpathqlineto{178.1991bp}{145.4976bp}
    \pgfpathqlineto{179.1469bp}{145.9716bp}
    \pgfpathqlineto{180.0948bp}{146.4455bp}
    \pgfpathqlineto{181.0427bp}{146.9194bp}
    \pgfpathqlineto{181.9905bp}{147.3934bp}
    \pgfpathqlineto{182.9384bp}{147.8673bp}
    \pgfpathqlineto{183.8863bp}{148.3412bp}
    \pgfpathqlineto{184.8341bp}{148.8152bp}
    \pgfpathqlineto{185.7820bp}{149.2891bp}
    \pgfpathqlineto{186.7299bp}{149.7630bp}
    \pgfpathqlineto{187.6777bp}{149.7630bp}
    \pgfpathqlineto{188.6256bp}{150.2370bp}
    \pgfpathqlineto{189.5735bp}{129.3839bp}
    \pgfpathqlineto{190.5213bp}{129.3839bp}
    \pgfpathqlineto{191.4692bp}{129.8578bp}
    \pgfpathqlineto{192.4171bp}{126.0664bp}
    \pgfpathqlineto{193.3649bp}{124.1706bp}
    \pgfpathqlineto{194.3128bp}{118.0095bp}
    \pgfpathqlineto{195.2607bp}{115.6398bp}
    \pgfpathqlineto{196.2085bp}{71.0900bp}
    \pgfpathqlineto{197.1564bp}{71.0900bp}
    \pgfpathqlineto{198.1043bp}{71.0900bp}
    \pgfpathqlineto{199.0521bp}{71.5640bp}
    \pgfusepathqstroke
  \end{pgfscope}
  \begin{pgfscope}
    \pgfsetlinewidth{0.5670bp}
    \definecolor{sc}{rgb}{1.0000,1.0000,0.0000}
    \pgfsetstrokecolor{sc}
    \pgfsetmiterjoin
    \pgfsetbuttcap
    \pgfpathqmoveto{1.8957bp}{87.2038bp}
    \pgfpathqlineto{2.8436bp}{87.2038bp}
    \pgfpathqlineto{3.7915bp}{87.2038bp}
    \pgfpathqlineto{4.7393bp}{87.2038bp}
    \pgfpathqlineto{5.6872bp}{87.2038bp}
    \pgfpathqlineto{6.6351bp}{87.2038bp}
    \pgfpathqlineto{7.5829bp}{87.2038bp}
    \pgfpathqlineto{8.5308bp}{87.2038bp}
    \pgfpathqlineto{9.4787bp}{87.2038bp}
    \pgfpathqlineto{10.4265bp}{87.2038bp}
    \pgfpathqlineto{11.3744bp}{87.2038bp}
    \pgfpathqlineto{12.3223bp}{87.2038bp}
    \pgfpathqlineto{13.2701bp}{87.2038bp}
    \pgfpathqlineto{14.2180bp}{87.2038bp}
    \pgfpathqlineto{15.1659bp}{87.2038bp}
    \pgfpathqlineto{16.1137bp}{87.2038bp}
    \pgfpathqlineto{17.0616bp}{87.2038bp}
    \pgfpathqlineto{18.0095bp}{87.2038bp}
    \pgfpathqlineto{18.9573bp}{87.2038bp}
    \pgfpathqlineto{19.9052bp}{87.2038bp}
    \pgfpathqlineto{20.8531bp}{87.2038bp}
    \pgfpathqlineto{21.8009bp}{87.2038bp}
    \pgfpathqlineto{22.7488bp}{87.2038bp}
    \pgfpathqlineto{23.6967bp}{87.2038bp}
    \pgfpathqlineto{24.6445bp}{87.2038bp}
    \pgfpathqlineto{25.5924bp}{87.2038bp}
    \pgfpathqlineto{26.5403bp}{87.2038bp}
    \pgfpathqlineto{27.4882bp}{87.2038bp}
    \pgfpathqlineto{28.4360bp}{87.2038bp}
    \pgfpathqlineto{29.3839bp}{87.2038bp}
    \pgfpathqlineto{30.3318bp}{87.2038bp}
    \pgfpathqlineto{31.2796bp}{87.2038bp}
    \pgfpathqlineto{32.2275bp}{87.2038bp}
    \pgfpathqlineto{33.1754bp}{87.2038bp}
    \pgfpathqlineto{34.1232bp}{87.2038bp}
    \pgfpathqlineto{35.0711bp}{87.2038bp}
    \pgfpathqlineto{36.0190bp}{87.2038bp}
    \pgfpathqlineto{36.9668bp}{87.2038bp}
    \pgfpathqlineto{37.9147bp}{87.2038bp}
    \pgfpathqlineto{38.8626bp}{87.2038bp}
    \pgfpathqlineto{39.8104bp}{87.2038bp}
    \pgfpathqlineto{40.7583bp}{87.2038bp}
    \pgfpathqlineto{41.7062bp}{87.2038bp}
    \pgfpathqlineto{42.6540bp}{87.2038bp}
    \pgfpathqlineto{43.6019bp}{87.2038bp}
    \pgfpathqlineto{44.5498bp}{87.2038bp}
    \pgfpathqlineto{45.4976bp}{87.2038bp}
    \pgfpathqlineto{46.4455bp}{87.2038bp}
    \pgfpathqlineto{47.3934bp}{87.2038bp}
    \pgfpathqlineto{48.3412bp}{87.2038bp}
    \pgfpathqlineto{49.2891bp}{87.2038bp}
    \pgfpathqlineto{50.2370bp}{87.2038bp}
    \pgfpathqlineto{51.1848bp}{87.2038bp}
    \pgfpathqlineto{52.1327bp}{87.2038bp}
    \pgfpathqlineto{53.0806bp}{87.2038bp}
    \pgfpathqlineto{54.0284bp}{87.2038bp}
    \pgfpathqlineto{54.9763bp}{87.2038bp}
    \pgfpathqlineto{55.9242bp}{87.2038bp}
    \pgfpathqlineto{56.8720bp}{87.2038bp}
    \pgfpathqlineto{57.8199bp}{87.2038bp}
    \pgfpathqlineto{58.7678bp}{87.2038bp}
    \pgfpathqlineto{59.7156bp}{87.2038bp}
    \pgfpathqlineto{60.6635bp}{87.2038bp}
    \pgfpathqlineto{61.6114bp}{87.2038bp}
    \pgfpathqlineto{62.5592bp}{87.2038bp}
    \pgfpathqlineto{63.5071bp}{87.2038bp}
    \pgfpathqlineto{64.4550bp}{87.2038bp}
    \pgfpathqlineto{65.4028bp}{87.2038bp}
    \pgfpathqlineto{66.3507bp}{87.2038bp}
    \pgfpathqlineto{67.2986bp}{87.2038bp}
    \pgfpathqlineto{68.2464bp}{87.2038bp}
    \pgfpathqlineto{69.1943bp}{87.2038bp}
    \pgfpathqlineto{70.1422bp}{87.2038bp}
    \pgfpathqlineto{71.0900bp}{87.2038bp}
    \pgfpathqlineto{72.0379bp}{87.2038bp}
    \pgfpathqlineto{72.9858bp}{87.2038bp}
    \pgfpathqlineto{73.9336bp}{87.2038bp}
    \pgfpathqlineto{74.8815bp}{87.2038bp}
    \pgfpathqlineto{75.8294bp}{87.2038bp}
    \pgfpathqlineto{76.7773bp}{87.2038bp}
    \pgfpathqlineto{77.7251bp}{87.2038bp}
    \pgfpathqlineto{78.6730bp}{87.2038bp}
    \pgfpathqlineto{79.6209bp}{87.2038bp}
    \pgfpathqlineto{80.5687bp}{87.2038bp}
    \pgfpathqlineto{81.5166bp}{87.2038bp}
    \pgfpathqlineto{82.4645bp}{87.2038bp}
    \pgfpathqlineto{83.4123bp}{87.2038bp}
    \pgfpathqlineto{84.3602bp}{87.2038bp}
    \pgfpathqlineto{85.3081bp}{87.2038bp}
    \pgfpathqlineto{86.2559bp}{87.2038bp}
    \pgfpathqlineto{87.2038bp}{87.2038bp}
    \pgfpathqlineto{88.1517bp}{87.2038bp}
    \pgfpathqlineto{89.0995bp}{87.2038bp}
    \pgfpathqlineto{90.0474bp}{87.2038bp}
    \pgfpathqlineto{90.9953bp}{87.2038bp}
    \pgfpathqlineto{91.9431bp}{87.2038bp}
    \pgfpathqlineto{92.8910bp}{87.2038bp}
    \pgfpathqlineto{93.8389bp}{87.2038bp}
    \pgfpathqlineto{94.7867bp}{87.2038bp}
    \pgfpathqlineto{95.7346bp}{87.2038bp}
    \pgfpathqlineto{96.6825bp}{87.2038bp}
    \pgfpathqlineto{97.6303bp}{87.2038bp}
    \pgfpathqlineto{98.5782bp}{87.2038bp}
    \pgfpathqlineto{99.5261bp}{87.2038bp}
    \pgfpathqlineto{100.4739bp}{87.2038bp}
    \pgfpathqlineto{101.4218bp}{87.2038bp}
    \pgfpathqlineto{102.3697bp}{87.2038bp}
    \pgfpathqlineto{103.3175bp}{87.2038bp}
    \pgfpathqlineto{104.2654bp}{87.2038bp}
    \pgfpathqlineto{105.2133bp}{87.2038bp}
    \pgfpathqlineto{106.1611bp}{87.2038bp}
    \pgfpathqlineto{107.1090bp}{87.2038bp}
    \pgfpathqlineto{108.0569bp}{87.2038bp}
    \pgfpathqlineto{109.0047bp}{87.2038bp}
    \pgfpathqlineto{109.9526bp}{87.2038bp}
    \pgfpathqlineto{110.9005bp}{87.2038bp}
    \pgfpathqlineto{111.8483bp}{87.2038bp}
    \pgfpathqlineto{112.7962bp}{87.2038bp}
    \pgfpathqlineto{113.7441bp}{87.2038bp}
    \pgfpathqlineto{114.6919bp}{87.2038bp}
    \pgfpathqlineto{115.6398bp}{87.2038bp}
    \pgfpathqlineto{116.5877bp}{87.2038bp}
    \pgfpathqlineto{117.5355bp}{87.2038bp}
    \pgfpathqlineto{118.4834bp}{87.2038bp}
    \pgfpathqlineto{119.4313bp}{87.2038bp}
    \pgfpathqlineto{120.3791bp}{87.2038bp}
    \pgfpathqlineto{121.3270bp}{87.2038bp}
    \pgfpathqlineto{122.2749bp}{87.2038bp}
    \pgfpathqlineto{123.2227bp}{87.2038bp}
    \pgfpathqlineto{124.1706bp}{87.2038bp}
    \pgfpathqlineto{125.1185bp}{87.2038bp}
    \pgfpathqlineto{126.0664bp}{87.2038bp}
    \pgfpathqlineto{127.0142bp}{87.2038bp}
    \pgfpathqlineto{127.9621bp}{87.2038bp}
    \pgfpathqlineto{128.9100bp}{87.2038bp}
    \pgfpathqlineto{129.8578bp}{87.2038bp}
    \pgfpathqlineto{130.8057bp}{87.2038bp}
    \pgfpathqlineto{131.7536bp}{87.2038bp}
    \pgfpathqlineto{132.7014bp}{87.2038bp}
    \pgfpathqlineto{133.6493bp}{87.2038bp}
    \pgfpathqlineto{134.5972bp}{87.2038bp}
    \pgfpathqlineto{135.5450bp}{87.2038bp}
    \pgfpathqlineto{136.4929bp}{87.2038bp}
    \pgfpathqlineto{137.4408bp}{87.2038bp}
    \pgfpathqlineto{138.3886bp}{87.2038bp}
    \pgfpathqlineto{139.3365bp}{87.2038bp}
    \pgfpathqlineto{140.2844bp}{87.2038bp}
    \pgfpathqlineto{141.2322bp}{87.2038bp}
    \pgfpathqlineto{142.1801bp}{87.2038bp}
    \pgfpathqlineto{143.1280bp}{87.2038bp}
    \pgfpathqlineto{144.0758bp}{87.2038bp}
    \pgfpathqlineto{145.0237bp}{87.2038bp}
    \pgfpathqlineto{145.9716bp}{87.2038bp}
    \pgfpathqlineto{146.9194bp}{87.2038bp}
    \pgfpathqlineto{147.8673bp}{87.2038bp}
    \pgfpathqlineto{148.8152bp}{87.2038bp}
    \pgfpathqlineto{149.7630bp}{87.2038bp}
    \pgfpathqlineto{150.7109bp}{87.2038bp}
    \pgfpathqlineto{151.6588bp}{87.2038bp}
    \pgfpathqlineto{152.6066bp}{87.2038bp}
    \pgfpathqlineto{153.5545bp}{87.2038bp}
    \pgfpathqlineto{154.5024bp}{87.2038bp}
    \pgfpathqlineto{155.4502bp}{87.2038bp}
    \pgfpathqlineto{156.3981bp}{87.2038bp}
    \pgfpathqlineto{157.3460bp}{87.2038bp}
    \pgfpathqlineto{158.2938bp}{87.2038bp}
    \pgfpathqlineto{159.2417bp}{87.2038bp}
    \pgfpathqlineto{160.1896bp}{87.2038bp}
    \pgfpathqlineto{161.1374bp}{87.2038bp}
    \pgfpathqlineto{162.0853bp}{87.2038bp}
    \pgfpathqlineto{163.0332bp}{87.2038bp}
    \pgfpathqlineto{163.9810bp}{87.2038bp}
    \pgfpathqlineto{164.9289bp}{87.2038bp}
    \pgfpathqlineto{165.8768bp}{87.2038bp}
    \pgfpathqlineto{166.8246bp}{87.2038bp}
    \pgfpathqlineto{167.7725bp}{87.2038bp}
    \pgfpathqlineto{168.7204bp}{87.2038bp}
    \pgfpathqlineto{169.6682bp}{87.2038bp}
    \pgfpathqlineto{170.6161bp}{87.2038bp}
    \pgfpathqlineto{171.5640bp}{87.2038bp}
    \pgfpathqlineto{172.5118bp}{87.2038bp}
    \pgfpathqlineto{173.4597bp}{87.2038bp}
    \pgfpathqlineto{174.4076bp}{87.2038bp}
    \pgfpathqlineto{175.3555bp}{87.2038bp}
    \pgfpathqlineto{176.3033bp}{87.2038bp}
    \pgfpathqlineto{177.2512bp}{87.2038bp}
    \pgfpathqlineto{178.1991bp}{87.2038bp}
    \pgfpathqlineto{179.1469bp}{87.2038bp}
    \pgfpathqlineto{180.0948bp}{87.2038bp}
    \pgfpathqlineto{181.0427bp}{87.2038bp}
    \pgfpathqlineto{181.9905bp}{87.2038bp}
    \pgfpathqlineto{182.9384bp}{87.2038bp}
    \pgfpathqlineto{183.8863bp}{87.2038bp}
    \pgfpathqlineto{184.8341bp}{87.2038bp}
    \pgfpathqlineto{185.7820bp}{87.2038bp}
    \pgfpathqlineto{186.7299bp}{87.2038bp}
    \pgfpathqlineto{187.6777bp}{83.8863bp}
    \pgfpathqlineto{188.6256bp}{83.8863bp}
    \pgfpathqlineto{189.5735bp}{83.8863bp}
    \pgfpathqlineto{190.5213bp}{83.8863bp}
    \pgfpathqlineto{191.4692bp}{83.8863bp}
    \pgfpathqlineto{192.4171bp}{83.8863bp}
    \pgfpathqlineto{193.3649bp}{83.8863bp}
    \pgfpathqlineto{194.3128bp}{83.8863bp}
    \pgfpathqlineto{195.2607bp}{83.8863bp}
    \pgfpathqlineto{196.2085bp}{83.8863bp}
    \pgfpathqlineto{197.1564bp}{83.8863bp}
    \pgfpathqlineto{198.1043bp}{83.8863bp}
    \pgfpathqlineto{199.0521bp}{83.8863bp}
    \pgfusepathqstroke
  \end{pgfscope}
  \begin{pgfscope}
    \pgfsetlinewidth{0.5670bp}
    \definecolor{sc}{rgb}{0.0000,0.5020,0.0000}
    \pgfsetstrokecolor{sc}
    \pgfsetmiterjoin
    \pgfsetbuttcap
    \pgfpathqmoveto{1.8957bp}{56.8720bp}
    \pgfpathqlineto{2.8436bp}{56.8720bp}
    \pgfpathqlineto{3.7915bp}{56.8720bp}
    \pgfpathqlineto{4.7393bp}{56.8720bp}
    \pgfpathqlineto{5.6872bp}{56.8720bp}
    \pgfpathqlineto{6.6351bp}{56.8720bp}
    \pgfpathqlineto{7.5829bp}{56.8720bp}
    \pgfpathqlineto{8.5308bp}{56.8720bp}
    \pgfpathqlineto{9.4787bp}{56.8720bp}
    \pgfpathqlineto{10.4265bp}{56.8720bp}
    \pgfpathqlineto{11.3744bp}{56.8720bp}
    \pgfpathqlineto{12.3223bp}{56.8720bp}
    \pgfpathqlineto{13.2701bp}{56.8720bp}
    \pgfpathqlineto{14.2180bp}{56.8720bp}
    \pgfpathqlineto{15.1659bp}{56.8720bp}
    \pgfpathqlineto{16.1137bp}{56.8720bp}
    \pgfpathqlineto{17.0616bp}{56.8720bp}
    \pgfpathqlineto{18.0095bp}{56.8720bp}
    \pgfpathqlineto{18.9573bp}{56.8720bp}
    \pgfpathqlineto{19.9052bp}{56.8720bp}
    \pgfpathqlineto{20.8531bp}{56.8720bp}
    \pgfpathqlineto{21.8009bp}{56.8720bp}
    \pgfpathqlineto{22.7488bp}{56.8720bp}
    \pgfpathqlineto{23.6967bp}{56.8720bp}
    \pgfpathqlineto{24.6445bp}{56.8720bp}
    \pgfpathqlineto{25.5924bp}{56.8720bp}
    \pgfpathqlineto{26.5403bp}{56.8720bp}
    \pgfpathqlineto{27.4882bp}{56.8720bp}
    \pgfpathqlineto{28.4360bp}{56.8720bp}
    \pgfpathqlineto{29.3839bp}{56.8720bp}
    \pgfpathqlineto{30.3318bp}{56.8720bp}
    \pgfpathqlineto{31.2796bp}{56.8720bp}
    \pgfpathqlineto{32.2275bp}{56.8720bp}
    \pgfpathqlineto{33.1754bp}{56.8720bp}
    \pgfpathqlineto{34.1232bp}{56.8720bp}
    \pgfpathqlineto{35.0711bp}{56.8720bp}
    \pgfpathqlineto{36.0190bp}{56.8720bp}
    \pgfpathqlineto{36.9668bp}{56.8720bp}
    \pgfpathqlineto{37.9147bp}{56.8720bp}
    \pgfpathqlineto{38.8626bp}{56.8720bp}
    \pgfpathqlineto{39.8104bp}{56.8720bp}
    \pgfpathqlineto{40.7583bp}{56.8720bp}
    \pgfpathqlineto{41.7062bp}{56.8720bp}
    \pgfpathqlineto{42.6540bp}{56.8720bp}
    \pgfpathqlineto{43.6019bp}{56.8720bp}
    \pgfpathqlineto{44.5498bp}{56.8720bp}
    \pgfpathqlineto{45.4976bp}{56.8720bp}
    \pgfpathqlineto{46.4455bp}{56.8720bp}
    \pgfpathqlineto{47.3934bp}{56.8720bp}
    \pgfpathqlineto{48.3412bp}{56.8720bp}
    \pgfpathqlineto{49.2891bp}{56.8720bp}
    \pgfpathqlineto{50.2370bp}{56.8720bp}
    \pgfpathqlineto{51.1848bp}{56.8720bp}
    \pgfpathqlineto{52.1327bp}{56.8720bp}
    \pgfpathqlineto{53.0806bp}{56.8720bp}
    \pgfpathqlineto{54.0284bp}{56.8720bp}
    \pgfpathqlineto{54.9763bp}{56.8720bp}
    \pgfpathqlineto{55.9242bp}{56.8720bp}
    \pgfpathqlineto{56.8720bp}{56.8720bp}
    \pgfpathqlineto{57.8199bp}{56.8720bp}
    \pgfpathqlineto{58.7678bp}{56.8720bp}
    \pgfpathqlineto{59.7156bp}{56.8720bp}
    \pgfpathqlineto{60.6635bp}{56.8720bp}
    \pgfpathqlineto{61.6114bp}{56.8720bp}
    \pgfpathqlineto{62.5592bp}{56.8720bp}
    \pgfpathqlineto{63.5071bp}{56.8720bp}
    \pgfpathqlineto{64.4550bp}{56.8720bp}
    \pgfpathqlineto{65.4028bp}{56.8720bp}
    \pgfpathqlineto{66.3507bp}{56.8720bp}
    \pgfpathqlineto{67.2986bp}{56.8720bp}
    \pgfpathqlineto{68.2464bp}{56.8720bp}
    \pgfpathqlineto{69.1943bp}{56.8720bp}
    \pgfpathqlineto{70.1422bp}{56.8720bp}
    \pgfpathqlineto{71.0900bp}{56.8720bp}
    \pgfpathqlineto{72.0379bp}{56.8720bp}
    \pgfpathqlineto{72.9858bp}{56.8720bp}
    \pgfpathqlineto{73.9336bp}{56.8720bp}
    \pgfpathqlineto{74.8815bp}{56.8720bp}
    \pgfpathqlineto{75.8294bp}{56.8720bp}
    \pgfpathqlineto{76.7773bp}{56.8720bp}
    \pgfpathqlineto{77.7251bp}{56.8720bp}
    \pgfpathqlineto{78.6730bp}{56.8720bp}
    \pgfpathqlineto{79.6209bp}{56.8720bp}
    \pgfpathqlineto{80.5687bp}{56.8720bp}
    \pgfpathqlineto{81.5166bp}{56.8720bp}
    \pgfpathqlineto{82.4645bp}{56.8720bp}
    \pgfpathqlineto{83.4123bp}{56.8720bp}
    \pgfpathqlineto{84.3602bp}{56.8720bp}
    \pgfpathqlineto{85.3081bp}{56.8720bp}
    \pgfpathqlineto{86.2559bp}{56.8720bp}
    \pgfpathqlineto{87.2038bp}{56.8720bp}
    \pgfpathqlineto{88.1517bp}{56.8720bp}
    \pgfpathqlineto{89.0995bp}{56.8720bp}
    \pgfpathqlineto{90.0474bp}{56.8720bp}
    \pgfpathqlineto{90.9953bp}{56.8720bp}
    \pgfpathqlineto{91.9431bp}{56.8720bp}
    \pgfpathqlineto{92.8910bp}{56.8720bp}
    \pgfpathqlineto{93.8389bp}{56.8720bp}
    \pgfpathqlineto{94.7867bp}{56.8720bp}
    \pgfpathqlineto{95.7346bp}{56.8720bp}
    \pgfpathqlineto{96.6825bp}{56.8720bp}
    \pgfpathqlineto{97.6303bp}{56.8720bp}
    \pgfpathqlineto{98.5782bp}{56.8720bp}
    \pgfpathqlineto{99.5261bp}{56.8720bp}
    \pgfpathqlineto{100.4739bp}{56.8720bp}
    \pgfpathqlineto{101.4218bp}{56.8720bp}
    \pgfpathqlineto{102.3697bp}{56.8720bp}
    \pgfpathqlineto{103.3175bp}{56.8720bp}
    \pgfpathqlineto{104.2654bp}{56.8720bp}
    \pgfpathqlineto{105.2133bp}{56.8720bp}
    \pgfpathqlineto{106.1611bp}{56.8720bp}
    \pgfpathqlineto{107.1090bp}{56.8720bp}
    \pgfpathqlineto{108.0569bp}{56.8720bp}
    \pgfpathqlineto{109.0047bp}{56.8720bp}
    \pgfpathqlineto{109.9526bp}{56.8720bp}
    \pgfpathqlineto{110.9005bp}{56.8720bp}
    \pgfpathqlineto{111.8483bp}{56.8720bp}
    \pgfpathqlineto{112.7962bp}{56.8720bp}
    \pgfpathqlineto{113.7441bp}{56.8720bp}
    \pgfpathqlineto{114.6919bp}{56.8720bp}
    \pgfpathqlineto{115.6398bp}{56.8720bp}
    \pgfpathqlineto{116.5877bp}{56.8720bp}
    \pgfpathqlineto{117.5355bp}{56.8720bp}
    \pgfpathqlineto{118.4834bp}{56.8720bp}
    \pgfpathqlineto{119.4313bp}{56.8720bp}
    \pgfpathqlineto{120.3791bp}{56.8720bp}
    \pgfpathqlineto{121.3270bp}{56.8720bp}
    \pgfpathqlineto{122.2749bp}{56.8720bp}
    \pgfpathqlineto{123.2227bp}{56.8720bp}
    \pgfpathqlineto{124.1706bp}{56.8720bp}
    \pgfpathqlineto{125.1185bp}{56.8720bp}
    \pgfpathqlineto{126.0664bp}{56.8720bp}
    \pgfpathqlineto{127.0142bp}{56.8720bp}
    \pgfpathqlineto{127.9621bp}{56.8720bp}
    \pgfpathqlineto{128.9100bp}{56.8720bp}
    \pgfpathqlineto{129.8578bp}{56.8720bp}
    \pgfpathqlineto{130.8057bp}{56.8720bp}
    \pgfpathqlineto{131.7536bp}{56.8720bp}
    \pgfpathqlineto{132.7014bp}{56.8720bp}
    \pgfpathqlineto{133.6493bp}{56.8720bp}
    \pgfpathqlineto{134.5972bp}{56.8720bp}
    \pgfpathqlineto{135.5450bp}{56.8720bp}
    \pgfpathqlineto{136.4929bp}{56.8720bp}
    \pgfpathqlineto{137.4408bp}{56.8720bp}
    \pgfpathqlineto{138.3886bp}{56.8720bp}
    \pgfpathqlineto{139.3365bp}{56.8720bp}
    \pgfpathqlineto{140.2844bp}{56.8720bp}
    \pgfpathqlineto{141.2322bp}{56.8720bp}
    \pgfpathqlineto{142.1801bp}{56.8720bp}
    \pgfpathqlineto{143.1280bp}{56.8720bp}
    \pgfpathqlineto{144.0758bp}{56.8720bp}
    \pgfpathqlineto{145.0237bp}{56.8720bp}
    \pgfpathqlineto{145.9716bp}{56.8720bp}
    \pgfpathqlineto{146.9194bp}{56.8720bp}
    \pgfpathqlineto{147.8673bp}{56.8720bp}
    \pgfpathqlineto{148.8152bp}{56.8720bp}
    \pgfpathqlineto{149.7630bp}{56.8720bp}
    \pgfpathqlineto{150.7109bp}{56.8720bp}
    \pgfpathqlineto{151.6588bp}{56.8720bp}
    \pgfpathqlineto{152.6066bp}{56.8720bp}
    \pgfpathqlineto{153.5545bp}{56.8720bp}
    \pgfpathqlineto{154.5024bp}{56.8720bp}
    \pgfpathqlineto{155.4502bp}{56.8720bp}
    \pgfpathqlineto{156.3981bp}{56.8720bp}
    \pgfpathqlineto{157.3460bp}{56.8720bp}
    \pgfpathqlineto{158.2938bp}{56.8720bp}
    \pgfpathqlineto{159.2417bp}{56.8720bp}
    \pgfpathqlineto{160.1896bp}{56.8720bp}
    \pgfpathqlineto{161.1374bp}{56.8720bp}
    \pgfpathqlineto{162.0853bp}{56.8720bp}
    \pgfpathqlineto{163.0332bp}{56.8720bp}
    \pgfpathqlineto{163.9810bp}{56.8720bp}
    \pgfpathqlineto{164.9289bp}{56.8720bp}
    \pgfpathqlineto{165.8768bp}{56.8720bp}
    \pgfpathqlineto{166.8246bp}{56.8720bp}
    \pgfpathqlineto{167.7725bp}{56.8720bp}
    \pgfpathqlineto{168.7204bp}{56.8720bp}
    \pgfpathqlineto{169.6682bp}{56.8720bp}
    \pgfpathqlineto{170.6161bp}{56.8720bp}
    \pgfpathqlineto{171.5640bp}{56.8720bp}
    \pgfpathqlineto{172.5118bp}{56.8720bp}
    \pgfpathqlineto{173.4597bp}{56.8720bp}
    \pgfpathqlineto{174.4076bp}{56.8720bp}
    \pgfpathqlineto{175.3555bp}{56.8720bp}
    \pgfpathqlineto{176.3033bp}{56.8720bp}
    \pgfpathqlineto{177.2512bp}{56.8720bp}
    \pgfpathqlineto{178.1991bp}{56.8720bp}
    \pgfpathqlineto{179.1469bp}{56.8720bp}
    \pgfpathqlineto{180.0948bp}{56.8720bp}
    \pgfpathqlineto{181.0427bp}{56.8720bp}
    \pgfpathqlineto{181.9905bp}{56.8720bp}
    \pgfpathqlineto{182.9384bp}{56.8720bp}
    \pgfpathqlineto{183.8863bp}{56.8720bp}
    \pgfpathqlineto{184.8341bp}{56.8720bp}
    \pgfpathqlineto{185.7820bp}{56.8720bp}
    \pgfpathqlineto{186.7299bp}{56.8720bp}
    \pgfpathqlineto{187.6777bp}{56.8720bp}
    \pgfpathqlineto{188.6256bp}{56.8720bp}
    \pgfpathqlineto{189.5735bp}{56.8720bp}
    \pgfpathqlineto{190.5213bp}{56.8720bp}
    \pgfpathqlineto{191.4692bp}{56.8720bp}
    \pgfpathqlineto{192.4171bp}{56.8720bp}
    \pgfpathqlineto{193.3649bp}{56.8720bp}
    \pgfpathqlineto{194.3128bp}{56.8720bp}
    \pgfpathqlineto{195.2607bp}{56.8720bp}
    \pgfpathqlineto{196.2085bp}{56.8720bp}
    \pgfpathqlineto{197.1564bp}{56.8720bp}
    \pgfpathqlineto{198.1043bp}{56.8720bp}
    \pgfpathqlineto{199.0521bp}{56.8720bp}
    \pgfusepathqstroke
  \end{pgfscope}
  \begin{pgfscope}
    \pgfsetlinewidth{0.5670bp}
    \definecolor{sc}{rgb}{1.0000,0.0000,0.0000}
    \pgfsetstrokecolor{sc}
    \pgfsetmiterjoin
    \pgfsetbuttcap
    \pgfpathqmoveto{1.8957bp}{87.2038bp}
    \pgfpathqlineto{2.8436bp}{87.2038bp}
    \pgfpathqlineto{3.7915bp}{87.2038bp}
    \pgfpathqlineto{4.7393bp}{87.2038bp}
    \pgfpathqlineto{5.6872bp}{87.2038bp}
    \pgfpathqlineto{6.6351bp}{87.2038bp}
    \pgfpathqlineto{7.5829bp}{87.2038bp}
    \pgfpathqlineto{8.5308bp}{87.2038bp}
    \pgfpathqlineto{9.4787bp}{87.2038bp}
    \pgfpathqlineto{10.4265bp}{87.2038bp}
    \pgfpathqlineto{11.3744bp}{87.2038bp}
    \pgfpathqlineto{12.3223bp}{87.2038bp}
    \pgfpathqlineto{13.2701bp}{87.2038bp}
    \pgfpathqlineto{14.2180bp}{87.2038bp}
    \pgfpathqlineto{15.1659bp}{87.2038bp}
    \pgfpathqlineto{16.1137bp}{87.2038bp}
    \pgfpathqlineto{17.0616bp}{87.2038bp}
    \pgfpathqlineto{18.0095bp}{87.2038bp}
    \pgfpathqlineto{18.9573bp}{87.2038bp}
    \pgfpathqlineto{19.9052bp}{87.2038bp}
    \pgfpathqlineto{20.8531bp}{87.2038bp}
    \pgfpathqlineto{21.8009bp}{87.2038bp}
    \pgfpathqlineto{22.7488bp}{87.2038bp}
    \pgfpathqlineto{23.6967bp}{87.2038bp}
    \pgfpathqlineto{24.6445bp}{87.2038bp}
    \pgfpathqlineto{25.5924bp}{87.2038bp}
    \pgfpathqlineto{26.5403bp}{87.2038bp}
    \pgfpathqlineto{27.4882bp}{87.2038bp}
    \pgfpathqlineto{28.4360bp}{87.2038bp}
    \pgfpathqlineto{29.3839bp}{87.2038bp}
    \pgfpathqlineto{30.3318bp}{87.2038bp}
    \pgfpathqlineto{31.2796bp}{87.2038bp}
    \pgfpathqlineto{32.2275bp}{87.2038bp}
    \pgfpathqlineto{33.1754bp}{87.2038bp}
    \pgfpathqlineto{34.1232bp}{87.2038bp}
    \pgfpathqlineto{35.0711bp}{87.2038bp}
    \pgfpathqlineto{36.0190bp}{87.2038bp}
    \pgfpathqlineto{36.9668bp}{87.2038bp}
    \pgfpathqlineto{37.9147bp}{87.2038bp}
    \pgfpathqlineto{38.8626bp}{87.2038bp}
    \pgfpathqlineto{39.8104bp}{87.2038bp}
    \pgfpathqlineto{40.7583bp}{87.2038bp}
    \pgfpathqlineto{41.7062bp}{87.2038bp}
    \pgfpathqlineto{42.6540bp}{87.2038bp}
    \pgfpathqlineto{43.6019bp}{87.2038bp}
    \pgfpathqlineto{44.5498bp}{87.2038bp}
    \pgfpathqlineto{45.4976bp}{87.2038bp}
    \pgfpathqlineto{46.4455bp}{87.2038bp}
    \pgfpathqlineto{47.3934bp}{87.2038bp}
    \pgfpathqlineto{48.3412bp}{87.2038bp}
    \pgfpathqlineto{49.2891bp}{87.2038bp}
    \pgfpathqlineto{50.2370bp}{87.2038bp}
    \pgfpathqlineto{51.1848bp}{87.2038bp}
    \pgfpathqlineto{52.1327bp}{87.2038bp}
    \pgfpathqlineto{53.0806bp}{87.2038bp}
    \pgfpathqlineto{54.0284bp}{87.2038bp}
    \pgfpathqlineto{54.9763bp}{87.2038bp}
    \pgfpathqlineto{55.9242bp}{87.2038bp}
    \pgfpathqlineto{56.8720bp}{87.2038bp}
    \pgfpathqlineto{57.8199bp}{87.2038bp}
    \pgfpathqlineto{58.7678bp}{87.2038bp}
    \pgfpathqlineto{59.7156bp}{87.2038bp}
    \pgfpathqlineto{60.6635bp}{87.2038bp}
    \pgfpathqlineto{61.6114bp}{87.2038bp}
    \pgfpathqlineto{62.5592bp}{87.2038bp}
    \pgfpathqlineto{63.5071bp}{87.2038bp}
    \pgfpathqlineto{64.4550bp}{87.2038bp}
    \pgfpathqlineto{65.4028bp}{87.2038bp}
    \pgfpathqlineto{66.3507bp}{87.2038bp}
    \pgfpathqlineto{67.2986bp}{87.2038bp}
    \pgfpathqlineto{68.2464bp}{87.2038bp}
    \pgfpathqlineto{69.1943bp}{87.2038bp}
    \pgfpathqlineto{70.1422bp}{87.2038bp}
    \pgfpathqlineto{71.0900bp}{87.2038bp}
    \pgfpathqlineto{72.0379bp}{87.2038bp}
    \pgfpathqlineto{72.9858bp}{87.2038bp}
    \pgfpathqlineto{73.9336bp}{87.2038bp}
    \pgfpathqlineto{74.8815bp}{87.2038bp}
    \pgfpathqlineto{75.8294bp}{87.2038bp}
    \pgfpathqlineto{76.7773bp}{87.2038bp}
    \pgfpathqlineto{77.7251bp}{87.2038bp}
    \pgfpathqlineto{78.6730bp}{87.2038bp}
    \pgfpathqlineto{79.6209bp}{87.2038bp}
    \pgfpathqlineto{80.5687bp}{87.2038bp}
    \pgfpathqlineto{81.5166bp}{87.2038bp}
    \pgfpathqlineto{82.4645bp}{87.2038bp}
    \pgfpathqlineto{83.4123bp}{87.2038bp}
    \pgfpathqlineto{84.3602bp}{87.2038bp}
    \pgfpathqlineto{85.3081bp}{87.2038bp}
    \pgfpathqlineto{86.2559bp}{87.2038bp}
    \pgfpathqlineto{87.2038bp}{87.2038bp}
    \pgfpathqlineto{88.1517bp}{87.2038bp}
    \pgfpathqlineto{89.0995bp}{87.2038bp}
    \pgfpathqlineto{90.0474bp}{87.2038bp}
    \pgfpathqlineto{90.9953bp}{87.2038bp}
    \pgfpathqlineto{91.9431bp}{87.2038bp}
    \pgfpathqlineto{92.8910bp}{87.2038bp}
    \pgfpathqlineto{93.8389bp}{87.2038bp}
    \pgfpathqlineto{94.7867bp}{87.2038bp}
    \pgfpathqlineto{95.7346bp}{87.2038bp}
    \pgfpathqlineto{96.6825bp}{87.2038bp}
    \pgfpathqlineto{97.6303bp}{87.2038bp}
    \pgfpathqlineto{98.5782bp}{87.2038bp}
    \pgfpathqlineto{99.5261bp}{87.2038bp}
    \pgfpathqlineto{100.4739bp}{87.2038bp}
    \pgfpathqlineto{101.4218bp}{87.2038bp}
    \pgfpathqlineto{102.3697bp}{87.2038bp}
    \pgfpathqlineto{103.3175bp}{87.2038bp}
    \pgfpathqlineto{104.2654bp}{87.2038bp}
    \pgfpathqlineto{105.2133bp}{87.2038bp}
    \pgfpathqlineto{106.1611bp}{87.2038bp}
    \pgfpathqlineto{107.1090bp}{87.2038bp}
    \pgfpathqlineto{108.0569bp}{87.2038bp}
    \pgfpathqlineto{109.0047bp}{87.2038bp}
    \pgfpathqlineto{109.9526bp}{87.2038bp}
    \pgfpathqlineto{110.9005bp}{87.2038bp}
    \pgfpathqlineto{111.8483bp}{87.2038bp}
    \pgfpathqlineto{112.7962bp}{87.2038bp}
    \pgfpathqlineto{113.7441bp}{87.2038bp}
    \pgfpathqlineto{114.6919bp}{87.2038bp}
    \pgfpathqlineto{115.6398bp}{87.2038bp}
    \pgfpathqlineto{116.5877bp}{87.2038bp}
    \pgfpathqlineto{117.5355bp}{87.2038bp}
    \pgfpathqlineto{118.4834bp}{87.2038bp}
    \pgfpathqlineto{119.4313bp}{87.2038bp}
    \pgfpathqlineto{120.3791bp}{87.2038bp}
    \pgfpathqlineto{121.3270bp}{87.2038bp}
    \pgfpathqlineto{122.2749bp}{87.2038bp}
    \pgfpathqlineto{123.2227bp}{87.2038bp}
    \pgfpathqlineto{124.1706bp}{87.2038bp}
    \pgfpathqlineto{125.1185bp}{87.2038bp}
    \pgfpathqlineto{126.0664bp}{87.2038bp}
    \pgfpathqlineto{127.0142bp}{87.2038bp}
    \pgfpathqlineto{127.9621bp}{87.2038bp}
    \pgfpathqlineto{128.9100bp}{87.2038bp}
    \pgfpathqlineto{129.8578bp}{87.2038bp}
    \pgfpathqlineto{130.8057bp}{87.2038bp}
    \pgfpathqlineto{131.7536bp}{87.2038bp}
    \pgfpathqlineto{132.7014bp}{87.2038bp}
    \pgfpathqlineto{133.6493bp}{87.2038bp}
    \pgfpathqlineto{134.5972bp}{87.2038bp}
    \pgfpathqlineto{135.5450bp}{87.2038bp}
    \pgfpathqlineto{136.4929bp}{87.2038bp}
    \pgfpathqlineto{137.4408bp}{87.2038bp}
    \pgfpathqlineto{138.3886bp}{87.2038bp}
    \pgfpathqlineto{139.3365bp}{87.2038bp}
    \pgfpathqlineto{140.2844bp}{87.2038bp}
    \pgfpathqlineto{141.2322bp}{87.2038bp}
    \pgfpathqlineto{142.1801bp}{87.2038bp}
    \pgfpathqlineto{143.1280bp}{87.2038bp}
    \pgfpathqlineto{144.0758bp}{87.2038bp}
    \pgfpathqlineto{145.0237bp}{87.2038bp}
    \pgfpathqlineto{145.9716bp}{87.2038bp}
    \pgfpathqlineto{146.9194bp}{87.2038bp}
    \pgfpathqlineto{147.8673bp}{87.2038bp}
    \pgfpathqlineto{148.8152bp}{87.2038bp}
    \pgfpathqlineto{149.7630bp}{87.2038bp}
    \pgfpathqlineto{150.7109bp}{87.2038bp}
    \pgfpathqlineto{151.6588bp}{87.2038bp}
    \pgfpathqlineto{152.6066bp}{87.2038bp}
    \pgfpathqlineto{153.5545bp}{87.2038bp}
    \pgfpathqlineto{154.5024bp}{87.2038bp}
    \pgfpathqlineto{155.4502bp}{87.2038bp}
    \pgfpathqlineto{156.3981bp}{87.2038bp}
    \pgfpathqlineto{157.3460bp}{87.2038bp}
    \pgfpathqlineto{158.2938bp}{87.2038bp}
    \pgfpathqlineto{159.2417bp}{87.2038bp}
    \pgfpathqlineto{160.1896bp}{87.2038bp}
    \pgfpathqlineto{161.1374bp}{87.2038bp}
    \pgfpathqlineto{162.0853bp}{87.2038bp}
    \pgfpathqlineto{163.0332bp}{87.2038bp}
    \pgfpathqlineto{163.9810bp}{87.2038bp}
    \pgfpathqlineto{164.9289bp}{87.2038bp}
    \pgfpathqlineto{165.8768bp}{87.2038bp}
    \pgfpathqlineto{166.8246bp}{87.2038bp}
    \pgfpathqlineto{167.7725bp}{87.2038bp}
    \pgfpathqlineto{168.7204bp}{87.2038bp}
    \pgfpathqlineto{169.6682bp}{87.2038bp}
    \pgfpathqlineto{170.6161bp}{87.2038bp}
    \pgfpathqlineto{171.5640bp}{87.2038bp}
    \pgfpathqlineto{172.5118bp}{87.2038bp}
    \pgfpathqlineto{173.4597bp}{87.2038bp}
    \pgfpathqlineto{174.4076bp}{87.2038bp}
    \pgfpathqlineto{175.3555bp}{87.2038bp}
    \pgfpathqlineto{176.3033bp}{87.2038bp}
    \pgfpathqlineto{177.2512bp}{87.2038bp}
    \pgfpathqlineto{178.1991bp}{87.2038bp}
    \pgfpathqlineto{179.1469bp}{87.2038bp}
    \pgfpathqlineto{180.0948bp}{87.2038bp}
    \pgfpathqlineto{181.0427bp}{87.2038bp}
    \pgfpathqlineto{181.9905bp}{87.2038bp}
    \pgfpathqlineto{182.9384bp}{87.2038bp}
    \pgfpathqlineto{183.8863bp}{87.2038bp}
    \pgfpathqlineto{184.8341bp}{87.2038bp}
    \pgfpathqlineto{185.7820bp}{87.2038bp}
    \pgfpathqlineto{186.7299bp}{87.2038bp}
    \pgfpathqlineto{187.6777bp}{60.1896bp}
    \pgfpathqlineto{188.6256bp}{60.1896bp}
    \pgfpathqlineto{189.5735bp}{60.1896bp}
    \pgfpathqlineto{190.5213bp}{60.1896bp}
    \pgfpathqlineto{191.4692bp}{60.1896bp}
    \pgfpathqlineto{192.4171bp}{60.1896bp}
    \pgfpathqlineto{193.3649bp}{60.1896bp}
    \pgfpathqlineto{194.3128bp}{60.1896bp}
    \pgfpathqlineto{195.2607bp}{60.1896bp}
    \pgfpathqlineto{196.2085bp}{59.7156bp}
    \pgfpathqlineto{197.1564bp}{59.7156bp}
    \pgfpathqlineto{198.1043bp}{59.7156bp}
    \pgfpathqlineto{199.0521bp}{59.7156bp}
    \pgfusepathqstroke
  \end{pgfscope}
  \begin{pgfscope}
    \pgfsetlinewidth{0.5670bp}
    \definecolor{sc}{rgb}{0.0000,0.0000,1.0000}
    \pgfsetstrokecolor{sc}
    \pgfsetmiterjoin
    \pgfsetbuttcap
    \pgfpathqmoveto{1.8957bp}{56.8720bp}
    \pgfpathqlineto{2.8436bp}{56.8720bp}
    \pgfpathqlineto{3.7915bp}{56.8720bp}
    \pgfpathqlineto{4.7393bp}{56.8720bp}
    \pgfpathqlineto{5.6872bp}{56.8720bp}
    \pgfpathqlineto{6.6351bp}{56.8720bp}
    \pgfpathqlineto{7.5829bp}{56.8720bp}
    \pgfpathqlineto{8.5308bp}{56.8720bp}
    \pgfpathqlineto{9.4787bp}{56.8720bp}
    \pgfpathqlineto{10.4265bp}{56.8720bp}
    \pgfpathqlineto{11.3744bp}{56.8720bp}
    \pgfpathqlineto{12.3223bp}{56.8720bp}
    \pgfpathqlineto{13.2701bp}{56.8720bp}
    \pgfpathqlineto{14.2180bp}{56.8720bp}
    \pgfpathqlineto{15.1659bp}{56.8720bp}
    \pgfpathqlineto{16.1137bp}{56.8720bp}
    \pgfpathqlineto{17.0616bp}{56.8720bp}
    \pgfpathqlineto{18.0095bp}{56.8720bp}
    \pgfpathqlineto{18.9573bp}{56.8720bp}
    \pgfpathqlineto{19.9052bp}{56.8720bp}
    \pgfpathqlineto{20.8531bp}{56.8720bp}
    \pgfpathqlineto{21.8009bp}{56.8720bp}
    \pgfpathqlineto{22.7488bp}{56.8720bp}
    \pgfpathqlineto{23.6967bp}{56.8720bp}
    \pgfpathqlineto{24.6445bp}{56.8720bp}
    \pgfpathqlineto{25.5924bp}{56.8720bp}
    \pgfpathqlineto{26.5403bp}{56.8720bp}
    \pgfpathqlineto{27.4882bp}{56.8720bp}
    \pgfpathqlineto{28.4360bp}{56.8720bp}
    \pgfpathqlineto{29.3839bp}{56.8720bp}
    \pgfpathqlineto{30.3318bp}{56.8720bp}
    \pgfpathqlineto{31.2796bp}{56.8720bp}
    \pgfpathqlineto{32.2275bp}{56.8720bp}
    \pgfpathqlineto{33.1754bp}{56.8720bp}
    \pgfpathqlineto{34.1232bp}{56.8720bp}
    \pgfpathqlineto{35.0711bp}{56.8720bp}
    \pgfpathqlineto{36.0190bp}{56.8720bp}
    \pgfpathqlineto{36.9668bp}{56.8720bp}
    \pgfpathqlineto{37.9147bp}{56.8720bp}
    \pgfpathqlineto{38.8626bp}{56.8720bp}
    \pgfpathqlineto{39.8104bp}{56.8720bp}
    \pgfpathqlineto{40.7583bp}{56.8720bp}
    \pgfpathqlineto{41.7062bp}{56.8720bp}
    \pgfpathqlineto{42.6540bp}{56.8720bp}
    \pgfpathqlineto{43.6019bp}{56.8720bp}
    \pgfpathqlineto{44.5498bp}{56.8720bp}
    \pgfpathqlineto{45.4976bp}{56.8720bp}
    \pgfpathqlineto{46.4455bp}{56.8720bp}
    \pgfpathqlineto{47.3934bp}{56.8720bp}
    \pgfpathqlineto{48.3412bp}{56.8720bp}
    \pgfpathqlineto{49.2891bp}{56.8720bp}
    \pgfpathqlineto{50.2370bp}{56.8720bp}
    \pgfpathqlineto{51.1848bp}{56.8720bp}
    \pgfpathqlineto{52.1327bp}{56.8720bp}
    \pgfpathqlineto{53.0806bp}{56.8720bp}
    \pgfpathqlineto{54.0284bp}{56.8720bp}
    \pgfpathqlineto{54.9763bp}{56.8720bp}
    \pgfpathqlineto{55.9242bp}{56.8720bp}
    \pgfpathqlineto{56.8720bp}{56.8720bp}
    \pgfpathqlineto{57.8199bp}{56.8720bp}
    \pgfpathqlineto{58.7678bp}{56.8720bp}
    \pgfpathqlineto{59.7156bp}{56.8720bp}
    \pgfpathqlineto{60.6635bp}{56.8720bp}
    \pgfpathqlineto{61.6114bp}{56.8720bp}
    \pgfpathqlineto{62.5592bp}{56.8720bp}
    \pgfpathqlineto{63.5071bp}{56.8720bp}
    \pgfpathqlineto{64.4550bp}{56.8720bp}
    \pgfpathqlineto{65.4028bp}{56.8720bp}
    \pgfpathqlineto{66.3507bp}{56.8720bp}
    \pgfpathqlineto{67.2986bp}{56.8720bp}
    \pgfpathqlineto{68.2464bp}{56.8720bp}
    \pgfpathqlineto{69.1943bp}{56.8720bp}
    \pgfpathqlineto{70.1422bp}{56.8720bp}
    \pgfpathqlineto{71.0900bp}{56.8720bp}
    \pgfpathqlineto{72.0379bp}{56.8720bp}
    \pgfpathqlineto{72.9858bp}{56.8720bp}
    \pgfpathqlineto{73.9336bp}{56.8720bp}
    \pgfpathqlineto{74.8815bp}{56.8720bp}
    \pgfpathqlineto{75.8294bp}{56.8720bp}
    \pgfpathqlineto{76.7773bp}{56.8720bp}
    \pgfpathqlineto{77.7251bp}{56.8720bp}
    \pgfpathqlineto{78.6730bp}{56.8720bp}
    \pgfpathqlineto{79.6209bp}{56.8720bp}
    \pgfpathqlineto{80.5687bp}{56.8720bp}
    \pgfpathqlineto{81.5166bp}{56.8720bp}
    \pgfpathqlineto{82.4645bp}{56.8720bp}
    \pgfpathqlineto{83.4123bp}{56.8720bp}
    \pgfpathqlineto{84.3602bp}{56.8720bp}
    \pgfpathqlineto{85.3081bp}{56.8720bp}
    \pgfpathqlineto{86.2559bp}{56.8720bp}
    \pgfpathqlineto{87.2038bp}{56.8720bp}
    \pgfpathqlineto{88.1517bp}{56.8720bp}
    \pgfpathqlineto{89.0995bp}{56.8720bp}
    \pgfpathqlineto{90.0474bp}{56.8720bp}
    \pgfpathqlineto{90.9953bp}{56.8720bp}
    \pgfpathqlineto{91.9431bp}{56.8720bp}
    \pgfpathqlineto{92.8910bp}{56.8720bp}
    \pgfpathqlineto{93.8389bp}{56.8720bp}
    \pgfpathqlineto{94.7867bp}{56.8720bp}
    \pgfpathqlineto{95.7346bp}{56.8720bp}
    \pgfpathqlineto{96.6825bp}{56.8720bp}
    \pgfpathqlineto{97.6303bp}{56.8720bp}
    \pgfpathqlineto{98.5782bp}{56.8720bp}
    \pgfpathqlineto{99.5261bp}{56.8720bp}
    \pgfpathqlineto{100.4739bp}{56.8720bp}
    \pgfpathqlineto{101.4218bp}{56.8720bp}
    \pgfpathqlineto{102.3697bp}{56.8720bp}
    \pgfpathqlineto{103.3175bp}{56.8720bp}
    \pgfpathqlineto{104.2654bp}{56.8720bp}
    \pgfpathqlineto{105.2133bp}{56.8720bp}
    \pgfpathqlineto{106.1611bp}{56.8720bp}
    \pgfpathqlineto{107.1090bp}{56.8720bp}
    \pgfpathqlineto{108.0569bp}{56.8720bp}
    \pgfpathqlineto{109.0047bp}{56.8720bp}
    \pgfpathqlineto{109.9526bp}{56.8720bp}
    \pgfpathqlineto{110.9005bp}{56.8720bp}
    \pgfpathqlineto{111.8483bp}{56.8720bp}
    \pgfpathqlineto{112.7962bp}{56.8720bp}
    \pgfpathqlineto{113.7441bp}{56.8720bp}
    \pgfpathqlineto{114.6919bp}{56.8720bp}
    \pgfpathqlineto{115.6398bp}{56.8720bp}
    \pgfpathqlineto{116.5877bp}{56.8720bp}
    \pgfpathqlineto{117.5355bp}{56.8720bp}
    \pgfpathqlineto{118.4834bp}{56.8720bp}
    \pgfpathqlineto{119.4313bp}{56.8720bp}
    \pgfpathqlineto{120.3791bp}{56.8720bp}
    \pgfpathqlineto{121.3270bp}{56.8720bp}
    \pgfpathqlineto{122.2749bp}{56.8720bp}
    \pgfpathqlineto{123.2227bp}{56.8720bp}
    \pgfpathqlineto{124.1706bp}{56.8720bp}
    \pgfpathqlineto{125.1185bp}{56.8720bp}
    \pgfpathqlineto{126.0664bp}{56.8720bp}
    \pgfpathqlineto{127.0142bp}{56.8720bp}
    \pgfpathqlineto{127.9621bp}{56.8720bp}
    \pgfpathqlineto{128.9100bp}{56.8720bp}
    \pgfpathqlineto{129.8578bp}{56.8720bp}
    \pgfpathqlineto{130.8057bp}{56.8720bp}
    \pgfpathqlineto{131.7536bp}{56.8720bp}
    \pgfpathqlineto{132.7014bp}{56.8720bp}
    \pgfpathqlineto{133.6493bp}{56.8720bp}
    \pgfpathqlineto{134.5972bp}{56.8720bp}
    \pgfpathqlineto{135.5450bp}{56.8720bp}
    \pgfpathqlineto{136.4929bp}{56.8720bp}
    \pgfpathqlineto{137.4408bp}{56.8720bp}
    \pgfpathqlineto{138.3886bp}{56.8720bp}
    \pgfpathqlineto{139.3365bp}{56.8720bp}
    \pgfpathqlineto{140.2844bp}{56.8720bp}
    \pgfpathqlineto{141.2322bp}{56.8720bp}
    \pgfpathqlineto{142.1801bp}{56.8720bp}
    \pgfpathqlineto{143.1280bp}{56.8720bp}
    \pgfpathqlineto{144.0758bp}{56.8720bp}
    \pgfpathqlineto{145.0237bp}{56.8720bp}
    \pgfpathqlineto{145.9716bp}{56.8720bp}
    \pgfpathqlineto{146.9194bp}{56.8720bp}
    \pgfpathqlineto{147.8673bp}{56.8720bp}
    \pgfpathqlineto{148.8152bp}{56.8720bp}
    \pgfpathqlineto{149.7630bp}{56.8720bp}
    \pgfpathqlineto{150.7109bp}{56.8720bp}
    \pgfpathqlineto{151.6588bp}{56.8720bp}
    \pgfpathqlineto{152.6066bp}{56.8720bp}
    \pgfpathqlineto{153.5545bp}{56.8720bp}
    \pgfpathqlineto{154.5024bp}{56.8720bp}
    \pgfpathqlineto{155.4502bp}{56.8720bp}
    \pgfpathqlineto{156.3981bp}{56.8720bp}
    \pgfpathqlineto{157.3460bp}{56.8720bp}
    \pgfpathqlineto{158.2938bp}{56.8720bp}
    \pgfpathqlineto{159.2417bp}{56.8720bp}
    \pgfpathqlineto{160.1896bp}{56.8720bp}
    \pgfpathqlineto{161.1374bp}{56.8720bp}
    \pgfpathqlineto{162.0853bp}{56.8720bp}
    \pgfpathqlineto{163.0332bp}{56.8720bp}
    \pgfpathqlineto{163.9810bp}{56.8720bp}
    \pgfpathqlineto{164.9289bp}{56.8720bp}
    \pgfpathqlineto{165.8768bp}{56.8720bp}
    \pgfpathqlineto{166.8246bp}{56.8720bp}
    \pgfpathqlineto{167.7725bp}{56.8720bp}
    \pgfpathqlineto{168.7204bp}{56.8720bp}
    \pgfpathqlineto{169.6682bp}{56.8720bp}
    \pgfpathqlineto{170.6161bp}{56.8720bp}
    \pgfpathqlineto{171.5640bp}{56.8720bp}
    \pgfpathqlineto{172.5118bp}{56.8720bp}
    \pgfpathqlineto{173.4597bp}{56.8720bp}
    \pgfpathqlineto{174.4076bp}{56.8720bp}
    \pgfpathqlineto{175.3555bp}{56.8720bp}
    \pgfpathqlineto{176.3033bp}{56.8720bp}
    \pgfpathqlineto{177.2512bp}{56.8720bp}
    \pgfpathqlineto{178.1991bp}{56.8720bp}
    \pgfpathqlineto{179.1469bp}{56.8720bp}
    \pgfpathqlineto{180.0948bp}{56.8720bp}
    \pgfpathqlineto{181.0427bp}{56.8720bp}
    \pgfpathqlineto{181.9905bp}{56.8720bp}
    \pgfpathqlineto{182.9384bp}{56.8720bp}
    \pgfpathqlineto{183.8863bp}{56.8720bp}
    \pgfpathqlineto{184.8341bp}{56.8720bp}
    \pgfpathqlineto{185.7820bp}{56.8720bp}
    \pgfpathqlineto{186.7299bp}{56.8720bp}
    \pgfpathqlineto{187.6777bp}{56.8720bp}
    \pgfpathqlineto{188.6256bp}{56.8720bp}
    \pgfpathqlineto{189.5735bp}{56.8720bp}
    \pgfpathqlineto{190.5213bp}{56.8720bp}
    \pgfpathqlineto{191.4692bp}{56.8720bp}
    \pgfpathqlineto{192.4171bp}{56.8720bp}
    \pgfpathqlineto{193.3649bp}{56.8720bp}
    \pgfpathqlineto{194.3128bp}{56.8720bp}
    \pgfpathqlineto{195.2607bp}{56.8720bp}
    \pgfpathqlineto{196.2085bp}{57.3460bp}
    \pgfpathqlineto{197.1564bp}{57.3460bp}
    \pgfpathqlineto{198.1043bp}{57.3460bp}
    \pgfpathqlineto{199.0521bp}{57.3460bp}
    \pgfusepathqstroke
  \end{pgfscope}
  \begin{pgfscope}
    \pgfsetlinewidth{0.5670bp}
    \definecolor{sc}{rgb}{1.0000,0.0000,0.0000}
    \pgfsetstrokecolor{sc}
    \pgfsetmiterjoin
    \pgfsetbuttcap
    \pgfpathqmoveto{200.0000bp}{56.8720bp}
    \pgfpathqlineto{200.0000bp}{56.3981bp}
    \pgfusepathqstroke
  \end{pgfscope}
  \begin{pgfscope}
    \pgfsetlinewidth{0.5670bp}
    \definecolor{sc}{rgb}{1.0000,0.0000,0.0000}
    \pgfsetstrokecolor{sc}
    \pgfsetmiterjoin
    \pgfsetbuttcap
    \pgfpathqmoveto{188.6256bp}{56.8720bp}
    \pgfpathqlineto{188.6256bp}{56.3981bp}
    \pgfusepathqstroke
  \end{pgfscope}
  \begin{pgfscope}
    \pgfsetlinewidth{0.5670bp}
    \definecolor{sc}{rgb}{1.0000,0.0000,0.0000}
    \pgfsetstrokecolor{sc}
    \pgfsetmiterjoin
    \pgfsetbuttcap
    \pgfpathqmoveto{119.4313bp}{56.8720bp}
    \pgfpathqlineto{119.4313bp}{56.3981bp}
    \pgfusepathqstroke
  \end{pgfscope}
  \begin{pgfscope}
    \pgfsetlinewidth{0.5670bp}
    \definecolor{sc}{rgb}{1.0000,0.0000,0.0000}
    \pgfsetstrokecolor{sc}
    \pgfsetmiterjoin
    \pgfsetbuttcap
    \pgfpathqmoveto{1.8957bp}{56.8720bp}
    \pgfpathqlineto{1.8957bp}{56.3981bp}
    \pgfusepathqstroke
  \end{pgfscope}
  \begin{pgfscope}
    \pgfsetlinewidth{0.5670bp}
    \definecolor{sc}{rgb}{0.0000,0.0000,0.0000}
    \pgfsetstrokecolor{sc}
    \pgfsetmiterjoin
    \pgfsetbuttcap
    \pgfpathqmoveto{196.2085bp}{56.8720bp}
    \pgfpathqlineto{196.2085bp}{56.3981bp}
    \pgfusepathqstroke
  \end{pgfscope}
  \begin{pgfscope}
    \pgfsetlinewidth{0.5670bp}
    \definecolor{sc}{rgb}{0.0000,0.0000,0.0000}
    \pgfsetstrokecolor{sc}
    \pgfsetmiterjoin
    \pgfsetbuttcap
    \pgfpathqmoveto{191.4692bp}{56.8720bp}
    \pgfpathqlineto{191.4692bp}{56.3981bp}
    \pgfusepathqstroke
  \end{pgfscope}
  \begin{pgfscope}
    \pgfsetlinewidth{0.5670bp}
    \definecolor{sc}{rgb}{0.0000,0.0000,0.0000}
    \pgfsetstrokecolor{sc}
    \pgfsetmiterjoin
    \pgfsetbuttcap
    \pgfpathqmoveto{186.7299bp}{56.8720bp}
    \pgfpathqlineto{186.7299bp}{56.3981bp}
    \pgfusepathqstroke
  \end{pgfscope}
  \begin{pgfscope}
    \pgfsetlinewidth{0.5670bp}
    \definecolor{sc}{rgb}{0.0000,0.0000,0.0000}
    \pgfsetstrokecolor{sc}
    \pgfsetmiterjoin
    \pgfsetbuttcap
    \pgfpathqmoveto{181.9905bp}{56.8720bp}
    \pgfpathqlineto{181.9905bp}{56.3981bp}
    \pgfusepathqstroke
  \end{pgfscope}
  \begin{pgfscope}
    \pgfsetlinewidth{0.5670bp}
    \definecolor{sc}{rgb}{0.0000,0.0000,0.0000}
    \pgfsetstrokecolor{sc}
    \pgfsetmiterjoin
    \pgfsetbuttcap
    \pgfpathqmoveto{177.2512bp}{56.8720bp}
    \pgfpathqlineto{177.2512bp}{56.3981bp}
    \pgfusepathqstroke
  \end{pgfscope}
  \begin{pgfscope}
    \pgfsetlinewidth{0.5670bp}
    \definecolor{sc}{rgb}{0.0000,0.0000,0.0000}
    \pgfsetstrokecolor{sc}
    \pgfsetmiterjoin
    \pgfsetbuttcap
    \pgfpathqmoveto{172.5118bp}{56.8720bp}
    \pgfpathqlineto{172.5118bp}{56.3981bp}
    \pgfusepathqstroke
  \end{pgfscope}
  \begin{pgfscope}
    \pgfsetlinewidth{0.5670bp}
    \definecolor{sc}{rgb}{0.0000,0.0000,0.0000}
    \pgfsetstrokecolor{sc}
    \pgfsetmiterjoin
    \pgfsetbuttcap
    \pgfpathqmoveto{167.7725bp}{56.8720bp}
    \pgfpathqlineto{167.7725bp}{56.3981bp}
    \pgfusepathqstroke
  \end{pgfscope}
  \begin{pgfscope}
    \pgfsetlinewidth{0.5670bp}
    \definecolor{sc}{rgb}{0.0000,0.0000,0.0000}
    \pgfsetstrokecolor{sc}
    \pgfsetmiterjoin
    \pgfsetbuttcap
    \pgfpathqmoveto{163.0332bp}{56.8720bp}
    \pgfpathqlineto{163.0332bp}{56.3981bp}
    \pgfusepathqstroke
  \end{pgfscope}
  \begin{pgfscope}
    \pgfsetlinewidth{0.5670bp}
    \definecolor{sc}{rgb}{0.0000,0.0000,0.0000}
    \pgfsetstrokecolor{sc}
    \pgfsetmiterjoin
    \pgfsetbuttcap
    \pgfpathqmoveto{158.2938bp}{56.8720bp}
    \pgfpathqlineto{158.2938bp}{56.3981bp}
    \pgfusepathqstroke
  \end{pgfscope}
  \begin{pgfscope}
    \pgfsetlinewidth{0.5670bp}
    \definecolor{sc}{rgb}{0.0000,0.0000,0.0000}
    \pgfsetstrokecolor{sc}
    \pgfsetmiterjoin
    \pgfsetbuttcap
    \pgfpathqmoveto{153.5545bp}{56.8720bp}
    \pgfpathqlineto{153.5545bp}{56.3981bp}
    \pgfusepathqstroke
  \end{pgfscope}
  \begin{pgfscope}
    \pgfsetlinewidth{0.5670bp}
    \definecolor{sc}{rgb}{0.0000,0.0000,0.0000}
    \pgfsetstrokecolor{sc}
    \pgfsetmiterjoin
    \pgfsetbuttcap
    \pgfpathqmoveto{148.8152bp}{56.8720bp}
    \pgfpathqlineto{148.8152bp}{56.3981bp}
    \pgfusepathqstroke
  \end{pgfscope}
  \begin{pgfscope}
    \pgfsetlinewidth{0.5670bp}
    \definecolor{sc}{rgb}{0.0000,0.0000,0.0000}
    \pgfsetstrokecolor{sc}
    \pgfsetmiterjoin
    \pgfsetbuttcap
    \pgfpathqmoveto{144.0758bp}{56.8720bp}
    \pgfpathqlineto{144.0758bp}{56.3981bp}
    \pgfusepathqstroke
  \end{pgfscope}
  \begin{pgfscope}
    \pgfsetlinewidth{0.5670bp}
    \definecolor{sc}{rgb}{0.0000,0.0000,0.0000}
    \pgfsetstrokecolor{sc}
    \pgfsetmiterjoin
    \pgfsetbuttcap
    \pgfpathqmoveto{139.3365bp}{56.8720bp}
    \pgfpathqlineto{139.3365bp}{56.3981bp}
    \pgfusepathqstroke
  \end{pgfscope}
  \begin{pgfscope}
    \pgfsetlinewidth{0.5670bp}
    \definecolor{sc}{rgb}{0.0000,0.0000,0.0000}
    \pgfsetstrokecolor{sc}
    \pgfsetmiterjoin
    \pgfsetbuttcap
    \pgfpathqmoveto{134.5972bp}{56.8720bp}
    \pgfpathqlineto{134.5972bp}{56.3981bp}
    \pgfusepathqstroke
  \end{pgfscope}
  \begin{pgfscope}
    \pgfsetlinewidth{0.5670bp}
    \definecolor{sc}{rgb}{0.0000,0.0000,0.0000}
    \pgfsetstrokecolor{sc}
    \pgfsetmiterjoin
    \pgfsetbuttcap
    \pgfpathqmoveto{129.8578bp}{56.8720bp}
    \pgfpathqlineto{129.8578bp}{56.3981bp}
    \pgfusepathqstroke
  \end{pgfscope}
  \begin{pgfscope}
    \pgfsetlinewidth{0.5670bp}
    \definecolor{sc}{rgb}{0.0000,0.0000,0.0000}
    \pgfsetstrokecolor{sc}
    \pgfsetmiterjoin
    \pgfsetbuttcap
    \pgfpathqmoveto{125.1185bp}{56.8720bp}
    \pgfpathqlineto{125.1185bp}{56.3981bp}
    \pgfusepathqstroke
  \end{pgfscope}
  \begin{pgfscope}
    \pgfsetlinewidth{0.5670bp}
    \definecolor{sc}{rgb}{0.0000,0.0000,0.0000}
    \pgfsetstrokecolor{sc}
    \pgfsetmiterjoin
    \pgfsetbuttcap
    \pgfpathqmoveto{120.3791bp}{56.8720bp}
    \pgfpathqlineto{120.3791bp}{56.3981bp}
    \pgfusepathqstroke
  \end{pgfscope}
  \begin{pgfscope}
    \pgfsetlinewidth{0.5670bp}
    \definecolor{sc}{rgb}{0.0000,0.0000,0.0000}
    \pgfsetstrokecolor{sc}
    \pgfsetmiterjoin
    \pgfsetbuttcap
    \pgfpathqmoveto{115.6398bp}{56.8720bp}
    \pgfpathqlineto{115.6398bp}{56.3981bp}
    \pgfusepathqstroke
  \end{pgfscope}
  \begin{pgfscope}
    \pgfsetlinewidth{0.5670bp}
    \definecolor{sc}{rgb}{0.0000,0.0000,0.0000}
    \pgfsetstrokecolor{sc}
    \pgfsetmiterjoin
    \pgfsetbuttcap
    \pgfpathqmoveto{110.9005bp}{56.8720bp}
    \pgfpathqlineto{110.9005bp}{56.3981bp}
    \pgfusepathqstroke
  \end{pgfscope}
  \begin{pgfscope}
    \pgfsetlinewidth{0.5670bp}
    \definecolor{sc}{rgb}{0.0000,0.0000,0.0000}
    \pgfsetstrokecolor{sc}
    \pgfsetmiterjoin
    \pgfsetbuttcap
    \pgfpathqmoveto{106.1611bp}{56.8720bp}
    \pgfpathqlineto{106.1611bp}{56.3981bp}
    \pgfusepathqstroke
  \end{pgfscope}
  \begin{pgfscope}
    \pgfsetlinewidth{0.5670bp}
    \definecolor{sc}{rgb}{0.0000,0.0000,0.0000}
    \pgfsetstrokecolor{sc}
    \pgfsetmiterjoin
    \pgfsetbuttcap
    \pgfpathqmoveto{101.4218bp}{56.8720bp}
    \pgfpathqlineto{101.4218bp}{56.3981bp}
    \pgfusepathqstroke
  \end{pgfscope}
  \begin{pgfscope}
    \pgfsetlinewidth{0.5670bp}
    \definecolor{sc}{rgb}{0.0000,0.0000,0.0000}
    \pgfsetstrokecolor{sc}
    \pgfsetmiterjoin
    \pgfsetbuttcap
    \pgfpathqmoveto{96.6825bp}{56.8720bp}
    \pgfpathqlineto{96.6825bp}{56.3981bp}
    \pgfusepathqstroke
  \end{pgfscope}
  \begin{pgfscope}
    \pgfsetlinewidth{0.5670bp}
    \definecolor{sc}{rgb}{0.0000,0.0000,0.0000}
    \pgfsetstrokecolor{sc}
    \pgfsetmiterjoin
    \pgfsetbuttcap
    \pgfpathqmoveto{91.9431bp}{56.8720bp}
    \pgfpathqlineto{91.9431bp}{56.3981bp}
    \pgfusepathqstroke
  \end{pgfscope}
  \begin{pgfscope}
    \pgfsetlinewidth{0.5670bp}
    \definecolor{sc}{rgb}{0.0000,0.0000,0.0000}
    \pgfsetstrokecolor{sc}
    \pgfsetmiterjoin
    \pgfsetbuttcap
    \pgfpathqmoveto{87.2038bp}{56.8720bp}
    \pgfpathqlineto{87.2038bp}{56.3981bp}
    \pgfusepathqstroke
  \end{pgfscope}
  \begin{pgfscope}
    \pgfsetlinewidth{0.5670bp}
    \definecolor{sc}{rgb}{0.0000,0.0000,0.0000}
    \pgfsetstrokecolor{sc}
    \pgfsetmiterjoin
    \pgfsetbuttcap
    \pgfpathqmoveto{82.4645bp}{56.8720bp}
    \pgfpathqlineto{82.4645bp}{56.3981bp}
    \pgfusepathqstroke
  \end{pgfscope}
  \begin{pgfscope}
    \pgfsetlinewidth{0.5670bp}
    \definecolor{sc}{rgb}{0.0000,0.0000,0.0000}
    \pgfsetstrokecolor{sc}
    \pgfsetmiterjoin
    \pgfsetbuttcap
    \pgfpathqmoveto{77.7251bp}{56.8720bp}
    \pgfpathqlineto{77.7251bp}{56.3981bp}
    \pgfusepathqstroke
  \end{pgfscope}
  \begin{pgfscope}
    \pgfsetlinewidth{0.5670bp}
    \definecolor{sc}{rgb}{0.0000,0.0000,0.0000}
    \pgfsetstrokecolor{sc}
    \pgfsetmiterjoin
    \pgfsetbuttcap
    \pgfpathqmoveto{72.9858bp}{56.8720bp}
    \pgfpathqlineto{72.9858bp}{56.3981bp}
    \pgfusepathqstroke
  \end{pgfscope}
  \begin{pgfscope}
    \pgfsetlinewidth{0.5670bp}
    \definecolor{sc}{rgb}{0.0000,0.0000,0.0000}
    \pgfsetstrokecolor{sc}
    \pgfsetmiterjoin
    \pgfsetbuttcap
    \pgfpathqmoveto{68.2464bp}{56.8720bp}
    \pgfpathqlineto{68.2464bp}{56.3981bp}
    \pgfusepathqstroke
  \end{pgfscope}
  \begin{pgfscope}
    \pgfsetlinewidth{0.5670bp}
    \definecolor{sc}{rgb}{0.0000,0.0000,0.0000}
    \pgfsetstrokecolor{sc}
    \pgfsetmiterjoin
    \pgfsetbuttcap
    \pgfpathqmoveto{63.5071bp}{56.8720bp}
    \pgfpathqlineto{63.5071bp}{56.3981bp}
    \pgfusepathqstroke
  \end{pgfscope}
  \begin{pgfscope}
    \pgfsetlinewidth{0.5670bp}
    \definecolor{sc}{rgb}{0.0000,0.0000,0.0000}
    \pgfsetstrokecolor{sc}
    \pgfsetmiterjoin
    \pgfsetbuttcap
    \pgfpathqmoveto{58.7678bp}{56.8720bp}
    \pgfpathqlineto{58.7678bp}{56.3981bp}
    \pgfusepathqstroke
  \end{pgfscope}
  \begin{pgfscope}
    \pgfsetlinewidth{0.5670bp}
    \definecolor{sc}{rgb}{0.0000,0.0000,0.0000}
    \pgfsetstrokecolor{sc}
    \pgfsetmiterjoin
    \pgfsetbuttcap
    \pgfpathqmoveto{54.0284bp}{56.8720bp}
    \pgfpathqlineto{54.0284bp}{56.3981bp}
    \pgfusepathqstroke
  \end{pgfscope}
  \begin{pgfscope}
    \pgfsetlinewidth{0.5670bp}
    \definecolor{sc}{rgb}{0.0000,0.0000,0.0000}
    \pgfsetstrokecolor{sc}
    \pgfsetmiterjoin
    \pgfsetbuttcap
    \pgfpathqmoveto{49.2891bp}{56.8720bp}
    \pgfpathqlineto{49.2891bp}{56.3981bp}
    \pgfusepathqstroke
  \end{pgfscope}
  \begin{pgfscope}
    \pgfsetlinewidth{0.5670bp}
    \definecolor{sc}{rgb}{0.0000,0.0000,0.0000}
    \pgfsetstrokecolor{sc}
    \pgfsetmiterjoin
    \pgfsetbuttcap
    \pgfpathqmoveto{44.5498bp}{56.8720bp}
    \pgfpathqlineto{44.5498bp}{56.3981bp}
    \pgfusepathqstroke
  \end{pgfscope}
  \begin{pgfscope}
    \pgfsetlinewidth{0.5670bp}
    \definecolor{sc}{rgb}{0.0000,0.0000,0.0000}
    \pgfsetstrokecolor{sc}
    \pgfsetmiterjoin
    \pgfsetbuttcap
    \pgfpathqmoveto{39.8104bp}{56.8720bp}
    \pgfpathqlineto{39.8104bp}{56.3981bp}
    \pgfusepathqstroke
  \end{pgfscope}
  \begin{pgfscope}
    \pgfsetlinewidth{0.5670bp}
    \definecolor{sc}{rgb}{0.0000,0.0000,0.0000}
    \pgfsetstrokecolor{sc}
    \pgfsetmiterjoin
    \pgfsetbuttcap
    \pgfpathqmoveto{35.0711bp}{56.8720bp}
    \pgfpathqlineto{35.0711bp}{56.3981bp}
    \pgfusepathqstroke
  \end{pgfscope}
  \begin{pgfscope}
    \pgfsetlinewidth{0.5670bp}
    \definecolor{sc}{rgb}{0.0000,0.0000,0.0000}
    \pgfsetstrokecolor{sc}
    \pgfsetmiterjoin
    \pgfsetbuttcap
    \pgfpathqmoveto{30.3318bp}{56.8720bp}
    \pgfpathqlineto{30.3318bp}{56.3981bp}
    \pgfusepathqstroke
  \end{pgfscope}
  \begin{pgfscope}
    \pgfsetlinewidth{0.5670bp}
    \definecolor{sc}{rgb}{0.0000,0.0000,0.0000}
    \pgfsetstrokecolor{sc}
    \pgfsetmiterjoin
    \pgfsetbuttcap
    \pgfpathqmoveto{25.5924bp}{56.8720bp}
    \pgfpathqlineto{25.5924bp}{56.3981bp}
    \pgfusepathqstroke
  \end{pgfscope}
  \begin{pgfscope}
    \pgfsetlinewidth{0.5670bp}
    \definecolor{sc}{rgb}{0.0000,0.0000,0.0000}
    \pgfsetstrokecolor{sc}
    \pgfsetmiterjoin
    \pgfsetbuttcap
    \pgfpathqmoveto{20.8531bp}{56.8720bp}
    \pgfpathqlineto{20.8531bp}{56.3981bp}
    \pgfusepathqstroke
  \end{pgfscope}
  \begin{pgfscope}
    \pgfsetlinewidth{0.5670bp}
    \definecolor{sc}{rgb}{0.0000,0.0000,0.0000}
    \pgfsetstrokecolor{sc}
    \pgfsetmiterjoin
    \pgfsetbuttcap
    \pgfpathqmoveto{16.1137bp}{56.8720bp}
    \pgfpathqlineto{16.1137bp}{56.3981bp}
    \pgfusepathqstroke
  \end{pgfscope}
  \begin{pgfscope}
    \pgfsetlinewidth{0.5670bp}
    \definecolor{sc}{rgb}{0.0000,0.0000,0.0000}
    \pgfsetstrokecolor{sc}
    \pgfsetmiterjoin
    \pgfsetbuttcap
    \pgfpathqmoveto{11.3744bp}{56.8720bp}
    \pgfpathqlineto{11.3744bp}{56.3981bp}
    \pgfusepathqstroke
  \end{pgfscope}
  \begin{pgfscope}
    \pgfsetlinewidth{0.5670bp}
    \definecolor{sc}{rgb}{0.0000,0.0000,0.0000}
    \pgfsetstrokecolor{sc}
    \pgfsetmiterjoin
    \pgfsetbuttcap
    \pgfpathqmoveto{6.6351bp}{56.8720bp}
    \pgfpathqlineto{6.6351bp}{56.3981bp}
    \pgfusepathqstroke
  \end{pgfscope}
  \begin{pgfscope}
    \definecolor{fc}{rgb}{0.0000,0.0000,0.0000}
    \pgfsetfillcolor{fc}
    \pgftransformshift{\pgfqpoint{-0.0000bp}{86.9194bp}}
    \pgftransformscale{0.1185}
    \pgftext[base,left]{$\mathbb{F}_A$}
  \end{pgfscope}
  \begin{pgfscope}
    \pgfsetlinewidth{0.5670bp}
    \definecolor{sc}{rgb}{0.0000,0.0000,0.0000}
    \pgfsetstrokecolor{sc}
    \pgfsetmiterjoin
    \pgfsetbuttcap
    \pgfpathqmoveto{1.8957bp}{87.2038bp}
    \pgfpathqlineto{1.7062bp}{87.2038bp}
    \pgfusepathqstroke
  \end{pgfscope}
  \begin{pgfscope}
    \pgfsetlinewidth{0.5670bp}
    \definecolor{sc}{rgb}{0.0000,0.0000,0.0000}
    \pgfsetstrokecolor{sc}
    \pgfsetmiterjoin
    \pgfsetbuttcap
    \pgfpathqmoveto{1.8957bp}{56.8720bp}
    \pgfpathqlineto{1.8957bp}{150.2370bp}
    \pgfusepathqstroke
  \end{pgfscope}
  \begin{pgfscope}
    \pgfsetlinewidth{0.5670bp}
    \definecolor{sc}{rgb}{0.0000,0.0000,0.0000}
    \pgfsetstrokecolor{sc}
    \pgfsetmiterjoin
    \pgfsetbuttcap
    \pgfpathqmoveto{1.8957bp}{56.8720bp}
    \pgfpathqlineto{200.0000bp}{56.8720bp}
    \pgfusepathqstroke
  \end{pgfscope}
\end{pgfpicture}

        \label{fig:ex:ca:hgma:ex:move-h}
    \caption{push-v preconditions}\label{fig:ex:ca:hgma:ex:disconnected}
\end{figure}

\begin{figure}
    \centering
    \begin{pgfpicture}
  \pgfpathrectangle{\pgfpointorigin}{\pgfqpoint{200.0000bp}{200.0000bp}}
  \pgfusepath{use as bounding box}
  \begin{pgfscope}
    \definecolor{fc}{rgb}{0.0000,0.0000,0.0000}
    \pgfsetfillcolor{fc}
    \pgftransformshift{\pgfqpoint{5.1515bp}{49.7727bp}}
    \pgftransformscale{0.1894}
    \pgftext[base,left]{candidates}
  \end{pgfscope}
  \begin{pgfscope}
    \definecolor{fc}{rgb}{0.0000,0.0000,0.0000}
    \pgfsetfillcolor{fc}
    \pgfsetlinewidth{0.5678bp}
    \definecolor{sc}{rgb}{0.0000,0.0000,0.0000}
    \pgfsetstrokecolor{sc}
    \pgfsetmiterjoin
    \pgfsetbuttcap
    \pgfpathqmoveto{3.6364bp}{50.2273bp}
    \pgfpathqcurveto{3.6364bp}{50.5620bp}{3.3650bp}{50.8333bp}{3.0303bp}{50.8333bp}
    \pgfpathqcurveto{2.6956bp}{50.8333bp}{2.4242bp}{50.5620bp}{2.4242bp}{50.2273bp}
    \pgfpathqcurveto{2.4242bp}{49.8926bp}{2.6956bp}{49.6212bp}{3.0303bp}{49.6212bp}
    \pgfpathqcurveto{3.3650bp}{49.6212bp}{3.6364bp}{49.8926bp}{3.6364bp}{50.2273bp}
    \pgfpathclose
    \pgfusepathqfillstroke
  \end{pgfscope}
  \begin{pgfscope}
    \definecolor{fc}{rgb}{0.0000,0.0000,0.0000}
    \pgfsetfillcolor{fc}
    \pgftransformshift{\pgfqpoint{5.1515bp}{51.7424bp}}
    \pgftransformscale{0.1894}
    \pgftext[base,left]{negative unproven}
  \end{pgfscope}
  \begin{pgfscope}
    \definecolor{fc}{rgb}{1.0000,1.0000,0.0000}
    \pgfsetfillcolor{fc}
    \pgfsetlinewidth{0.5678bp}
    \definecolor{sc}{rgb}{1.0000,1.0000,0.0000}
    \pgfsetstrokecolor{sc}
    \pgfsetmiterjoin
    \pgfsetbuttcap
    \pgfpathqmoveto{3.6364bp}{52.1970bp}
    \pgfpathqcurveto{3.6364bp}{52.5317bp}{3.3650bp}{52.8030bp}{3.0303bp}{52.8030bp}
    \pgfpathqcurveto{2.6956bp}{52.8030bp}{2.4242bp}{52.5317bp}{2.4242bp}{52.1970bp}
    \pgfpathqcurveto{2.4242bp}{51.8623bp}{2.6956bp}{51.5909bp}{3.0303bp}{51.5909bp}
    \pgfpathqcurveto{3.3650bp}{51.5909bp}{3.6364bp}{51.8623bp}{3.6364bp}{52.1970bp}
    \pgfpathclose
    \pgfusepathqfillstroke
  \end{pgfscope}
  \begin{pgfscope}
    \definecolor{fc}{rgb}{0.0000,0.0000,0.0000}
    \pgfsetfillcolor{fc}
    \pgftransformshift{\pgfqpoint{5.1515bp}{53.7121bp}}
    \pgftransformscale{0.1894}
    \pgftext[base,left]{negative proven}
  \end{pgfscope}
  \begin{pgfscope}
    \definecolor{fc}{rgb}{0.0000,0.5020,0.0000}
    \pgfsetfillcolor{fc}
    \pgfsetlinewidth{0.5678bp}
    \definecolor{sc}{rgb}{0.0000,0.5020,0.0000}
    \pgfsetstrokecolor{sc}
    \pgfsetmiterjoin
    \pgfsetbuttcap
    \pgfpathqmoveto{3.6364bp}{54.1667bp}
    \pgfpathqcurveto{3.6364bp}{54.5014bp}{3.3650bp}{54.7727bp}{3.0303bp}{54.7727bp}
    \pgfpathqcurveto{2.6956bp}{54.7727bp}{2.4242bp}{54.5014bp}{2.4242bp}{54.1667bp}
    \pgfpathqcurveto{2.4242bp}{53.8319bp}{2.6956bp}{53.5606bp}{3.0303bp}{53.5606bp}
    \pgfpathqcurveto{3.3650bp}{53.5606bp}{3.6364bp}{53.8319bp}{3.6364bp}{54.1667bp}
    \pgfpathclose
    \pgfusepathqfillstroke
  \end{pgfscope}
  \begin{pgfscope}
    \definecolor{fc}{rgb}{0.0000,0.0000,0.0000}
    \pgfsetfillcolor{fc}
    \pgftransformshift{\pgfqpoint{5.1515bp}{55.6818bp}}
    \pgftransformscale{0.1894}
    \pgftext[base,left]{positive unproven}
  \end{pgfscope}
  \begin{pgfscope}
    \definecolor{fc}{rgb}{1.0000,0.0000,0.0000}
    \pgfsetfillcolor{fc}
    \pgfsetlinewidth{0.5678bp}
    \definecolor{sc}{rgb}{1.0000,0.0000,0.0000}
    \pgfsetstrokecolor{sc}
    \pgfsetmiterjoin
    \pgfsetbuttcap
    \pgfpathqmoveto{3.6364bp}{56.1364bp}
    \pgfpathqcurveto{3.6364bp}{56.4711bp}{3.3650bp}{56.7424bp}{3.0303bp}{56.7424bp}
    \pgfpathqcurveto{2.6956bp}{56.7424bp}{2.4242bp}{56.4711bp}{2.4242bp}{56.1364bp}
    \pgfpathqcurveto{2.4242bp}{55.8016bp}{2.6956bp}{55.5303bp}{3.0303bp}{55.5303bp}
    \pgfpathqcurveto{3.3650bp}{55.5303bp}{3.6364bp}{55.8016bp}{3.6364bp}{56.1364bp}
    \pgfpathclose
    \pgfusepathqfillstroke
  \end{pgfscope}
  \begin{pgfscope}
    \definecolor{fc}{rgb}{0.0000,0.0000,0.0000}
    \pgfsetfillcolor{fc}
    \pgftransformshift{\pgfqpoint{5.1515bp}{57.6515bp}}
    \pgftransformscale{0.1894}
    \pgftext[base,left]{positive proven}
  \end{pgfscope}
  \begin{pgfscope}
    \definecolor{fc}{rgb}{0.0000,0.0000,1.0000}
    \pgfsetfillcolor{fc}
    \pgfsetlinewidth{0.5678bp}
    \definecolor{sc}{rgb}{0.0000,0.0000,1.0000}
    \pgfsetstrokecolor{sc}
    \pgfsetmiterjoin
    \pgfsetbuttcap
    \pgfpathqmoveto{3.6364bp}{58.1061bp}
    \pgfpathqcurveto{3.6364bp}{58.4408bp}{3.3650bp}{58.7121bp}{3.0303bp}{58.7121bp}
    \pgfpathqcurveto{2.6956bp}{58.7121bp}{2.4242bp}{58.4408bp}{2.4242bp}{58.1061bp}
    \pgfpathqcurveto{2.4242bp}{57.7713bp}{2.6956bp}{57.5000bp}{3.0303bp}{57.5000bp}
    \pgfpathqcurveto{3.3650bp}{57.5000bp}{3.6364bp}{57.7713bp}{3.6364bp}{58.1061bp}
    \pgfpathclose
    \pgfusepathqfillstroke
  \end{pgfscope}
  \begin{pgfscope}
    \pgfsetlinewidth{0.5678bp}
    \definecolor{sc}{rgb}{0.0000,0.0000,0.0000}
    \pgfsetstrokecolor{sc}
    \pgfsetmiterjoin
    \pgfsetbuttcap
    \pgfpathqmoveto{3.0303bp}{61.7424bp}
    \pgfpathqlineto{4.5455bp}{62.5000bp}
    \pgfpathqlineto{6.0606bp}{63.2576bp}
    \pgfpathqlineto{7.5758bp}{64.0152bp}
    \pgfpathqlineto{9.0909bp}{64.7727bp}
    \pgfpathqlineto{10.6061bp}{65.5303bp}
    \pgfpathqlineto{12.1212bp}{66.2879bp}
    \pgfpathqlineto{13.6364bp}{67.0455bp}
    \pgfpathqlineto{15.1515bp}{67.8030bp}
    \pgfpathqlineto{16.6667bp}{68.5606bp}
    \pgfpathqlineto{18.1818bp}{69.3182bp}
    \pgfpathqlineto{19.6970bp}{70.0758bp}
    \pgfpathqlineto{21.2121bp}{70.8333bp}
    \pgfpathqlineto{22.7273bp}{71.5909bp}
    \pgfpathqlineto{24.2424bp}{72.3485bp}
    \pgfpathqlineto{25.7576bp}{73.1061bp}
    \pgfpathqlineto{27.2727bp}{73.8636bp}
    \pgfpathqlineto{28.7879bp}{74.6212bp}
    \pgfpathqlineto{30.3030bp}{75.3788bp}
    \pgfpathqlineto{31.8182bp}{76.1364bp}
    \pgfpathqlineto{33.3333bp}{76.8939bp}
    \pgfpathqlineto{34.8485bp}{77.6515bp}
    \pgfpathqlineto{36.3636bp}{78.4091bp}
    \pgfpathqlineto{37.8788bp}{79.1667bp}
    \pgfpathqlineto{39.3939bp}{79.9242bp}
    \pgfpathqlineto{40.9091bp}{80.6818bp}
    \pgfpathqlineto{42.4242bp}{81.4394bp}
    \pgfpathqlineto{43.9394bp}{82.1970bp}
    \pgfpathqlineto{45.4545bp}{82.9545bp}
    \pgfpathqlineto{46.9697bp}{83.7121bp}
    \pgfpathqlineto{48.4848bp}{84.4697bp}
    \pgfpathqlineto{50.0000bp}{85.2273bp}
    \pgfpathqlineto{51.5152bp}{85.9848bp}
    \pgfpathqlineto{53.0303bp}{86.7424bp}
    \pgfpathqlineto{54.5455bp}{87.5000bp}
    \pgfpathqlineto{56.0606bp}{88.2576bp}
    \pgfpathqlineto{57.5758bp}{89.0152bp}
    \pgfpathqlineto{59.0909bp}{89.7727bp}
    \pgfpathqlineto{60.6061bp}{90.5303bp}
    \pgfpathqlineto{62.1212bp}{91.2879bp}
    \pgfpathqlineto{63.6364bp}{92.0455bp}
    \pgfpathqlineto{65.1515bp}{92.8030bp}
    \pgfpathqlineto{66.6667bp}{93.5606bp}
    \pgfpathqlineto{68.1818bp}{94.3182bp}
    \pgfpathqlineto{69.6970bp}{95.0758bp}
    \pgfpathqlineto{71.2121bp}{95.8333bp}
    \pgfpathqlineto{72.7273bp}{96.5909bp}
    \pgfpathqlineto{74.2424bp}{97.3485bp}
    \pgfpathqlineto{75.7576bp}{98.1061bp}
    \pgfpathqlineto{77.2727bp}{98.8636bp}
    \pgfpathqlineto{78.7879bp}{99.6212bp}
    \pgfpathqlineto{80.3030bp}{100.3788bp}
    \pgfpathqlineto{81.8182bp}{101.1364bp}
    \pgfpathqlineto{83.3333bp}{101.8939bp}
    \pgfpathqlineto{84.8485bp}{102.6515bp}
    \pgfpathqlineto{86.3636bp}{103.4091bp}
    \pgfpathqlineto{87.8788bp}{104.1667bp}
    \pgfpathqlineto{89.3939bp}{104.9242bp}
    \pgfpathqlineto{90.9091bp}{105.6818bp}
    \pgfpathqlineto{92.4242bp}{106.4394bp}
    \pgfpathqlineto{93.9394bp}{107.1970bp}
    \pgfpathqlineto{95.4545bp}{107.9545bp}
    \pgfpathqlineto{96.9697bp}{108.7121bp}
    \pgfpathqlineto{98.4848bp}{109.4697bp}
    \pgfpathqlineto{100.0000bp}{110.2273bp}
    \pgfpathqlineto{101.5152bp}{110.9848bp}
    \pgfpathqlineto{103.0303bp}{111.7424bp}
    \pgfpathqlineto{104.5455bp}{112.5000bp}
    \pgfpathqlineto{106.0606bp}{113.2576bp}
    \pgfpathqlineto{107.5758bp}{114.0152bp}
    \pgfpathqlineto{109.0909bp}{114.7727bp}
    \pgfpathqlineto{110.6061bp}{115.5303bp}
    \pgfpathqlineto{112.1212bp}{116.2879bp}
    \pgfpathqlineto{113.6364bp}{117.0455bp}
    \pgfpathqlineto{115.1515bp}{117.8030bp}
    \pgfpathqlineto{116.6667bp}{118.5606bp}
    \pgfpathqlineto{118.1818bp}{119.3182bp}
    \pgfpathqlineto{119.6970bp}{120.0758bp}
    \pgfpathqlineto{121.2121bp}{120.8333bp}
    \pgfpathqlineto{122.7273bp}{121.5909bp}
    \pgfpathqlineto{124.2424bp}{122.3485bp}
    \pgfpathqlineto{125.7576bp}{123.1061bp}
    \pgfpathqlineto{127.2727bp}{123.8636bp}
    \pgfpathqlineto{128.7879bp}{124.6212bp}
    \pgfpathqlineto{130.3030bp}{125.3788bp}
    \pgfpathqlineto{131.8182bp}{126.1364bp}
    \pgfpathqlineto{133.3333bp}{126.8939bp}
    \pgfpathqlineto{134.8485bp}{127.6515bp}
    \pgfpathqlineto{136.3636bp}{128.4091bp}
    \pgfpathqlineto{137.8788bp}{129.1667bp}
    \pgfpathqlineto{139.3939bp}{129.9242bp}
    \pgfpathqlineto{140.9091bp}{130.6818bp}
    \pgfpathqlineto{142.4242bp}{131.4394bp}
    \pgfpathqlineto{143.9394bp}{132.1970bp}
    \pgfpathqlineto{145.4545bp}{132.9545bp}
    \pgfpathqlineto{146.9697bp}{133.7121bp}
    \pgfpathqlineto{148.4848bp}{134.4697bp}
    \pgfpathqlineto{150.0000bp}{135.2273bp}
    \pgfpathqlineto{151.5152bp}{135.9848bp}
    \pgfpathqlineto{153.0303bp}{136.7424bp}
    \pgfpathqlineto{154.5455bp}{137.5000bp}
    \pgfpathqlineto{156.0606bp}{138.2576bp}
    \pgfpathqlineto{157.5758bp}{139.0152bp}
    \pgfpathqlineto{159.0909bp}{139.7727bp}
    \pgfpathqlineto{160.6061bp}{140.5303bp}
    \pgfpathqlineto{162.1212bp}{141.2879bp}
    \pgfpathqlineto{163.6364bp}{142.0455bp}
    \pgfpathqlineto{165.1515bp}{142.8030bp}
    \pgfpathqlineto{166.6667bp}{143.5606bp}
    \pgfpathqlineto{168.1818bp}{144.3182bp}
    \pgfpathqlineto{169.6970bp}{145.0758bp}
    \pgfpathqlineto{171.2121bp}{145.8333bp}
    \pgfpathqlineto{172.7273bp}{146.5909bp}
    \pgfpathqlineto{174.2424bp}{147.3485bp}
    \pgfpathqlineto{175.7576bp}{148.1061bp}
    \pgfpathqlineto{177.2727bp}{148.8636bp}
    \pgfpathqlineto{178.7879bp}{149.6212bp}
    \pgfpathqlineto{180.3030bp}{150.3788bp}
    \pgfpathqlineto{181.8182bp}{150.3788bp}
    \pgfpathqlineto{183.3333bp}{135.2273bp}
    \pgfpathqlineto{184.8485bp}{128.4091bp}
    \pgfpathqlineto{186.3636bp}{125.3788bp}
    \pgfpathqlineto{187.8788bp}{96.5909bp}
    \pgfpathqlineto{189.3939bp}{87.5000bp}
    \pgfpathqlineto{190.9091bp}{83.7121bp}
    \pgfpathqlineto{192.4242bp}{73.8636bp}
    \pgfpathqlineto{193.9394bp}{73.8636bp}
    \pgfpathqlineto{195.4545bp}{66.2879bp}
    \pgfpathqlineto{196.9697bp}{64.7727bp}
    \pgfpathqlineto{198.4848bp}{64.7727bp}
    \pgfusepathqstroke
  \end{pgfscope}
  \begin{pgfscope}
    \pgfsetlinewidth{0.5678bp}
    \definecolor{sc}{rgb}{1.0000,1.0000,0.0000}
    \pgfsetstrokecolor{sc}
    \pgfsetmiterjoin
    \pgfsetbuttcap
    \pgfpathqmoveto{3.0303bp}{109.4697bp}
    \pgfpathqlineto{4.5455bp}{109.4697bp}
    \pgfpathqlineto{6.0606bp}{109.4697bp}
    \pgfpathqlineto{7.5758bp}{109.4697bp}
    \pgfpathqlineto{9.0909bp}{109.4697bp}
    \pgfpathqlineto{10.6061bp}{109.4697bp}
    \pgfpathqlineto{12.1212bp}{109.4697bp}
    \pgfpathqlineto{13.6364bp}{109.4697bp}
    \pgfpathqlineto{15.1515bp}{109.4697bp}
    \pgfpathqlineto{16.6667bp}{109.4697bp}
    \pgfpathqlineto{18.1818bp}{109.4697bp}
    \pgfpathqlineto{19.6970bp}{109.4697bp}
    \pgfpathqlineto{21.2121bp}{109.4697bp}
    \pgfpathqlineto{22.7273bp}{109.4697bp}
    \pgfpathqlineto{24.2424bp}{109.4697bp}
    \pgfpathqlineto{25.7576bp}{109.4697bp}
    \pgfpathqlineto{27.2727bp}{109.4697bp}
    \pgfpathqlineto{28.7879bp}{109.4697bp}
    \pgfpathqlineto{30.3030bp}{109.4697bp}
    \pgfpathqlineto{31.8182bp}{109.4697bp}
    \pgfpathqlineto{33.3333bp}{109.4697bp}
    \pgfpathqlineto{34.8485bp}{109.4697bp}
    \pgfpathqlineto{36.3636bp}{109.4697bp}
    \pgfpathqlineto{37.8788bp}{109.4697bp}
    \pgfpathqlineto{39.3939bp}{109.4697bp}
    \pgfpathqlineto{40.9091bp}{109.4697bp}
    \pgfpathqlineto{42.4242bp}{109.4697bp}
    \pgfpathqlineto{43.9394bp}{109.4697bp}
    \pgfpathqlineto{45.4545bp}{109.4697bp}
    \pgfpathqlineto{46.9697bp}{109.4697bp}
    \pgfpathqlineto{48.4848bp}{109.4697bp}
    \pgfpathqlineto{50.0000bp}{109.4697bp}
    \pgfpathqlineto{51.5152bp}{109.4697bp}
    \pgfpathqlineto{53.0303bp}{109.4697bp}
    \pgfpathqlineto{54.5455bp}{109.4697bp}
    \pgfpathqlineto{56.0606bp}{109.4697bp}
    \pgfpathqlineto{57.5758bp}{109.4697bp}
    \pgfpathqlineto{59.0909bp}{109.4697bp}
    \pgfpathqlineto{60.6061bp}{109.4697bp}
    \pgfpathqlineto{62.1212bp}{109.4697bp}
    \pgfpathqlineto{63.6364bp}{109.4697bp}
    \pgfpathqlineto{65.1515bp}{109.4697bp}
    \pgfpathqlineto{66.6667bp}{109.4697bp}
    \pgfpathqlineto{68.1818bp}{109.4697bp}
    \pgfpathqlineto{69.6970bp}{109.4697bp}
    \pgfpathqlineto{71.2121bp}{109.4697bp}
    \pgfpathqlineto{72.7273bp}{109.4697bp}
    \pgfpathqlineto{74.2424bp}{109.4697bp}
    \pgfpathqlineto{75.7576bp}{109.4697bp}
    \pgfpathqlineto{77.2727bp}{109.4697bp}
    \pgfpathqlineto{78.7879bp}{109.4697bp}
    \pgfpathqlineto{80.3030bp}{109.4697bp}
    \pgfpathqlineto{81.8182bp}{109.4697bp}
    \pgfpathqlineto{83.3333bp}{109.4697bp}
    \pgfpathqlineto{84.8485bp}{109.4697bp}
    \pgfpathqlineto{86.3636bp}{109.4697bp}
    \pgfpathqlineto{87.8788bp}{109.4697bp}
    \pgfpathqlineto{89.3939bp}{109.4697bp}
    \pgfpathqlineto{90.9091bp}{109.4697bp}
    \pgfpathqlineto{92.4242bp}{109.4697bp}
    \pgfpathqlineto{93.9394bp}{109.4697bp}
    \pgfpathqlineto{95.4545bp}{109.4697bp}
    \pgfpathqlineto{96.9697bp}{109.4697bp}
    \pgfpathqlineto{98.4848bp}{109.4697bp}
    \pgfpathqlineto{100.0000bp}{109.4697bp}
    \pgfpathqlineto{101.5152bp}{109.4697bp}
    \pgfpathqlineto{103.0303bp}{109.4697bp}
    \pgfpathqlineto{104.5455bp}{109.4697bp}
    \pgfpathqlineto{106.0606bp}{109.4697bp}
    \pgfpathqlineto{107.5758bp}{109.4697bp}
    \pgfpathqlineto{109.0909bp}{109.4697bp}
    \pgfpathqlineto{110.6061bp}{109.4697bp}
    \pgfpathqlineto{112.1212bp}{109.4697bp}
    \pgfpathqlineto{113.6364bp}{109.4697bp}
    \pgfpathqlineto{115.1515bp}{109.4697bp}
    \pgfpathqlineto{116.6667bp}{109.4697bp}
    \pgfpathqlineto{118.1818bp}{109.4697bp}
    \pgfpathqlineto{119.6970bp}{109.4697bp}
    \pgfpathqlineto{121.2121bp}{109.4697bp}
    \pgfpathqlineto{122.7273bp}{109.4697bp}
    \pgfpathqlineto{124.2424bp}{109.4697bp}
    \pgfpathqlineto{125.7576bp}{109.4697bp}
    \pgfpathqlineto{127.2727bp}{109.4697bp}
    \pgfpathqlineto{128.7879bp}{109.4697bp}
    \pgfpathqlineto{130.3030bp}{109.4697bp}
    \pgfpathqlineto{131.8182bp}{109.4697bp}
    \pgfpathqlineto{133.3333bp}{109.4697bp}
    \pgfpathqlineto{134.8485bp}{109.4697bp}
    \pgfpathqlineto{136.3636bp}{109.4697bp}
    \pgfpathqlineto{137.8788bp}{109.4697bp}
    \pgfpathqlineto{139.3939bp}{109.4697bp}
    \pgfpathqlineto{140.9091bp}{109.4697bp}
    \pgfpathqlineto{142.4242bp}{109.4697bp}
    \pgfpathqlineto{143.9394bp}{109.4697bp}
    \pgfpathqlineto{145.4545bp}{109.4697bp}
    \pgfpathqlineto{146.9697bp}{109.4697bp}
    \pgfpathqlineto{148.4848bp}{109.4697bp}
    \pgfpathqlineto{150.0000bp}{109.4697bp}
    \pgfpathqlineto{151.5152bp}{109.4697bp}
    \pgfpathqlineto{153.0303bp}{109.4697bp}
    \pgfpathqlineto{154.5455bp}{109.4697bp}
    \pgfpathqlineto{156.0606bp}{109.4697bp}
    \pgfpathqlineto{157.5758bp}{109.4697bp}
    \pgfpathqlineto{159.0909bp}{109.4697bp}
    \pgfpathqlineto{160.6061bp}{109.4697bp}
    \pgfpathqlineto{162.1212bp}{109.4697bp}
    \pgfpathqlineto{163.6364bp}{109.4697bp}
    \pgfpathqlineto{165.1515bp}{109.4697bp}
    \pgfpathqlineto{166.6667bp}{109.4697bp}
    \pgfpathqlineto{168.1818bp}{109.4697bp}
    \pgfpathqlineto{169.6970bp}{109.4697bp}
    \pgfpathqlineto{171.2121bp}{109.4697bp}
    \pgfpathqlineto{172.7273bp}{109.4697bp}
    \pgfpathqlineto{174.2424bp}{109.4697bp}
    \pgfpathqlineto{175.7576bp}{109.4697bp}
    \pgfpathqlineto{177.2727bp}{109.4697bp}
    \pgfpathqlineto{178.7879bp}{109.4697bp}
    \pgfpathqlineto{180.3030bp}{109.4697bp}
    \pgfpathqlineto{181.8182bp}{103.4091bp}
    \pgfpathqlineto{183.3333bp}{103.4091bp}
    \pgfpathqlineto{184.8485bp}{103.4091bp}
    \pgfpathqlineto{186.3636bp}{103.4091bp}
    \pgfpathqlineto{187.8788bp}{103.4091bp}
    \pgfpathqlineto{189.3939bp}{103.4091bp}
    \pgfpathqlineto{190.9091bp}{103.4091bp}
    \pgfpathqlineto{192.4242bp}{103.4091bp}
    \pgfpathqlineto{193.9394bp}{103.4091bp}
    \pgfpathqlineto{195.4545bp}{103.4091bp}
    \pgfpathqlineto{196.9697bp}{103.4091bp}
    \pgfpathqlineto{198.4848bp}{103.4091bp}
    \pgfusepathqstroke
  \end{pgfscope}
  \begin{pgfscope}
    \pgfsetlinewidth{0.5678bp}
    \definecolor{sc}{rgb}{0.0000,0.5020,0.0000}
    \pgfsetstrokecolor{sc}
    \pgfsetmiterjoin
    \pgfsetbuttcap
    \pgfpathqmoveto{3.0303bp}{60.9848bp}
    \pgfpathqlineto{4.5455bp}{60.9848bp}
    \pgfpathqlineto{6.0606bp}{60.9848bp}
    \pgfpathqlineto{7.5758bp}{60.9848bp}
    \pgfpathqlineto{9.0909bp}{60.9848bp}
    \pgfpathqlineto{10.6061bp}{60.9848bp}
    \pgfpathqlineto{12.1212bp}{60.9848bp}
    \pgfpathqlineto{13.6364bp}{60.9848bp}
    \pgfpathqlineto{15.1515bp}{60.9848bp}
    \pgfpathqlineto{16.6667bp}{60.9848bp}
    \pgfpathqlineto{18.1818bp}{60.9848bp}
    \pgfpathqlineto{19.6970bp}{60.9848bp}
    \pgfpathqlineto{21.2121bp}{60.9848bp}
    \pgfpathqlineto{22.7273bp}{60.9848bp}
    \pgfpathqlineto{24.2424bp}{60.9848bp}
    \pgfpathqlineto{25.7576bp}{60.9848bp}
    \pgfpathqlineto{27.2727bp}{60.9848bp}
    \pgfpathqlineto{28.7879bp}{60.9848bp}
    \pgfpathqlineto{30.3030bp}{60.9848bp}
    \pgfpathqlineto{31.8182bp}{60.9848bp}
    \pgfpathqlineto{33.3333bp}{60.9848bp}
    \pgfpathqlineto{34.8485bp}{60.9848bp}
    \pgfpathqlineto{36.3636bp}{60.9848bp}
    \pgfpathqlineto{37.8788bp}{60.9848bp}
    \pgfpathqlineto{39.3939bp}{60.9848bp}
    \pgfpathqlineto{40.9091bp}{60.9848bp}
    \pgfpathqlineto{42.4242bp}{60.9848bp}
    \pgfpathqlineto{43.9394bp}{60.9848bp}
    \pgfpathqlineto{45.4545bp}{60.9848bp}
    \pgfpathqlineto{46.9697bp}{60.9848bp}
    \pgfpathqlineto{48.4848bp}{60.9848bp}
    \pgfpathqlineto{50.0000bp}{60.9848bp}
    \pgfpathqlineto{51.5152bp}{60.9848bp}
    \pgfpathqlineto{53.0303bp}{60.9848bp}
    \pgfpathqlineto{54.5455bp}{60.9848bp}
    \pgfpathqlineto{56.0606bp}{60.9848bp}
    \pgfpathqlineto{57.5758bp}{60.9848bp}
    \pgfpathqlineto{59.0909bp}{60.9848bp}
    \pgfpathqlineto{60.6061bp}{60.9848bp}
    \pgfpathqlineto{62.1212bp}{60.9848bp}
    \pgfpathqlineto{63.6364bp}{60.9848bp}
    \pgfpathqlineto{65.1515bp}{60.9848bp}
    \pgfpathqlineto{66.6667bp}{60.9848bp}
    \pgfpathqlineto{68.1818bp}{60.9848bp}
    \pgfpathqlineto{69.6970bp}{60.9848bp}
    \pgfpathqlineto{71.2121bp}{60.9848bp}
    \pgfpathqlineto{72.7273bp}{60.9848bp}
    \pgfpathqlineto{74.2424bp}{60.9848bp}
    \pgfpathqlineto{75.7576bp}{60.9848bp}
    \pgfpathqlineto{77.2727bp}{60.9848bp}
    \pgfpathqlineto{78.7879bp}{60.9848bp}
    \pgfpathqlineto{80.3030bp}{60.9848bp}
    \pgfpathqlineto{81.8182bp}{60.9848bp}
    \pgfpathqlineto{83.3333bp}{60.9848bp}
    \pgfpathqlineto{84.8485bp}{60.9848bp}
    \pgfpathqlineto{86.3636bp}{60.9848bp}
    \pgfpathqlineto{87.8788bp}{60.9848bp}
    \pgfpathqlineto{89.3939bp}{60.9848bp}
    \pgfpathqlineto{90.9091bp}{60.9848bp}
    \pgfpathqlineto{92.4242bp}{60.9848bp}
    \pgfpathqlineto{93.9394bp}{60.9848bp}
    \pgfpathqlineto{95.4545bp}{60.9848bp}
    \pgfpathqlineto{96.9697bp}{60.9848bp}
    \pgfpathqlineto{98.4848bp}{60.9848bp}
    \pgfpathqlineto{100.0000bp}{60.9848bp}
    \pgfpathqlineto{101.5152bp}{60.9848bp}
    \pgfpathqlineto{103.0303bp}{60.9848bp}
    \pgfpathqlineto{104.5455bp}{60.9848bp}
    \pgfpathqlineto{106.0606bp}{60.9848bp}
    \pgfpathqlineto{107.5758bp}{60.9848bp}
    \pgfpathqlineto{109.0909bp}{60.9848bp}
    \pgfpathqlineto{110.6061bp}{60.9848bp}
    \pgfpathqlineto{112.1212bp}{60.9848bp}
    \pgfpathqlineto{113.6364bp}{60.9848bp}
    \pgfpathqlineto{115.1515bp}{60.9848bp}
    \pgfpathqlineto{116.6667bp}{60.9848bp}
    \pgfpathqlineto{118.1818bp}{60.9848bp}
    \pgfpathqlineto{119.6970bp}{60.9848bp}
    \pgfpathqlineto{121.2121bp}{60.9848bp}
    \pgfpathqlineto{122.7273bp}{60.9848bp}
    \pgfpathqlineto{124.2424bp}{60.9848bp}
    \pgfpathqlineto{125.7576bp}{60.9848bp}
    \pgfpathqlineto{127.2727bp}{60.9848bp}
    \pgfpathqlineto{128.7879bp}{60.9848bp}
    \pgfpathqlineto{130.3030bp}{60.9848bp}
    \pgfpathqlineto{131.8182bp}{60.9848bp}
    \pgfpathqlineto{133.3333bp}{60.9848bp}
    \pgfpathqlineto{134.8485bp}{60.9848bp}
    \pgfpathqlineto{136.3636bp}{60.9848bp}
    \pgfpathqlineto{137.8788bp}{60.9848bp}
    \pgfpathqlineto{139.3939bp}{60.9848bp}
    \pgfpathqlineto{140.9091bp}{60.9848bp}
    \pgfpathqlineto{142.4242bp}{60.9848bp}
    \pgfpathqlineto{143.9394bp}{60.9848bp}
    \pgfpathqlineto{145.4545bp}{60.9848bp}
    \pgfpathqlineto{146.9697bp}{60.9848bp}
    \pgfpathqlineto{148.4848bp}{60.9848bp}
    \pgfpathqlineto{150.0000bp}{60.9848bp}
    \pgfpathqlineto{151.5152bp}{60.9848bp}
    \pgfpathqlineto{153.0303bp}{60.9848bp}
    \pgfpathqlineto{154.5455bp}{60.9848bp}
    \pgfpathqlineto{156.0606bp}{60.9848bp}
    \pgfpathqlineto{157.5758bp}{60.9848bp}
    \pgfpathqlineto{159.0909bp}{60.9848bp}
    \pgfpathqlineto{160.6061bp}{60.9848bp}
    \pgfpathqlineto{162.1212bp}{60.9848bp}
    \pgfpathqlineto{163.6364bp}{60.9848bp}
    \pgfpathqlineto{165.1515bp}{60.9848bp}
    \pgfpathqlineto{166.6667bp}{60.9848bp}
    \pgfpathqlineto{168.1818bp}{60.9848bp}
    \pgfpathqlineto{169.6970bp}{60.9848bp}
    \pgfpathqlineto{171.2121bp}{60.9848bp}
    \pgfpathqlineto{172.7273bp}{60.9848bp}
    \pgfpathqlineto{174.2424bp}{60.9848bp}
    \pgfpathqlineto{175.7576bp}{60.9848bp}
    \pgfpathqlineto{177.2727bp}{60.9848bp}
    \pgfpathqlineto{178.7879bp}{60.9848bp}
    \pgfpathqlineto{180.3030bp}{60.9848bp}
    \pgfpathqlineto{181.8182bp}{60.9848bp}
    \pgfpathqlineto{183.3333bp}{60.9848bp}
    \pgfpathqlineto{184.8485bp}{60.9848bp}
    \pgfpathqlineto{186.3636bp}{60.9848bp}
    \pgfpathqlineto{187.8788bp}{60.9848bp}
    \pgfpathqlineto{189.3939bp}{60.9848bp}
    \pgfpathqlineto{190.9091bp}{60.9848bp}
    \pgfpathqlineto{192.4242bp}{60.9848bp}
    \pgfpathqlineto{193.9394bp}{60.9848bp}
    \pgfpathqlineto{195.4545bp}{60.9848bp}
    \pgfpathqlineto{196.9697bp}{60.9848bp}
    \pgfpathqlineto{198.4848bp}{60.9848bp}
    \pgfusepathqstroke
  \end{pgfscope}
  \begin{pgfscope}
    \pgfsetlinewidth{0.5678bp}
    \definecolor{sc}{rgb}{1.0000,0.0000,0.0000}
    \pgfsetstrokecolor{sc}
    \pgfsetmiterjoin
    \pgfsetbuttcap
    \pgfpathqmoveto{3.0303bp}{109.4697bp}
    \pgfpathqlineto{4.5455bp}{109.4697bp}
    \pgfpathqlineto{6.0606bp}{109.4697bp}
    \pgfpathqlineto{7.5758bp}{109.4697bp}
    \pgfpathqlineto{9.0909bp}{109.4697bp}
    \pgfpathqlineto{10.6061bp}{109.4697bp}
    \pgfpathqlineto{12.1212bp}{109.4697bp}
    \pgfpathqlineto{13.6364bp}{109.4697bp}
    \pgfpathqlineto{15.1515bp}{109.4697bp}
    \pgfpathqlineto{16.6667bp}{109.4697bp}
    \pgfpathqlineto{18.1818bp}{109.4697bp}
    \pgfpathqlineto{19.6970bp}{109.4697bp}
    \pgfpathqlineto{21.2121bp}{109.4697bp}
    \pgfpathqlineto{22.7273bp}{109.4697bp}
    \pgfpathqlineto{24.2424bp}{109.4697bp}
    \pgfpathqlineto{25.7576bp}{109.4697bp}
    \pgfpathqlineto{27.2727bp}{109.4697bp}
    \pgfpathqlineto{28.7879bp}{109.4697bp}
    \pgfpathqlineto{30.3030bp}{109.4697bp}
    \pgfpathqlineto{31.8182bp}{109.4697bp}
    \pgfpathqlineto{33.3333bp}{109.4697bp}
    \pgfpathqlineto{34.8485bp}{109.4697bp}
    \pgfpathqlineto{36.3636bp}{109.4697bp}
    \pgfpathqlineto{37.8788bp}{109.4697bp}
    \pgfpathqlineto{39.3939bp}{109.4697bp}
    \pgfpathqlineto{40.9091bp}{109.4697bp}
    \pgfpathqlineto{42.4242bp}{109.4697bp}
    \pgfpathqlineto{43.9394bp}{109.4697bp}
    \pgfpathqlineto{45.4545bp}{109.4697bp}
    \pgfpathqlineto{46.9697bp}{109.4697bp}
    \pgfpathqlineto{48.4848bp}{109.4697bp}
    \pgfpathqlineto{50.0000bp}{109.4697bp}
    \pgfpathqlineto{51.5152bp}{109.4697bp}
    \pgfpathqlineto{53.0303bp}{109.4697bp}
    \pgfpathqlineto{54.5455bp}{109.4697bp}
    \pgfpathqlineto{56.0606bp}{109.4697bp}
    \pgfpathqlineto{57.5758bp}{109.4697bp}
    \pgfpathqlineto{59.0909bp}{109.4697bp}
    \pgfpathqlineto{60.6061bp}{109.4697bp}
    \pgfpathqlineto{62.1212bp}{109.4697bp}
    \pgfpathqlineto{63.6364bp}{109.4697bp}
    \pgfpathqlineto{65.1515bp}{109.4697bp}
    \pgfpathqlineto{66.6667bp}{109.4697bp}
    \pgfpathqlineto{68.1818bp}{109.4697bp}
    \pgfpathqlineto{69.6970bp}{109.4697bp}
    \pgfpathqlineto{71.2121bp}{109.4697bp}
    \pgfpathqlineto{72.7273bp}{109.4697bp}
    \pgfpathqlineto{74.2424bp}{109.4697bp}
    \pgfpathqlineto{75.7576bp}{109.4697bp}
    \pgfpathqlineto{77.2727bp}{109.4697bp}
    \pgfpathqlineto{78.7879bp}{109.4697bp}
    \pgfpathqlineto{80.3030bp}{109.4697bp}
    \pgfpathqlineto{81.8182bp}{109.4697bp}
    \pgfpathqlineto{83.3333bp}{109.4697bp}
    \pgfpathqlineto{84.8485bp}{109.4697bp}
    \pgfpathqlineto{86.3636bp}{109.4697bp}
    \pgfpathqlineto{87.8788bp}{109.4697bp}
    \pgfpathqlineto{89.3939bp}{109.4697bp}
    \pgfpathqlineto{90.9091bp}{109.4697bp}
    \pgfpathqlineto{92.4242bp}{109.4697bp}
    \pgfpathqlineto{93.9394bp}{109.4697bp}
    \pgfpathqlineto{95.4545bp}{109.4697bp}
    \pgfpathqlineto{96.9697bp}{109.4697bp}
    \pgfpathqlineto{98.4848bp}{109.4697bp}
    \pgfpathqlineto{100.0000bp}{109.4697bp}
    \pgfpathqlineto{101.5152bp}{109.4697bp}
    \pgfpathqlineto{103.0303bp}{109.4697bp}
    \pgfpathqlineto{104.5455bp}{109.4697bp}
    \pgfpathqlineto{106.0606bp}{109.4697bp}
    \pgfpathqlineto{107.5758bp}{109.4697bp}
    \pgfpathqlineto{109.0909bp}{109.4697bp}
    \pgfpathqlineto{110.6061bp}{109.4697bp}
    \pgfpathqlineto{112.1212bp}{109.4697bp}
    \pgfpathqlineto{113.6364bp}{109.4697bp}
    \pgfpathqlineto{115.1515bp}{109.4697bp}
    \pgfpathqlineto{116.6667bp}{109.4697bp}
    \pgfpathqlineto{118.1818bp}{109.4697bp}
    \pgfpathqlineto{119.6970bp}{109.4697bp}
    \pgfpathqlineto{121.2121bp}{109.4697bp}
    \pgfpathqlineto{122.7273bp}{109.4697bp}
    \pgfpathqlineto{124.2424bp}{109.4697bp}
    \pgfpathqlineto{125.7576bp}{109.4697bp}
    \pgfpathqlineto{127.2727bp}{109.4697bp}
    \pgfpathqlineto{128.7879bp}{109.4697bp}
    \pgfpathqlineto{130.3030bp}{109.4697bp}
    \pgfpathqlineto{131.8182bp}{109.4697bp}
    \pgfpathqlineto{133.3333bp}{109.4697bp}
    \pgfpathqlineto{134.8485bp}{109.4697bp}
    \pgfpathqlineto{136.3636bp}{109.4697bp}
    \pgfpathqlineto{137.8788bp}{109.4697bp}
    \pgfpathqlineto{139.3939bp}{109.4697bp}
    \pgfpathqlineto{140.9091bp}{109.4697bp}
    \pgfpathqlineto{142.4242bp}{109.4697bp}
    \pgfpathqlineto{143.9394bp}{109.4697bp}
    \pgfpathqlineto{145.4545bp}{109.4697bp}
    \pgfpathqlineto{146.9697bp}{109.4697bp}
    \pgfpathqlineto{148.4848bp}{109.4697bp}
    \pgfpathqlineto{150.0000bp}{109.4697bp}
    \pgfpathqlineto{151.5152bp}{109.4697bp}
    \pgfpathqlineto{153.0303bp}{109.4697bp}
    \pgfpathqlineto{154.5455bp}{109.4697bp}
    \pgfpathqlineto{156.0606bp}{109.4697bp}
    \pgfpathqlineto{157.5758bp}{109.4697bp}
    \pgfpathqlineto{159.0909bp}{109.4697bp}
    \pgfpathqlineto{160.6061bp}{109.4697bp}
    \pgfpathqlineto{162.1212bp}{109.4697bp}
    \pgfpathqlineto{163.6364bp}{109.4697bp}
    \pgfpathqlineto{165.1515bp}{109.4697bp}
    \pgfpathqlineto{166.6667bp}{109.4697bp}
    \pgfpathqlineto{168.1818bp}{109.4697bp}
    \pgfpathqlineto{169.6970bp}{109.4697bp}
    \pgfpathqlineto{171.2121bp}{109.4697bp}
    \pgfpathqlineto{172.7273bp}{109.4697bp}
    \pgfpathqlineto{174.2424bp}{109.4697bp}
    \pgfpathqlineto{175.7576bp}{109.4697bp}
    \pgfpathqlineto{177.2727bp}{109.4697bp}
    \pgfpathqlineto{178.7879bp}{109.4697bp}
    \pgfpathqlineto{180.3030bp}{109.4697bp}
    \pgfpathqlineto{181.8182bp}{67.0455bp}
    \pgfpathqlineto{183.3333bp}{67.0455bp}
    \pgfpathqlineto{184.8485bp}{67.0455bp}
    \pgfpathqlineto{186.3636bp}{67.0455bp}
    \pgfpathqlineto{187.8788bp}{67.0455bp}
    \pgfpathqlineto{189.3939bp}{67.0455bp}
    \pgfpathqlineto{190.9091bp}{67.0455bp}
    \pgfpathqlineto{192.4242bp}{66.2879bp}
    \pgfpathqlineto{193.9394bp}{66.2879bp}
    \pgfpathqlineto{195.4545bp}{65.5303bp}
    \pgfpathqlineto{196.9697bp}{65.5303bp}
    \pgfpathqlineto{198.4848bp}{65.5303bp}
    \pgfusepathqstroke
  \end{pgfscope}
  \begin{pgfscope}
    \pgfsetlinewidth{0.5678bp}
    \definecolor{sc}{rgb}{0.0000,0.0000,1.0000}
    \pgfsetstrokecolor{sc}
    \pgfsetmiterjoin
    \pgfsetbuttcap
    \pgfpathqmoveto{3.0303bp}{60.9848bp}
    \pgfpathqlineto{4.5455bp}{60.9848bp}
    \pgfpathqlineto{6.0606bp}{60.9848bp}
    \pgfpathqlineto{7.5758bp}{60.9848bp}
    \pgfpathqlineto{9.0909bp}{60.9848bp}
    \pgfpathqlineto{10.6061bp}{60.9848bp}
    \pgfpathqlineto{12.1212bp}{60.9848bp}
    \pgfpathqlineto{13.6364bp}{60.9848bp}
    \pgfpathqlineto{15.1515bp}{60.9848bp}
    \pgfpathqlineto{16.6667bp}{60.9848bp}
    \pgfpathqlineto{18.1818bp}{60.9848bp}
    \pgfpathqlineto{19.6970bp}{60.9848bp}
    \pgfpathqlineto{21.2121bp}{60.9848bp}
    \pgfpathqlineto{22.7273bp}{60.9848bp}
    \pgfpathqlineto{24.2424bp}{60.9848bp}
    \pgfpathqlineto{25.7576bp}{60.9848bp}
    \pgfpathqlineto{27.2727bp}{60.9848bp}
    \pgfpathqlineto{28.7879bp}{60.9848bp}
    \pgfpathqlineto{30.3030bp}{60.9848bp}
    \pgfpathqlineto{31.8182bp}{60.9848bp}
    \pgfpathqlineto{33.3333bp}{60.9848bp}
    \pgfpathqlineto{34.8485bp}{60.9848bp}
    \pgfpathqlineto{36.3636bp}{60.9848bp}
    \pgfpathqlineto{37.8788bp}{60.9848bp}
    \pgfpathqlineto{39.3939bp}{60.9848bp}
    \pgfpathqlineto{40.9091bp}{60.9848bp}
    \pgfpathqlineto{42.4242bp}{60.9848bp}
    \pgfpathqlineto{43.9394bp}{60.9848bp}
    \pgfpathqlineto{45.4545bp}{60.9848bp}
    \pgfpathqlineto{46.9697bp}{60.9848bp}
    \pgfpathqlineto{48.4848bp}{60.9848bp}
    \pgfpathqlineto{50.0000bp}{60.9848bp}
    \pgfpathqlineto{51.5152bp}{60.9848bp}
    \pgfpathqlineto{53.0303bp}{60.9848bp}
    \pgfpathqlineto{54.5455bp}{60.9848bp}
    \pgfpathqlineto{56.0606bp}{60.9848bp}
    \pgfpathqlineto{57.5758bp}{60.9848bp}
    \pgfpathqlineto{59.0909bp}{60.9848bp}
    \pgfpathqlineto{60.6061bp}{60.9848bp}
    \pgfpathqlineto{62.1212bp}{60.9848bp}
    \pgfpathqlineto{63.6364bp}{60.9848bp}
    \pgfpathqlineto{65.1515bp}{60.9848bp}
    \pgfpathqlineto{66.6667bp}{60.9848bp}
    \pgfpathqlineto{68.1818bp}{60.9848bp}
    \pgfpathqlineto{69.6970bp}{60.9848bp}
    \pgfpathqlineto{71.2121bp}{60.9848bp}
    \pgfpathqlineto{72.7273bp}{60.9848bp}
    \pgfpathqlineto{74.2424bp}{60.9848bp}
    \pgfpathqlineto{75.7576bp}{60.9848bp}
    \pgfpathqlineto{77.2727bp}{60.9848bp}
    \pgfpathqlineto{78.7879bp}{60.9848bp}
    \pgfpathqlineto{80.3030bp}{60.9848bp}
    \pgfpathqlineto{81.8182bp}{60.9848bp}
    \pgfpathqlineto{83.3333bp}{60.9848bp}
    \pgfpathqlineto{84.8485bp}{60.9848bp}
    \pgfpathqlineto{86.3636bp}{60.9848bp}
    \pgfpathqlineto{87.8788bp}{60.9848bp}
    \pgfpathqlineto{89.3939bp}{60.9848bp}
    \pgfpathqlineto{90.9091bp}{60.9848bp}
    \pgfpathqlineto{92.4242bp}{60.9848bp}
    \pgfpathqlineto{93.9394bp}{60.9848bp}
    \pgfpathqlineto{95.4545bp}{60.9848bp}
    \pgfpathqlineto{96.9697bp}{60.9848bp}
    \pgfpathqlineto{98.4848bp}{60.9848bp}
    \pgfpathqlineto{100.0000bp}{60.9848bp}
    \pgfpathqlineto{101.5152bp}{60.9848bp}
    \pgfpathqlineto{103.0303bp}{60.9848bp}
    \pgfpathqlineto{104.5455bp}{60.9848bp}
    \pgfpathqlineto{106.0606bp}{60.9848bp}
    \pgfpathqlineto{107.5758bp}{60.9848bp}
    \pgfpathqlineto{109.0909bp}{60.9848bp}
    \pgfpathqlineto{110.6061bp}{60.9848bp}
    \pgfpathqlineto{112.1212bp}{60.9848bp}
    \pgfpathqlineto{113.6364bp}{60.9848bp}
    \pgfpathqlineto{115.1515bp}{60.9848bp}
    \pgfpathqlineto{116.6667bp}{60.9848bp}
    \pgfpathqlineto{118.1818bp}{60.9848bp}
    \pgfpathqlineto{119.6970bp}{60.9848bp}
    \pgfpathqlineto{121.2121bp}{60.9848bp}
    \pgfpathqlineto{122.7273bp}{60.9848bp}
    \pgfpathqlineto{124.2424bp}{60.9848bp}
    \pgfpathqlineto{125.7576bp}{60.9848bp}
    \pgfpathqlineto{127.2727bp}{60.9848bp}
    \pgfpathqlineto{128.7879bp}{60.9848bp}
    \pgfpathqlineto{130.3030bp}{60.9848bp}
    \pgfpathqlineto{131.8182bp}{60.9848bp}
    \pgfpathqlineto{133.3333bp}{60.9848bp}
    \pgfpathqlineto{134.8485bp}{60.9848bp}
    \pgfpathqlineto{136.3636bp}{60.9848bp}
    \pgfpathqlineto{137.8788bp}{60.9848bp}
    \pgfpathqlineto{139.3939bp}{60.9848bp}
    \pgfpathqlineto{140.9091bp}{60.9848bp}
    \pgfpathqlineto{142.4242bp}{60.9848bp}
    \pgfpathqlineto{143.9394bp}{60.9848bp}
    \pgfpathqlineto{145.4545bp}{60.9848bp}
    \pgfpathqlineto{146.9697bp}{60.9848bp}
    \pgfpathqlineto{148.4848bp}{60.9848bp}
    \pgfpathqlineto{150.0000bp}{60.9848bp}
    \pgfpathqlineto{151.5152bp}{60.9848bp}
    \pgfpathqlineto{153.0303bp}{60.9848bp}
    \pgfpathqlineto{154.5455bp}{60.9848bp}
    \pgfpathqlineto{156.0606bp}{60.9848bp}
    \pgfpathqlineto{157.5758bp}{60.9848bp}
    \pgfpathqlineto{159.0909bp}{60.9848bp}
    \pgfpathqlineto{160.6061bp}{60.9848bp}
    \pgfpathqlineto{162.1212bp}{60.9848bp}
    \pgfpathqlineto{163.6364bp}{60.9848bp}
    \pgfpathqlineto{165.1515bp}{60.9848bp}
    \pgfpathqlineto{166.6667bp}{60.9848bp}
    \pgfpathqlineto{168.1818bp}{60.9848bp}
    \pgfpathqlineto{169.6970bp}{60.9848bp}
    \pgfpathqlineto{171.2121bp}{60.9848bp}
    \pgfpathqlineto{172.7273bp}{60.9848bp}
    \pgfpathqlineto{174.2424bp}{60.9848bp}
    \pgfpathqlineto{175.7576bp}{60.9848bp}
    \pgfpathqlineto{177.2727bp}{60.9848bp}
    \pgfpathqlineto{178.7879bp}{60.9848bp}
    \pgfpathqlineto{180.3030bp}{60.9848bp}
    \pgfpathqlineto{181.8182bp}{60.9848bp}
    \pgfpathqlineto{183.3333bp}{60.9848bp}
    \pgfpathqlineto{184.8485bp}{60.9848bp}
    \pgfpathqlineto{186.3636bp}{60.9848bp}
    \pgfpathqlineto{187.8788bp}{60.9848bp}
    \pgfpathqlineto{189.3939bp}{60.9848bp}
    \pgfpathqlineto{190.9091bp}{60.9848bp}
    \pgfpathqlineto{192.4242bp}{61.7424bp}
    \pgfpathqlineto{193.9394bp}{61.7424bp}
    \pgfpathqlineto{195.4545bp}{62.5000bp}
    \pgfpathqlineto{196.9697bp}{62.5000bp}
    \pgfpathqlineto{198.4848bp}{62.5000bp}
    \pgfusepathqstroke
  \end{pgfscope}
  \begin{pgfscope}
    \pgfsetlinewidth{0.5678bp}
    \definecolor{sc}{rgb}{1.0000,0.0000,0.0000}
    \pgfsetstrokecolor{sc}
    \pgfsetmiterjoin
    \pgfsetbuttcap
    \pgfpathqmoveto{200.0000bp}{60.9848bp}
    \pgfpathqlineto{200.0000bp}{60.2273bp}
    \pgfusepathqstroke
  \end{pgfscope}
  \begin{pgfscope}
    \pgfsetlinewidth{0.5678bp}
    \definecolor{sc}{rgb}{1.0000,0.0000,0.0000}
    \pgfsetstrokecolor{sc}
    \pgfsetmiterjoin
    \pgfsetbuttcap
    \pgfpathqmoveto{183.3333bp}{60.9848bp}
    \pgfpathqlineto{183.3333bp}{60.2273bp}
    \pgfusepathqstroke
  \end{pgfscope}
  \begin{pgfscope}
    \pgfsetlinewidth{0.5678bp}
    \definecolor{sc}{rgb}{1.0000,0.0000,0.0000}
    \pgfsetstrokecolor{sc}
    \pgfsetmiterjoin
    \pgfsetbuttcap
    \pgfpathqmoveto{3.0303bp}{60.9848bp}
    \pgfpathqlineto{3.0303bp}{60.2273bp}
    \pgfusepathqstroke
  \end{pgfscope}
  \begin{pgfscope}
    \pgfsetlinewidth{0.5678bp}
    \definecolor{sc}{rgb}{0.0000,0.0000,0.0000}
    \pgfsetstrokecolor{sc}
    \pgfsetmiterjoin
    \pgfsetbuttcap
    \pgfpathqmoveto{200.0000bp}{60.9848bp}
    \pgfpathqlineto{200.0000bp}{60.2273bp}
    \pgfusepathqstroke
  \end{pgfscope}
  \begin{pgfscope}
    \pgfsetlinewidth{0.5678bp}
    \definecolor{sc}{rgb}{0.0000,0.0000,0.0000}
    \pgfsetstrokecolor{sc}
    \pgfsetmiterjoin
    \pgfsetbuttcap
    \pgfpathqmoveto{192.4242bp}{60.9848bp}
    \pgfpathqlineto{192.4242bp}{60.2273bp}
    \pgfusepathqstroke
  \end{pgfscope}
  \begin{pgfscope}
    \pgfsetlinewidth{0.5678bp}
    \definecolor{sc}{rgb}{0.0000,0.0000,0.0000}
    \pgfsetstrokecolor{sc}
    \pgfsetmiterjoin
    \pgfsetbuttcap
    \pgfpathqmoveto{184.8485bp}{60.9848bp}
    \pgfpathqlineto{184.8485bp}{60.2273bp}
    \pgfusepathqstroke
  \end{pgfscope}
  \begin{pgfscope}
    \pgfsetlinewidth{0.5678bp}
    \definecolor{sc}{rgb}{0.0000,0.0000,0.0000}
    \pgfsetstrokecolor{sc}
    \pgfsetmiterjoin
    \pgfsetbuttcap
    \pgfpathqmoveto{177.2727bp}{60.9848bp}
    \pgfpathqlineto{177.2727bp}{60.2273bp}
    \pgfusepathqstroke
  \end{pgfscope}
  \begin{pgfscope}
    \pgfsetlinewidth{0.5678bp}
    \definecolor{sc}{rgb}{0.0000,0.0000,0.0000}
    \pgfsetstrokecolor{sc}
    \pgfsetmiterjoin
    \pgfsetbuttcap
    \pgfpathqmoveto{169.6970bp}{60.9848bp}
    \pgfpathqlineto{169.6970bp}{60.2273bp}
    \pgfusepathqstroke
  \end{pgfscope}
  \begin{pgfscope}
    \pgfsetlinewidth{0.5678bp}
    \definecolor{sc}{rgb}{0.0000,0.0000,0.0000}
    \pgfsetstrokecolor{sc}
    \pgfsetmiterjoin
    \pgfsetbuttcap
    \pgfpathqmoveto{162.1212bp}{60.9848bp}
    \pgfpathqlineto{162.1212bp}{60.2273bp}
    \pgfusepathqstroke
  \end{pgfscope}
  \begin{pgfscope}
    \pgfsetlinewidth{0.5678bp}
    \definecolor{sc}{rgb}{0.0000,0.0000,0.0000}
    \pgfsetstrokecolor{sc}
    \pgfsetmiterjoin
    \pgfsetbuttcap
    \pgfpathqmoveto{154.5455bp}{60.9848bp}
    \pgfpathqlineto{154.5455bp}{60.2273bp}
    \pgfusepathqstroke
  \end{pgfscope}
  \begin{pgfscope}
    \pgfsetlinewidth{0.5678bp}
    \definecolor{sc}{rgb}{0.0000,0.0000,0.0000}
    \pgfsetstrokecolor{sc}
    \pgfsetmiterjoin
    \pgfsetbuttcap
    \pgfpathqmoveto{146.9697bp}{60.9848bp}
    \pgfpathqlineto{146.9697bp}{60.2273bp}
    \pgfusepathqstroke
  \end{pgfscope}
  \begin{pgfscope}
    \pgfsetlinewidth{0.5678bp}
    \definecolor{sc}{rgb}{0.0000,0.0000,0.0000}
    \pgfsetstrokecolor{sc}
    \pgfsetmiterjoin
    \pgfsetbuttcap
    \pgfpathqmoveto{139.3939bp}{60.9848bp}
    \pgfpathqlineto{139.3939bp}{60.2273bp}
    \pgfusepathqstroke
  \end{pgfscope}
  \begin{pgfscope}
    \pgfsetlinewidth{0.5678bp}
    \definecolor{sc}{rgb}{0.0000,0.0000,0.0000}
    \pgfsetstrokecolor{sc}
    \pgfsetmiterjoin
    \pgfsetbuttcap
    \pgfpathqmoveto{131.8182bp}{60.9848bp}
    \pgfpathqlineto{131.8182bp}{60.2273bp}
    \pgfusepathqstroke
  \end{pgfscope}
  \begin{pgfscope}
    \pgfsetlinewidth{0.5678bp}
    \definecolor{sc}{rgb}{0.0000,0.0000,0.0000}
    \pgfsetstrokecolor{sc}
    \pgfsetmiterjoin
    \pgfsetbuttcap
    \pgfpathqmoveto{124.2424bp}{60.9848bp}
    \pgfpathqlineto{124.2424bp}{60.2273bp}
    \pgfusepathqstroke
  \end{pgfscope}
  \begin{pgfscope}
    \pgfsetlinewidth{0.5678bp}
    \definecolor{sc}{rgb}{0.0000,0.0000,0.0000}
    \pgfsetstrokecolor{sc}
    \pgfsetmiterjoin
    \pgfsetbuttcap
    \pgfpathqmoveto{116.6667bp}{60.9848bp}
    \pgfpathqlineto{116.6667bp}{60.2273bp}
    \pgfusepathqstroke
  \end{pgfscope}
  \begin{pgfscope}
    \pgfsetlinewidth{0.5678bp}
    \definecolor{sc}{rgb}{0.0000,0.0000,0.0000}
    \pgfsetstrokecolor{sc}
    \pgfsetmiterjoin
    \pgfsetbuttcap
    \pgfpathqmoveto{109.0909bp}{60.9848bp}
    \pgfpathqlineto{109.0909bp}{60.2273bp}
    \pgfusepathqstroke
  \end{pgfscope}
  \begin{pgfscope}
    \pgfsetlinewidth{0.5678bp}
    \definecolor{sc}{rgb}{0.0000,0.0000,0.0000}
    \pgfsetstrokecolor{sc}
    \pgfsetmiterjoin
    \pgfsetbuttcap
    \pgfpathqmoveto{101.5152bp}{60.9848bp}
    \pgfpathqlineto{101.5152bp}{60.2273bp}
    \pgfusepathqstroke
  \end{pgfscope}
  \begin{pgfscope}
    \pgfsetlinewidth{0.5678bp}
    \definecolor{sc}{rgb}{0.0000,0.0000,0.0000}
    \pgfsetstrokecolor{sc}
    \pgfsetmiterjoin
    \pgfsetbuttcap
    \pgfpathqmoveto{93.9394bp}{60.9848bp}
    \pgfpathqlineto{93.9394bp}{60.2273bp}
    \pgfusepathqstroke
  \end{pgfscope}
  \begin{pgfscope}
    \pgfsetlinewidth{0.5678bp}
    \definecolor{sc}{rgb}{0.0000,0.0000,0.0000}
    \pgfsetstrokecolor{sc}
    \pgfsetmiterjoin
    \pgfsetbuttcap
    \pgfpathqmoveto{86.3636bp}{60.9848bp}
    \pgfpathqlineto{86.3636bp}{60.2273bp}
    \pgfusepathqstroke
  \end{pgfscope}
  \begin{pgfscope}
    \pgfsetlinewidth{0.5678bp}
    \definecolor{sc}{rgb}{0.0000,0.0000,0.0000}
    \pgfsetstrokecolor{sc}
    \pgfsetmiterjoin
    \pgfsetbuttcap
    \pgfpathqmoveto{78.7879bp}{60.9848bp}
    \pgfpathqlineto{78.7879bp}{60.2273bp}
    \pgfusepathqstroke
  \end{pgfscope}
  \begin{pgfscope}
    \pgfsetlinewidth{0.5678bp}
    \definecolor{sc}{rgb}{0.0000,0.0000,0.0000}
    \pgfsetstrokecolor{sc}
    \pgfsetmiterjoin
    \pgfsetbuttcap
    \pgfpathqmoveto{71.2121bp}{60.9848bp}
    \pgfpathqlineto{71.2121bp}{60.2273bp}
    \pgfusepathqstroke
  \end{pgfscope}
  \begin{pgfscope}
    \pgfsetlinewidth{0.5678bp}
    \definecolor{sc}{rgb}{0.0000,0.0000,0.0000}
    \pgfsetstrokecolor{sc}
    \pgfsetmiterjoin
    \pgfsetbuttcap
    \pgfpathqmoveto{63.6364bp}{60.9848bp}
    \pgfpathqlineto{63.6364bp}{60.2273bp}
    \pgfusepathqstroke
  \end{pgfscope}
  \begin{pgfscope}
    \pgfsetlinewidth{0.5678bp}
    \definecolor{sc}{rgb}{0.0000,0.0000,0.0000}
    \pgfsetstrokecolor{sc}
    \pgfsetmiterjoin
    \pgfsetbuttcap
    \pgfpathqmoveto{56.0606bp}{60.9848bp}
    \pgfpathqlineto{56.0606bp}{60.2273bp}
    \pgfusepathqstroke
  \end{pgfscope}
  \begin{pgfscope}
    \pgfsetlinewidth{0.5678bp}
    \definecolor{sc}{rgb}{0.0000,0.0000,0.0000}
    \pgfsetstrokecolor{sc}
    \pgfsetmiterjoin
    \pgfsetbuttcap
    \pgfpathqmoveto{48.4848bp}{60.9848bp}
    \pgfpathqlineto{48.4848bp}{60.2273bp}
    \pgfusepathqstroke
  \end{pgfscope}
  \begin{pgfscope}
    \pgfsetlinewidth{0.5678bp}
    \definecolor{sc}{rgb}{0.0000,0.0000,0.0000}
    \pgfsetstrokecolor{sc}
    \pgfsetmiterjoin
    \pgfsetbuttcap
    \pgfpathqmoveto{40.9091bp}{60.9848bp}
    \pgfpathqlineto{40.9091bp}{60.2273bp}
    \pgfusepathqstroke
  \end{pgfscope}
  \begin{pgfscope}
    \pgfsetlinewidth{0.5678bp}
    \definecolor{sc}{rgb}{0.0000,0.0000,0.0000}
    \pgfsetstrokecolor{sc}
    \pgfsetmiterjoin
    \pgfsetbuttcap
    \pgfpathqmoveto{33.3333bp}{60.9848bp}
    \pgfpathqlineto{33.3333bp}{60.2273bp}
    \pgfusepathqstroke
  \end{pgfscope}
  \begin{pgfscope}
    \pgfsetlinewidth{0.5678bp}
    \definecolor{sc}{rgb}{0.0000,0.0000,0.0000}
    \pgfsetstrokecolor{sc}
    \pgfsetmiterjoin
    \pgfsetbuttcap
    \pgfpathqmoveto{25.7576bp}{60.9848bp}
    \pgfpathqlineto{25.7576bp}{60.2273bp}
    \pgfusepathqstroke
  \end{pgfscope}
  \begin{pgfscope}
    \pgfsetlinewidth{0.5678bp}
    \definecolor{sc}{rgb}{0.0000,0.0000,0.0000}
    \pgfsetstrokecolor{sc}
    \pgfsetmiterjoin
    \pgfsetbuttcap
    \pgfpathqmoveto{18.1818bp}{60.9848bp}
    \pgfpathqlineto{18.1818bp}{60.2273bp}
    \pgfusepathqstroke
  \end{pgfscope}
  \begin{pgfscope}
    \pgfsetlinewidth{0.5678bp}
    \definecolor{sc}{rgb}{0.0000,0.0000,0.0000}
    \pgfsetstrokecolor{sc}
    \pgfsetmiterjoin
    \pgfsetbuttcap
    \pgfpathqmoveto{10.6061bp}{60.9848bp}
    \pgfpathqlineto{10.6061bp}{60.2273bp}
    \pgfusepathqstroke
  \end{pgfscope}
  \begin{pgfscope}
    \definecolor{fc}{rgb}{0.0000,0.0000,0.0000}
    \pgfsetfillcolor{fc}
    \pgftransformshift{\pgfqpoint{0.0000bp}{109.0152bp}}
    \pgftransformscale{0.1894}
    \pgftext[base,left]{$\mathbb{F}_A$}
  \end{pgfscope}
  \begin{pgfscope}
    \pgfsetlinewidth{0.5678bp}
    \definecolor{sc}{rgb}{0.0000,0.0000,0.0000}
    \pgfsetstrokecolor{sc}
    \pgfsetmiterjoin
    \pgfsetbuttcap
    \pgfpathqmoveto{3.0303bp}{109.4697bp}
    \pgfpathqlineto{2.7273bp}{109.4697bp}
    \pgfusepathqstroke
  \end{pgfscope}
  \begin{pgfscope}
    \pgfsetlinewidth{0.5678bp}
    \definecolor{sc}{rgb}{0.0000,0.0000,0.0000}
    \pgfsetstrokecolor{sc}
    \pgfsetmiterjoin
    \pgfsetbuttcap
    \pgfpathqmoveto{3.0303bp}{60.9848bp}
    \pgfpathqlineto{3.0303bp}{150.3788bp}
    \pgfusepathqstroke
  \end{pgfscope}
  \begin{pgfscope}
    \pgfsetlinewidth{0.5678bp}
    \definecolor{sc}{rgb}{0.0000,0.0000,0.0000}
    \pgfsetstrokecolor{sc}
    \pgfsetmiterjoin
    \pgfsetbuttcap
    \pgfpathqmoveto{3.0303bp}{60.9848bp}
    \pgfpathqlineto{200.0000bp}{60.9848bp}
    \pgfusepathqstroke
  \end{pgfscope}
\end{pgfpicture}

        \label{fig:ex:ca:hgma:ex:move-h}
    \caption{push-h-goal preconditions}\label{fig:ex:ca:hgma:ex:disconnected}
\end{figure}

\begin{figure}
    \centering
    \begin{pgfpicture}
  \pgfpathrectangle{\pgfpointorigin}{\pgfqpoint{200.0000bp}{200.0000bp}}
  \pgfusepath{use as bounding box}
  \begin{pgfscope}
    \definecolor{fc}{rgb}{0.0000,0.0000,0.0000}
    \pgfsetfillcolor{fc}
    \pgftransformshift{\pgfqpoint{3.4000bp}{48.8500bp}}
    \pgftransformscale{0.1250}
    \pgftext[base,left]{candidates}
  \end{pgfscope}
  \begin{pgfscope}
    \definecolor{fc}{rgb}{0.0000,0.0000,0.0000}
    \pgfsetfillcolor{fc}
    \pgfsetlinewidth{0.5727bp}
    \definecolor{sc}{rgb}{0.0000,0.0000,0.0000}
    \pgfsetstrokecolor{sc}
    \pgfsetmiterjoin
    \pgfsetbuttcap
    \pgfpathqmoveto{2.4000bp}{49.1500bp}
    \pgfpathqcurveto{2.4000bp}{49.3709bp}{2.2209bp}{49.5500bp}{2.0000bp}{49.5500bp}
    \pgfpathqcurveto{1.7791bp}{49.5500bp}{1.6000bp}{49.3709bp}{1.6000bp}{49.1500bp}
    \pgfpathqcurveto{1.6000bp}{48.9291bp}{1.7791bp}{48.7500bp}{2.0000bp}{48.7500bp}
    \pgfpathqcurveto{2.2209bp}{48.7500bp}{2.4000bp}{48.9291bp}{2.4000bp}{49.1500bp}
    \pgfpathclose
    \pgfusepathqfillstroke
  \end{pgfscope}
  \begin{pgfscope}
    \definecolor{fc}{rgb}{0.0000,0.0000,0.0000}
    \pgfsetfillcolor{fc}
    \pgftransformshift{\pgfqpoint{3.4000bp}{50.1500bp}}
    \pgftransformscale{0.1250}
    \pgftext[base,left]{negative unproven}
  \end{pgfscope}
  \begin{pgfscope}
    \definecolor{fc}{rgb}{1.0000,1.0000,0.0000}
    \pgfsetfillcolor{fc}
    \pgfsetlinewidth{0.5727bp}
    \definecolor{sc}{rgb}{1.0000,1.0000,0.0000}
    \pgfsetstrokecolor{sc}
    \pgfsetmiterjoin
    \pgfsetbuttcap
    \pgfpathqmoveto{2.4000bp}{50.4500bp}
    \pgfpathqcurveto{2.4000bp}{50.6709bp}{2.2209bp}{50.8500bp}{2.0000bp}{50.8500bp}
    \pgfpathqcurveto{1.7791bp}{50.8500bp}{1.6000bp}{50.6709bp}{1.6000bp}{50.4500bp}
    \pgfpathqcurveto{1.6000bp}{50.2291bp}{1.7791bp}{50.0500bp}{2.0000bp}{50.0500bp}
    \pgfpathqcurveto{2.2209bp}{50.0500bp}{2.4000bp}{50.2291bp}{2.4000bp}{50.4500bp}
    \pgfpathclose
    \pgfusepathqfillstroke
  \end{pgfscope}
  \begin{pgfscope}
    \definecolor{fc}{rgb}{0.0000,0.0000,0.0000}
    \pgfsetfillcolor{fc}
    \pgftransformshift{\pgfqpoint{3.4000bp}{51.4500bp}}
    \pgftransformscale{0.1250}
    \pgftext[base,left]{negative proven}
  \end{pgfscope}
  \begin{pgfscope}
    \definecolor{fc}{rgb}{0.0000,0.5020,0.0000}
    \pgfsetfillcolor{fc}
    \pgfsetlinewidth{0.5727bp}
    \definecolor{sc}{rgb}{0.0000,0.5020,0.0000}
    \pgfsetstrokecolor{sc}
    \pgfsetmiterjoin
    \pgfsetbuttcap
    \pgfpathqmoveto{2.4000bp}{51.7500bp}
    \pgfpathqcurveto{2.4000bp}{51.9709bp}{2.2209bp}{52.1500bp}{2.0000bp}{52.1500bp}
    \pgfpathqcurveto{1.7791bp}{52.1500bp}{1.6000bp}{51.9709bp}{1.6000bp}{51.7500bp}
    \pgfpathqcurveto{1.6000bp}{51.5291bp}{1.7791bp}{51.3500bp}{2.0000bp}{51.3500bp}
    \pgfpathqcurveto{2.2209bp}{51.3500bp}{2.4000bp}{51.5291bp}{2.4000bp}{51.7500bp}
    \pgfpathclose
    \pgfusepathqfillstroke
  \end{pgfscope}
  \begin{pgfscope}
    \definecolor{fc}{rgb}{0.0000,0.0000,0.0000}
    \pgfsetfillcolor{fc}
    \pgftransformshift{\pgfqpoint{3.4000bp}{52.7500bp}}
    \pgftransformscale{0.1250}
    \pgftext[base,left]{positive unproven}
  \end{pgfscope}
  \begin{pgfscope}
    \definecolor{fc}{rgb}{1.0000,0.0000,0.0000}
    \pgfsetfillcolor{fc}
    \pgfsetlinewidth{0.5727bp}
    \definecolor{sc}{rgb}{1.0000,0.0000,0.0000}
    \pgfsetstrokecolor{sc}
    \pgfsetmiterjoin
    \pgfsetbuttcap
    \pgfpathqmoveto{2.4000bp}{53.0500bp}
    \pgfpathqcurveto{2.4000bp}{53.2709bp}{2.2209bp}{53.4500bp}{2.0000bp}{53.4500bp}
    \pgfpathqcurveto{1.7791bp}{53.4500bp}{1.6000bp}{53.2709bp}{1.6000bp}{53.0500bp}
    \pgfpathqcurveto{1.6000bp}{52.8291bp}{1.7791bp}{52.6500bp}{2.0000bp}{52.6500bp}
    \pgfpathqcurveto{2.2209bp}{52.6500bp}{2.4000bp}{52.8291bp}{2.4000bp}{53.0500bp}
    \pgfpathclose
    \pgfusepathqfillstroke
  \end{pgfscope}
  \begin{pgfscope}
    \definecolor{fc}{rgb}{0.0000,0.0000,0.0000}
    \pgfsetfillcolor{fc}
    \pgftransformshift{\pgfqpoint{3.4000bp}{54.0500bp}}
    \pgftransformscale{0.1250}
    \pgftext[base,left]{positive proven}
  \end{pgfscope}
  \begin{pgfscope}
    \definecolor{fc}{rgb}{0.0000,0.0000,1.0000}
    \pgfsetfillcolor{fc}
    \pgfsetlinewidth{0.5727bp}
    \definecolor{sc}{rgb}{0.0000,0.0000,1.0000}
    \pgfsetstrokecolor{sc}
    \pgfsetmiterjoin
    \pgfsetbuttcap
    \pgfpathqmoveto{2.4000bp}{54.3500bp}
    \pgfpathqcurveto{2.4000bp}{54.5709bp}{2.2209bp}{54.7500bp}{2.0000bp}{54.7500bp}
    \pgfpathqcurveto{1.7791bp}{54.7500bp}{1.6000bp}{54.5709bp}{1.6000bp}{54.3500bp}
    \pgfpathqcurveto{1.6000bp}{54.1291bp}{1.7791bp}{53.9500bp}{2.0000bp}{53.9500bp}
    \pgfpathqcurveto{2.2209bp}{53.9500bp}{2.4000bp}{54.1291bp}{2.4000bp}{54.3500bp}
    \pgfpathclose
    \pgfusepathqfillstroke
  \end{pgfscope}
  \begin{pgfscope}
    \pgfsetlinewidth{0.5727bp}
    \definecolor{sc}{rgb}{0.0000,0.0000,0.0000}
    \pgfsetstrokecolor{sc}
    \pgfsetmiterjoin
    \pgfsetbuttcap
    \pgfpathqmoveto{2.0000bp}{56.7500bp}
    \pgfpathqlineto{3.0000bp}{57.2500bp}
    \pgfpathqlineto{4.0000bp}{57.7500bp}
    \pgfpathqlineto{5.0000bp}{58.2500bp}
    \pgfpathqlineto{6.0000bp}{58.7500bp}
    \pgfpathqlineto{7.0000bp}{59.2500bp}
    \pgfpathqlineto{8.0000bp}{59.7500bp}
    \pgfpathqlineto{9.0000bp}{60.2500bp}
    \pgfpathqlineto{10.0000bp}{60.7500bp}
    \pgfpathqlineto{11.0000bp}{61.2500bp}
    \pgfpathqlineto{12.0000bp}{61.7500bp}
    \pgfpathqlineto{13.0000bp}{62.2500bp}
    \pgfpathqlineto{14.0000bp}{62.7500bp}
    \pgfpathqlineto{15.0000bp}{63.2500bp}
    \pgfpathqlineto{16.0000bp}{63.7500bp}
    \pgfpathqlineto{17.0000bp}{64.2500bp}
    \pgfpathqlineto{18.0000bp}{64.7500bp}
    \pgfpathqlineto{19.0000bp}{65.2500bp}
    \pgfpathqlineto{20.0000bp}{65.7500bp}
    \pgfpathqlineto{21.0000bp}{66.2500bp}
    \pgfpathqlineto{22.0000bp}{66.7500bp}
    \pgfpathqlineto{23.0000bp}{67.2500bp}
    \pgfpathqlineto{24.0000bp}{67.7500bp}
    \pgfpathqlineto{25.0000bp}{68.2500bp}
    \pgfpathqlineto{26.0000bp}{68.7500bp}
    \pgfpathqlineto{27.0000bp}{69.2500bp}
    \pgfpathqlineto{28.0000bp}{69.7500bp}
    \pgfpathqlineto{29.0000bp}{70.2500bp}
    \pgfpathqlineto{30.0000bp}{70.7500bp}
    \pgfpathqlineto{31.0000bp}{71.2500bp}
    \pgfpathqlineto{32.0000bp}{71.7500bp}
    \pgfpathqlineto{33.0000bp}{72.2500bp}
    \pgfpathqlineto{34.0000bp}{72.7500bp}
    \pgfpathqlineto{35.0000bp}{73.2500bp}
    \pgfpathqlineto{36.0000bp}{73.7500bp}
    \pgfpathqlineto{37.0000bp}{74.2500bp}
    \pgfpathqlineto{38.0000bp}{74.7500bp}
    \pgfpathqlineto{39.0000bp}{75.2500bp}
    \pgfpathqlineto{40.0000bp}{75.7500bp}
    \pgfpathqlineto{41.0000bp}{76.2500bp}
    \pgfpathqlineto{42.0000bp}{76.7500bp}
    \pgfpathqlineto{43.0000bp}{77.2500bp}
    \pgfpathqlineto{44.0000bp}{77.7500bp}
    \pgfpathqlineto{45.0000bp}{78.2500bp}
    \pgfpathqlineto{46.0000bp}{78.7500bp}
    \pgfpathqlineto{47.0000bp}{79.2500bp}
    \pgfpathqlineto{48.0000bp}{79.7500bp}
    \pgfpathqlineto{49.0000bp}{80.2500bp}
    \pgfpathqlineto{50.0000bp}{80.7500bp}
    \pgfpathqlineto{51.0000bp}{81.2500bp}
    \pgfpathqlineto{52.0000bp}{81.7500bp}
    \pgfpathqlineto{53.0000bp}{82.2500bp}
    \pgfpathqlineto{54.0000bp}{82.7500bp}
    \pgfpathqlineto{55.0000bp}{83.2500bp}
    \pgfpathqlineto{56.0000bp}{83.7500bp}
    \pgfpathqlineto{57.0000bp}{84.2500bp}
    \pgfpathqlineto{58.0000bp}{84.7500bp}
    \pgfpathqlineto{59.0000bp}{85.2500bp}
    \pgfpathqlineto{60.0000bp}{85.7500bp}
    \pgfpathqlineto{61.0000bp}{86.2500bp}
    \pgfpathqlineto{62.0000bp}{86.7500bp}
    \pgfpathqlineto{63.0000bp}{87.2500bp}
    \pgfpathqlineto{64.0000bp}{87.7500bp}
    \pgfpathqlineto{65.0000bp}{88.2500bp}
    \pgfpathqlineto{66.0000bp}{88.7500bp}
    \pgfpathqlineto{67.0000bp}{89.2500bp}
    \pgfpathqlineto{68.0000bp}{89.7500bp}
    \pgfpathqlineto{69.0000bp}{90.2500bp}
    \pgfpathqlineto{70.0000bp}{90.7500bp}
    \pgfpathqlineto{71.0000bp}{91.2500bp}
    \pgfpathqlineto{72.0000bp}{91.7500bp}
    \pgfpathqlineto{73.0000bp}{92.2500bp}
    \pgfpathqlineto{74.0000bp}{92.7500bp}
    \pgfpathqlineto{75.0000bp}{93.2500bp}
    \pgfpathqlineto{76.0000bp}{93.7500bp}
    \pgfpathqlineto{77.0000bp}{94.2500bp}
    \pgfpathqlineto{78.0000bp}{94.7500bp}
    \pgfpathqlineto{79.0000bp}{95.2500bp}
    \pgfpathqlineto{80.0000bp}{95.7500bp}
    \pgfpathqlineto{81.0000bp}{96.2500bp}
    \pgfpathqlineto{82.0000bp}{96.7500bp}
    \pgfpathqlineto{83.0000bp}{97.2500bp}
    \pgfpathqlineto{84.0000bp}{97.7500bp}
    \pgfpathqlineto{85.0000bp}{98.2500bp}
    \pgfpathqlineto{86.0000bp}{98.7500bp}
    \pgfpathqlineto{87.0000bp}{99.2500bp}
    \pgfpathqlineto{88.0000bp}{99.7500bp}
    \pgfpathqlineto{89.0000bp}{100.2500bp}
    \pgfpathqlineto{90.0000bp}{100.7500bp}
    \pgfpathqlineto{91.0000bp}{101.2500bp}
    \pgfpathqlineto{92.0000bp}{101.7500bp}
    \pgfpathqlineto{93.0000bp}{102.2500bp}
    \pgfpathqlineto{94.0000bp}{102.7500bp}
    \pgfpathqlineto{95.0000bp}{103.2500bp}
    \pgfpathqlineto{96.0000bp}{103.7500bp}
    \pgfpathqlineto{97.0000bp}{104.2500bp}
    \pgfpathqlineto{98.0000bp}{104.7500bp}
    \pgfpathqlineto{99.0000bp}{105.2500bp}
    \pgfpathqlineto{100.0000bp}{105.7500bp}
    \pgfpathqlineto{101.0000bp}{106.2500bp}
    \pgfpathqlineto{102.0000bp}{106.7500bp}
    \pgfpathqlineto{103.0000bp}{107.2500bp}
    \pgfpathqlineto{104.0000bp}{107.7500bp}
    \pgfpathqlineto{105.0000bp}{108.2500bp}
    \pgfpathqlineto{106.0000bp}{108.7500bp}
    \pgfpathqlineto{107.0000bp}{109.2500bp}
    \pgfpathqlineto{108.0000bp}{109.7500bp}
    \pgfpathqlineto{109.0000bp}{110.2500bp}
    \pgfpathqlineto{110.0000bp}{110.7500bp}
    \pgfpathqlineto{111.0000bp}{111.2500bp}
    \pgfpathqlineto{112.0000bp}{111.7500bp}
    \pgfpathqlineto{113.0000bp}{112.2500bp}
    \pgfpathqlineto{114.0000bp}{112.7500bp}
    \pgfpathqlineto{115.0000bp}{113.2500bp}
    \pgfpathqlineto{116.0000bp}{113.7500bp}
    \pgfpathqlineto{117.0000bp}{114.2500bp}
    \pgfpathqlineto{118.0000bp}{114.7500bp}
    \pgfpathqlineto{119.0000bp}{115.2500bp}
    \pgfpathqlineto{120.0000bp}{115.7500bp}
    \pgfpathqlineto{121.0000bp}{116.2500bp}
    \pgfpathqlineto{122.0000bp}{116.7500bp}
    \pgfpathqlineto{123.0000bp}{117.2500bp}
    \pgfpathqlineto{124.0000bp}{117.7500bp}
    \pgfpathqlineto{125.0000bp}{118.2500bp}
    \pgfpathqlineto{126.0000bp}{118.7500bp}
    \pgfpathqlineto{127.0000bp}{119.2500bp}
    \pgfpathqlineto{128.0000bp}{119.7500bp}
    \pgfpathqlineto{129.0000bp}{120.2500bp}
    \pgfpathqlineto{130.0000bp}{120.7500bp}
    \pgfpathqlineto{131.0000bp}{121.2500bp}
    \pgfpathqlineto{132.0000bp}{121.7500bp}
    \pgfpathqlineto{133.0000bp}{122.2500bp}
    \pgfpathqlineto{134.0000bp}{122.7500bp}
    \pgfpathqlineto{135.0000bp}{123.2500bp}
    \pgfpathqlineto{136.0000bp}{123.7500bp}
    \pgfpathqlineto{137.0000bp}{124.2500bp}
    \pgfpathqlineto{138.0000bp}{124.7500bp}
    \pgfpathqlineto{139.0000bp}{125.2500bp}
    \pgfpathqlineto{140.0000bp}{125.7500bp}
    \pgfpathqlineto{141.0000bp}{126.2500bp}
    \pgfpathqlineto{142.0000bp}{126.7500bp}
    \pgfpathqlineto{143.0000bp}{127.2500bp}
    \pgfpathqlineto{144.0000bp}{127.7500bp}
    \pgfpathqlineto{145.0000bp}{128.2500bp}
    \pgfpathqlineto{146.0000bp}{128.7500bp}
    \pgfpathqlineto{147.0000bp}{129.2500bp}
    \pgfpathqlineto{148.0000bp}{129.7500bp}
    \pgfpathqlineto{149.0000bp}{130.2500bp}
    \pgfpathqlineto{150.0000bp}{130.7500bp}
    \pgfpathqlineto{151.0000bp}{131.2500bp}
    \pgfpathqlineto{152.0000bp}{131.7500bp}
    \pgfpathqlineto{153.0000bp}{132.2500bp}
    \pgfpathqlineto{154.0000bp}{132.7500bp}
    \pgfpathqlineto{155.0000bp}{133.2500bp}
    \pgfpathqlineto{156.0000bp}{133.7500bp}
    \pgfpathqlineto{157.0000bp}{134.2500bp}
    \pgfpathqlineto{158.0000bp}{134.7500bp}
    \pgfpathqlineto{159.0000bp}{135.2500bp}
    \pgfpathqlineto{160.0000bp}{135.7500bp}
    \pgfpathqlineto{161.0000bp}{136.2500bp}
    \pgfpathqlineto{162.0000bp}{136.7500bp}
    \pgfpathqlineto{163.0000bp}{137.2500bp}
    \pgfpathqlineto{164.0000bp}{137.7500bp}
    \pgfpathqlineto{165.0000bp}{138.2500bp}
    \pgfpathqlineto{166.0000bp}{138.7500bp}
    \pgfpathqlineto{167.0000bp}{139.2500bp}
    \pgfpathqlineto{168.0000bp}{139.7500bp}
    \pgfpathqlineto{169.0000bp}{140.2500bp}
    \pgfpathqlineto{170.0000bp}{140.7500bp}
    \pgfpathqlineto{171.0000bp}{141.2500bp}
    \pgfpathqlineto{172.0000bp}{141.7500bp}
    \pgfpathqlineto{173.0000bp}{142.2500bp}
    \pgfpathqlineto{174.0000bp}{142.7500bp}
    \pgfpathqlineto{175.0000bp}{143.2500bp}
    \pgfpathqlineto{176.0000bp}{143.7500bp}
    \pgfpathqlineto{177.0000bp}{144.2500bp}
    \pgfpathqlineto{178.0000bp}{144.7500bp}
    \pgfpathqlineto{179.0000bp}{145.2500bp}
    \pgfpathqlineto{180.0000bp}{145.7500bp}
    \pgfpathqlineto{181.0000bp}{146.2500bp}
    \pgfpathqlineto{182.0000bp}{146.7500bp}
    \pgfpathqlineto{183.0000bp}{147.2500bp}
    \pgfpathqlineto{184.0000bp}{147.7500bp}
    \pgfpathqlineto{185.0000bp}{148.2500bp}
    \pgfpathqlineto{186.0000bp}{148.7500bp}
    \pgfpathqlineto{187.0000bp}{149.2500bp}
    \pgfpathqlineto{188.0000bp}{149.7500bp}
    \pgfpathqlineto{189.0000bp}{150.2500bp}
    \pgfpathqlineto{190.0000bp}{150.7500bp}
    \pgfpathqlineto{191.0000bp}{151.2500bp}
    \pgfpathqlineto{192.0000bp}{151.2500bp}
    \pgfpathqlineto{193.0000bp}{130.7500bp}
    \pgfpathqlineto{194.0000bp}{123.2500bp}
    \pgfpathqlineto{195.0000bp}{119.7500bp}
    \pgfpathqlineto{196.0000bp}{88.2500bp}
    \pgfpathqlineto{197.0000bp}{78.7500bp}
    \pgfpathqlineto{198.0000bp}{68.7500bp}
    \pgfpathqlineto{199.0000bp}{68.7500bp}
    \pgfusepathqstroke
  \end{pgfscope}
  \begin{pgfscope}
    \pgfsetlinewidth{0.5727bp}
    \definecolor{sc}{rgb}{1.0000,1.0000,0.0000}
    \pgfsetstrokecolor{sc}
    \pgfsetmiterjoin
    \pgfsetbuttcap
    \pgfpathqmoveto{2.0000bp}{88.2500bp}
    \pgfpathqlineto{3.0000bp}{88.2500bp}
    \pgfpathqlineto{4.0000bp}{88.2500bp}
    \pgfpathqlineto{5.0000bp}{88.2500bp}
    \pgfpathqlineto{6.0000bp}{88.2500bp}
    \pgfpathqlineto{7.0000bp}{88.2500bp}
    \pgfpathqlineto{8.0000bp}{88.2500bp}
    \pgfpathqlineto{9.0000bp}{88.2500bp}
    \pgfpathqlineto{10.0000bp}{88.2500bp}
    \pgfpathqlineto{11.0000bp}{88.2500bp}
    \pgfpathqlineto{12.0000bp}{88.2500bp}
    \pgfpathqlineto{13.0000bp}{88.2500bp}
    \pgfpathqlineto{14.0000bp}{88.2500bp}
    \pgfpathqlineto{15.0000bp}{88.2500bp}
    \pgfpathqlineto{16.0000bp}{88.2500bp}
    \pgfpathqlineto{17.0000bp}{88.2500bp}
    \pgfpathqlineto{18.0000bp}{88.2500bp}
    \pgfpathqlineto{19.0000bp}{88.2500bp}
    \pgfpathqlineto{20.0000bp}{88.2500bp}
    \pgfpathqlineto{21.0000bp}{88.2500bp}
    \pgfpathqlineto{22.0000bp}{88.2500bp}
    \pgfpathqlineto{23.0000bp}{88.2500bp}
    \pgfpathqlineto{24.0000bp}{88.2500bp}
    \pgfpathqlineto{25.0000bp}{88.2500bp}
    \pgfpathqlineto{26.0000bp}{88.2500bp}
    \pgfpathqlineto{27.0000bp}{88.2500bp}
    \pgfpathqlineto{28.0000bp}{88.2500bp}
    \pgfpathqlineto{29.0000bp}{88.2500bp}
    \pgfpathqlineto{30.0000bp}{88.2500bp}
    \pgfpathqlineto{31.0000bp}{88.2500bp}
    \pgfpathqlineto{32.0000bp}{88.2500bp}
    \pgfpathqlineto{33.0000bp}{88.2500bp}
    \pgfpathqlineto{34.0000bp}{88.2500bp}
    \pgfpathqlineto{35.0000bp}{88.2500bp}
    \pgfpathqlineto{36.0000bp}{88.2500bp}
    \pgfpathqlineto{37.0000bp}{88.2500bp}
    \pgfpathqlineto{38.0000bp}{88.2500bp}
    \pgfpathqlineto{39.0000bp}{88.2500bp}
    \pgfpathqlineto{40.0000bp}{88.2500bp}
    \pgfpathqlineto{41.0000bp}{88.2500bp}
    \pgfpathqlineto{42.0000bp}{88.2500bp}
    \pgfpathqlineto{43.0000bp}{88.2500bp}
    \pgfpathqlineto{44.0000bp}{88.2500bp}
    \pgfpathqlineto{45.0000bp}{88.2500bp}
    \pgfpathqlineto{46.0000bp}{88.2500bp}
    \pgfpathqlineto{47.0000bp}{88.2500bp}
    \pgfpathqlineto{48.0000bp}{88.2500bp}
    \pgfpathqlineto{49.0000bp}{88.2500bp}
    \pgfpathqlineto{50.0000bp}{88.2500bp}
    \pgfpathqlineto{51.0000bp}{88.2500bp}
    \pgfpathqlineto{52.0000bp}{88.2500bp}
    \pgfpathqlineto{53.0000bp}{88.2500bp}
    \pgfpathqlineto{54.0000bp}{88.2500bp}
    \pgfpathqlineto{55.0000bp}{88.2500bp}
    \pgfpathqlineto{56.0000bp}{88.2500bp}
    \pgfpathqlineto{57.0000bp}{88.2500bp}
    \pgfpathqlineto{58.0000bp}{88.2500bp}
    \pgfpathqlineto{59.0000bp}{88.2500bp}
    \pgfpathqlineto{60.0000bp}{88.2500bp}
    \pgfpathqlineto{61.0000bp}{88.2500bp}
    \pgfpathqlineto{62.0000bp}{88.2500bp}
    \pgfpathqlineto{63.0000bp}{88.2500bp}
    \pgfpathqlineto{64.0000bp}{88.2500bp}
    \pgfpathqlineto{65.0000bp}{88.2500bp}
    \pgfpathqlineto{66.0000bp}{88.2500bp}
    \pgfpathqlineto{67.0000bp}{88.2500bp}
    \pgfpathqlineto{68.0000bp}{88.2500bp}
    \pgfpathqlineto{69.0000bp}{88.2500bp}
    \pgfpathqlineto{70.0000bp}{88.2500bp}
    \pgfpathqlineto{71.0000bp}{88.2500bp}
    \pgfpathqlineto{72.0000bp}{88.2500bp}
    \pgfpathqlineto{73.0000bp}{88.2500bp}
    \pgfpathqlineto{74.0000bp}{88.2500bp}
    \pgfpathqlineto{75.0000bp}{88.2500bp}
    \pgfpathqlineto{76.0000bp}{88.2500bp}
    \pgfpathqlineto{77.0000bp}{88.2500bp}
    \pgfpathqlineto{78.0000bp}{88.2500bp}
    \pgfpathqlineto{79.0000bp}{88.2500bp}
    \pgfpathqlineto{80.0000bp}{88.2500bp}
    \pgfpathqlineto{81.0000bp}{88.2500bp}
    \pgfpathqlineto{82.0000bp}{88.2500bp}
    \pgfpathqlineto{83.0000bp}{88.2500bp}
    \pgfpathqlineto{84.0000bp}{88.2500bp}
    \pgfpathqlineto{85.0000bp}{88.2500bp}
    \pgfpathqlineto{86.0000bp}{88.2500bp}
    \pgfpathqlineto{87.0000bp}{88.2500bp}
    \pgfpathqlineto{88.0000bp}{88.2500bp}
    \pgfpathqlineto{89.0000bp}{88.2500bp}
    \pgfpathqlineto{90.0000bp}{88.2500bp}
    \pgfpathqlineto{91.0000bp}{88.2500bp}
    \pgfpathqlineto{92.0000bp}{88.2500bp}
    \pgfpathqlineto{93.0000bp}{88.2500bp}
    \pgfpathqlineto{94.0000bp}{88.2500bp}
    \pgfpathqlineto{95.0000bp}{88.2500bp}
    \pgfpathqlineto{96.0000bp}{88.2500bp}
    \pgfpathqlineto{97.0000bp}{88.2500bp}
    \pgfpathqlineto{98.0000bp}{88.2500bp}
    \pgfpathqlineto{99.0000bp}{88.2500bp}
    \pgfpathqlineto{100.0000bp}{88.2500bp}
    \pgfpathqlineto{101.0000bp}{88.2500bp}
    \pgfpathqlineto{102.0000bp}{88.2500bp}
    \pgfpathqlineto{103.0000bp}{88.2500bp}
    \pgfpathqlineto{104.0000bp}{88.2500bp}
    \pgfpathqlineto{105.0000bp}{88.2500bp}
    \pgfpathqlineto{106.0000bp}{88.2500bp}
    \pgfpathqlineto{107.0000bp}{88.2500bp}
    \pgfpathqlineto{108.0000bp}{88.2500bp}
    \pgfpathqlineto{109.0000bp}{88.2500bp}
    \pgfpathqlineto{110.0000bp}{88.2500bp}
    \pgfpathqlineto{111.0000bp}{88.2500bp}
    \pgfpathqlineto{112.0000bp}{88.2500bp}
    \pgfpathqlineto{113.0000bp}{88.2500bp}
    \pgfpathqlineto{114.0000bp}{88.2500bp}
    \pgfpathqlineto{115.0000bp}{88.2500bp}
    \pgfpathqlineto{116.0000bp}{88.2500bp}
    \pgfpathqlineto{117.0000bp}{88.2500bp}
    \pgfpathqlineto{118.0000bp}{88.2500bp}
    \pgfpathqlineto{119.0000bp}{88.2500bp}
    \pgfpathqlineto{120.0000bp}{88.2500bp}
    \pgfpathqlineto{121.0000bp}{88.2500bp}
    \pgfpathqlineto{122.0000bp}{88.2500bp}
    \pgfpathqlineto{123.0000bp}{88.2500bp}
    \pgfpathqlineto{124.0000bp}{88.2500bp}
    \pgfpathqlineto{125.0000bp}{88.2500bp}
    \pgfpathqlineto{126.0000bp}{88.2500bp}
    \pgfpathqlineto{127.0000bp}{88.2500bp}
    \pgfpathqlineto{128.0000bp}{88.2500bp}
    \pgfpathqlineto{129.0000bp}{88.2500bp}
    \pgfpathqlineto{130.0000bp}{88.2500bp}
    \pgfpathqlineto{131.0000bp}{88.2500bp}
    \pgfpathqlineto{132.0000bp}{88.2500bp}
    \pgfpathqlineto{133.0000bp}{88.2500bp}
    \pgfpathqlineto{134.0000bp}{88.2500bp}
    \pgfpathqlineto{135.0000bp}{88.2500bp}
    \pgfpathqlineto{136.0000bp}{88.2500bp}
    \pgfpathqlineto{137.0000bp}{88.2500bp}
    \pgfpathqlineto{138.0000bp}{88.2500bp}
    \pgfpathqlineto{139.0000bp}{88.2500bp}
    \pgfpathqlineto{140.0000bp}{88.2500bp}
    \pgfpathqlineto{141.0000bp}{88.2500bp}
    \pgfpathqlineto{142.0000bp}{88.2500bp}
    \pgfpathqlineto{143.0000bp}{88.2500bp}
    \pgfpathqlineto{144.0000bp}{88.2500bp}
    \pgfpathqlineto{145.0000bp}{88.2500bp}
    \pgfpathqlineto{146.0000bp}{88.2500bp}
    \pgfpathqlineto{147.0000bp}{88.2500bp}
    \pgfpathqlineto{148.0000bp}{88.2500bp}
    \pgfpathqlineto{149.0000bp}{88.2500bp}
    \pgfpathqlineto{150.0000bp}{88.2500bp}
    \pgfpathqlineto{151.0000bp}{88.2500bp}
    \pgfpathqlineto{152.0000bp}{88.2500bp}
    \pgfpathqlineto{153.0000bp}{88.2500bp}
    \pgfpathqlineto{154.0000bp}{88.2500bp}
    \pgfpathqlineto{155.0000bp}{88.2500bp}
    \pgfpathqlineto{156.0000bp}{88.2500bp}
    \pgfpathqlineto{157.0000bp}{88.2500bp}
    \pgfpathqlineto{158.0000bp}{88.2500bp}
    \pgfpathqlineto{159.0000bp}{88.2500bp}
    \pgfpathqlineto{160.0000bp}{88.2500bp}
    \pgfpathqlineto{161.0000bp}{88.2500bp}
    \pgfpathqlineto{162.0000bp}{88.2500bp}
    \pgfpathqlineto{163.0000bp}{88.2500bp}
    \pgfpathqlineto{164.0000bp}{88.2500bp}
    \pgfpathqlineto{165.0000bp}{88.2500bp}
    \pgfpathqlineto{166.0000bp}{88.2500bp}
    \pgfpathqlineto{167.0000bp}{88.2500bp}
    \pgfpathqlineto{168.0000bp}{88.2500bp}
    \pgfpathqlineto{169.0000bp}{88.2500bp}
    \pgfpathqlineto{170.0000bp}{88.2500bp}
    \pgfpathqlineto{171.0000bp}{88.2500bp}
    \pgfpathqlineto{172.0000bp}{88.2500bp}
    \pgfpathqlineto{173.0000bp}{88.2500bp}
    \pgfpathqlineto{174.0000bp}{88.2500bp}
    \pgfpathqlineto{175.0000bp}{88.2500bp}
    \pgfpathqlineto{176.0000bp}{88.2500bp}
    \pgfpathqlineto{177.0000bp}{88.2500bp}
    \pgfpathqlineto{178.0000bp}{88.2500bp}
    \pgfpathqlineto{179.0000bp}{88.2500bp}
    \pgfpathqlineto{180.0000bp}{88.2500bp}
    \pgfpathqlineto{181.0000bp}{88.2500bp}
    \pgfpathqlineto{182.0000bp}{88.2500bp}
    \pgfpathqlineto{183.0000bp}{88.2500bp}
    \pgfpathqlineto{184.0000bp}{88.2500bp}
    \pgfpathqlineto{185.0000bp}{88.2500bp}
    \pgfpathqlineto{186.0000bp}{88.2500bp}
    \pgfpathqlineto{187.0000bp}{88.2500bp}
    \pgfpathqlineto{188.0000bp}{88.2500bp}
    \pgfpathqlineto{189.0000bp}{88.2500bp}
    \pgfpathqlineto{190.0000bp}{88.2500bp}
    \pgfpathqlineto{191.0000bp}{88.2500bp}
    \pgfpathqlineto{192.0000bp}{84.2500bp}
    \pgfpathqlineto{193.0000bp}{84.2500bp}
    \pgfpathqlineto{194.0000bp}{84.2500bp}
    \pgfpathqlineto{195.0000bp}{84.2500bp}
    \pgfpathqlineto{196.0000bp}{84.2500bp}
    \pgfpathqlineto{197.0000bp}{84.2500bp}
    \pgfpathqlineto{198.0000bp}{84.2500bp}
    \pgfpathqlineto{199.0000bp}{84.2500bp}
    \pgfusepathqstroke
  \end{pgfscope}
  \begin{pgfscope}
    \pgfsetlinewidth{0.5727bp}
    \definecolor{sc}{rgb}{0.0000,0.5020,0.0000}
    \pgfsetstrokecolor{sc}
    \pgfsetmiterjoin
    \pgfsetbuttcap
    \pgfpathqmoveto{2.0000bp}{56.2500bp}
    \pgfpathqlineto{3.0000bp}{56.2500bp}
    \pgfpathqlineto{4.0000bp}{56.2500bp}
    \pgfpathqlineto{5.0000bp}{56.2500bp}
    \pgfpathqlineto{6.0000bp}{56.2500bp}
    \pgfpathqlineto{7.0000bp}{56.2500bp}
    \pgfpathqlineto{8.0000bp}{56.2500bp}
    \pgfpathqlineto{9.0000bp}{56.2500bp}
    \pgfpathqlineto{10.0000bp}{56.2500bp}
    \pgfpathqlineto{11.0000bp}{56.2500bp}
    \pgfpathqlineto{12.0000bp}{56.2500bp}
    \pgfpathqlineto{13.0000bp}{56.2500bp}
    \pgfpathqlineto{14.0000bp}{56.2500bp}
    \pgfpathqlineto{15.0000bp}{56.2500bp}
    \pgfpathqlineto{16.0000bp}{56.2500bp}
    \pgfpathqlineto{17.0000bp}{56.2500bp}
    \pgfpathqlineto{18.0000bp}{56.2500bp}
    \pgfpathqlineto{19.0000bp}{56.2500bp}
    \pgfpathqlineto{20.0000bp}{56.2500bp}
    \pgfpathqlineto{21.0000bp}{56.2500bp}
    \pgfpathqlineto{22.0000bp}{56.2500bp}
    \pgfpathqlineto{23.0000bp}{56.2500bp}
    \pgfpathqlineto{24.0000bp}{56.2500bp}
    \pgfpathqlineto{25.0000bp}{56.2500bp}
    \pgfpathqlineto{26.0000bp}{56.2500bp}
    \pgfpathqlineto{27.0000bp}{56.2500bp}
    \pgfpathqlineto{28.0000bp}{56.2500bp}
    \pgfpathqlineto{29.0000bp}{56.2500bp}
    \pgfpathqlineto{30.0000bp}{56.2500bp}
    \pgfpathqlineto{31.0000bp}{56.2500bp}
    \pgfpathqlineto{32.0000bp}{56.2500bp}
    \pgfpathqlineto{33.0000bp}{56.2500bp}
    \pgfpathqlineto{34.0000bp}{56.2500bp}
    \pgfpathqlineto{35.0000bp}{56.2500bp}
    \pgfpathqlineto{36.0000bp}{56.2500bp}
    \pgfpathqlineto{37.0000bp}{56.2500bp}
    \pgfpathqlineto{38.0000bp}{56.2500bp}
    \pgfpathqlineto{39.0000bp}{56.2500bp}
    \pgfpathqlineto{40.0000bp}{56.2500bp}
    \pgfpathqlineto{41.0000bp}{56.2500bp}
    \pgfpathqlineto{42.0000bp}{56.2500bp}
    \pgfpathqlineto{43.0000bp}{56.2500bp}
    \pgfpathqlineto{44.0000bp}{56.2500bp}
    \pgfpathqlineto{45.0000bp}{56.2500bp}
    \pgfpathqlineto{46.0000bp}{56.2500bp}
    \pgfpathqlineto{47.0000bp}{56.2500bp}
    \pgfpathqlineto{48.0000bp}{56.2500bp}
    \pgfpathqlineto{49.0000bp}{56.2500bp}
    \pgfpathqlineto{50.0000bp}{56.2500bp}
    \pgfpathqlineto{51.0000bp}{56.2500bp}
    \pgfpathqlineto{52.0000bp}{56.2500bp}
    \pgfpathqlineto{53.0000bp}{56.2500bp}
    \pgfpathqlineto{54.0000bp}{56.2500bp}
    \pgfpathqlineto{55.0000bp}{56.2500bp}
    \pgfpathqlineto{56.0000bp}{56.2500bp}
    \pgfpathqlineto{57.0000bp}{56.2500bp}
    \pgfpathqlineto{58.0000bp}{56.2500bp}
    \pgfpathqlineto{59.0000bp}{56.2500bp}
    \pgfpathqlineto{60.0000bp}{56.2500bp}
    \pgfpathqlineto{61.0000bp}{56.2500bp}
    \pgfpathqlineto{62.0000bp}{56.2500bp}
    \pgfpathqlineto{63.0000bp}{56.2500bp}
    \pgfpathqlineto{64.0000bp}{56.2500bp}
    \pgfpathqlineto{65.0000bp}{56.2500bp}
    \pgfpathqlineto{66.0000bp}{56.2500bp}
    \pgfpathqlineto{67.0000bp}{56.2500bp}
    \pgfpathqlineto{68.0000bp}{56.2500bp}
    \pgfpathqlineto{69.0000bp}{56.2500bp}
    \pgfpathqlineto{70.0000bp}{56.2500bp}
    \pgfpathqlineto{71.0000bp}{56.2500bp}
    \pgfpathqlineto{72.0000bp}{56.2500bp}
    \pgfpathqlineto{73.0000bp}{56.2500bp}
    \pgfpathqlineto{74.0000bp}{56.2500bp}
    \pgfpathqlineto{75.0000bp}{56.2500bp}
    \pgfpathqlineto{76.0000bp}{56.2500bp}
    \pgfpathqlineto{77.0000bp}{56.2500bp}
    \pgfpathqlineto{78.0000bp}{56.2500bp}
    \pgfpathqlineto{79.0000bp}{56.2500bp}
    \pgfpathqlineto{80.0000bp}{56.2500bp}
    \pgfpathqlineto{81.0000bp}{56.2500bp}
    \pgfpathqlineto{82.0000bp}{56.2500bp}
    \pgfpathqlineto{83.0000bp}{56.2500bp}
    \pgfpathqlineto{84.0000bp}{56.2500bp}
    \pgfpathqlineto{85.0000bp}{56.2500bp}
    \pgfpathqlineto{86.0000bp}{56.2500bp}
    \pgfpathqlineto{87.0000bp}{56.2500bp}
    \pgfpathqlineto{88.0000bp}{56.2500bp}
    \pgfpathqlineto{89.0000bp}{56.2500bp}
    \pgfpathqlineto{90.0000bp}{56.2500bp}
    \pgfpathqlineto{91.0000bp}{56.2500bp}
    \pgfpathqlineto{92.0000bp}{56.2500bp}
    \pgfpathqlineto{93.0000bp}{56.2500bp}
    \pgfpathqlineto{94.0000bp}{56.2500bp}
    \pgfpathqlineto{95.0000bp}{56.2500bp}
    \pgfpathqlineto{96.0000bp}{56.2500bp}
    \pgfpathqlineto{97.0000bp}{56.2500bp}
    \pgfpathqlineto{98.0000bp}{56.2500bp}
    \pgfpathqlineto{99.0000bp}{56.2500bp}
    \pgfpathqlineto{100.0000bp}{56.2500bp}
    \pgfpathqlineto{101.0000bp}{56.2500bp}
    \pgfpathqlineto{102.0000bp}{56.2500bp}
    \pgfpathqlineto{103.0000bp}{56.2500bp}
    \pgfpathqlineto{104.0000bp}{56.2500bp}
    \pgfpathqlineto{105.0000bp}{56.2500bp}
    \pgfpathqlineto{106.0000bp}{56.2500bp}
    \pgfpathqlineto{107.0000bp}{56.2500bp}
    \pgfpathqlineto{108.0000bp}{56.2500bp}
    \pgfpathqlineto{109.0000bp}{56.2500bp}
    \pgfpathqlineto{110.0000bp}{56.2500bp}
    \pgfpathqlineto{111.0000bp}{56.2500bp}
    \pgfpathqlineto{112.0000bp}{56.2500bp}
    \pgfpathqlineto{113.0000bp}{56.2500bp}
    \pgfpathqlineto{114.0000bp}{56.2500bp}
    \pgfpathqlineto{115.0000bp}{56.2500bp}
    \pgfpathqlineto{116.0000bp}{56.2500bp}
    \pgfpathqlineto{117.0000bp}{56.2500bp}
    \pgfpathqlineto{118.0000bp}{56.2500bp}
    \pgfpathqlineto{119.0000bp}{56.2500bp}
    \pgfpathqlineto{120.0000bp}{56.2500bp}
    \pgfpathqlineto{121.0000bp}{56.2500bp}
    \pgfpathqlineto{122.0000bp}{56.2500bp}
    \pgfpathqlineto{123.0000bp}{56.2500bp}
    \pgfpathqlineto{124.0000bp}{56.2500bp}
    \pgfpathqlineto{125.0000bp}{56.2500bp}
    \pgfpathqlineto{126.0000bp}{56.2500bp}
    \pgfpathqlineto{127.0000bp}{56.2500bp}
    \pgfpathqlineto{128.0000bp}{56.2500bp}
    \pgfpathqlineto{129.0000bp}{56.2500bp}
    \pgfpathqlineto{130.0000bp}{56.2500bp}
    \pgfpathqlineto{131.0000bp}{56.2500bp}
    \pgfpathqlineto{132.0000bp}{56.2500bp}
    \pgfpathqlineto{133.0000bp}{56.2500bp}
    \pgfpathqlineto{134.0000bp}{56.2500bp}
    \pgfpathqlineto{135.0000bp}{56.2500bp}
    \pgfpathqlineto{136.0000bp}{56.2500bp}
    \pgfpathqlineto{137.0000bp}{56.2500bp}
    \pgfpathqlineto{138.0000bp}{56.2500bp}
    \pgfpathqlineto{139.0000bp}{56.2500bp}
    \pgfpathqlineto{140.0000bp}{56.2500bp}
    \pgfpathqlineto{141.0000bp}{56.2500bp}
    \pgfpathqlineto{142.0000bp}{56.2500bp}
    \pgfpathqlineto{143.0000bp}{56.2500bp}
    \pgfpathqlineto{144.0000bp}{56.2500bp}
    \pgfpathqlineto{145.0000bp}{56.2500bp}
    \pgfpathqlineto{146.0000bp}{56.2500bp}
    \pgfpathqlineto{147.0000bp}{56.2500bp}
    \pgfpathqlineto{148.0000bp}{56.2500bp}
    \pgfpathqlineto{149.0000bp}{56.2500bp}
    \pgfpathqlineto{150.0000bp}{56.2500bp}
    \pgfpathqlineto{151.0000bp}{56.2500bp}
    \pgfpathqlineto{152.0000bp}{56.2500bp}
    \pgfpathqlineto{153.0000bp}{56.2500bp}
    \pgfpathqlineto{154.0000bp}{56.2500bp}
    \pgfpathqlineto{155.0000bp}{56.2500bp}
    \pgfpathqlineto{156.0000bp}{56.2500bp}
    \pgfpathqlineto{157.0000bp}{56.2500bp}
    \pgfpathqlineto{158.0000bp}{56.2500bp}
    \pgfpathqlineto{159.0000bp}{56.2500bp}
    \pgfpathqlineto{160.0000bp}{56.2500bp}
    \pgfpathqlineto{161.0000bp}{56.2500bp}
    \pgfpathqlineto{162.0000bp}{56.2500bp}
    \pgfpathqlineto{163.0000bp}{56.2500bp}
    \pgfpathqlineto{164.0000bp}{56.2500bp}
    \pgfpathqlineto{165.0000bp}{56.2500bp}
    \pgfpathqlineto{166.0000bp}{56.2500bp}
    \pgfpathqlineto{167.0000bp}{56.2500bp}
    \pgfpathqlineto{168.0000bp}{56.2500bp}
    \pgfpathqlineto{169.0000bp}{56.2500bp}
    \pgfpathqlineto{170.0000bp}{56.2500bp}
    \pgfpathqlineto{171.0000bp}{56.2500bp}
    \pgfpathqlineto{172.0000bp}{56.2500bp}
    \pgfpathqlineto{173.0000bp}{56.2500bp}
    \pgfpathqlineto{174.0000bp}{56.2500bp}
    \pgfpathqlineto{175.0000bp}{56.2500bp}
    \pgfpathqlineto{176.0000bp}{56.2500bp}
    \pgfpathqlineto{177.0000bp}{56.2500bp}
    \pgfpathqlineto{178.0000bp}{56.2500bp}
    \pgfpathqlineto{179.0000bp}{56.2500bp}
    \pgfpathqlineto{180.0000bp}{56.2500bp}
    \pgfpathqlineto{181.0000bp}{56.2500bp}
    \pgfpathqlineto{182.0000bp}{56.2500bp}
    \pgfpathqlineto{183.0000bp}{56.2500bp}
    \pgfpathqlineto{184.0000bp}{56.2500bp}
    \pgfpathqlineto{185.0000bp}{56.2500bp}
    \pgfpathqlineto{186.0000bp}{56.2500bp}
    \pgfpathqlineto{187.0000bp}{56.2500bp}
    \pgfpathqlineto{188.0000bp}{56.2500bp}
    \pgfpathqlineto{189.0000bp}{56.2500bp}
    \pgfpathqlineto{190.0000bp}{56.2500bp}
    \pgfpathqlineto{191.0000bp}{56.2500bp}
    \pgfpathqlineto{192.0000bp}{56.2500bp}
    \pgfpathqlineto{193.0000bp}{56.2500bp}
    \pgfpathqlineto{194.0000bp}{56.2500bp}
    \pgfpathqlineto{195.0000bp}{56.2500bp}
    \pgfpathqlineto{196.0000bp}{56.2500bp}
    \pgfpathqlineto{197.0000bp}{56.2500bp}
    \pgfpathqlineto{198.0000bp}{56.2500bp}
    \pgfpathqlineto{199.0000bp}{56.2500bp}
    \pgfusepathqstroke
  \end{pgfscope}
  \begin{pgfscope}
    \pgfsetlinewidth{0.5727bp}
    \definecolor{sc}{rgb}{1.0000,0.0000,0.0000}
    \pgfsetstrokecolor{sc}
    \pgfsetmiterjoin
    \pgfsetbuttcap
    \pgfpathqmoveto{2.0000bp}{88.2500bp}
    \pgfpathqlineto{3.0000bp}{88.2500bp}
    \pgfpathqlineto{4.0000bp}{88.2500bp}
    \pgfpathqlineto{5.0000bp}{88.2500bp}
    \pgfpathqlineto{6.0000bp}{88.2500bp}
    \pgfpathqlineto{7.0000bp}{88.2500bp}
    \pgfpathqlineto{8.0000bp}{88.2500bp}
    \pgfpathqlineto{9.0000bp}{88.2500bp}
    \pgfpathqlineto{10.0000bp}{88.2500bp}
    \pgfpathqlineto{11.0000bp}{88.2500bp}
    \pgfpathqlineto{12.0000bp}{88.2500bp}
    \pgfpathqlineto{13.0000bp}{88.2500bp}
    \pgfpathqlineto{14.0000bp}{88.2500bp}
    \pgfpathqlineto{15.0000bp}{88.2500bp}
    \pgfpathqlineto{16.0000bp}{88.2500bp}
    \pgfpathqlineto{17.0000bp}{88.2500bp}
    \pgfpathqlineto{18.0000bp}{88.2500bp}
    \pgfpathqlineto{19.0000bp}{88.2500bp}
    \pgfpathqlineto{20.0000bp}{88.2500bp}
    \pgfpathqlineto{21.0000bp}{88.2500bp}
    \pgfpathqlineto{22.0000bp}{88.2500bp}
    \pgfpathqlineto{23.0000bp}{88.2500bp}
    \pgfpathqlineto{24.0000bp}{88.2500bp}
    \pgfpathqlineto{25.0000bp}{88.2500bp}
    \pgfpathqlineto{26.0000bp}{88.2500bp}
    \pgfpathqlineto{27.0000bp}{88.2500bp}
    \pgfpathqlineto{28.0000bp}{88.2500bp}
    \pgfpathqlineto{29.0000bp}{88.2500bp}
    \pgfpathqlineto{30.0000bp}{88.2500bp}
    \pgfpathqlineto{31.0000bp}{88.2500bp}
    \pgfpathqlineto{32.0000bp}{88.2500bp}
    \pgfpathqlineto{33.0000bp}{88.2500bp}
    \pgfpathqlineto{34.0000bp}{88.2500bp}
    \pgfpathqlineto{35.0000bp}{88.2500bp}
    \pgfpathqlineto{36.0000bp}{88.2500bp}
    \pgfpathqlineto{37.0000bp}{88.2500bp}
    \pgfpathqlineto{38.0000bp}{88.2500bp}
    \pgfpathqlineto{39.0000bp}{88.2500bp}
    \pgfpathqlineto{40.0000bp}{88.2500bp}
    \pgfpathqlineto{41.0000bp}{88.2500bp}
    \pgfpathqlineto{42.0000bp}{88.2500bp}
    \pgfpathqlineto{43.0000bp}{88.2500bp}
    \pgfpathqlineto{44.0000bp}{88.2500bp}
    \pgfpathqlineto{45.0000bp}{88.2500bp}
    \pgfpathqlineto{46.0000bp}{88.2500bp}
    \pgfpathqlineto{47.0000bp}{88.2500bp}
    \pgfpathqlineto{48.0000bp}{88.2500bp}
    \pgfpathqlineto{49.0000bp}{88.2500bp}
    \pgfpathqlineto{50.0000bp}{88.2500bp}
    \pgfpathqlineto{51.0000bp}{88.2500bp}
    \pgfpathqlineto{52.0000bp}{88.2500bp}
    \pgfpathqlineto{53.0000bp}{88.2500bp}
    \pgfpathqlineto{54.0000bp}{88.2500bp}
    \pgfpathqlineto{55.0000bp}{88.2500bp}
    \pgfpathqlineto{56.0000bp}{88.2500bp}
    \pgfpathqlineto{57.0000bp}{88.2500bp}
    \pgfpathqlineto{58.0000bp}{88.2500bp}
    \pgfpathqlineto{59.0000bp}{88.2500bp}
    \pgfpathqlineto{60.0000bp}{88.2500bp}
    \pgfpathqlineto{61.0000bp}{88.2500bp}
    \pgfpathqlineto{62.0000bp}{88.2500bp}
    \pgfpathqlineto{63.0000bp}{88.2500bp}
    \pgfpathqlineto{64.0000bp}{88.2500bp}
    \pgfpathqlineto{65.0000bp}{88.2500bp}
    \pgfpathqlineto{66.0000bp}{88.2500bp}
    \pgfpathqlineto{67.0000bp}{88.2500bp}
    \pgfpathqlineto{68.0000bp}{88.2500bp}
    \pgfpathqlineto{69.0000bp}{88.2500bp}
    \pgfpathqlineto{70.0000bp}{88.2500bp}
    \pgfpathqlineto{71.0000bp}{88.2500bp}
    \pgfpathqlineto{72.0000bp}{88.2500bp}
    \pgfpathqlineto{73.0000bp}{88.2500bp}
    \pgfpathqlineto{74.0000bp}{88.2500bp}
    \pgfpathqlineto{75.0000bp}{88.2500bp}
    \pgfpathqlineto{76.0000bp}{88.2500bp}
    \pgfpathqlineto{77.0000bp}{88.2500bp}
    \pgfpathqlineto{78.0000bp}{88.2500bp}
    \pgfpathqlineto{79.0000bp}{88.2500bp}
    \pgfpathqlineto{80.0000bp}{88.2500bp}
    \pgfpathqlineto{81.0000bp}{88.2500bp}
    \pgfpathqlineto{82.0000bp}{88.2500bp}
    \pgfpathqlineto{83.0000bp}{88.2500bp}
    \pgfpathqlineto{84.0000bp}{88.2500bp}
    \pgfpathqlineto{85.0000bp}{88.2500bp}
    \pgfpathqlineto{86.0000bp}{88.2500bp}
    \pgfpathqlineto{87.0000bp}{88.2500bp}
    \pgfpathqlineto{88.0000bp}{88.2500bp}
    \pgfpathqlineto{89.0000bp}{88.2500bp}
    \pgfpathqlineto{90.0000bp}{88.2500bp}
    \pgfpathqlineto{91.0000bp}{88.2500bp}
    \pgfpathqlineto{92.0000bp}{88.2500bp}
    \pgfpathqlineto{93.0000bp}{88.2500bp}
    \pgfpathqlineto{94.0000bp}{88.2500bp}
    \pgfpathqlineto{95.0000bp}{88.2500bp}
    \pgfpathqlineto{96.0000bp}{88.2500bp}
    \pgfpathqlineto{97.0000bp}{88.2500bp}
    \pgfpathqlineto{98.0000bp}{88.2500bp}
    \pgfpathqlineto{99.0000bp}{88.2500bp}
    \pgfpathqlineto{100.0000bp}{88.2500bp}
    \pgfpathqlineto{101.0000bp}{88.2500bp}
    \pgfpathqlineto{102.0000bp}{88.2500bp}
    \pgfpathqlineto{103.0000bp}{88.2500bp}
    \pgfpathqlineto{104.0000bp}{88.2500bp}
    \pgfpathqlineto{105.0000bp}{88.2500bp}
    \pgfpathqlineto{106.0000bp}{88.2500bp}
    \pgfpathqlineto{107.0000bp}{88.2500bp}
    \pgfpathqlineto{108.0000bp}{88.2500bp}
    \pgfpathqlineto{109.0000bp}{88.2500bp}
    \pgfpathqlineto{110.0000bp}{88.2500bp}
    \pgfpathqlineto{111.0000bp}{88.2500bp}
    \pgfpathqlineto{112.0000bp}{88.2500bp}
    \pgfpathqlineto{113.0000bp}{88.2500bp}
    \pgfpathqlineto{114.0000bp}{88.2500bp}
    \pgfpathqlineto{115.0000bp}{88.2500bp}
    \pgfpathqlineto{116.0000bp}{88.2500bp}
    \pgfpathqlineto{117.0000bp}{88.2500bp}
    \pgfpathqlineto{118.0000bp}{88.2500bp}
    \pgfpathqlineto{119.0000bp}{88.2500bp}
    \pgfpathqlineto{120.0000bp}{88.2500bp}
    \pgfpathqlineto{121.0000bp}{88.2500bp}
    \pgfpathqlineto{122.0000bp}{88.2500bp}
    \pgfpathqlineto{123.0000bp}{88.2500bp}
    \pgfpathqlineto{124.0000bp}{88.2500bp}
    \pgfpathqlineto{125.0000bp}{88.2500bp}
    \pgfpathqlineto{126.0000bp}{88.2500bp}
    \pgfpathqlineto{127.0000bp}{88.2500bp}
    \pgfpathqlineto{128.0000bp}{88.2500bp}
    \pgfpathqlineto{129.0000bp}{88.2500bp}
    \pgfpathqlineto{130.0000bp}{88.2500bp}
    \pgfpathqlineto{131.0000bp}{88.2500bp}
    \pgfpathqlineto{132.0000bp}{88.2500bp}
    \pgfpathqlineto{133.0000bp}{88.2500bp}
    \pgfpathqlineto{134.0000bp}{88.2500bp}
    \pgfpathqlineto{135.0000bp}{88.2500bp}
    \pgfpathqlineto{136.0000bp}{88.2500bp}
    \pgfpathqlineto{137.0000bp}{88.2500bp}
    \pgfpathqlineto{138.0000bp}{88.2500bp}
    \pgfpathqlineto{139.0000bp}{88.2500bp}
    \pgfpathqlineto{140.0000bp}{88.2500bp}
    \pgfpathqlineto{141.0000bp}{88.2500bp}
    \pgfpathqlineto{142.0000bp}{88.2500bp}
    \pgfpathqlineto{143.0000bp}{88.2500bp}
    \pgfpathqlineto{144.0000bp}{88.2500bp}
    \pgfpathqlineto{145.0000bp}{88.2500bp}
    \pgfpathqlineto{146.0000bp}{88.2500bp}
    \pgfpathqlineto{147.0000bp}{88.2500bp}
    \pgfpathqlineto{148.0000bp}{88.2500bp}
    \pgfpathqlineto{149.0000bp}{88.2500bp}
    \pgfpathqlineto{150.0000bp}{88.2500bp}
    \pgfpathqlineto{151.0000bp}{88.2500bp}
    \pgfpathqlineto{152.0000bp}{88.2500bp}
    \pgfpathqlineto{153.0000bp}{88.2500bp}
    \pgfpathqlineto{154.0000bp}{88.2500bp}
    \pgfpathqlineto{155.0000bp}{88.2500bp}
    \pgfpathqlineto{156.0000bp}{88.2500bp}
    \pgfpathqlineto{157.0000bp}{88.2500bp}
    \pgfpathqlineto{158.0000bp}{88.2500bp}
    \pgfpathqlineto{159.0000bp}{88.2500bp}
    \pgfpathqlineto{160.0000bp}{88.2500bp}
    \pgfpathqlineto{161.0000bp}{88.2500bp}
    \pgfpathqlineto{162.0000bp}{88.2500bp}
    \pgfpathqlineto{163.0000bp}{88.2500bp}
    \pgfpathqlineto{164.0000bp}{88.2500bp}
    \pgfpathqlineto{165.0000bp}{88.2500bp}
    \pgfpathqlineto{166.0000bp}{88.2500bp}
    \pgfpathqlineto{167.0000bp}{88.2500bp}
    \pgfpathqlineto{168.0000bp}{88.2500bp}
    \pgfpathqlineto{169.0000bp}{88.2500bp}
    \pgfpathqlineto{170.0000bp}{88.2500bp}
    \pgfpathqlineto{171.0000bp}{88.2500bp}
    \pgfpathqlineto{172.0000bp}{88.2500bp}
    \pgfpathqlineto{173.0000bp}{88.2500bp}
    \pgfpathqlineto{174.0000bp}{88.2500bp}
    \pgfpathqlineto{175.0000bp}{88.2500bp}
    \pgfpathqlineto{176.0000bp}{88.2500bp}
    \pgfpathqlineto{177.0000bp}{88.2500bp}
    \pgfpathqlineto{178.0000bp}{88.2500bp}
    \pgfpathqlineto{179.0000bp}{88.2500bp}
    \pgfpathqlineto{180.0000bp}{88.2500bp}
    \pgfpathqlineto{181.0000bp}{88.2500bp}
    \pgfpathqlineto{182.0000bp}{88.2500bp}
    \pgfpathqlineto{183.0000bp}{88.2500bp}
    \pgfpathqlineto{184.0000bp}{88.2500bp}
    \pgfpathqlineto{185.0000bp}{88.2500bp}
    \pgfpathqlineto{186.0000bp}{88.2500bp}
    \pgfpathqlineto{187.0000bp}{88.2500bp}
    \pgfpathqlineto{188.0000bp}{88.2500bp}
    \pgfpathqlineto{189.0000bp}{88.2500bp}
    \pgfpathqlineto{190.0000bp}{88.2500bp}
    \pgfpathqlineto{191.0000bp}{88.2500bp}
    \pgfpathqlineto{192.0000bp}{60.2500bp}
    \pgfpathqlineto{193.0000bp}{60.2500bp}
    \pgfpathqlineto{194.0000bp}{60.2500bp}
    \pgfpathqlineto{195.0000bp}{60.2500bp}
    \pgfpathqlineto{196.0000bp}{60.2500bp}
    \pgfpathqlineto{197.0000bp}{60.2500bp}
    \pgfpathqlineto{198.0000bp}{59.7500bp}
    \pgfpathqlineto{199.0000bp}{59.7500bp}
    \pgfusepathqstroke
  \end{pgfscope}
  \begin{pgfscope}
    \pgfsetlinewidth{0.5727bp}
    \definecolor{sc}{rgb}{0.0000,0.0000,1.0000}
    \pgfsetstrokecolor{sc}
    \pgfsetmiterjoin
    \pgfsetbuttcap
    \pgfpathqmoveto{2.0000bp}{56.2500bp}
    \pgfpathqlineto{3.0000bp}{56.2500bp}
    \pgfpathqlineto{4.0000bp}{56.2500bp}
    \pgfpathqlineto{5.0000bp}{56.2500bp}
    \pgfpathqlineto{6.0000bp}{56.2500bp}
    \pgfpathqlineto{7.0000bp}{56.2500bp}
    \pgfpathqlineto{8.0000bp}{56.2500bp}
    \pgfpathqlineto{9.0000bp}{56.2500bp}
    \pgfpathqlineto{10.0000bp}{56.2500bp}
    \pgfpathqlineto{11.0000bp}{56.2500bp}
    \pgfpathqlineto{12.0000bp}{56.2500bp}
    \pgfpathqlineto{13.0000bp}{56.2500bp}
    \pgfpathqlineto{14.0000bp}{56.2500bp}
    \pgfpathqlineto{15.0000bp}{56.2500bp}
    \pgfpathqlineto{16.0000bp}{56.2500bp}
    \pgfpathqlineto{17.0000bp}{56.2500bp}
    \pgfpathqlineto{18.0000bp}{56.2500bp}
    \pgfpathqlineto{19.0000bp}{56.2500bp}
    \pgfpathqlineto{20.0000bp}{56.2500bp}
    \pgfpathqlineto{21.0000bp}{56.2500bp}
    \pgfpathqlineto{22.0000bp}{56.2500bp}
    \pgfpathqlineto{23.0000bp}{56.2500bp}
    \pgfpathqlineto{24.0000bp}{56.2500bp}
    \pgfpathqlineto{25.0000bp}{56.2500bp}
    \pgfpathqlineto{26.0000bp}{56.2500bp}
    \pgfpathqlineto{27.0000bp}{56.2500bp}
    \pgfpathqlineto{28.0000bp}{56.2500bp}
    \pgfpathqlineto{29.0000bp}{56.2500bp}
    \pgfpathqlineto{30.0000bp}{56.2500bp}
    \pgfpathqlineto{31.0000bp}{56.2500bp}
    \pgfpathqlineto{32.0000bp}{56.2500bp}
    \pgfpathqlineto{33.0000bp}{56.2500bp}
    \pgfpathqlineto{34.0000bp}{56.2500bp}
    \pgfpathqlineto{35.0000bp}{56.2500bp}
    \pgfpathqlineto{36.0000bp}{56.2500bp}
    \pgfpathqlineto{37.0000bp}{56.2500bp}
    \pgfpathqlineto{38.0000bp}{56.2500bp}
    \pgfpathqlineto{39.0000bp}{56.2500bp}
    \pgfpathqlineto{40.0000bp}{56.2500bp}
    \pgfpathqlineto{41.0000bp}{56.2500bp}
    \pgfpathqlineto{42.0000bp}{56.2500bp}
    \pgfpathqlineto{43.0000bp}{56.2500bp}
    \pgfpathqlineto{44.0000bp}{56.2500bp}
    \pgfpathqlineto{45.0000bp}{56.2500bp}
    \pgfpathqlineto{46.0000bp}{56.2500bp}
    \pgfpathqlineto{47.0000bp}{56.2500bp}
    \pgfpathqlineto{48.0000bp}{56.2500bp}
    \pgfpathqlineto{49.0000bp}{56.2500bp}
    \pgfpathqlineto{50.0000bp}{56.2500bp}
    \pgfpathqlineto{51.0000bp}{56.2500bp}
    \pgfpathqlineto{52.0000bp}{56.2500bp}
    \pgfpathqlineto{53.0000bp}{56.2500bp}
    \pgfpathqlineto{54.0000bp}{56.2500bp}
    \pgfpathqlineto{55.0000bp}{56.2500bp}
    \pgfpathqlineto{56.0000bp}{56.2500bp}
    \pgfpathqlineto{57.0000bp}{56.2500bp}
    \pgfpathqlineto{58.0000bp}{56.2500bp}
    \pgfpathqlineto{59.0000bp}{56.2500bp}
    \pgfpathqlineto{60.0000bp}{56.2500bp}
    \pgfpathqlineto{61.0000bp}{56.2500bp}
    \pgfpathqlineto{62.0000bp}{56.2500bp}
    \pgfpathqlineto{63.0000bp}{56.2500bp}
    \pgfpathqlineto{64.0000bp}{56.2500bp}
    \pgfpathqlineto{65.0000bp}{56.2500bp}
    \pgfpathqlineto{66.0000bp}{56.2500bp}
    \pgfpathqlineto{67.0000bp}{56.2500bp}
    \pgfpathqlineto{68.0000bp}{56.2500bp}
    \pgfpathqlineto{69.0000bp}{56.2500bp}
    \pgfpathqlineto{70.0000bp}{56.2500bp}
    \pgfpathqlineto{71.0000bp}{56.2500bp}
    \pgfpathqlineto{72.0000bp}{56.2500bp}
    \pgfpathqlineto{73.0000bp}{56.2500bp}
    \pgfpathqlineto{74.0000bp}{56.2500bp}
    \pgfpathqlineto{75.0000bp}{56.2500bp}
    \pgfpathqlineto{76.0000bp}{56.2500bp}
    \pgfpathqlineto{77.0000bp}{56.2500bp}
    \pgfpathqlineto{78.0000bp}{56.2500bp}
    \pgfpathqlineto{79.0000bp}{56.2500bp}
    \pgfpathqlineto{80.0000bp}{56.2500bp}
    \pgfpathqlineto{81.0000bp}{56.2500bp}
    \pgfpathqlineto{82.0000bp}{56.2500bp}
    \pgfpathqlineto{83.0000bp}{56.2500bp}
    \pgfpathqlineto{84.0000bp}{56.2500bp}
    \pgfpathqlineto{85.0000bp}{56.2500bp}
    \pgfpathqlineto{86.0000bp}{56.2500bp}
    \pgfpathqlineto{87.0000bp}{56.2500bp}
    \pgfpathqlineto{88.0000bp}{56.2500bp}
    \pgfpathqlineto{89.0000bp}{56.2500bp}
    \pgfpathqlineto{90.0000bp}{56.2500bp}
    \pgfpathqlineto{91.0000bp}{56.2500bp}
    \pgfpathqlineto{92.0000bp}{56.2500bp}
    \pgfpathqlineto{93.0000bp}{56.2500bp}
    \pgfpathqlineto{94.0000bp}{56.2500bp}
    \pgfpathqlineto{95.0000bp}{56.2500bp}
    \pgfpathqlineto{96.0000bp}{56.2500bp}
    \pgfpathqlineto{97.0000bp}{56.2500bp}
    \pgfpathqlineto{98.0000bp}{56.2500bp}
    \pgfpathqlineto{99.0000bp}{56.2500bp}
    \pgfpathqlineto{100.0000bp}{56.2500bp}
    \pgfpathqlineto{101.0000bp}{56.2500bp}
    \pgfpathqlineto{102.0000bp}{56.2500bp}
    \pgfpathqlineto{103.0000bp}{56.2500bp}
    \pgfpathqlineto{104.0000bp}{56.2500bp}
    \pgfpathqlineto{105.0000bp}{56.2500bp}
    \pgfpathqlineto{106.0000bp}{56.2500bp}
    \pgfpathqlineto{107.0000bp}{56.2500bp}
    \pgfpathqlineto{108.0000bp}{56.2500bp}
    \pgfpathqlineto{109.0000bp}{56.2500bp}
    \pgfpathqlineto{110.0000bp}{56.2500bp}
    \pgfpathqlineto{111.0000bp}{56.2500bp}
    \pgfpathqlineto{112.0000bp}{56.2500bp}
    \pgfpathqlineto{113.0000bp}{56.2500bp}
    \pgfpathqlineto{114.0000bp}{56.2500bp}
    \pgfpathqlineto{115.0000bp}{56.2500bp}
    \pgfpathqlineto{116.0000bp}{56.2500bp}
    \pgfpathqlineto{117.0000bp}{56.2500bp}
    \pgfpathqlineto{118.0000bp}{56.2500bp}
    \pgfpathqlineto{119.0000bp}{56.2500bp}
    \pgfpathqlineto{120.0000bp}{56.2500bp}
    \pgfpathqlineto{121.0000bp}{56.2500bp}
    \pgfpathqlineto{122.0000bp}{56.2500bp}
    \pgfpathqlineto{123.0000bp}{56.2500bp}
    \pgfpathqlineto{124.0000bp}{56.2500bp}
    \pgfpathqlineto{125.0000bp}{56.2500bp}
    \pgfpathqlineto{126.0000bp}{56.2500bp}
    \pgfpathqlineto{127.0000bp}{56.2500bp}
    \pgfpathqlineto{128.0000bp}{56.2500bp}
    \pgfpathqlineto{129.0000bp}{56.2500bp}
    \pgfpathqlineto{130.0000bp}{56.2500bp}
    \pgfpathqlineto{131.0000bp}{56.2500bp}
    \pgfpathqlineto{132.0000bp}{56.2500bp}
    \pgfpathqlineto{133.0000bp}{56.2500bp}
    \pgfpathqlineto{134.0000bp}{56.2500bp}
    \pgfpathqlineto{135.0000bp}{56.2500bp}
    \pgfpathqlineto{136.0000bp}{56.2500bp}
    \pgfpathqlineto{137.0000bp}{56.2500bp}
    \pgfpathqlineto{138.0000bp}{56.2500bp}
    \pgfpathqlineto{139.0000bp}{56.2500bp}
    \pgfpathqlineto{140.0000bp}{56.2500bp}
    \pgfpathqlineto{141.0000bp}{56.2500bp}
    \pgfpathqlineto{142.0000bp}{56.2500bp}
    \pgfpathqlineto{143.0000bp}{56.2500bp}
    \pgfpathqlineto{144.0000bp}{56.2500bp}
    \pgfpathqlineto{145.0000bp}{56.2500bp}
    \pgfpathqlineto{146.0000bp}{56.2500bp}
    \pgfpathqlineto{147.0000bp}{56.2500bp}
    \pgfpathqlineto{148.0000bp}{56.2500bp}
    \pgfpathqlineto{149.0000bp}{56.2500bp}
    \pgfpathqlineto{150.0000bp}{56.2500bp}
    \pgfpathqlineto{151.0000bp}{56.2500bp}
    \pgfpathqlineto{152.0000bp}{56.2500bp}
    \pgfpathqlineto{153.0000bp}{56.2500bp}
    \pgfpathqlineto{154.0000bp}{56.2500bp}
    \pgfpathqlineto{155.0000bp}{56.2500bp}
    \pgfpathqlineto{156.0000bp}{56.2500bp}
    \pgfpathqlineto{157.0000bp}{56.2500bp}
    \pgfpathqlineto{158.0000bp}{56.2500bp}
    \pgfpathqlineto{159.0000bp}{56.2500bp}
    \pgfpathqlineto{160.0000bp}{56.2500bp}
    \pgfpathqlineto{161.0000bp}{56.2500bp}
    \pgfpathqlineto{162.0000bp}{56.2500bp}
    \pgfpathqlineto{163.0000bp}{56.2500bp}
    \pgfpathqlineto{164.0000bp}{56.2500bp}
    \pgfpathqlineto{165.0000bp}{56.2500bp}
    \pgfpathqlineto{166.0000bp}{56.2500bp}
    \pgfpathqlineto{167.0000bp}{56.2500bp}
    \pgfpathqlineto{168.0000bp}{56.2500bp}
    \pgfpathqlineto{169.0000bp}{56.2500bp}
    \pgfpathqlineto{170.0000bp}{56.2500bp}
    \pgfpathqlineto{171.0000bp}{56.2500bp}
    \pgfpathqlineto{172.0000bp}{56.2500bp}
    \pgfpathqlineto{173.0000bp}{56.2500bp}
    \pgfpathqlineto{174.0000bp}{56.2500bp}
    \pgfpathqlineto{175.0000bp}{56.2500bp}
    \pgfpathqlineto{176.0000bp}{56.2500bp}
    \pgfpathqlineto{177.0000bp}{56.2500bp}
    \pgfpathqlineto{178.0000bp}{56.2500bp}
    \pgfpathqlineto{179.0000bp}{56.2500bp}
    \pgfpathqlineto{180.0000bp}{56.2500bp}
    \pgfpathqlineto{181.0000bp}{56.2500bp}
    \pgfpathqlineto{182.0000bp}{56.2500bp}
    \pgfpathqlineto{183.0000bp}{56.2500bp}
    \pgfpathqlineto{184.0000bp}{56.2500bp}
    \pgfpathqlineto{185.0000bp}{56.2500bp}
    \pgfpathqlineto{186.0000bp}{56.2500bp}
    \pgfpathqlineto{187.0000bp}{56.2500bp}
    \pgfpathqlineto{188.0000bp}{56.2500bp}
    \pgfpathqlineto{189.0000bp}{56.2500bp}
    \pgfpathqlineto{190.0000bp}{56.2500bp}
    \pgfpathqlineto{191.0000bp}{56.2500bp}
    \pgfpathqlineto{192.0000bp}{56.2500bp}
    \pgfpathqlineto{193.0000bp}{56.2500bp}
    \pgfpathqlineto{194.0000bp}{56.2500bp}
    \pgfpathqlineto{195.0000bp}{56.2500bp}
    \pgfpathqlineto{196.0000bp}{56.2500bp}
    \pgfpathqlineto{197.0000bp}{56.2500bp}
    \pgfpathqlineto{198.0000bp}{56.7500bp}
    \pgfpathqlineto{199.0000bp}{56.7500bp}
    \pgfusepathqstroke
  \end{pgfscope}
  \begin{pgfscope}
    \pgfsetlinewidth{0.5727bp}
    \definecolor{sc}{rgb}{1.0000,0.0000,0.0000}
    \pgfsetstrokecolor{sc}
    \pgfsetmiterjoin
    \pgfsetbuttcap
    \pgfpathqmoveto{200.0000bp}{56.2500bp}
    \pgfpathqlineto{200.0000bp}{55.7500bp}
    \pgfusepathqstroke
  \end{pgfscope}
  \begin{pgfscope}
    \pgfsetlinewidth{0.5727bp}
    \definecolor{sc}{rgb}{1.0000,0.0000,0.0000}
    \pgfsetstrokecolor{sc}
    \pgfsetmiterjoin
    \pgfsetbuttcap
    \pgfpathqmoveto{193.0000bp}{56.2500bp}
    \pgfpathqlineto{193.0000bp}{55.7500bp}
    \pgfusepathqstroke
  \end{pgfscope}
  \begin{pgfscope}
    \pgfsetlinewidth{0.5727bp}
    \definecolor{sc}{rgb}{1.0000,0.0000,0.0000}
    \pgfsetstrokecolor{sc}
    \pgfsetmiterjoin
    \pgfsetbuttcap
    \pgfpathqmoveto{120.0000bp}{56.2500bp}
    \pgfpathqlineto{120.0000bp}{55.7500bp}
    \pgfusepathqstroke
  \end{pgfscope}
  \begin{pgfscope}
    \pgfsetlinewidth{0.5727bp}
    \definecolor{sc}{rgb}{1.0000,0.0000,0.0000}
    \pgfsetstrokecolor{sc}
    \pgfsetmiterjoin
    \pgfsetbuttcap
    \pgfpathqmoveto{2.0000bp}{56.2500bp}
    \pgfpathqlineto{2.0000bp}{55.7500bp}
    \pgfusepathqstroke
  \end{pgfscope}
  \begin{pgfscope}
    \pgfsetlinewidth{0.5727bp}
    \definecolor{sc}{rgb}{0.0000,0.0000,0.0000}
    \pgfsetstrokecolor{sc}
    \pgfsetmiterjoin
    \pgfsetbuttcap
    \pgfpathqmoveto{197.0000bp}{56.2500bp}
    \pgfpathqlineto{197.0000bp}{55.7500bp}
    \pgfusepathqstroke
  \end{pgfscope}
  \begin{pgfscope}
    \pgfsetlinewidth{0.5727bp}
    \definecolor{sc}{rgb}{0.0000,0.0000,0.0000}
    \pgfsetstrokecolor{sc}
    \pgfsetmiterjoin
    \pgfsetbuttcap
    \pgfpathqmoveto{192.0000bp}{56.2500bp}
    \pgfpathqlineto{192.0000bp}{55.7500bp}
    \pgfusepathqstroke
  \end{pgfscope}
  \begin{pgfscope}
    \pgfsetlinewidth{0.5727bp}
    \definecolor{sc}{rgb}{0.0000,0.0000,0.0000}
    \pgfsetstrokecolor{sc}
    \pgfsetmiterjoin
    \pgfsetbuttcap
    \pgfpathqmoveto{187.0000bp}{56.2500bp}
    \pgfpathqlineto{187.0000bp}{55.7500bp}
    \pgfusepathqstroke
  \end{pgfscope}
  \begin{pgfscope}
    \pgfsetlinewidth{0.5727bp}
    \definecolor{sc}{rgb}{0.0000,0.0000,0.0000}
    \pgfsetstrokecolor{sc}
    \pgfsetmiterjoin
    \pgfsetbuttcap
    \pgfpathqmoveto{182.0000bp}{56.2500bp}
    \pgfpathqlineto{182.0000bp}{55.7500bp}
    \pgfusepathqstroke
  \end{pgfscope}
  \begin{pgfscope}
    \pgfsetlinewidth{0.5727bp}
    \definecolor{sc}{rgb}{0.0000,0.0000,0.0000}
    \pgfsetstrokecolor{sc}
    \pgfsetmiterjoin
    \pgfsetbuttcap
    \pgfpathqmoveto{177.0000bp}{56.2500bp}
    \pgfpathqlineto{177.0000bp}{55.7500bp}
    \pgfusepathqstroke
  \end{pgfscope}
  \begin{pgfscope}
    \pgfsetlinewidth{0.5727bp}
    \definecolor{sc}{rgb}{0.0000,0.0000,0.0000}
    \pgfsetstrokecolor{sc}
    \pgfsetmiterjoin
    \pgfsetbuttcap
    \pgfpathqmoveto{172.0000bp}{56.2500bp}
    \pgfpathqlineto{172.0000bp}{55.7500bp}
    \pgfusepathqstroke
  \end{pgfscope}
  \begin{pgfscope}
    \pgfsetlinewidth{0.5727bp}
    \definecolor{sc}{rgb}{0.0000,0.0000,0.0000}
    \pgfsetstrokecolor{sc}
    \pgfsetmiterjoin
    \pgfsetbuttcap
    \pgfpathqmoveto{167.0000bp}{56.2500bp}
    \pgfpathqlineto{167.0000bp}{55.7500bp}
    \pgfusepathqstroke
  \end{pgfscope}
  \begin{pgfscope}
    \pgfsetlinewidth{0.5727bp}
    \definecolor{sc}{rgb}{0.0000,0.0000,0.0000}
    \pgfsetstrokecolor{sc}
    \pgfsetmiterjoin
    \pgfsetbuttcap
    \pgfpathqmoveto{162.0000bp}{56.2500bp}
    \pgfpathqlineto{162.0000bp}{55.7500bp}
    \pgfusepathqstroke
  \end{pgfscope}
  \begin{pgfscope}
    \pgfsetlinewidth{0.5727bp}
    \definecolor{sc}{rgb}{0.0000,0.0000,0.0000}
    \pgfsetstrokecolor{sc}
    \pgfsetmiterjoin
    \pgfsetbuttcap
    \pgfpathqmoveto{157.0000bp}{56.2500bp}
    \pgfpathqlineto{157.0000bp}{55.7500bp}
    \pgfusepathqstroke
  \end{pgfscope}
  \begin{pgfscope}
    \pgfsetlinewidth{0.5727bp}
    \definecolor{sc}{rgb}{0.0000,0.0000,0.0000}
    \pgfsetstrokecolor{sc}
    \pgfsetmiterjoin
    \pgfsetbuttcap
    \pgfpathqmoveto{152.0000bp}{56.2500bp}
    \pgfpathqlineto{152.0000bp}{55.7500bp}
    \pgfusepathqstroke
  \end{pgfscope}
  \begin{pgfscope}
    \pgfsetlinewidth{0.5727bp}
    \definecolor{sc}{rgb}{0.0000,0.0000,0.0000}
    \pgfsetstrokecolor{sc}
    \pgfsetmiterjoin
    \pgfsetbuttcap
    \pgfpathqmoveto{147.0000bp}{56.2500bp}
    \pgfpathqlineto{147.0000bp}{55.7500bp}
    \pgfusepathqstroke
  \end{pgfscope}
  \begin{pgfscope}
    \pgfsetlinewidth{0.5727bp}
    \definecolor{sc}{rgb}{0.0000,0.0000,0.0000}
    \pgfsetstrokecolor{sc}
    \pgfsetmiterjoin
    \pgfsetbuttcap
    \pgfpathqmoveto{142.0000bp}{56.2500bp}
    \pgfpathqlineto{142.0000bp}{55.7500bp}
    \pgfusepathqstroke
  \end{pgfscope}
  \begin{pgfscope}
    \pgfsetlinewidth{0.5727bp}
    \definecolor{sc}{rgb}{0.0000,0.0000,0.0000}
    \pgfsetstrokecolor{sc}
    \pgfsetmiterjoin
    \pgfsetbuttcap
    \pgfpathqmoveto{137.0000bp}{56.2500bp}
    \pgfpathqlineto{137.0000bp}{55.7500bp}
    \pgfusepathqstroke
  \end{pgfscope}
  \begin{pgfscope}
    \pgfsetlinewidth{0.5727bp}
    \definecolor{sc}{rgb}{0.0000,0.0000,0.0000}
    \pgfsetstrokecolor{sc}
    \pgfsetmiterjoin
    \pgfsetbuttcap
    \pgfpathqmoveto{132.0000bp}{56.2500bp}
    \pgfpathqlineto{132.0000bp}{55.7500bp}
    \pgfusepathqstroke
  \end{pgfscope}
  \begin{pgfscope}
    \pgfsetlinewidth{0.5727bp}
    \definecolor{sc}{rgb}{0.0000,0.0000,0.0000}
    \pgfsetstrokecolor{sc}
    \pgfsetmiterjoin
    \pgfsetbuttcap
    \pgfpathqmoveto{127.0000bp}{56.2500bp}
    \pgfpathqlineto{127.0000bp}{55.7500bp}
    \pgfusepathqstroke
  \end{pgfscope}
  \begin{pgfscope}
    \pgfsetlinewidth{0.5727bp}
    \definecolor{sc}{rgb}{0.0000,0.0000,0.0000}
    \pgfsetstrokecolor{sc}
    \pgfsetmiterjoin
    \pgfsetbuttcap
    \pgfpathqmoveto{122.0000bp}{56.2500bp}
    \pgfpathqlineto{122.0000bp}{55.7500bp}
    \pgfusepathqstroke
  \end{pgfscope}
  \begin{pgfscope}
    \pgfsetlinewidth{0.5727bp}
    \definecolor{sc}{rgb}{0.0000,0.0000,0.0000}
    \pgfsetstrokecolor{sc}
    \pgfsetmiterjoin
    \pgfsetbuttcap
    \pgfpathqmoveto{117.0000bp}{56.2500bp}
    \pgfpathqlineto{117.0000bp}{55.7500bp}
    \pgfusepathqstroke
  \end{pgfscope}
  \begin{pgfscope}
    \pgfsetlinewidth{0.5727bp}
    \definecolor{sc}{rgb}{0.0000,0.0000,0.0000}
    \pgfsetstrokecolor{sc}
    \pgfsetmiterjoin
    \pgfsetbuttcap
    \pgfpathqmoveto{112.0000bp}{56.2500bp}
    \pgfpathqlineto{112.0000bp}{55.7500bp}
    \pgfusepathqstroke
  \end{pgfscope}
  \begin{pgfscope}
    \pgfsetlinewidth{0.5727bp}
    \definecolor{sc}{rgb}{0.0000,0.0000,0.0000}
    \pgfsetstrokecolor{sc}
    \pgfsetmiterjoin
    \pgfsetbuttcap
    \pgfpathqmoveto{107.0000bp}{56.2500bp}
    \pgfpathqlineto{107.0000bp}{55.7500bp}
    \pgfusepathqstroke
  \end{pgfscope}
  \begin{pgfscope}
    \pgfsetlinewidth{0.5727bp}
    \definecolor{sc}{rgb}{0.0000,0.0000,0.0000}
    \pgfsetstrokecolor{sc}
    \pgfsetmiterjoin
    \pgfsetbuttcap
    \pgfpathqmoveto{102.0000bp}{56.2500bp}
    \pgfpathqlineto{102.0000bp}{55.7500bp}
    \pgfusepathqstroke
  \end{pgfscope}
  \begin{pgfscope}
    \pgfsetlinewidth{0.5727bp}
    \definecolor{sc}{rgb}{0.0000,0.0000,0.0000}
    \pgfsetstrokecolor{sc}
    \pgfsetmiterjoin
    \pgfsetbuttcap
    \pgfpathqmoveto{97.0000bp}{56.2500bp}
    \pgfpathqlineto{97.0000bp}{55.7500bp}
    \pgfusepathqstroke
  \end{pgfscope}
  \begin{pgfscope}
    \pgfsetlinewidth{0.5727bp}
    \definecolor{sc}{rgb}{0.0000,0.0000,0.0000}
    \pgfsetstrokecolor{sc}
    \pgfsetmiterjoin
    \pgfsetbuttcap
    \pgfpathqmoveto{92.0000bp}{56.2500bp}
    \pgfpathqlineto{92.0000bp}{55.7500bp}
    \pgfusepathqstroke
  \end{pgfscope}
  \begin{pgfscope}
    \pgfsetlinewidth{0.5727bp}
    \definecolor{sc}{rgb}{0.0000,0.0000,0.0000}
    \pgfsetstrokecolor{sc}
    \pgfsetmiterjoin
    \pgfsetbuttcap
    \pgfpathqmoveto{87.0000bp}{56.2500bp}
    \pgfpathqlineto{87.0000bp}{55.7500bp}
    \pgfusepathqstroke
  \end{pgfscope}
  \begin{pgfscope}
    \pgfsetlinewidth{0.5727bp}
    \definecolor{sc}{rgb}{0.0000,0.0000,0.0000}
    \pgfsetstrokecolor{sc}
    \pgfsetmiterjoin
    \pgfsetbuttcap
    \pgfpathqmoveto{82.0000bp}{56.2500bp}
    \pgfpathqlineto{82.0000bp}{55.7500bp}
    \pgfusepathqstroke
  \end{pgfscope}
  \begin{pgfscope}
    \pgfsetlinewidth{0.5727bp}
    \definecolor{sc}{rgb}{0.0000,0.0000,0.0000}
    \pgfsetstrokecolor{sc}
    \pgfsetmiterjoin
    \pgfsetbuttcap
    \pgfpathqmoveto{77.0000bp}{56.2500bp}
    \pgfpathqlineto{77.0000bp}{55.7500bp}
    \pgfusepathqstroke
  \end{pgfscope}
  \begin{pgfscope}
    \pgfsetlinewidth{0.5727bp}
    \definecolor{sc}{rgb}{0.0000,0.0000,0.0000}
    \pgfsetstrokecolor{sc}
    \pgfsetmiterjoin
    \pgfsetbuttcap
    \pgfpathqmoveto{72.0000bp}{56.2500bp}
    \pgfpathqlineto{72.0000bp}{55.7500bp}
    \pgfusepathqstroke
  \end{pgfscope}
  \begin{pgfscope}
    \pgfsetlinewidth{0.5727bp}
    \definecolor{sc}{rgb}{0.0000,0.0000,0.0000}
    \pgfsetstrokecolor{sc}
    \pgfsetmiterjoin
    \pgfsetbuttcap
    \pgfpathqmoveto{67.0000bp}{56.2500bp}
    \pgfpathqlineto{67.0000bp}{55.7500bp}
    \pgfusepathqstroke
  \end{pgfscope}
  \begin{pgfscope}
    \pgfsetlinewidth{0.5727bp}
    \definecolor{sc}{rgb}{0.0000,0.0000,0.0000}
    \pgfsetstrokecolor{sc}
    \pgfsetmiterjoin
    \pgfsetbuttcap
    \pgfpathqmoveto{62.0000bp}{56.2500bp}
    \pgfpathqlineto{62.0000bp}{55.7500bp}
    \pgfusepathqstroke
  \end{pgfscope}
  \begin{pgfscope}
    \pgfsetlinewidth{0.5727bp}
    \definecolor{sc}{rgb}{0.0000,0.0000,0.0000}
    \pgfsetstrokecolor{sc}
    \pgfsetmiterjoin
    \pgfsetbuttcap
    \pgfpathqmoveto{57.0000bp}{56.2500bp}
    \pgfpathqlineto{57.0000bp}{55.7500bp}
    \pgfusepathqstroke
  \end{pgfscope}
  \begin{pgfscope}
    \pgfsetlinewidth{0.5727bp}
    \definecolor{sc}{rgb}{0.0000,0.0000,0.0000}
    \pgfsetstrokecolor{sc}
    \pgfsetmiterjoin
    \pgfsetbuttcap
    \pgfpathqmoveto{52.0000bp}{56.2500bp}
    \pgfpathqlineto{52.0000bp}{55.7500bp}
    \pgfusepathqstroke
  \end{pgfscope}
  \begin{pgfscope}
    \pgfsetlinewidth{0.5727bp}
    \definecolor{sc}{rgb}{0.0000,0.0000,0.0000}
    \pgfsetstrokecolor{sc}
    \pgfsetmiterjoin
    \pgfsetbuttcap
    \pgfpathqmoveto{47.0000bp}{56.2500bp}
    \pgfpathqlineto{47.0000bp}{55.7500bp}
    \pgfusepathqstroke
  \end{pgfscope}
  \begin{pgfscope}
    \pgfsetlinewidth{0.5727bp}
    \definecolor{sc}{rgb}{0.0000,0.0000,0.0000}
    \pgfsetstrokecolor{sc}
    \pgfsetmiterjoin
    \pgfsetbuttcap
    \pgfpathqmoveto{42.0000bp}{56.2500bp}
    \pgfpathqlineto{42.0000bp}{55.7500bp}
    \pgfusepathqstroke
  \end{pgfscope}
  \begin{pgfscope}
    \pgfsetlinewidth{0.5727bp}
    \definecolor{sc}{rgb}{0.0000,0.0000,0.0000}
    \pgfsetstrokecolor{sc}
    \pgfsetmiterjoin
    \pgfsetbuttcap
    \pgfpathqmoveto{37.0000bp}{56.2500bp}
    \pgfpathqlineto{37.0000bp}{55.7500bp}
    \pgfusepathqstroke
  \end{pgfscope}
  \begin{pgfscope}
    \pgfsetlinewidth{0.5727bp}
    \definecolor{sc}{rgb}{0.0000,0.0000,0.0000}
    \pgfsetstrokecolor{sc}
    \pgfsetmiterjoin
    \pgfsetbuttcap
    \pgfpathqmoveto{32.0000bp}{56.2500bp}
    \pgfpathqlineto{32.0000bp}{55.7500bp}
    \pgfusepathqstroke
  \end{pgfscope}
  \begin{pgfscope}
    \pgfsetlinewidth{0.5727bp}
    \definecolor{sc}{rgb}{0.0000,0.0000,0.0000}
    \pgfsetstrokecolor{sc}
    \pgfsetmiterjoin
    \pgfsetbuttcap
    \pgfpathqmoveto{27.0000bp}{56.2500bp}
    \pgfpathqlineto{27.0000bp}{55.7500bp}
    \pgfusepathqstroke
  \end{pgfscope}
  \begin{pgfscope}
    \pgfsetlinewidth{0.5727bp}
    \definecolor{sc}{rgb}{0.0000,0.0000,0.0000}
    \pgfsetstrokecolor{sc}
    \pgfsetmiterjoin
    \pgfsetbuttcap
    \pgfpathqmoveto{22.0000bp}{56.2500bp}
    \pgfpathqlineto{22.0000bp}{55.7500bp}
    \pgfusepathqstroke
  \end{pgfscope}
  \begin{pgfscope}
    \pgfsetlinewidth{0.5727bp}
    \definecolor{sc}{rgb}{0.0000,0.0000,0.0000}
    \pgfsetstrokecolor{sc}
    \pgfsetmiterjoin
    \pgfsetbuttcap
    \pgfpathqmoveto{17.0000bp}{56.2500bp}
    \pgfpathqlineto{17.0000bp}{55.7500bp}
    \pgfusepathqstroke
  \end{pgfscope}
  \begin{pgfscope}
    \pgfsetlinewidth{0.5727bp}
    \definecolor{sc}{rgb}{0.0000,0.0000,0.0000}
    \pgfsetstrokecolor{sc}
    \pgfsetmiterjoin
    \pgfsetbuttcap
    \pgfpathqmoveto{12.0000bp}{56.2500bp}
    \pgfpathqlineto{12.0000bp}{55.7500bp}
    \pgfusepathqstroke
  \end{pgfscope}
  \begin{pgfscope}
    \pgfsetlinewidth{0.5727bp}
    \definecolor{sc}{rgb}{0.0000,0.0000,0.0000}
    \pgfsetstrokecolor{sc}
    \pgfsetmiterjoin
    \pgfsetbuttcap
    \pgfpathqmoveto{7.0000bp}{56.2500bp}
    \pgfpathqlineto{7.0000bp}{55.7500bp}
    \pgfusepathqstroke
  \end{pgfscope}
  \begin{pgfscope}
    \definecolor{fc}{rgb}{0.0000,0.0000,0.0000}
    \pgfsetfillcolor{fc}
    \pgftransformshift{\pgfqpoint{0.0000bp}{87.9500bp}}
    \pgftransformscale{0.1250}
    \pgftext[base,left]{$\mathbb{L}_A$}
  \end{pgfscope}
  \begin{pgfscope}
    \pgfsetlinewidth{0.5727bp}
    \definecolor{sc}{rgb}{0.0000,0.0000,0.0000}
    \pgfsetstrokecolor{sc}
    \pgfsetmiterjoin
    \pgfsetbuttcap
    \pgfpathqmoveto{2.0000bp}{88.2500bp}
    \pgfpathqlineto{1.8000bp}{88.2500bp}
    \pgfusepathqstroke
  \end{pgfscope}
  \begin{pgfscope}
    \pgfsetlinewidth{0.5727bp}
    \definecolor{sc}{rgb}{0.0000,0.0000,0.0000}
    \pgfsetstrokecolor{sc}
    \pgfsetmiterjoin
    \pgfsetbuttcap
    \pgfpathqmoveto{2.0000bp}{56.2500bp}
    \pgfpathqlineto{2.0000bp}{151.2500bp}
    \pgfusepathqstroke
  \end{pgfscope}
  \begin{pgfscope}
    \pgfsetlinewidth{0.5727bp}
    \definecolor{sc}{rgb}{0.0000,0.0000,0.0000}
    \pgfsetstrokecolor{sc}
    \pgfsetmiterjoin
    \pgfsetbuttcap
    \pgfpathqmoveto{2.0000bp}{56.2500bp}
    \pgfpathqlineto{200.0000bp}{56.2500bp}
    \pgfusepathqstroke
  \end{pgfscope}
\end{pgfpicture}

        \label{fig:ex:ca:hgma:ex:move-h}
    \caption{push-v-goal preconditions}\label{fig:ex:ca:hgma:ex:disconnected}
\end{figure}
\end{document}
