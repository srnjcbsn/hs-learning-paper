\documentclass[../Master.tex]{subfiles}
\providecommand{\master}{..}
\begin{document}


Some conditional effects can be reduced to a smaller expression that is equally restrictive.

For all binding sets $b$ of a hypergraph $H$, the following holds:

For any two subsets $b_1$ and $b_2$ of $b$ it holds that: If there exists a vertex transformation function $t$ such that $t\left[b_1\right] \subseteq b_2$, then $b_1$ is redundant, ie. $H$ is equally expressive with or withouth the subgraph described by $b_1$.

\begin{example}\label{ex:ca:hgma:ex:collapsing}

	Consider an action $a$ applied in state $s_0$, resulting in state $s_1$:
	\begin{equation*}
		s_0 = \left\{ 
			p\left(o_1, o_2\right), p\left(o_1, o_3\right)
		\right\}
	\end{equation*}

	\begin{equation*}
		s_1 = s_0 \cup \left\{
			q\left( o_1 \right)
		\right\}
	\end{equation*}

	Inspecting this state transition will reveal a single pattern $H$ for the effect $q$ (disregarding negative preconditions), visualized in~\figref{fig:ex:ca:hgma:ex:isomorphic}.

    \begin{equation*}
        H = \forall x, y, z : q(x) \quad \textit{when} \quad
            p(x,y) \land p(x,z)
    \end{equation*}

	As the variables $y$ and $z$ are both leaves, and are bound to the effect in the same manner, the subgraphs $p(x,y)$ and $p(x,z)$ are isomorphic, ie.\ identical under identifier renaming. Collapsing the two subgraphs results in the hypergraph $H'$ (visualized in~\figref{fig:ex:ca:hgma:ex:isomorphicReduced}).
	
    \begin{equation*}
        H' = \forall x, y : q(x) \quad \textit{when} \quad p(x,y)
    \end{equation*}

	Notice that $H$ and $H'$ are equally restrictive: If, for a given set of objects $O$, and a state $s$, there exists a substitution $\delta_1 : \left\{x, y, z\right\} \rightarrow O$ such that $\delta\left[p(x,y), p(x,z)\right] \vdash s$, then there also exists a substitution $\delta_1 : \left\{x, y\right\} \rightarrow O$ such that $\delta_2\left[p(x,y)\right] \vdash s$

	\begin{figure}
        \centering
        \hfill
        \begin{subfigure}[b]{0.4\textwidth}
            \centering
            \resizebox{0.7\linewidth}{!}{\begin{pgfpicture}
  \pgfpathrectangle{\pgfpointorigin}{\pgfqpoint{200.0000bp}{200.0000bp}}
  \pgfusepath{use as bounding box}
  \begin{pgfscope}
    \definecolor{fc}{rgb}{0.0000,0.0000,0.0000}
    \pgfsetfillcolor{fc}
    \pgfsetlinewidth{0.8000bp}
    \definecolor{sc}{rgb}{0.0000,0.0000,0.0000}
    \pgfsetstrokecolor{sc}
    \pgfsetmiterjoin
    \pgfsetbuttcap
    \pgfpathqmoveto{57.1429bp}{28.5714bp}
    \pgfpathqcurveto{57.1429bp}{44.3510bp}{44.3510bp}{57.1429bp}{28.5714bp}{57.1429bp}
    \pgfpathqcurveto{12.7919bp}{57.1429bp}{0.0000bp}{44.3510bp}{0.0000bp}{28.5714bp}
    \pgfpathqcurveto{-0.0000bp}{12.7919bp}{12.7919bp}{0.0000bp}{28.5714bp}{0.0000bp}
    \pgfpathqcurveto{44.3510bp}{-0.0000bp}{57.1429bp}{12.7919bp}{57.1429bp}{28.5714bp}
    \pgfpathclose
    \pgfusepathqfillstroke
  \end{pgfscope}
  \begin{pgfscope}
    \definecolor{fc}{rgb}{1.0000,1.0000,1.0000}
    \pgfsetfillcolor{fc}
    \pgfsetlinewidth{0.8000bp}
    \definecolor{sc}{rgb}{1.0000,1.0000,1.0000}
    \pgfsetstrokecolor{sc}
    \pgfsetmiterjoin
    \pgfsetbuttcap
    \pgfpathqmoveto{55.7143bp}{28.5714bp}
    \pgfpathqcurveto{55.7143bp}{43.5620bp}{43.5620bp}{55.7143bp}{28.5714bp}{55.7143bp}
    \pgfpathqcurveto{13.5808bp}{55.7143bp}{1.4286bp}{43.5620bp}{1.4286bp}{28.5714bp}
    \pgfpathqcurveto{1.4286bp}{13.5808bp}{13.5808bp}{1.4286bp}{28.5714bp}{1.4286bp}
    \pgfpathqcurveto{43.5620bp}{1.4286bp}{55.7143bp}{13.5808bp}{55.7143bp}{28.5714bp}
    \pgfpathclose
    \pgfusepathqfillstroke
  \end{pgfscope}
  \begin{pgfscope}
    \definecolor{fc}{rgb}{0.0000,0.0000,0.0000}
    \pgfsetfillcolor{fc}
    \pgfsetlinewidth{0.8000bp}
    \definecolor{sc}{rgb}{0.0000,0.0000,0.0000}
    \pgfsetstrokecolor{sc}
    \pgfsetmiterjoin
    \pgfsetbuttcap
    \pgfpathqmoveto{200.0000bp}{171.4286bp}
    \pgfpathqcurveto{200.0000bp}{187.2081bp}{187.2081bp}{200.0000bp}{171.4286bp}{200.0000bp}
    \pgfpathqcurveto{155.6490bp}{200.0000bp}{142.8571bp}{187.2081bp}{142.8571bp}{171.4286bp}
    \pgfpathqcurveto{142.8571bp}{155.6490bp}{155.6490bp}{142.8571bp}{171.4286bp}{142.8571bp}
    \pgfpathqcurveto{187.2081bp}{142.8571bp}{200.0000bp}{155.6490bp}{200.0000bp}{171.4286bp}
    \pgfpathclose
    \pgfusepathqfillstroke
  \end{pgfscope}
  \begin{pgfscope}
    \definecolor{fc}{rgb}{1.0000,1.0000,1.0000}
    \pgfsetfillcolor{fc}
    \pgfsetlinewidth{0.8000bp}
    \definecolor{sc}{rgb}{1.0000,1.0000,1.0000}
    \pgfsetstrokecolor{sc}
    \pgfsetmiterjoin
    \pgfsetbuttcap
    \pgfpathqmoveto{198.5714bp}{171.4286bp}
    \pgfpathqcurveto{198.5714bp}{186.4192bp}{186.4192bp}{198.5714bp}{171.4286bp}{198.5714bp}
    \pgfpathqcurveto{156.4380bp}{198.5714bp}{144.2857bp}{186.4192bp}{144.2857bp}{171.4286bp}
    \pgfpathqcurveto{144.2857bp}{156.4380bp}{156.4380bp}{144.2857bp}{171.4286bp}{144.2857bp}
    \pgfpathqcurveto{186.4192bp}{144.2857bp}{198.5714bp}{156.4380bp}{198.5714bp}{171.4286bp}
    \pgfpathclose
    \pgfusepathqfillstroke
  \end{pgfscope}
  \begin{pgfscope}
    \definecolor{fc}{rgb}{0.0000,0.0000,0.0000}
    \pgfsetfillcolor{fc}
    \pgfsetlinewidth{0.8000bp}
    \definecolor{sc}{rgb}{0.0000,0.0000,0.0000}
    \pgfsetstrokecolor{sc}
    \pgfsetmiterjoin
    \pgfsetbuttcap
    \pgfpathqmoveto{-0.0000bp}{100.0000bp}
    \pgfpathqcurveto{-0.0000bp}{84.2204bp}{12.7919bp}{71.4286bp}{28.5714bp}{71.4286bp}
    \pgfpathqcurveto{44.3510bp}{71.4286bp}{57.1429bp}{84.2204bp}{57.1429bp}{100.0000bp}
    \pgfpathqlineto{57.1429bp}{171.4286bp}
    \pgfpathqcurveto{57.1429bp}{187.2081bp}{44.3510bp}{200.0000bp}{28.5714bp}{200.0000bp}
    \pgfpathqcurveto{12.7919bp}{200.0000bp}{0.0000bp}{187.2081bp}{0.0000bp}{171.4286bp}
    \pgfpathqcurveto{-0.0000bp}{155.6490bp}{12.7919bp}{142.8571bp}{28.5714bp}{142.8571bp}
    \pgfpathqlineto{100.0000bp}{142.8571bp}
    \pgfpathqcurveto{115.7796bp}{142.8571bp}{128.5714bp}{155.6490bp}{128.5714bp}{171.4286bp}
    \pgfpathqcurveto{128.5714bp}{187.2081bp}{115.7796bp}{200.0000bp}{100.0000bp}{200.0000bp}
    \pgfpathqlineto{28.5714bp}{200.0000bp}
    \pgfpathqcurveto{12.7919bp}{200.0000bp}{0.0000bp}{187.2081bp}{0.0000bp}{171.4286bp}
    \pgfpathqlineto{-0.0000bp}{100.0000bp}
    \pgfpathclose
    \pgfusepathqfillstroke
  \end{pgfscope}
  \begin{pgfscope}
    \definecolor{fc}{rgb}{1.0000,1.0000,1.0000}
    \pgfsetfillcolor{fc}
    \pgfsetlinewidth{0.8000bp}
    \definecolor{sc}{rgb}{1.0000,1.0000,1.0000}
    \pgfsetstrokecolor{sc}
    \pgfsetmiterjoin
    \pgfsetbuttcap
    \pgfpathqmoveto{1.4286bp}{100.0000bp}
    \pgfpathqcurveto{1.4286bp}{85.0094bp}{13.5808bp}{72.8571bp}{28.5714bp}{72.8571bp}
    \pgfpathqcurveto{43.5620bp}{72.8571bp}{55.7143bp}{85.0094bp}{55.7143bp}{100.0000bp}
    \pgfpathqlineto{55.7143bp}{171.4286bp}
    \pgfpathqcurveto{55.7143bp}{186.4192bp}{43.5620bp}{198.5714bp}{28.5714bp}{198.5714bp}
    \pgfpathqcurveto{13.5808bp}{198.5714bp}{1.4286bp}{186.4192bp}{1.4286bp}{171.4286bp}
    \pgfpathqcurveto{1.4286bp}{156.4380bp}{13.5808bp}{144.2857bp}{28.5714bp}{144.2857bp}
    \pgfpathqlineto{100.0000bp}{144.2857bp}
    \pgfpathqcurveto{114.9906bp}{144.2857bp}{127.1429bp}{156.4380bp}{127.1429bp}{171.4286bp}
    \pgfpathqcurveto{127.1429bp}{186.4192bp}{114.9906bp}{198.5714bp}{100.0000bp}{198.5714bp}
    \pgfpathqlineto{28.5714bp}{198.5714bp}
    \pgfpathqcurveto{13.5808bp}{198.5714bp}{1.4286bp}{186.4192bp}{1.4286bp}{171.4286bp}
    \pgfpathqlineto{1.4286bp}{100.0000bp}
    \pgfpathclose
    \pgfusepathqfillstroke
  \end{pgfscope}
  \begin{pgfscope}
    \definecolor{fc}{rgb}{0.0000,0.0000,0.0000}
    \pgfsetfillcolor{fc}
    \pgftransformshift{\pgfqpoint{35.7143bp}{35.7143bp}}
    \pgftransformscale{1.7857}
    \pgftext[base,left]{$p_2$}
  \end{pgfscope}
  \begin{pgfscope}
    \definecolor{fc}{rgb}{0.0000,0.0000,0.0000}
    \pgfsetfillcolor{fc}
    \pgfsetlinewidth{0.8000bp}
    \definecolor{sc}{rgb}{0.0000,0.0000,0.0000}
    \pgfsetstrokecolor{sc}
    \pgfsetmiterjoin
    \pgfsetbuttcap
    \pgfpathqmoveto{32.1429bp}{28.5714bp}
    \pgfpathqcurveto{32.1429bp}{30.5439bp}{30.5439bp}{32.1429bp}{28.5714bp}{32.1429bp}
    \pgfpathqcurveto{26.5990bp}{32.1429bp}{25.0000bp}{30.5439bp}{25.0000bp}{28.5714bp}
    \pgfpathqcurveto{25.0000bp}{26.5990bp}{26.5990bp}{25.0000bp}{28.5714bp}{25.0000bp}
    \pgfpathqcurveto{30.5439bp}{25.0000bp}{32.1429bp}{26.5990bp}{32.1429bp}{28.5714bp}
    \pgfpathclose
    \pgfusepathqfillstroke
  \end{pgfscope}
  \begin{pgfscope}
    \definecolor{fc}{rgb}{0.0000,0.0000,0.0000}
    \pgfsetfillcolor{fc}
    \pgftransformshift{\pgfqpoint{35.7143bp}{107.1429bp}}
    \pgftransformscale{1.7857}
    \pgftext[base,left]{$p_1$}
  \end{pgfscope}
  \begin{pgfscope}
    \definecolor{fc}{rgb}{0.0000,0.0000,0.0000}
    \pgfsetfillcolor{fc}
    \pgfsetlinewidth{0.8000bp}
    \definecolor{sc}{rgb}{0.0000,0.0000,0.0000}
    \pgfsetstrokecolor{sc}
    \pgfsetmiterjoin
    \pgfsetbuttcap
    \pgfpathqmoveto{32.1429bp}{100.0000bp}
    \pgfpathqcurveto{32.1429bp}{101.9724bp}{30.5439bp}{103.5714bp}{28.5714bp}{103.5714bp}
    \pgfpathqcurveto{26.5990bp}{103.5714bp}{25.0000bp}{101.9724bp}{25.0000bp}{100.0000bp}
    \pgfpathqcurveto{25.0000bp}{98.0276bp}{26.5990bp}{96.4286bp}{28.5714bp}{96.4286bp}
    \pgfpathqcurveto{30.5439bp}{96.4286bp}{32.1429bp}{98.0276bp}{32.1429bp}{100.0000bp}
    \pgfpathclose
    \pgfusepathqfillstroke
  \end{pgfscope}
  \begin{pgfscope}
    \pgfsetlinewidth{1.5000bp}
    \definecolor{sc}{rgb}{0.0000,0.0000,0.0000}
    \pgfsetstrokecolor{sc}
    \pgfsetmiterjoin
    \pgfsetbuttcap
    \pgfpathqmoveto{28.5714bp}{100.0000bp}
    \pgfpathqlineto{28.5714bp}{28.5714bp}
    \pgfusepathqstroke
  \end{pgfscope}
  \begin{pgfscope}
    \definecolor{fc}{rgb}{0.0000,0.0000,0.0000}
    \pgfsetfillcolor{fc}
    \pgfusepathqfill
  \end{pgfscope}
  \begin{pgfscope}
    \definecolor{fc}{rgb}{0.0000,0.0000,0.0000}
    \pgfsetfillcolor{fc}
    \pgfusepathqfill
  \end{pgfscope}
  \begin{pgfscope}
    \definecolor{fc}{rgb}{0.0000,0.0000,0.0000}
    \pgfsetfillcolor{fc}
    \pgfusepathqfill
  \end{pgfscope}
  \begin{pgfscope}
    \definecolor{fc}{rgb}{0.0000,0.0000,0.0000}
    \pgfsetfillcolor{fc}
    \pgfusepathqfill
  \end{pgfscope}
  \begin{pgfscope}
    \definecolor{fc}{rgb}{0.0000,0.0000,0.0000}
    \pgfsetfillcolor{fc}
    \pgftransformshift{\pgfqpoint{178.5714bp}{178.5714bp}}
    \pgftransformscale{1.7857}
    \pgftext[base,left]{$p_2$}
  \end{pgfscope}
  \begin{pgfscope}
    \definecolor{fc}{rgb}{0.0000,0.0000,0.0000}
    \pgfsetfillcolor{fc}
    \pgfsetlinewidth{0.8000bp}
    \definecolor{sc}{rgb}{0.0000,0.0000,0.0000}
    \pgfsetstrokecolor{sc}
    \pgfsetmiterjoin
    \pgfsetbuttcap
    \pgfpathqmoveto{175.0000bp}{171.4286bp}
    \pgfpathqcurveto{175.0000bp}{173.4010bp}{173.4010bp}{175.0000bp}{171.4286bp}{175.0000bp}
    \pgfpathqcurveto{169.4561bp}{175.0000bp}{167.8571bp}{173.4010bp}{167.8571bp}{171.4286bp}
    \pgfpathqcurveto{167.8571bp}{169.4561bp}{169.4561bp}{167.8571bp}{171.4286bp}{167.8571bp}
    \pgfpathqcurveto{173.4010bp}{167.8571bp}{175.0000bp}{169.4561bp}{175.0000bp}{171.4286bp}
    \pgfpathclose
    \pgfusepathqfillstroke
  \end{pgfscope}
  \begin{pgfscope}
    \definecolor{fc}{rgb}{0.0000,0.0000,0.0000}
    \pgfsetfillcolor{fc}
    \pgftransformshift{\pgfqpoint{107.1429bp}{178.5714bp}}
    \pgftransformscale{1.7857}
    \pgftext[base,left]{$p_1$}
  \end{pgfscope}
  \begin{pgfscope}
    \definecolor{fc}{rgb}{0.0000,0.0000,0.0000}
    \pgfsetfillcolor{fc}
    \pgfsetlinewidth{0.8000bp}
    \definecolor{sc}{rgb}{0.0000,0.0000,0.0000}
    \pgfsetstrokecolor{sc}
    \pgfsetmiterjoin
    \pgfsetbuttcap
    \pgfpathqmoveto{103.5714bp}{171.4286bp}
    \pgfpathqcurveto{103.5714bp}{173.4010bp}{101.9724bp}{175.0000bp}{100.0000bp}{175.0000bp}
    \pgfpathqcurveto{98.0276bp}{175.0000bp}{96.4286bp}{173.4010bp}{96.4286bp}{171.4286bp}
    \pgfpathqcurveto{96.4286bp}{169.4561bp}{98.0276bp}{167.8571bp}{100.0000bp}{167.8571bp}
    \pgfpathqcurveto{101.9724bp}{167.8571bp}{103.5714bp}{169.4561bp}{103.5714bp}{171.4286bp}
    \pgfpathclose
    \pgfusepathqfillstroke
  \end{pgfscope}
  \begin{pgfscope}
    \pgfsetlinewidth{1.5000bp}
    \definecolor{sc}{rgb}{0.0000,0.0000,0.0000}
    \pgfsetstrokecolor{sc}
    \pgfsetmiterjoin
    \pgfsetbuttcap
    \pgfpathqmoveto{100.0000bp}{171.4286bp}
    \pgfpathqlineto{171.4286bp}{171.4286bp}
    \pgfusepathqstroke
  \end{pgfscope}
  \begin{pgfscope}
    \definecolor{fc}{rgb}{0.0000,0.0000,0.0000}
    \pgfsetfillcolor{fc}
    \pgfusepathqfill
  \end{pgfscope}
  \begin{pgfscope}
    \definecolor{fc}{rgb}{0.0000,0.0000,0.0000}
    \pgfsetfillcolor{fc}
    \pgfusepathqfill
  \end{pgfscope}
  \begin{pgfscope}
    \definecolor{fc}{rgb}{0.0000,0.0000,0.0000}
    \pgfsetfillcolor{fc}
    \pgfusepathqfill
  \end{pgfscope}
  \begin{pgfscope}
    \definecolor{fc}{rgb}{0.0000,0.0000,0.0000}
    \pgfsetfillcolor{fc}
    \pgfusepathqfill
  \end{pgfscope}
  \begin{pgfscope}
    \definecolor{fc}{rgb}{0.0000,0.0000,0.0000}
    \pgfsetfillcolor{fc}
    \pgftransformshift{\pgfqpoint{35.7143bp}{178.5714bp}}
    \pgftransformscale{1.7857}
    \pgftext[base,left]{$q_1$}
  \end{pgfscope}
  \begin{pgfscope}
    \definecolor{fc}{rgb}{0.0000,0.0000,0.0000}
    \pgfsetfillcolor{fc}
    \pgfsetlinewidth{0.8000bp}
    \definecolor{sc}{rgb}{0.0000,0.0000,0.0000}
    \pgfsetstrokecolor{sc}
    \pgfsetmiterjoin
    \pgfsetbuttcap
    \pgfpathqmoveto{32.1429bp}{171.4286bp}
    \pgfpathqcurveto{32.1429bp}{173.4010bp}{30.5439bp}{175.0000bp}{28.5714bp}{175.0000bp}
    \pgfpathqcurveto{26.5990bp}{175.0000bp}{25.0000bp}{173.4010bp}{25.0000bp}{171.4286bp}
    \pgfpathqcurveto{25.0000bp}{169.4561bp}{26.5990bp}{167.8571bp}{28.5714bp}{167.8571bp}
    \pgfpathqcurveto{30.5439bp}{167.8571bp}{32.1429bp}{169.4561bp}{32.1429bp}{171.4286bp}
    \pgfpathclose
    \pgfusepathqfillstroke
  \end{pgfscope}
\end{pgfpicture}
}
            \caption{Hypergraph $H$ before collapsing}
            \label{fig:ex:ca:hgma:ex:isomorphic}
        \end{subfigure}%
        \hfill%
        \begin{subfigure}[b]{0.4\textwidth}
            \centering
            \resizebox{0.75\linewidth}{!}{\begin{pgfpicture}
  \pgfpathrectangle{\pgfpointorigin}{\pgfqpoint{200.0000bp}{200.0000bp}}
  \pgfusepath{use as bounding box}
  \begin{pgfscope}
    \definecolor{fc}{rgb}{0.0000,0.0000,0.0000}
    \pgfsetfillcolor{fc}
    \pgfsetlinewidth{0.5000bp}
    \definecolor{sc}{rgb}{0.0000,0.0000,0.0000}
    \pgfsetstrokecolor{sc}
    \pgfsetmiterjoin
    \pgfsetbuttcap
    \pgfpathqmoveto{200.0000bp}{100.0000bp}
    \pgfpathqcurveto{200.0000bp}{120.7107bp}{183.2107bp}{137.5000bp}{162.5000bp}{137.5000bp}
    \pgfpathqcurveto{141.7893bp}{137.5000bp}{125.0000bp}{120.7107bp}{125.0000bp}{100.0000bp}
    \pgfpathqcurveto{125.0000bp}{79.2893bp}{141.7893bp}{62.5000bp}{162.5000bp}{62.5000bp}
    \pgfpathqcurveto{183.2107bp}{62.5000bp}{200.0000bp}{79.2893bp}{200.0000bp}{100.0000bp}
    \pgfpathclose
    \pgfusepathqfillstroke
  \end{pgfscope}
  \begin{pgfscope}
    \definecolor{fc}{rgb}{1.0000,1.0000,1.0000}
    \pgfsetfillcolor{fc}
    \pgfsetlinewidth{0.5000bp}
    \definecolor{sc}{rgb}{1.0000,1.0000,1.0000}
    \pgfsetstrokecolor{sc}
    \pgfsetmiterjoin
    \pgfsetbuttcap
    \pgfpathqmoveto{198.7500bp}{100.0000bp}
    \pgfpathqcurveto{198.7500bp}{120.0203bp}{182.5203bp}{136.2500bp}{162.5000bp}{136.2500bp}
    \pgfpathqcurveto{142.4797bp}{136.2500bp}{126.2500bp}{120.0203bp}{126.2500bp}{100.0000bp}
    \pgfpathqcurveto{126.2500bp}{79.9797bp}{142.4797bp}{63.7500bp}{162.5000bp}{63.7500bp}
    \pgfpathqcurveto{182.5203bp}{63.7500bp}{198.7500bp}{79.9797bp}{198.7500bp}{100.0000bp}
    \pgfpathclose
    \pgfusepathqfillstroke
  \end{pgfscope}
  \begin{pgfscope}
    \definecolor{fc}{rgb}{0.0000,0.0000,0.0000}
    \pgfsetfillcolor{fc}
    \pgfsetlinewidth{0.5000bp}
    \definecolor{sc}{rgb}{0.0000,0.0000,0.0000}
    \pgfsetstrokecolor{sc}
    \pgfsetmiterjoin
    \pgfsetbuttcap
    \pgfpathqmoveto{37.5000bp}{137.5000bp}
    \pgfpathqcurveto{16.7893bp}{137.5000bp}{0.0000bp}{120.7107bp}{0.0000bp}{100.0000bp}
    \pgfpathqcurveto{-0.0000bp}{79.2893bp}{16.7893bp}{62.5000bp}{37.5000bp}{62.5000bp}
    \pgfpathqlineto{100.0000bp}{62.5000bp}
    \pgfpathqcurveto{120.7107bp}{62.5000bp}{137.5000bp}{79.2893bp}{137.5000bp}{100.0000bp}
    \pgfpathqcurveto{137.5000bp}{120.7107bp}{120.7107bp}{137.5000bp}{100.0000bp}{137.5000bp}
    \pgfpathqlineto{37.5000bp}{137.5000bp}
    \pgfpathclose
    \pgfusepathqfillstroke
  \end{pgfscope}
  \begin{pgfscope}
    \definecolor{fc}{rgb}{1.0000,1.0000,1.0000}
    \pgfsetfillcolor{fc}
    \pgfsetlinewidth{0.5000bp}
    \definecolor{sc}{rgb}{1.0000,1.0000,1.0000}
    \pgfsetstrokecolor{sc}
    \pgfsetmiterjoin
    \pgfsetbuttcap
    \pgfpathqmoveto{37.5000bp}{136.2500bp}
    \pgfpathqcurveto{17.4797bp}{136.2500bp}{1.2500bp}{120.0203bp}{1.2500bp}{100.0000bp}
    \pgfpathqcurveto{1.2500bp}{79.9797bp}{17.4797bp}{63.7500bp}{37.5000bp}{63.7500bp}
    \pgfpathqlineto{100.0000bp}{63.7500bp}
    \pgfpathqcurveto{120.0203bp}{63.7500bp}{136.2500bp}{79.9797bp}{136.2500bp}{100.0000bp}
    \pgfpathqcurveto{136.2500bp}{120.0203bp}{120.0203bp}{136.2500bp}{100.0000bp}{136.2500bp}
    \pgfpathqlineto{37.5000bp}{136.2500bp}
    \pgfpathclose
    \pgfusepathqfillstroke
  \end{pgfscope}
  \begin{pgfscope}
    \definecolor{fc}{rgb}{0.0000,0.0000,0.0000}
    \pgfsetfillcolor{fc}
    \pgftransformshift{\pgfqpoint{168.7500bp}{106.2500bp}}
    \pgftransformscale{1.5625}
    \pgftext[base,left]{$p_2$}
  \end{pgfscope}
  \begin{pgfscope}
    \definecolor{fc}{rgb}{0.0000,0.0000,0.0000}
    \pgfsetfillcolor{fc}
    \pgfsetlinewidth{0.5000bp}
    \definecolor{sc}{rgb}{0.0000,0.0000,0.0000}
    \pgfsetstrokecolor{sc}
    \pgfsetmiterjoin
    \pgfsetbuttcap
    \pgfpathqmoveto{165.6250bp}{100.0000bp}
    \pgfpathqcurveto{165.6250bp}{101.7259bp}{164.2259bp}{103.1250bp}{162.5000bp}{103.1250bp}
    \pgfpathqcurveto{160.7741bp}{103.1250bp}{159.3750bp}{101.7259bp}{159.3750bp}{100.0000bp}
    \pgfpathqcurveto{159.3750bp}{98.2741bp}{160.7741bp}{96.8750bp}{162.5000bp}{96.8750bp}
    \pgfpathqcurveto{164.2259bp}{96.8750bp}{165.6250bp}{98.2741bp}{165.6250bp}{100.0000bp}
    \pgfpathclose
    \pgfusepathqfillstroke
  \end{pgfscope}
  \begin{pgfscope}
    \definecolor{fc}{rgb}{0.0000,0.0000,0.0000}
    \pgfsetfillcolor{fc}
    \pgftransformshift{\pgfqpoint{106.2500bp}{106.2500bp}}
    \pgftransformscale{1.5625}
    \pgftext[base,left]{$p_1$}
  \end{pgfscope}
  \begin{pgfscope}
    \definecolor{fc}{rgb}{0.0000,0.0000,0.0000}
    \pgfsetfillcolor{fc}
    \pgfsetlinewidth{0.5000bp}
    \definecolor{sc}{rgb}{0.0000,0.0000,0.0000}
    \pgfsetstrokecolor{sc}
    \pgfsetmiterjoin
    \pgfsetbuttcap
    \pgfpathqmoveto{103.1250bp}{100.0000bp}
    \pgfpathqcurveto{103.1250bp}{101.7259bp}{101.7259bp}{103.1250bp}{100.0000bp}{103.1250bp}
    \pgfpathqcurveto{98.2741bp}{103.1250bp}{96.8750bp}{101.7259bp}{96.8750bp}{100.0000bp}
    \pgfpathqcurveto{96.8750bp}{98.2741bp}{98.2741bp}{96.8750bp}{100.0000bp}{96.8750bp}
    \pgfpathqcurveto{101.7259bp}{96.8750bp}{103.1250bp}{98.2741bp}{103.1250bp}{100.0000bp}
    \pgfpathclose
    \pgfusepathqfillstroke
  \end{pgfscope}
  \begin{pgfscope}
    \pgfsetlinewidth{0.9186bp}
    \definecolor{sc}{rgb}{0.0000,0.0000,0.0000}
    \pgfsetstrokecolor{sc}
    \pgfsetmiterjoin
    \pgfsetbuttcap
    \pgfpathqmoveto{100.0000bp}{100.0000bp}
    \pgfpathqlineto{162.5000bp}{100.0000bp}
    \pgfusepathqstroke
  \end{pgfscope}
  \begin{pgfscope}
    \definecolor{fc}{rgb}{0.0000,0.0000,0.0000}
    \pgfsetfillcolor{fc}
    \pgfusepathqfill
  \end{pgfscope}
  \begin{pgfscope}
    \definecolor{fc}{rgb}{0.0000,0.0000,0.0000}
    \pgfsetfillcolor{fc}
    \pgfusepathqfill
  \end{pgfscope}
  \begin{pgfscope}
    \definecolor{fc}{rgb}{0.0000,0.0000,0.0000}
    \pgfsetfillcolor{fc}
    \pgfusepathqfill
  \end{pgfscope}
  \begin{pgfscope}
    \definecolor{fc}{rgb}{0.0000,0.0000,0.0000}
    \pgfsetfillcolor{fc}
    \pgfusepathqfill
  \end{pgfscope}
  \begin{pgfscope}
    \definecolor{fc}{rgb}{0.0000,0.0000,0.0000}
    \pgfsetfillcolor{fc}
    \pgftransformshift{\pgfqpoint{43.7500bp}{106.2500bp}}
    \pgftransformscale{1.5625}
    \pgftext[base,left]{$q_1$}
  \end{pgfscope}
  \begin{pgfscope}
    \definecolor{fc}{rgb}{0.0000,0.0000,0.0000}
    \pgfsetfillcolor{fc}
    \pgfsetlinewidth{0.5000bp}
    \definecolor{sc}{rgb}{0.0000,0.0000,0.0000}
    \pgfsetstrokecolor{sc}
    \pgfsetmiterjoin
    \pgfsetbuttcap
    \pgfpathqmoveto{40.6250bp}{100.0000bp}
    \pgfpathqcurveto{40.6250bp}{101.7259bp}{39.2259bp}{103.1250bp}{37.5000bp}{103.1250bp}
    \pgfpathqcurveto{35.7741bp}{103.1250bp}{34.3750bp}{101.7259bp}{34.3750bp}{100.0000bp}
    \pgfpathqcurveto{34.3750bp}{98.2741bp}{35.7741bp}{96.8750bp}{37.5000bp}{96.8750bp}
    \pgfpathqcurveto{39.2259bp}{96.8750bp}{40.6250bp}{98.2741bp}{40.6250bp}{100.0000bp}
    \pgfpathclose
    \pgfusepathqfillstroke
  \end{pgfscope}
\end{pgfpicture}
}
            \caption{Hypergraph $H$ after collapsing.}
            \label{fig:ex:ca:hgma:ex:isomorphicReduced}
        \end{subfigure}
		\caption{Hyper graphs for example~\ref{fig:ex:ca:hgma:ex:collapsing}}\label{fig:ex:ca:hgma:ex:collapsing}
        \hfill
    \end{figure}
\end{example}

\end{document}
