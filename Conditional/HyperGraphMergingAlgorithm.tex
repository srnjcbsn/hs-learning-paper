\documentclass[../Master.tex]{subfiles}
\providecommand{\master}{..}

\begin{document}

Formally, a vertex in a such a hypergraph is a pair $\left(n,p_i \right)$

Given two hypergraphs $H_1$ and $H_2$ describing the same conditional effect $q$:

Starting with the hyperedge describing the effect $q$, ie. $\left\{ q_1, \dots, q_{|q|}  \right\}$:

Each argument $q_i$ of $q$ is now member of exactly one \emph{arg}-hyperedge in  both $H_1$ and $H_2$, denoted $e_1(q_i)$ and $e_2(q_i)$. For all nodes $v_j \in e_1(q_i)$, it holds that  if $v_j$ does not exist in $e_2(q_i)$, then it is not a precondition that the $i$'th argument of $q$ is the same as the $j$'th argument of $v$.

\begin{proposition}
    All vertices in a hypergraph $H$ is member of exactly one predicate edge and one binding edge.
\end{proposition}

\begin{definition}[Hyper edge intersection]
    The intersection of two hyper edges $e_1$ and $e_2$ is a new hyperedge where the number of vertices with name $v_i$ is equal to the number of vertices with name $v_i$ in $e_1$ and $e_2$, multiplied. Formally:
    \begin{equation*}
        e_1 \sqcap e_2 = \left\{
            f(v, t) \; | \;
                v \in e_1 \land t \in e_2 \land v \approx t
        \right\}
    \end{equation*}
\end{definition}

\begin{definition}
    Given two vertices with the same name, ie $v = \left(n, p_i \right)$ and $t = \left( m, p_i \right)$, $f(v,t)$ yields a new vertex with that name and an identifier based on $v$'s and $t$'s. $f$ must be injective, such that $f^{-1}(f(v,t)) = (v,t)$. In the following, we will use the shorthands $f^{-1}_1(f(v,t)) = v$ and $f^{-1}_2(f(v,t)) = t$.
\end{definition}
 A simple implementation of $f$ is tupling the identifiers of the arguement, ie:

\begin{equation*}
    f\left( \left(n, p_i \right), \left(m, p_i \right) \right) =
        \left( (n,m), p_i \right)
\end{equation*}
Then, $f$ is clearly inversible, and $f^{-1}$ can be computed by decomposing the tuple. Note that with this implementation, repeated application of $f$ requires linear space.


\begin{proposition}
    Given two binding edges $e_1$ and $e_2$ from hypergraphs $H_1$ and $H_2$, respectively; any node $v$ that only exists in \emph{either} $e_1$ or $e_2$ is discarded as a precondition, formally:
    \begin{equation*}
        v \notin e_1 \lor v \notin e_2 \rightarrow v \notin e_1 \sqcap e_2
    \end{equation*}
\end{proposition}

\begin{example}
    \begin{equation*}
        H = \forall x, y, z : q(x) \quad \textit{when} \quad
            p(x,y) \land p(x,z)
    \end{equation*}

    \begin{equation}
        \forall x, y : q(x) \quad \textit{when} \quad p(x,y)
    \end{equation}

    \begin{figure}
        \centering
        \begin{subfigure}[b]{0.4\textwidth}
            \centering
            \resizebox{\linewidth}{!}{\begin{pgfpicture}
  \pgfpathrectangle{\pgfpointorigin}{\pgfqpoint{200.0000bp}{200.0000bp}}
  \pgfusepath{use as bounding box}
  \begin{pgfscope}
    \definecolor{fc}{rgb}{0.0000,0.0000,0.0000}
    \pgfsetfillcolor{fc}
    \pgfsetlinewidth{0.8000bp}
    \definecolor{sc}{rgb}{0.0000,0.0000,0.0000}
    \pgfsetstrokecolor{sc}
    \pgfsetmiterjoin
    \pgfsetbuttcap
    \pgfpathqmoveto{57.1429bp}{28.5714bp}
    \pgfpathqcurveto{57.1429bp}{44.3510bp}{44.3510bp}{57.1429bp}{28.5714bp}{57.1429bp}
    \pgfpathqcurveto{12.7919bp}{57.1429bp}{0.0000bp}{44.3510bp}{0.0000bp}{28.5714bp}
    \pgfpathqcurveto{-0.0000bp}{12.7919bp}{12.7919bp}{0.0000bp}{28.5714bp}{0.0000bp}
    \pgfpathqcurveto{44.3510bp}{-0.0000bp}{57.1429bp}{12.7919bp}{57.1429bp}{28.5714bp}
    \pgfpathclose
    \pgfusepathqfillstroke
  \end{pgfscope}
  \begin{pgfscope}
    \definecolor{fc}{rgb}{1.0000,1.0000,1.0000}
    \pgfsetfillcolor{fc}
    \pgfsetlinewidth{0.8000bp}
    \definecolor{sc}{rgb}{1.0000,1.0000,1.0000}
    \pgfsetstrokecolor{sc}
    \pgfsetmiterjoin
    \pgfsetbuttcap
    \pgfpathqmoveto{55.7143bp}{28.5714bp}
    \pgfpathqcurveto{55.7143bp}{43.5620bp}{43.5620bp}{55.7143bp}{28.5714bp}{55.7143bp}
    \pgfpathqcurveto{13.5808bp}{55.7143bp}{1.4286bp}{43.5620bp}{1.4286bp}{28.5714bp}
    \pgfpathqcurveto{1.4286bp}{13.5808bp}{13.5808bp}{1.4286bp}{28.5714bp}{1.4286bp}
    \pgfpathqcurveto{43.5620bp}{1.4286bp}{55.7143bp}{13.5808bp}{55.7143bp}{28.5714bp}
    \pgfpathclose
    \pgfusepathqfillstroke
  \end{pgfscope}
  \begin{pgfscope}
    \definecolor{fc}{rgb}{0.0000,0.0000,0.0000}
    \pgfsetfillcolor{fc}
    \pgfsetlinewidth{0.8000bp}
    \definecolor{sc}{rgb}{0.0000,0.0000,0.0000}
    \pgfsetstrokecolor{sc}
    \pgfsetmiterjoin
    \pgfsetbuttcap
    \pgfpathqmoveto{200.0000bp}{171.4286bp}
    \pgfpathqcurveto{200.0000bp}{187.2081bp}{187.2081bp}{200.0000bp}{171.4286bp}{200.0000bp}
    \pgfpathqcurveto{155.6490bp}{200.0000bp}{142.8571bp}{187.2081bp}{142.8571bp}{171.4286bp}
    \pgfpathqcurveto{142.8571bp}{155.6490bp}{155.6490bp}{142.8571bp}{171.4286bp}{142.8571bp}
    \pgfpathqcurveto{187.2081bp}{142.8571bp}{200.0000bp}{155.6490bp}{200.0000bp}{171.4286bp}
    \pgfpathclose
    \pgfusepathqfillstroke
  \end{pgfscope}
  \begin{pgfscope}
    \definecolor{fc}{rgb}{1.0000,1.0000,1.0000}
    \pgfsetfillcolor{fc}
    \pgfsetlinewidth{0.8000bp}
    \definecolor{sc}{rgb}{1.0000,1.0000,1.0000}
    \pgfsetstrokecolor{sc}
    \pgfsetmiterjoin
    \pgfsetbuttcap
    \pgfpathqmoveto{198.5714bp}{171.4286bp}
    \pgfpathqcurveto{198.5714bp}{186.4192bp}{186.4192bp}{198.5714bp}{171.4286bp}{198.5714bp}
    \pgfpathqcurveto{156.4380bp}{198.5714bp}{144.2857bp}{186.4192bp}{144.2857bp}{171.4286bp}
    \pgfpathqcurveto{144.2857bp}{156.4380bp}{156.4380bp}{144.2857bp}{171.4286bp}{144.2857bp}
    \pgfpathqcurveto{186.4192bp}{144.2857bp}{198.5714bp}{156.4380bp}{198.5714bp}{171.4286bp}
    \pgfpathclose
    \pgfusepathqfillstroke
  \end{pgfscope}
  \begin{pgfscope}
    \definecolor{fc}{rgb}{0.0000,0.0000,0.0000}
    \pgfsetfillcolor{fc}
    \pgfsetlinewidth{0.8000bp}
    \definecolor{sc}{rgb}{0.0000,0.0000,0.0000}
    \pgfsetstrokecolor{sc}
    \pgfsetmiterjoin
    \pgfsetbuttcap
    \pgfpathqmoveto{-0.0000bp}{100.0000bp}
    \pgfpathqcurveto{-0.0000bp}{84.2204bp}{12.7919bp}{71.4286bp}{28.5714bp}{71.4286bp}
    \pgfpathqcurveto{44.3510bp}{71.4286bp}{57.1429bp}{84.2204bp}{57.1429bp}{100.0000bp}
    \pgfpathqlineto{57.1429bp}{171.4286bp}
    \pgfpathqcurveto{57.1429bp}{187.2081bp}{44.3510bp}{200.0000bp}{28.5714bp}{200.0000bp}
    \pgfpathqcurveto{12.7919bp}{200.0000bp}{0.0000bp}{187.2081bp}{0.0000bp}{171.4286bp}
    \pgfpathqcurveto{-0.0000bp}{155.6490bp}{12.7919bp}{142.8571bp}{28.5714bp}{142.8571bp}
    \pgfpathqlineto{100.0000bp}{142.8571bp}
    \pgfpathqcurveto{115.7796bp}{142.8571bp}{128.5714bp}{155.6490bp}{128.5714bp}{171.4286bp}
    \pgfpathqcurveto{128.5714bp}{187.2081bp}{115.7796bp}{200.0000bp}{100.0000bp}{200.0000bp}
    \pgfpathqlineto{28.5714bp}{200.0000bp}
    \pgfpathqcurveto{12.7919bp}{200.0000bp}{0.0000bp}{187.2081bp}{0.0000bp}{171.4286bp}
    \pgfpathqlineto{-0.0000bp}{100.0000bp}
    \pgfpathclose
    \pgfusepathqfillstroke
  \end{pgfscope}
  \begin{pgfscope}
    \definecolor{fc}{rgb}{1.0000,1.0000,1.0000}
    \pgfsetfillcolor{fc}
    \pgfsetlinewidth{0.8000bp}
    \definecolor{sc}{rgb}{1.0000,1.0000,1.0000}
    \pgfsetstrokecolor{sc}
    \pgfsetmiterjoin
    \pgfsetbuttcap
    \pgfpathqmoveto{1.4286bp}{100.0000bp}
    \pgfpathqcurveto{1.4286bp}{85.0094bp}{13.5808bp}{72.8571bp}{28.5714bp}{72.8571bp}
    \pgfpathqcurveto{43.5620bp}{72.8571bp}{55.7143bp}{85.0094bp}{55.7143bp}{100.0000bp}
    \pgfpathqlineto{55.7143bp}{171.4286bp}
    \pgfpathqcurveto{55.7143bp}{186.4192bp}{43.5620bp}{198.5714bp}{28.5714bp}{198.5714bp}
    \pgfpathqcurveto{13.5808bp}{198.5714bp}{1.4286bp}{186.4192bp}{1.4286bp}{171.4286bp}
    \pgfpathqcurveto{1.4286bp}{156.4380bp}{13.5808bp}{144.2857bp}{28.5714bp}{144.2857bp}
    \pgfpathqlineto{100.0000bp}{144.2857bp}
    \pgfpathqcurveto{114.9906bp}{144.2857bp}{127.1429bp}{156.4380bp}{127.1429bp}{171.4286bp}
    \pgfpathqcurveto{127.1429bp}{186.4192bp}{114.9906bp}{198.5714bp}{100.0000bp}{198.5714bp}
    \pgfpathqlineto{28.5714bp}{198.5714bp}
    \pgfpathqcurveto{13.5808bp}{198.5714bp}{1.4286bp}{186.4192bp}{1.4286bp}{171.4286bp}
    \pgfpathqlineto{1.4286bp}{100.0000bp}
    \pgfpathclose
    \pgfusepathqfillstroke
  \end{pgfscope}
  \begin{pgfscope}
    \definecolor{fc}{rgb}{0.0000,0.0000,0.0000}
    \pgfsetfillcolor{fc}
    \pgftransformshift{\pgfqpoint{35.7143bp}{35.7143bp}}
    \pgftransformscale{1.7857}
    \pgftext[base,left]{$p_2$}
  \end{pgfscope}
  \begin{pgfscope}
    \definecolor{fc}{rgb}{0.0000,0.0000,0.0000}
    \pgfsetfillcolor{fc}
    \pgfsetlinewidth{0.8000bp}
    \definecolor{sc}{rgb}{0.0000,0.0000,0.0000}
    \pgfsetstrokecolor{sc}
    \pgfsetmiterjoin
    \pgfsetbuttcap
    \pgfpathqmoveto{32.1429bp}{28.5714bp}
    \pgfpathqcurveto{32.1429bp}{30.5439bp}{30.5439bp}{32.1429bp}{28.5714bp}{32.1429bp}
    \pgfpathqcurveto{26.5990bp}{32.1429bp}{25.0000bp}{30.5439bp}{25.0000bp}{28.5714bp}
    \pgfpathqcurveto{25.0000bp}{26.5990bp}{26.5990bp}{25.0000bp}{28.5714bp}{25.0000bp}
    \pgfpathqcurveto{30.5439bp}{25.0000bp}{32.1429bp}{26.5990bp}{32.1429bp}{28.5714bp}
    \pgfpathclose
    \pgfusepathqfillstroke
  \end{pgfscope}
  \begin{pgfscope}
    \definecolor{fc}{rgb}{0.0000,0.0000,0.0000}
    \pgfsetfillcolor{fc}
    \pgftransformshift{\pgfqpoint{35.7143bp}{107.1429bp}}
    \pgftransformscale{1.7857}
    \pgftext[base,left]{$p_1$}
  \end{pgfscope}
  \begin{pgfscope}
    \definecolor{fc}{rgb}{0.0000,0.0000,0.0000}
    \pgfsetfillcolor{fc}
    \pgfsetlinewidth{0.8000bp}
    \definecolor{sc}{rgb}{0.0000,0.0000,0.0000}
    \pgfsetstrokecolor{sc}
    \pgfsetmiterjoin
    \pgfsetbuttcap
    \pgfpathqmoveto{32.1429bp}{100.0000bp}
    \pgfpathqcurveto{32.1429bp}{101.9724bp}{30.5439bp}{103.5714bp}{28.5714bp}{103.5714bp}
    \pgfpathqcurveto{26.5990bp}{103.5714bp}{25.0000bp}{101.9724bp}{25.0000bp}{100.0000bp}
    \pgfpathqcurveto{25.0000bp}{98.0276bp}{26.5990bp}{96.4286bp}{28.5714bp}{96.4286bp}
    \pgfpathqcurveto{30.5439bp}{96.4286bp}{32.1429bp}{98.0276bp}{32.1429bp}{100.0000bp}
    \pgfpathclose
    \pgfusepathqfillstroke
  \end{pgfscope}
  \begin{pgfscope}
    \pgfsetlinewidth{1.5000bp}
    \definecolor{sc}{rgb}{0.0000,0.0000,0.0000}
    \pgfsetstrokecolor{sc}
    \pgfsetmiterjoin
    \pgfsetbuttcap
    \pgfpathqmoveto{28.5714bp}{100.0000bp}
    \pgfpathqlineto{28.5714bp}{28.5714bp}
    \pgfusepathqstroke
  \end{pgfscope}
  \begin{pgfscope}
    \definecolor{fc}{rgb}{0.0000,0.0000,0.0000}
    \pgfsetfillcolor{fc}
    \pgfusepathqfill
  \end{pgfscope}
  \begin{pgfscope}
    \definecolor{fc}{rgb}{0.0000,0.0000,0.0000}
    \pgfsetfillcolor{fc}
    \pgfusepathqfill
  \end{pgfscope}
  \begin{pgfscope}
    \definecolor{fc}{rgb}{0.0000,0.0000,0.0000}
    \pgfsetfillcolor{fc}
    \pgfusepathqfill
  \end{pgfscope}
  \begin{pgfscope}
    \definecolor{fc}{rgb}{0.0000,0.0000,0.0000}
    \pgfsetfillcolor{fc}
    \pgfusepathqfill
  \end{pgfscope}
  \begin{pgfscope}
    \definecolor{fc}{rgb}{0.0000,0.0000,0.0000}
    \pgfsetfillcolor{fc}
    \pgftransformshift{\pgfqpoint{178.5714bp}{178.5714bp}}
    \pgftransformscale{1.7857}
    \pgftext[base,left]{$p_2$}
  \end{pgfscope}
  \begin{pgfscope}
    \definecolor{fc}{rgb}{0.0000,0.0000,0.0000}
    \pgfsetfillcolor{fc}
    \pgfsetlinewidth{0.8000bp}
    \definecolor{sc}{rgb}{0.0000,0.0000,0.0000}
    \pgfsetstrokecolor{sc}
    \pgfsetmiterjoin
    \pgfsetbuttcap
    \pgfpathqmoveto{175.0000bp}{171.4286bp}
    \pgfpathqcurveto{175.0000bp}{173.4010bp}{173.4010bp}{175.0000bp}{171.4286bp}{175.0000bp}
    \pgfpathqcurveto{169.4561bp}{175.0000bp}{167.8571bp}{173.4010bp}{167.8571bp}{171.4286bp}
    \pgfpathqcurveto{167.8571bp}{169.4561bp}{169.4561bp}{167.8571bp}{171.4286bp}{167.8571bp}
    \pgfpathqcurveto{173.4010bp}{167.8571bp}{175.0000bp}{169.4561bp}{175.0000bp}{171.4286bp}
    \pgfpathclose
    \pgfusepathqfillstroke
  \end{pgfscope}
  \begin{pgfscope}
    \definecolor{fc}{rgb}{0.0000,0.0000,0.0000}
    \pgfsetfillcolor{fc}
    \pgftransformshift{\pgfqpoint{107.1429bp}{178.5714bp}}
    \pgftransformscale{1.7857}
    \pgftext[base,left]{$p_1$}
  \end{pgfscope}
  \begin{pgfscope}
    \definecolor{fc}{rgb}{0.0000,0.0000,0.0000}
    \pgfsetfillcolor{fc}
    \pgfsetlinewidth{0.8000bp}
    \definecolor{sc}{rgb}{0.0000,0.0000,0.0000}
    \pgfsetstrokecolor{sc}
    \pgfsetmiterjoin
    \pgfsetbuttcap
    \pgfpathqmoveto{103.5714bp}{171.4286bp}
    \pgfpathqcurveto{103.5714bp}{173.4010bp}{101.9724bp}{175.0000bp}{100.0000bp}{175.0000bp}
    \pgfpathqcurveto{98.0276bp}{175.0000bp}{96.4286bp}{173.4010bp}{96.4286bp}{171.4286bp}
    \pgfpathqcurveto{96.4286bp}{169.4561bp}{98.0276bp}{167.8571bp}{100.0000bp}{167.8571bp}
    \pgfpathqcurveto{101.9724bp}{167.8571bp}{103.5714bp}{169.4561bp}{103.5714bp}{171.4286bp}
    \pgfpathclose
    \pgfusepathqfillstroke
  \end{pgfscope}
  \begin{pgfscope}
    \pgfsetlinewidth{1.5000bp}
    \definecolor{sc}{rgb}{0.0000,0.0000,0.0000}
    \pgfsetstrokecolor{sc}
    \pgfsetmiterjoin
    \pgfsetbuttcap
    \pgfpathqmoveto{100.0000bp}{171.4286bp}
    \pgfpathqlineto{171.4286bp}{171.4286bp}
    \pgfusepathqstroke
  \end{pgfscope}
  \begin{pgfscope}
    \definecolor{fc}{rgb}{0.0000,0.0000,0.0000}
    \pgfsetfillcolor{fc}
    \pgfusepathqfill
  \end{pgfscope}
  \begin{pgfscope}
    \definecolor{fc}{rgb}{0.0000,0.0000,0.0000}
    \pgfsetfillcolor{fc}
    \pgfusepathqfill
  \end{pgfscope}
  \begin{pgfscope}
    \definecolor{fc}{rgb}{0.0000,0.0000,0.0000}
    \pgfsetfillcolor{fc}
    \pgfusepathqfill
  \end{pgfscope}
  \begin{pgfscope}
    \definecolor{fc}{rgb}{0.0000,0.0000,0.0000}
    \pgfsetfillcolor{fc}
    \pgfusepathqfill
  \end{pgfscope}
  \begin{pgfscope}
    \definecolor{fc}{rgb}{0.0000,0.0000,0.0000}
    \pgfsetfillcolor{fc}
    \pgftransformshift{\pgfqpoint{35.7143bp}{178.5714bp}}
    \pgftransformscale{1.7857}
    \pgftext[base,left]{$q_1$}
  \end{pgfscope}
  \begin{pgfscope}
    \definecolor{fc}{rgb}{0.0000,0.0000,0.0000}
    \pgfsetfillcolor{fc}
    \pgfsetlinewidth{0.8000bp}
    \definecolor{sc}{rgb}{0.0000,0.0000,0.0000}
    \pgfsetstrokecolor{sc}
    \pgfsetmiterjoin
    \pgfsetbuttcap
    \pgfpathqmoveto{32.1429bp}{171.4286bp}
    \pgfpathqcurveto{32.1429bp}{173.4010bp}{30.5439bp}{175.0000bp}{28.5714bp}{175.0000bp}
    \pgfpathqcurveto{26.5990bp}{175.0000bp}{25.0000bp}{173.4010bp}{25.0000bp}{171.4286bp}
    \pgfpathqcurveto{25.0000bp}{169.4561bp}{26.5990bp}{167.8571bp}{28.5714bp}{167.8571bp}
    \pgfpathqcurveto{30.5439bp}{167.8571bp}{32.1429bp}{169.4561bp}{32.1429bp}{171.4286bp}
    \pgfpathclose
    \pgfusepathqfillstroke
  \end{pgfscope}
\end{pgfpicture}
}
            \caption{}
            \label{fig:ex:ca:hgma:ex:isomorphic}
        \end{subfigure}%
        \hfill%
        \begin{subfigure}[b]{0.4\textwidth}
            \centering
            \resizebox{0.8\linewidth}{!}{\begin{pgfpicture}
  \pgfpathrectangle{\pgfpointorigin}{\pgfqpoint{200.0000bp}{200.0000bp}}
  \pgfusepath{use as bounding box}
  \begin{pgfscope}
    \definecolor{fc}{rgb}{0.0000,0.0000,0.0000}
    \pgfsetfillcolor{fc}
    \pgfsetlinewidth{0.5000bp}
    \definecolor{sc}{rgb}{0.0000,0.0000,0.0000}
    \pgfsetstrokecolor{sc}
    \pgfsetmiterjoin
    \pgfsetbuttcap
    \pgfpathqmoveto{200.0000bp}{100.0000bp}
    \pgfpathqcurveto{200.0000bp}{115.7796bp}{187.2081bp}{128.5714bp}{171.4286bp}{128.5714bp}
    \pgfpathqcurveto{155.6490bp}{128.5714bp}{142.8571bp}{115.7796bp}{142.8571bp}{100.0000bp}
    \pgfpathqcurveto{142.8571bp}{84.2204bp}{155.6490bp}{71.4286bp}{171.4286bp}{71.4286bp}
    \pgfpathqcurveto{187.2081bp}{71.4286bp}{200.0000bp}{84.2204bp}{200.0000bp}{100.0000bp}
    \pgfpathclose
    \pgfusepathqfillstroke
  \end{pgfscope}
  \begin{pgfscope}
    \definecolor{fc}{rgb}{1.0000,1.0000,1.0000}
    \pgfsetfillcolor{fc}
    \pgfsetlinewidth{0.5000bp}
    \definecolor{sc}{rgb}{1.0000,1.0000,1.0000}
    \pgfsetstrokecolor{sc}
    \pgfsetmiterjoin
    \pgfsetbuttcap
    \pgfpathqmoveto{198.5714bp}{100.0000bp}
    \pgfpathqcurveto{198.5714bp}{114.9906bp}{186.4192bp}{127.1429bp}{171.4286bp}{127.1429bp}
    \pgfpathqcurveto{156.4380bp}{127.1429bp}{144.2857bp}{114.9906bp}{144.2857bp}{100.0000bp}
    \pgfpathqcurveto{144.2857bp}{85.0094bp}{156.4380bp}{72.8571bp}{171.4286bp}{72.8571bp}
    \pgfpathqcurveto{186.4192bp}{72.8571bp}{198.5714bp}{85.0094bp}{198.5714bp}{100.0000bp}
    \pgfpathclose
    \pgfusepathqfillstroke
  \end{pgfscope}
  \begin{pgfscope}
    \definecolor{fc}{rgb}{0.0000,0.0000,0.0000}
    \pgfsetfillcolor{fc}
    \pgfsetlinewidth{0.5000bp}
    \definecolor{sc}{rgb}{0.0000,0.0000,0.0000}
    \pgfsetstrokecolor{sc}
    \pgfsetmiterjoin
    \pgfsetbuttcap
    \pgfpathqmoveto{28.5714bp}{128.5714bp}
    \pgfpathqcurveto{12.7919bp}{128.5714bp}{-0.0000bp}{115.7796bp}{-0.0000bp}{100.0000bp}
    \pgfpathqcurveto{-0.0000bp}{84.2204bp}{12.7919bp}{71.4286bp}{28.5714bp}{71.4286bp}
    \pgfpathqlineto{100.0000bp}{71.4286bp}
    \pgfpathqcurveto{115.7796bp}{71.4286bp}{128.5714bp}{84.2204bp}{128.5714bp}{100.0000bp}
    \pgfpathqcurveto{128.5714bp}{115.7796bp}{115.7796bp}{128.5714bp}{100.0000bp}{128.5714bp}
    \pgfpathqlineto{28.5714bp}{128.5714bp}
    \pgfpathqcurveto{12.7919bp}{128.5714bp}{-0.0000bp}{115.7796bp}{-0.0000bp}{100.0000bp}
    \pgfpathqcurveto{-0.0000bp}{84.2204bp}{12.7919bp}{71.4286bp}{28.5714bp}{71.4286bp}
    \pgfpathqlineto{100.0000bp}{71.4286bp}
    \pgfpathqcurveto{115.7796bp}{71.4286bp}{128.5714bp}{84.2204bp}{128.5714bp}{100.0000bp}
    \pgfpathqcurveto{128.5714bp}{115.7796bp}{115.7796bp}{128.5714bp}{100.0000bp}{128.5714bp}
    \pgfpathqlineto{28.5714bp}{128.5714bp}
    \pgfpathclose
    \pgfusepathqfillstroke
  \end{pgfscope}
  \begin{pgfscope}
    \definecolor{fc}{rgb}{1.0000,1.0000,1.0000}
    \pgfsetfillcolor{fc}
    \pgfsetlinewidth{0.5000bp}
    \definecolor{sc}{rgb}{1.0000,1.0000,1.0000}
    \pgfsetstrokecolor{sc}
    \pgfsetmiterjoin
    \pgfsetbuttcap
    \pgfpathqmoveto{28.5714bp}{127.1429bp}
    \pgfpathqcurveto{13.5808bp}{127.1429bp}{1.4286bp}{114.9906bp}{1.4286bp}{100.0000bp}
    \pgfpathqcurveto{1.4286bp}{85.0094bp}{13.5808bp}{72.8571bp}{28.5714bp}{72.8571bp}
    \pgfpathqlineto{100.0000bp}{72.8571bp}
    \pgfpathqcurveto{114.9906bp}{72.8571bp}{127.1429bp}{85.0094bp}{127.1429bp}{100.0000bp}
    \pgfpathqcurveto{127.1429bp}{114.9906bp}{114.9906bp}{127.1429bp}{100.0000bp}{127.1429bp}
    \pgfpathqlineto{28.5714bp}{127.1429bp}
    \pgfpathqcurveto{13.5808bp}{127.1429bp}{1.4286bp}{114.9906bp}{1.4286bp}{100.0000bp}
    \pgfpathqcurveto{1.4286bp}{85.0094bp}{13.5808bp}{72.8571bp}{28.5714bp}{72.8571bp}
    \pgfpathqlineto{100.0000bp}{72.8571bp}
    \pgfpathqcurveto{114.9906bp}{72.8571bp}{127.1429bp}{85.0094bp}{127.1429bp}{100.0000bp}
    \pgfpathqcurveto{127.1429bp}{114.9906bp}{114.9906bp}{127.1429bp}{100.0000bp}{127.1429bp}
    \pgfpathqlineto{28.5714bp}{127.1429bp}
    \pgfpathclose
    \pgfusepathqfillstroke
  \end{pgfscope}
  \begin{pgfscope}
    \definecolor{fc}{rgb}{0.0000,0.0000,0.0000}
    \pgfsetfillcolor{fc}
    \pgftransformshift{\pgfqpoint{178.5714bp}{107.1429bp}}
    \pgftransformscale{1.7857}
    \pgftext[base,left]{$p_2$}
  \end{pgfscope}
  \begin{pgfscope}
    \definecolor{fc}{rgb}{0.0000,0.0000,0.0000}
    \pgfsetfillcolor{fc}
    \pgfsetlinewidth{0.5000bp}
    \definecolor{sc}{rgb}{0.0000,0.0000,0.0000}
    \pgfsetstrokecolor{sc}
    \pgfsetmiterjoin
    \pgfsetbuttcap
    \pgfpathqmoveto{175.0000bp}{100.0000bp}
    \pgfpathqcurveto{175.0000bp}{101.9724bp}{173.4010bp}{103.5714bp}{171.4286bp}{103.5714bp}
    \pgfpathqcurveto{169.4561bp}{103.5714bp}{167.8571bp}{101.9724bp}{167.8571bp}{100.0000bp}
    \pgfpathqcurveto{167.8571bp}{98.0276bp}{169.4561bp}{96.4286bp}{171.4286bp}{96.4286bp}
    \pgfpathqcurveto{173.4010bp}{96.4286bp}{175.0000bp}{98.0276bp}{175.0000bp}{100.0000bp}
    \pgfpathclose
    \pgfusepathqfillstroke
  \end{pgfscope}
  \begin{pgfscope}
    \definecolor{fc}{rgb}{0.0000,0.0000,0.0000}
    \pgfsetfillcolor{fc}
    \pgftransformshift{\pgfqpoint{107.1429bp}{107.1429bp}}
    \pgftransformscale{1.7857}
    \pgftext[base,left]{$p_1$}
  \end{pgfscope}
  \begin{pgfscope}
    \definecolor{fc}{rgb}{0.0000,0.0000,0.0000}
    \pgfsetfillcolor{fc}
    \pgfsetlinewidth{0.5000bp}
    \definecolor{sc}{rgb}{0.0000,0.0000,0.0000}
    \pgfsetstrokecolor{sc}
    \pgfsetmiterjoin
    \pgfsetbuttcap
    \pgfpathqmoveto{103.5714bp}{100.0000bp}
    \pgfpathqcurveto{103.5714bp}{101.9724bp}{101.9724bp}{103.5714bp}{100.0000bp}{103.5714bp}
    \pgfpathqcurveto{98.0276bp}{103.5714bp}{96.4286bp}{101.9724bp}{96.4286bp}{100.0000bp}
    \pgfpathqcurveto{96.4286bp}{98.0276bp}{98.0276bp}{96.4286bp}{100.0000bp}{96.4286bp}
    \pgfpathqcurveto{101.9724bp}{96.4286bp}{103.5714bp}{98.0276bp}{103.5714bp}{100.0000bp}
    \pgfpathclose
    \pgfusepathqfillstroke
  \end{pgfscope}
  \begin{pgfscope}
    \pgfsetlinewidth{0.8018bp}
    \definecolor{sc}{rgb}{0.0000,0.0000,0.0000}
    \pgfsetstrokecolor{sc}
    \pgfsetmiterjoin
    \pgfsetbuttcap
    \pgfpathqmoveto{100.0000bp}{100.0000bp}
    \pgfpathqlineto{171.4286bp}{100.0000bp}
    \pgfusepathqstroke
  \end{pgfscope}
  \begin{pgfscope}
    \definecolor{fc}{rgb}{0.0000,0.0000,0.0000}
    \pgfsetfillcolor{fc}
    \pgfusepathqfill
  \end{pgfscope}
  \begin{pgfscope}
    \definecolor{fc}{rgb}{0.0000,0.0000,0.0000}
    \pgfsetfillcolor{fc}
    \pgfusepathqfill
  \end{pgfscope}
  \begin{pgfscope}
    \definecolor{fc}{rgb}{0.0000,0.0000,0.0000}
    \pgfsetfillcolor{fc}
    \pgfusepathqfill
  \end{pgfscope}
  \begin{pgfscope}
    \definecolor{fc}{rgb}{0.0000,0.0000,0.0000}
    \pgfsetfillcolor{fc}
    \pgfusepathqfill
  \end{pgfscope}
  \begin{pgfscope}
    \definecolor{fc}{rgb}{0.0000,0.0000,0.0000}
    \pgfsetfillcolor{fc}
    \pgftransformshift{\pgfqpoint{35.7143bp}{107.1429bp}}
    \pgftransformscale{1.7857}
    \pgftext[base,left]{$q_1$}
  \end{pgfscope}
  \begin{pgfscope}
    \definecolor{fc}{rgb}{0.0000,0.0000,0.0000}
    \pgfsetfillcolor{fc}
    \pgfsetlinewidth{0.5000bp}
    \definecolor{sc}{rgb}{0.0000,0.0000,0.0000}
    \pgfsetstrokecolor{sc}
    \pgfsetmiterjoin
    \pgfsetbuttcap
    \pgfpathqmoveto{32.1429bp}{100.0000bp}
    \pgfpathqcurveto{32.1429bp}{101.9724bp}{30.5439bp}{103.5714bp}{28.5714bp}{103.5714bp}
    \pgfpathqcurveto{26.5990bp}{103.5714bp}{25.0000bp}{101.9724bp}{25.0000bp}{100.0000bp}
    \pgfpathqcurveto{25.0000bp}{98.0276bp}{26.5990bp}{96.4286bp}{28.5714bp}{96.4286bp}
    \pgfpathqcurveto{30.5439bp}{96.4286bp}{32.1429bp}{98.0276bp}{32.1429bp}{100.0000bp}
    \pgfpathclose
    \pgfusepathqfillstroke
  \end{pgfscope}
\end{pgfpicture}
}
            \caption{}
            \label{fig:ex:ca:hgma:ex:isomorphic_reduced}
        \end{subfigure}
    \end{figure}
\end{example}

\begin{algorithm}
    \caption{Binding edge merging algorithm}
    \label{algo:bindingedgemerge}
    \begin{algorithmic}
        \Function {$\textsc{MergeBindingEdges}$} {$H_1, H_2, H', b_1, b_2$}
            \State $b' \gets b_1 \sqcap b_2$
            \State $H' \gets H' \cup b'$
            \ForAll {vertices $v \in b'$}
                \If {there is no predicate edge in $H'$ containing $v$}
                    \State $p_1 \gets$ predicate set in $H_1$ containing $f_1^{-1}(v)$
                    \State $p_2 \gets$ predicate set in $H_2$ containing $f_2^{-1}(v)$
                    \State $H' \gets \textsc{MergePredicateEdges}
                                        \left( H_1, H_2, H', p_1, p_2 \right)$
                \EndIf
            \EndFor
            \State \Return $H'$
        \EndFunction
    \end{algorithmic}
\end{algorithm}

\begin{algorithm}
    \caption{Predicate edge merging algorithm}
    \label{algo:prededgemerge}

    \begin{algorithmic}
        \Function {$\textsc{MergePredicateEdges}$} {$H_1, H_2, H', p_1, p_2$}
            \State $p' \gets p_1 \sqcap p_2$
            \State $H' \gets H' \cup p'$
            \ForAll {vertices $v \in p'$}
                \If {there is no binding edge in $H'$ containing $v$}
                    \State $b_1 \gets$ the binding edge in $H_1$ containing $f_1^{-1}(v)$
                    \State $b_2 \gets$ the binding edge in $H_2$ containing $f_2^{-1}(v)$
                    \State $H' \gets \textsc{MergeBindingEdges}
                        \left( H_1, H_2, H', b_1, b_2 \right)$
                \EndIf
            \EndFor
            \State \Return $H'$
        \EndFunction
    \end{algorithmic}
\end{algorithm}

\begin{algorithm}
    \caption{Hyper graph merging algorithm}
    \label{algo:hypergraphmerge}
    \begin{algorithmic}
        \Function {$\textsc{MergeHyperGraphs}$} {$H_1$, $H_2$}
            \State Let $q_1$ and $q_2$ denote the predicate hyper edges describing the effect in $H_1$ and $H_2$, respectively
            \State Let $H' = \emptyset$ be a new, empty hyper graph
            \State $H' \gets \textsc{MergePredicateEdges}(H_1,H_2,H',q_1,q_2)$
            \State $H' \gets \textsc{CollapseHyperGraph}(H')$
            \State \Return $H'$
        \EndFunction
    \end{algorithmic}
\end{algorithm}

\begin{example} \label{ex:ca:hgma:disconnected}
    \begin{equation*}
        H_1 = \forall x, y, z : q(x) \quad \text{when} \quad
            p(x,y) \land p(y, z)
    \end{equation*}

    \begin{equation*}
        H_2 = \forall \alpha, \beta, \gamma : q(\alpha) \quad \text{when} \quad
            p(\alpha, \beta) \land p(\gamma, \delta)
    \end{equation*}

    \begin{equation*}
        H_1 \sqcap H_2 = \forall x, y : q(x) \quad \text{when} \quad p(x, y)
    \end{equation*}

    \begin{figure}
        <placeholder>
        \caption{\label{fig:ex:ca:hgma:ex:disconnected} Figure of hypergraphs for example \ref{ex:ca:hgma:disconnected}.}
    \end{figure}

    Note that the precondition decribed by $H_1$ is more restrictive than that of $H_2$. Consequentially, if $H_1$ was the real precondition, the conditional effect would not have succeeded for $H_2$.
\end{example}

If, on the other hand, $v_j$ does exist in $e_2(q_i)$, then it cannot be discarded as a precondition. Furthermore, if there exists several nodes with name $v_j$ in $e_2(q_i)$, then it may be the case that

\begin{example} \label{ex:ca:hgma:generalization}
    \begin{equation*}
        H_1 = \forall x, y : q(x) \quad \text{when} \quad
            p(x,y) \land f(y) \land g(y)
    \end{equation*}

    \begin{equation*}
        H_2 = \forall x, y, z : q(x) \quad \text{when} \quad
            p(x, y) \land p(x,z) \land f(y) \land g(z)
    \end{equation*}

    \begin{equation*}
        H_1 \sqcap H_2 = H_2
    \end{equation*}

    \begin{figure}
        \centering
        \begin{subfigure}[b]{0.4\textwidth}
            \centering
            \resizebox{\linewidth}{!}{\begin{pgfpicture}
  \pgfpathrectangle{\pgfpointorigin}{\pgfqpoint{200.0000bp}{200.0000bp}}
  \pgfusepath{use as bounding box}
  \begin{pgfscope}
    \definecolor{fc}{rgb}{0.0000,0.0000,0.0000}
    \pgfsetfillcolor{fc}
    \pgfsetlinewidth{0.5790bp}
    \definecolor{sc}{rgb}{0.0000,0.0000,0.0000}
    \pgfsetstrokecolor{sc}
    \pgfsetmiterjoin
    \pgfsetbuttcap
    \pgfpathqmoveto{95.2381bp}{76.1905bp}
    \pgfpathqcurveto{95.2381bp}{60.4109bp}{108.0300bp}{47.6190bp}{123.8095bp}{47.6190bp}
    \pgfpathqcurveto{139.5891bp}{47.6190bp}{152.3810bp}{60.4109bp}{152.3810bp}{76.1905bp}
    \pgfpathqlineto{152.3810bp}{123.8095bp}
    \pgfpathqcurveto{152.3810bp}{139.5891bp}{139.5891bp}{152.3810bp}{123.8095bp}{152.3810bp}
    \pgfpathqcurveto{108.0300bp}{152.3810bp}{95.2381bp}{139.5891bp}{95.2381bp}{123.8095bp}
    \pgfpathqcurveto{95.2381bp}{108.0300bp}{108.0300bp}{95.2381bp}{123.8095bp}{95.2381bp}
    \pgfpathqlineto{171.4286bp}{95.2381bp}
    \pgfpathqcurveto{187.2081bp}{95.2381bp}{200.0000bp}{108.0300bp}{200.0000bp}{123.8095bp}
    \pgfpathqcurveto{200.0000bp}{139.5891bp}{187.2081bp}{152.3810bp}{171.4286bp}{152.3810bp}
    \pgfpathqlineto{123.8095bp}{152.3810bp}
    \pgfpathqcurveto{108.0300bp}{152.3810bp}{95.2381bp}{139.5891bp}{95.2381bp}{123.8095bp}
    \pgfpathqlineto{95.2381bp}{76.1905bp}
    \pgfpathclose
    \pgfusepathqfillstroke
  \end{pgfscope}
  \begin{pgfscope}
    \definecolor{fc}{rgb}{1.0000,1.0000,1.0000}
    \pgfsetfillcolor{fc}
    \pgfsetlinewidth{0.5790bp}
    \definecolor{sc}{rgb}{1.0000,1.0000,1.0000}
    \pgfsetstrokecolor{sc}
    \pgfsetmiterjoin
    \pgfsetbuttcap
    \pgfpathqmoveto{96.1905bp}{76.1905bp}
    \pgfpathqcurveto{96.1905bp}{60.9369bp}{108.5559bp}{48.5714bp}{123.8095bp}{48.5714bp}
    \pgfpathqcurveto{139.0631bp}{48.5714bp}{151.4286bp}{60.9369bp}{151.4286bp}{76.1905bp}
    \pgfpathqlineto{151.4286bp}{123.8095bp}
    \pgfpathqcurveto{151.4286bp}{139.0631bp}{139.0631bp}{151.4286bp}{123.8095bp}{151.4286bp}
    \pgfpathqcurveto{108.5559bp}{151.4286bp}{96.1905bp}{139.0631bp}{96.1905bp}{123.8095bp}
    \pgfpathqcurveto{96.1905bp}{108.5559bp}{108.5559bp}{96.1905bp}{123.8095bp}{96.1905bp}
    \pgfpathqlineto{171.4286bp}{96.1905bp}
    \pgfpathqcurveto{186.6822bp}{96.1905bp}{199.0476bp}{108.5559bp}{199.0476bp}{123.8095bp}
    \pgfpathqcurveto{199.0476bp}{139.0631bp}{186.6822bp}{151.4286bp}{171.4286bp}{151.4286bp}
    \pgfpathqlineto{123.8095bp}{151.4286bp}
    \pgfpathqcurveto{108.5559bp}{151.4286bp}{96.1905bp}{139.0631bp}{96.1905bp}{123.8095bp}
    \pgfpathqlineto{96.1905bp}{76.1905bp}
    \pgfpathclose
    \pgfusepathqfillstroke
  \end{pgfscope}
  \begin{pgfscope}
    \definecolor{fc}{rgb}{0.0000,0.0000,0.0000}
    \pgfsetfillcolor{fc}
    \pgfsetlinewidth{0.5790bp}
    \definecolor{sc}{rgb}{0.0000,0.0000,0.0000}
    \pgfsetstrokecolor{sc}
    \pgfsetmiterjoin
    \pgfsetbuttcap
    \pgfpathqmoveto{28.5714bp}{152.3810bp}
    \pgfpathqcurveto{12.7919bp}{152.3810bp}{-0.0000bp}{139.5891bp}{-0.0000bp}{123.8095bp}
    \pgfpathqcurveto{-0.0000bp}{108.0300bp}{12.7919bp}{95.2381bp}{28.5714bp}{95.2381bp}
    \pgfpathqlineto{76.1905bp}{95.2381bp}
    \pgfpathqcurveto{91.9700bp}{95.2381bp}{104.7619bp}{108.0300bp}{104.7619bp}{123.8095bp}
    \pgfpathqcurveto{104.7619bp}{139.5891bp}{91.9700bp}{152.3810bp}{76.1905bp}{152.3810bp}
    \pgfpathqlineto{28.5714bp}{152.3810bp}
    \pgfpathclose
    \pgfusepathqfillstroke
  \end{pgfscope}
  \begin{pgfscope}
    \definecolor{fc}{rgb}{1.0000,1.0000,1.0000}
    \pgfsetfillcolor{fc}
    \pgfsetlinewidth{0.5790bp}
    \definecolor{sc}{rgb}{1.0000,1.0000,1.0000}
    \pgfsetstrokecolor{sc}
    \pgfsetmiterjoin
    \pgfsetbuttcap
    \pgfpathqmoveto{28.5714bp}{151.4286bp}
    \pgfpathqcurveto{13.3178bp}{151.4286bp}{0.9524bp}{139.0631bp}{0.9524bp}{123.8095bp}
    \pgfpathqcurveto{0.9524bp}{108.5559bp}{13.3178bp}{96.1905bp}{28.5714bp}{96.1905bp}
    \pgfpathqlineto{76.1905bp}{96.1905bp}
    \pgfpathqcurveto{91.4441bp}{96.1905bp}{103.8095bp}{108.5559bp}{103.8095bp}{123.8095bp}
    \pgfpathqcurveto{103.8095bp}{139.0631bp}{91.4441bp}{151.4286bp}{76.1905bp}{151.4286bp}
    \pgfpathqlineto{28.5714bp}{151.4286bp}
    \pgfpathclose
    \pgfusepathqfillstroke
  \end{pgfscope}
  \begin{pgfscope}
    \definecolor{fc}{rgb}{0.0000,0.0000,0.0000}
    \pgfsetfillcolor{fc}
    \pgftransformshift{\pgfqpoint{176.1905bp}{128.5714bp}}
    \pgftransformscale{1.1905}
    \pgftext[base,left]{$g_1$}
  \end{pgfscope}
  \begin{pgfscope}
    \definecolor{fc}{rgb}{0.0000,0.0000,0.0000}
    \pgfsetfillcolor{fc}
    \pgfsetlinewidth{0.5790bp}
    \definecolor{sc}{rgb}{0.0000,0.0000,0.0000}
    \pgfsetstrokecolor{sc}
    \pgfsetmiterjoin
    \pgfsetbuttcap
    \pgfpathqmoveto{173.8095bp}{123.8095bp}
    \pgfpathqcurveto{173.8095bp}{125.1245bp}{172.7435bp}{126.1905bp}{171.4286bp}{126.1905bp}
    \pgfpathqcurveto{170.1136bp}{126.1905bp}{169.0476bp}{125.1245bp}{169.0476bp}{123.8095bp}
    \pgfpathqcurveto{169.0476bp}{122.4946bp}{170.1136bp}{121.4286bp}{171.4286bp}{121.4286bp}
    \pgfpathqcurveto{172.7435bp}{121.4286bp}{173.8095bp}{122.4946bp}{173.8095bp}{123.8095bp}
    \pgfpathclose
    \pgfusepathqfillstroke
  \end{pgfscope}
  \begin{pgfscope}
    \definecolor{fc}{rgb}{0.0000,0.0000,0.0000}
    \pgfsetfillcolor{fc}
    \pgftransformshift{\pgfqpoint{128.5714bp}{80.9524bp}}
    \pgftransformscale{1.1905}
    \pgftext[base,left]{$f_1$}
  \end{pgfscope}
  \begin{pgfscope}
    \definecolor{fc}{rgb}{0.0000,0.0000,0.0000}
    \pgfsetfillcolor{fc}
    \pgfsetlinewidth{0.5790bp}
    \definecolor{sc}{rgb}{0.0000,0.0000,0.0000}
    \pgfsetstrokecolor{sc}
    \pgfsetmiterjoin
    \pgfsetbuttcap
    \pgfpathqmoveto{126.1905bp}{76.1905bp}
    \pgfpathqcurveto{126.1905bp}{77.5054bp}{125.1245bp}{78.5714bp}{123.8095bp}{78.5714bp}
    \pgfpathqcurveto{122.4946bp}{78.5714bp}{121.4286bp}{77.5054bp}{121.4286bp}{76.1905bp}
    \pgfpathqcurveto{121.4286bp}{74.8755bp}{122.4946bp}{73.8095bp}{123.8095bp}{73.8095bp}
    \pgfpathqcurveto{125.1245bp}{73.8095bp}{126.1905bp}{74.8755bp}{126.1905bp}{76.1905bp}
    \pgfpathclose
    \pgfusepathqfillstroke
  \end{pgfscope}
  \begin{pgfscope}
    \definecolor{fc}{rgb}{0.0000,0.0000,0.0000}
    \pgfsetfillcolor{fc}
    \pgftransformshift{\pgfqpoint{128.5714bp}{128.5714bp}}
    \pgftransformscale{1.1905}
    \pgftext[base,left]{$p_2$}
  \end{pgfscope}
  \begin{pgfscope}
    \definecolor{fc}{rgb}{0.0000,0.0000,0.0000}
    \pgfsetfillcolor{fc}
    \pgfsetlinewidth{0.5790bp}
    \definecolor{sc}{rgb}{0.0000,0.0000,0.0000}
    \pgfsetstrokecolor{sc}
    \pgfsetmiterjoin
    \pgfsetbuttcap
    \pgfpathqmoveto{126.1905bp}{123.8095bp}
    \pgfpathqcurveto{126.1905bp}{125.1245bp}{125.1245bp}{126.1905bp}{123.8095bp}{126.1905bp}
    \pgfpathqcurveto{122.4946bp}{126.1905bp}{121.4286bp}{125.1245bp}{121.4286bp}{123.8095bp}
    \pgfpathqcurveto{121.4286bp}{122.4946bp}{122.4946bp}{121.4286bp}{123.8095bp}{121.4286bp}
    \pgfpathqcurveto{125.1245bp}{121.4286bp}{126.1905bp}{122.4946bp}{126.1905bp}{123.8095bp}
    \pgfpathclose
    \pgfusepathqfillstroke
  \end{pgfscope}
  \begin{pgfscope}
    \definecolor{fc}{rgb}{0.0000,0.0000,0.0000}
    \pgfsetfillcolor{fc}
    \pgftransformshift{\pgfqpoint{80.9524bp}{128.5714bp}}
    \pgftransformscale{1.1905}
    \pgftext[base,left]{$p_1$}
  \end{pgfscope}
  \begin{pgfscope}
    \definecolor{fc}{rgb}{0.0000,0.0000,0.0000}
    \pgfsetfillcolor{fc}
    \pgfsetlinewidth{0.5790bp}
    \definecolor{sc}{rgb}{0.0000,0.0000,0.0000}
    \pgfsetstrokecolor{sc}
    \pgfsetmiterjoin
    \pgfsetbuttcap
    \pgfpathqmoveto{78.5714bp}{123.8095bp}
    \pgfpathqcurveto{78.5714bp}{125.1245bp}{77.5054bp}{126.1905bp}{76.1905bp}{126.1905bp}
    \pgfpathqcurveto{74.8755bp}{126.1905bp}{73.8095bp}{125.1245bp}{73.8095bp}{123.8095bp}
    \pgfpathqcurveto{73.8095bp}{122.4946bp}{74.8755bp}{121.4286bp}{76.1905bp}{121.4286bp}
    \pgfpathqcurveto{77.5054bp}{121.4286bp}{78.5714bp}{122.4946bp}{78.5714bp}{123.8095bp}
    \pgfpathclose
    \pgfusepathqfillstroke
  \end{pgfscope}
  \begin{pgfscope}
    \pgfsetlinewidth{1.0856bp}
    \definecolor{sc}{rgb}{0.0000,0.0000,0.0000}
    \pgfsetstrokecolor{sc}
    \pgfsetmiterjoin
    \pgfsetbuttcap
    \pgfpathqmoveto{76.1905bp}{123.8095bp}
    \pgfpathqlineto{123.8095bp}{123.8095bp}
    \pgfusepathqstroke
  \end{pgfscope}
  \begin{pgfscope}
    \definecolor{fc}{rgb}{0.0000,0.0000,0.0000}
    \pgfsetfillcolor{fc}
    \pgfusepathqfill
  \end{pgfscope}
  \begin{pgfscope}
    \definecolor{fc}{rgb}{0.0000,0.0000,0.0000}
    \pgfsetfillcolor{fc}
    \pgfusepathqfill
  \end{pgfscope}
  \begin{pgfscope}
    \definecolor{fc}{rgb}{0.0000,0.0000,0.0000}
    \pgfsetfillcolor{fc}
    \pgfusepathqfill
  \end{pgfscope}
  \begin{pgfscope}
    \definecolor{fc}{rgb}{0.0000,0.0000,0.0000}
    \pgfsetfillcolor{fc}
    \pgfusepathqfill
  \end{pgfscope}
  \begin{pgfscope}
    \definecolor{fc}{rgb}{0.0000,0.0000,0.0000}
    \pgfsetfillcolor{fc}
    \pgftransformshift{\pgfqpoint{33.3333bp}{128.5714bp}}
    \pgftransformscale{1.1905}
    \pgftext[base,left]{$q_1$}
  \end{pgfscope}
  \begin{pgfscope}
    \definecolor{fc}{rgb}{0.0000,0.0000,0.0000}
    \pgfsetfillcolor{fc}
    \pgfsetlinewidth{0.5790bp}
    \definecolor{sc}{rgb}{0.0000,0.0000,0.0000}
    \pgfsetstrokecolor{sc}
    \pgfsetmiterjoin
    \pgfsetbuttcap
    \pgfpathqmoveto{30.9524bp}{123.8095bp}
    \pgfpathqcurveto{30.9524bp}{125.1245bp}{29.8864bp}{126.1905bp}{28.5714bp}{126.1905bp}
    \pgfpathqcurveto{27.2565bp}{126.1905bp}{26.1905bp}{125.1245bp}{26.1905bp}{123.8095bp}
    \pgfpathqcurveto{26.1905bp}{122.4946bp}{27.2565bp}{121.4286bp}{28.5714bp}{121.4286bp}
    \pgfpathqcurveto{29.8864bp}{121.4286bp}{30.9524bp}{122.4946bp}{30.9524bp}{123.8095bp}
    \pgfpathclose
    \pgfusepathqfillstroke
  \end{pgfscope}
\end{pgfpicture}
}
            \caption{}
            \label{fig:ex:ca:hgma:ex:generalization1}
        \end{subfigure}%
        \hfill%
        \begin{subfigure}[b]{0.4\textwidth}
            \centering
            \resizebox{0.8\linewidth}{!}{\input{\master/Graphics/hgEx2_2.pgf}}
            \caption{}
            \label{fig:ex:ca:hgma:ex:generalization1}
        \end{subfigure}
        \caption{} Figure of hypergraphs for example
    \end{figure}

\end{example}

\end{document}
