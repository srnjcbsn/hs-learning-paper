\documentclass[../Master.tex]{subfiles}
\providecommand{\master}{..}

\begin{document}

Formally, a vertex in a such a hypergraph is a pair $\left(n,p_i \right)$

Given two hypergraphs $H_1$ and $H_2$ describing the same conditional effect $q$:

Starting with the hyperedge describing the effect $q$, ie. $\left\{ q_1, \dots, q_{|q|}  \right\}$:

Each argument $q_i$ of $q$ is now member of exactly one \emph{arg}-hyperedge in  both $H_1$ and $H_2$, denoted $e_1(q_i)$ and $e_2(q_i)$. For all nodes $v_j \in e_1(q_i)$, it holds that  if $v_j$ does not exist in $e_2(q_i)$, then it is not a precondition that the $i$'th argument of $q$ is the same as the $j$'th argument of $v$.

\begin{proposition}
    All vertices in a hypergraph $H$ is member of exactly one predicate edge and one binding edge.
\end{proposition}

\begin{definition}[Hyper edge intersection]
    The intersection of two hyper edges $e_1$ and $e_2$ is a new hyperedge where the number of vertices with name $v_i$ is equal to the number of vertices with name $v_i$ in $e_1$ and $e_2$, multiplied. Formally:
    \begin{equation*}
        e_1 \sqcap e_2 = \left\{
            f(v, t) \; | \;
                v \in e_1 \land t \in e_2 \land v \approx t
        \right\}
    \end{equation*}
\end{definition}

\begin{definition}
    Given two vertices with the same name, ie $v = \left(n, p_i \right)$ and $t = \left( m, p_i \right)$, $f(v,t)$ yields a new vertex with that name and an identifier based on $v$'s and $t$'s. $f$ must be injective, such that $f^{-1}(f(v,t)) = (v,t)$. In the following, we will use the shorthands $f^{-1}_1(f(v,t)) = v$ and $f^{-1}_2(f(v,t)) = t$.
\end{definition}

 A simple implementation of $f$ is tupling the identifiers of the arguement, ie:

\begin{equation*}
    f\left( \left(n, p_i \right), \left(m, p_i \right) \right) =
        \left( (n,m), p_i \right)
\end{equation*}
Then, $f$ is clearly inversible, and $f^{-1}$ can be computed by decomposing the tuple. Note that with this implementation, repeated application of $f$ requires linear space.

\begin{proposition}
    If
\end{proposition}

\begin{proposition}
    Given two binding edges $e_1$ and $e_2$ from hypergraphs $H_1$ and $H_2$, respectively; any node $v$ that only exists in \emph{either} $e_1$ or $e_2$ is discarded as a precondition, formally:
    \begin{equation*}
        v \notin e_1 \lor v \notin e_2 \rightarrow v \notin e_1 \sqcap e_2
    \end{equation*}
\end{proposition}

\begin{example}
    \begin{equation*}
        H = \forall x, y, z : q(x) \quad \textit{when} \quad
            p(x,y) \land p(x,z)
    \end{equation*}

    \begin{equation}
        \forall x, y : q(x) \quad \textit{when} \quad p(x,y)
    \end{equation}

    \begin{figure}
        \centering
        \hfill
        \begin{subfigure}[b]{0.4\textwidth}
            \centering
            \resizebox{0.7\linewidth}{!}{\begin{pgfpicture}
  \pgfpathrectangle{\pgfpointorigin}{\pgfqpoint{200.0000bp}{200.0000bp}}
  \pgfusepath{use as bounding box}
  \begin{pgfscope}
    \definecolor{fc}{rgb}{0.0000,0.0000,0.0000}
    \pgfsetfillcolor{fc}
    \pgfsetlinewidth{0.8000bp}
    \definecolor{sc}{rgb}{0.0000,0.0000,0.0000}
    \pgfsetstrokecolor{sc}
    \pgfsetmiterjoin
    \pgfsetbuttcap
    \pgfpathqmoveto{57.1429bp}{28.5714bp}
    \pgfpathqcurveto{57.1429bp}{44.3510bp}{44.3510bp}{57.1429bp}{28.5714bp}{57.1429bp}
    \pgfpathqcurveto{12.7919bp}{57.1429bp}{0.0000bp}{44.3510bp}{0.0000bp}{28.5714bp}
    \pgfpathqcurveto{-0.0000bp}{12.7919bp}{12.7919bp}{0.0000bp}{28.5714bp}{0.0000bp}
    \pgfpathqcurveto{44.3510bp}{-0.0000bp}{57.1429bp}{12.7919bp}{57.1429bp}{28.5714bp}
    \pgfpathclose
    \pgfusepathqfillstroke
  \end{pgfscope}
  \begin{pgfscope}
    \definecolor{fc}{rgb}{1.0000,1.0000,1.0000}
    \pgfsetfillcolor{fc}
    \pgfsetlinewidth{0.8000bp}
    \definecolor{sc}{rgb}{1.0000,1.0000,1.0000}
    \pgfsetstrokecolor{sc}
    \pgfsetmiterjoin
    \pgfsetbuttcap
    \pgfpathqmoveto{55.7143bp}{28.5714bp}
    \pgfpathqcurveto{55.7143bp}{43.5620bp}{43.5620bp}{55.7143bp}{28.5714bp}{55.7143bp}
    \pgfpathqcurveto{13.5808bp}{55.7143bp}{1.4286bp}{43.5620bp}{1.4286bp}{28.5714bp}
    \pgfpathqcurveto{1.4286bp}{13.5808bp}{13.5808bp}{1.4286bp}{28.5714bp}{1.4286bp}
    \pgfpathqcurveto{43.5620bp}{1.4286bp}{55.7143bp}{13.5808bp}{55.7143bp}{28.5714bp}
    \pgfpathclose
    \pgfusepathqfillstroke
  \end{pgfscope}
  \begin{pgfscope}
    \definecolor{fc}{rgb}{0.0000,0.0000,0.0000}
    \pgfsetfillcolor{fc}
    \pgfsetlinewidth{0.8000bp}
    \definecolor{sc}{rgb}{0.0000,0.0000,0.0000}
    \pgfsetstrokecolor{sc}
    \pgfsetmiterjoin
    \pgfsetbuttcap
    \pgfpathqmoveto{200.0000bp}{171.4286bp}
    \pgfpathqcurveto{200.0000bp}{187.2081bp}{187.2081bp}{200.0000bp}{171.4286bp}{200.0000bp}
    \pgfpathqcurveto{155.6490bp}{200.0000bp}{142.8571bp}{187.2081bp}{142.8571bp}{171.4286bp}
    \pgfpathqcurveto{142.8571bp}{155.6490bp}{155.6490bp}{142.8571bp}{171.4286bp}{142.8571bp}
    \pgfpathqcurveto{187.2081bp}{142.8571bp}{200.0000bp}{155.6490bp}{200.0000bp}{171.4286bp}
    \pgfpathclose
    \pgfusepathqfillstroke
  \end{pgfscope}
  \begin{pgfscope}
    \definecolor{fc}{rgb}{1.0000,1.0000,1.0000}
    \pgfsetfillcolor{fc}
    \pgfsetlinewidth{0.8000bp}
    \definecolor{sc}{rgb}{1.0000,1.0000,1.0000}
    \pgfsetstrokecolor{sc}
    \pgfsetmiterjoin
    \pgfsetbuttcap
    \pgfpathqmoveto{198.5714bp}{171.4286bp}
    \pgfpathqcurveto{198.5714bp}{186.4192bp}{186.4192bp}{198.5714bp}{171.4286bp}{198.5714bp}
    \pgfpathqcurveto{156.4380bp}{198.5714bp}{144.2857bp}{186.4192bp}{144.2857bp}{171.4286bp}
    \pgfpathqcurveto{144.2857bp}{156.4380bp}{156.4380bp}{144.2857bp}{171.4286bp}{144.2857bp}
    \pgfpathqcurveto{186.4192bp}{144.2857bp}{198.5714bp}{156.4380bp}{198.5714bp}{171.4286bp}
    \pgfpathclose
    \pgfusepathqfillstroke
  \end{pgfscope}
  \begin{pgfscope}
    \definecolor{fc}{rgb}{0.0000,0.0000,0.0000}
    \pgfsetfillcolor{fc}
    \pgfsetlinewidth{0.8000bp}
    \definecolor{sc}{rgb}{0.0000,0.0000,0.0000}
    \pgfsetstrokecolor{sc}
    \pgfsetmiterjoin
    \pgfsetbuttcap
    \pgfpathqmoveto{-0.0000bp}{100.0000bp}
    \pgfpathqcurveto{-0.0000bp}{84.2204bp}{12.7919bp}{71.4286bp}{28.5714bp}{71.4286bp}
    \pgfpathqcurveto{44.3510bp}{71.4286bp}{57.1429bp}{84.2204bp}{57.1429bp}{100.0000bp}
    \pgfpathqlineto{57.1429bp}{171.4286bp}
    \pgfpathqcurveto{57.1429bp}{187.2081bp}{44.3510bp}{200.0000bp}{28.5714bp}{200.0000bp}
    \pgfpathqcurveto{12.7919bp}{200.0000bp}{0.0000bp}{187.2081bp}{0.0000bp}{171.4286bp}
    \pgfpathqcurveto{-0.0000bp}{155.6490bp}{12.7919bp}{142.8571bp}{28.5714bp}{142.8571bp}
    \pgfpathqlineto{100.0000bp}{142.8571bp}
    \pgfpathqcurveto{115.7796bp}{142.8571bp}{128.5714bp}{155.6490bp}{128.5714bp}{171.4286bp}
    \pgfpathqcurveto{128.5714bp}{187.2081bp}{115.7796bp}{200.0000bp}{100.0000bp}{200.0000bp}
    \pgfpathqlineto{28.5714bp}{200.0000bp}
    \pgfpathqcurveto{12.7919bp}{200.0000bp}{0.0000bp}{187.2081bp}{0.0000bp}{171.4286bp}
    \pgfpathqlineto{-0.0000bp}{100.0000bp}
    \pgfpathclose
    \pgfusepathqfillstroke
  \end{pgfscope}
  \begin{pgfscope}
    \definecolor{fc}{rgb}{1.0000,1.0000,1.0000}
    \pgfsetfillcolor{fc}
    \pgfsetlinewidth{0.8000bp}
    \definecolor{sc}{rgb}{1.0000,1.0000,1.0000}
    \pgfsetstrokecolor{sc}
    \pgfsetmiterjoin
    \pgfsetbuttcap
    \pgfpathqmoveto{1.4286bp}{100.0000bp}
    \pgfpathqcurveto{1.4286bp}{85.0094bp}{13.5808bp}{72.8571bp}{28.5714bp}{72.8571bp}
    \pgfpathqcurveto{43.5620bp}{72.8571bp}{55.7143bp}{85.0094bp}{55.7143bp}{100.0000bp}
    \pgfpathqlineto{55.7143bp}{171.4286bp}
    \pgfpathqcurveto{55.7143bp}{186.4192bp}{43.5620bp}{198.5714bp}{28.5714bp}{198.5714bp}
    \pgfpathqcurveto{13.5808bp}{198.5714bp}{1.4286bp}{186.4192bp}{1.4286bp}{171.4286bp}
    \pgfpathqcurveto{1.4286bp}{156.4380bp}{13.5808bp}{144.2857bp}{28.5714bp}{144.2857bp}
    \pgfpathqlineto{100.0000bp}{144.2857bp}
    \pgfpathqcurveto{114.9906bp}{144.2857bp}{127.1429bp}{156.4380bp}{127.1429bp}{171.4286bp}
    \pgfpathqcurveto{127.1429bp}{186.4192bp}{114.9906bp}{198.5714bp}{100.0000bp}{198.5714bp}
    \pgfpathqlineto{28.5714bp}{198.5714bp}
    \pgfpathqcurveto{13.5808bp}{198.5714bp}{1.4286bp}{186.4192bp}{1.4286bp}{171.4286bp}
    \pgfpathqlineto{1.4286bp}{100.0000bp}
    \pgfpathclose
    \pgfusepathqfillstroke
  \end{pgfscope}
  \begin{pgfscope}
    \definecolor{fc}{rgb}{0.0000,0.0000,0.0000}
    \pgfsetfillcolor{fc}
    \pgftransformshift{\pgfqpoint{35.7143bp}{35.7143bp}}
    \pgftransformscale{1.7857}
    \pgftext[base,left]{$p_2$}
  \end{pgfscope}
  \begin{pgfscope}
    \definecolor{fc}{rgb}{0.0000,0.0000,0.0000}
    \pgfsetfillcolor{fc}
    \pgfsetlinewidth{0.8000bp}
    \definecolor{sc}{rgb}{0.0000,0.0000,0.0000}
    \pgfsetstrokecolor{sc}
    \pgfsetmiterjoin
    \pgfsetbuttcap
    \pgfpathqmoveto{32.1429bp}{28.5714bp}
    \pgfpathqcurveto{32.1429bp}{30.5439bp}{30.5439bp}{32.1429bp}{28.5714bp}{32.1429bp}
    \pgfpathqcurveto{26.5990bp}{32.1429bp}{25.0000bp}{30.5439bp}{25.0000bp}{28.5714bp}
    \pgfpathqcurveto{25.0000bp}{26.5990bp}{26.5990bp}{25.0000bp}{28.5714bp}{25.0000bp}
    \pgfpathqcurveto{30.5439bp}{25.0000bp}{32.1429bp}{26.5990bp}{32.1429bp}{28.5714bp}
    \pgfpathclose
    \pgfusepathqfillstroke
  \end{pgfscope}
  \begin{pgfscope}
    \definecolor{fc}{rgb}{0.0000,0.0000,0.0000}
    \pgfsetfillcolor{fc}
    \pgftransformshift{\pgfqpoint{35.7143bp}{107.1429bp}}
    \pgftransformscale{1.7857}
    \pgftext[base,left]{$p_1$}
  \end{pgfscope}
  \begin{pgfscope}
    \definecolor{fc}{rgb}{0.0000,0.0000,0.0000}
    \pgfsetfillcolor{fc}
    \pgfsetlinewidth{0.8000bp}
    \definecolor{sc}{rgb}{0.0000,0.0000,0.0000}
    \pgfsetstrokecolor{sc}
    \pgfsetmiterjoin
    \pgfsetbuttcap
    \pgfpathqmoveto{32.1429bp}{100.0000bp}
    \pgfpathqcurveto{32.1429bp}{101.9724bp}{30.5439bp}{103.5714bp}{28.5714bp}{103.5714bp}
    \pgfpathqcurveto{26.5990bp}{103.5714bp}{25.0000bp}{101.9724bp}{25.0000bp}{100.0000bp}
    \pgfpathqcurveto{25.0000bp}{98.0276bp}{26.5990bp}{96.4286bp}{28.5714bp}{96.4286bp}
    \pgfpathqcurveto{30.5439bp}{96.4286bp}{32.1429bp}{98.0276bp}{32.1429bp}{100.0000bp}
    \pgfpathclose
    \pgfusepathqfillstroke
  \end{pgfscope}
  \begin{pgfscope}
    \pgfsetlinewidth{1.5000bp}
    \definecolor{sc}{rgb}{0.0000,0.0000,0.0000}
    \pgfsetstrokecolor{sc}
    \pgfsetmiterjoin
    \pgfsetbuttcap
    \pgfpathqmoveto{28.5714bp}{100.0000bp}
    \pgfpathqlineto{28.5714bp}{28.5714bp}
    \pgfusepathqstroke
  \end{pgfscope}
  \begin{pgfscope}
    \definecolor{fc}{rgb}{0.0000,0.0000,0.0000}
    \pgfsetfillcolor{fc}
    \pgfusepathqfill
  \end{pgfscope}
  \begin{pgfscope}
    \definecolor{fc}{rgb}{0.0000,0.0000,0.0000}
    \pgfsetfillcolor{fc}
    \pgfusepathqfill
  \end{pgfscope}
  \begin{pgfscope}
    \definecolor{fc}{rgb}{0.0000,0.0000,0.0000}
    \pgfsetfillcolor{fc}
    \pgfusepathqfill
  \end{pgfscope}
  \begin{pgfscope}
    \definecolor{fc}{rgb}{0.0000,0.0000,0.0000}
    \pgfsetfillcolor{fc}
    \pgfusepathqfill
  \end{pgfscope}
  \begin{pgfscope}
    \definecolor{fc}{rgb}{0.0000,0.0000,0.0000}
    \pgfsetfillcolor{fc}
    \pgftransformshift{\pgfqpoint{178.5714bp}{178.5714bp}}
    \pgftransformscale{1.7857}
    \pgftext[base,left]{$p_2$}
  \end{pgfscope}
  \begin{pgfscope}
    \definecolor{fc}{rgb}{0.0000,0.0000,0.0000}
    \pgfsetfillcolor{fc}
    \pgfsetlinewidth{0.8000bp}
    \definecolor{sc}{rgb}{0.0000,0.0000,0.0000}
    \pgfsetstrokecolor{sc}
    \pgfsetmiterjoin
    \pgfsetbuttcap
    \pgfpathqmoveto{175.0000bp}{171.4286bp}
    \pgfpathqcurveto{175.0000bp}{173.4010bp}{173.4010bp}{175.0000bp}{171.4286bp}{175.0000bp}
    \pgfpathqcurveto{169.4561bp}{175.0000bp}{167.8571bp}{173.4010bp}{167.8571bp}{171.4286bp}
    \pgfpathqcurveto{167.8571bp}{169.4561bp}{169.4561bp}{167.8571bp}{171.4286bp}{167.8571bp}
    \pgfpathqcurveto{173.4010bp}{167.8571bp}{175.0000bp}{169.4561bp}{175.0000bp}{171.4286bp}
    \pgfpathclose
    \pgfusepathqfillstroke
  \end{pgfscope}
  \begin{pgfscope}
    \definecolor{fc}{rgb}{0.0000,0.0000,0.0000}
    \pgfsetfillcolor{fc}
    \pgftransformshift{\pgfqpoint{107.1429bp}{178.5714bp}}
    \pgftransformscale{1.7857}
    \pgftext[base,left]{$p_1$}
  \end{pgfscope}
  \begin{pgfscope}
    \definecolor{fc}{rgb}{0.0000,0.0000,0.0000}
    \pgfsetfillcolor{fc}
    \pgfsetlinewidth{0.8000bp}
    \definecolor{sc}{rgb}{0.0000,0.0000,0.0000}
    \pgfsetstrokecolor{sc}
    \pgfsetmiterjoin
    \pgfsetbuttcap
    \pgfpathqmoveto{103.5714bp}{171.4286bp}
    \pgfpathqcurveto{103.5714bp}{173.4010bp}{101.9724bp}{175.0000bp}{100.0000bp}{175.0000bp}
    \pgfpathqcurveto{98.0276bp}{175.0000bp}{96.4286bp}{173.4010bp}{96.4286bp}{171.4286bp}
    \pgfpathqcurveto{96.4286bp}{169.4561bp}{98.0276bp}{167.8571bp}{100.0000bp}{167.8571bp}
    \pgfpathqcurveto{101.9724bp}{167.8571bp}{103.5714bp}{169.4561bp}{103.5714bp}{171.4286bp}
    \pgfpathclose
    \pgfusepathqfillstroke
  \end{pgfscope}
  \begin{pgfscope}
    \pgfsetlinewidth{1.5000bp}
    \definecolor{sc}{rgb}{0.0000,0.0000,0.0000}
    \pgfsetstrokecolor{sc}
    \pgfsetmiterjoin
    \pgfsetbuttcap
    \pgfpathqmoveto{100.0000bp}{171.4286bp}
    \pgfpathqlineto{171.4286bp}{171.4286bp}
    \pgfusepathqstroke
  \end{pgfscope}
  \begin{pgfscope}
    \definecolor{fc}{rgb}{0.0000,0.0000,0.0000}
    \pgfsetfillcolor{fc}
    \pgfusepathqfill
  \end{pgfscope}
  \begin{pgfscope}
    \definecolor{fc}{rgb}{0.0000,0.0000,0.0000}
    \pgfsetfillcolor{fc}
    \pgfusepathqfill
  \end{pgfscope}
  \begin{pgfscope}
    \definecolor{fc}{rgb}{0.0000,0.0000,0.0000}
    \pgfsetfillcolor{fc}
    \pgfusepathqfill
  \end{pgfscope}
  \begin{pgfscope}
    \definecolor{fc}{rgb}{0.0000,0.0000,0.0000}
    \pgfsetfillcolor{fc}
    \pgfusepathqfill
  \end{pgfscope}
  \begin{pgfscope}
    \definecolor{fc}{rgb}{0.0000,0.0000,0.0000}
    \pgfsetfillcolor{fc}
    \pgftransformshift{\pgfqpoint{35.7143bp}{178.5714bp}}
    \pgftransformscale{1.7857}
    \pgftext[base,left]{$q_1$}
  \end{pgfscope}
  \begin{pgfscope}
    \definecolor{fc}{rgb}{0.0000,0.0000,0.0000}
    \pgfsetfillcolor{fc}
    \pgfsetlinewidth{0.8000bp}
    \definecolor{sc}{rgb}{0.0000,0.0000,0.0000}
    \pgfsetstrokecolor{sc}
    \pgfsetmiterjoin
    \pgfsetbuttcap
    \pgfpathqmoveto{32.1429bp}{171.4286bp}
    \pgfpathqcurveto{32.1429bp}{173.4010bp}{30.5439bp}{175.0000bp}{28.5714bp}{175.0000bp}
    \pgfpathqcurveto{26.5990bp}{175.0000bp}{25.0000bp}{173.4010bp}{25.0000bp}{171.4286bp}
    \pgfpathqcurveto{25.0000bp}{169.4561bp}{26.5990bp}{167.8571bp}{28.5714bp}{167.8571bp}
    \pgfpathqcurveto{30.5439bp}{167.8571bp}{32.1429bp}{169.4561bp}{32.1429bp}{171.4286bp}
    \pgfpathclose
    \pgfusepathqfillstroke
  \end{pgfscope}
\end{pgfpicture}
}
            \caption{}
            \label{fig:ex:ca:hgma:ex:isomorphic}
        \end{subfigure}%
        \hfill%
        \begin{subfigure}[b]{0.4\textwidth}
            \centering
            \resizebox{0.75\linewidth}{!}{\begin{pgfpicture}
  \pgfpathrectangle{\pgfpointorigin}{\pgfqpoint{200.0000bp}{200.0000bp}}
  \pgfusepath{use as bounding box}
  \begin{pgfscope}
    \definecolor{fc}{rgb}{0.0000,0.0000,0.0000}
    \pgfsetfillcolor{fc}
    \pgfsetlinewidth{0.5000bp}
    \definecolor{sc}{rgb}{0.0000,0.0000,0.0000}
    \pgfsetstrokecolor{sc}
    \pgfsetmiterjoin
    \pgfsetbuttcap
    \pgfpathqmoveto{200.0000bp}{100.0000bp}
    \pgfpathqcurveto{200.0000bp}{120.7107bp}{183.2107bp}{137.5000bp}{162.5000bp}{137.5000bp}
    \pgfpathqcurveto{141.7893bp}{137.5000bp}{125.0000bp}{120.7107bp}{125.0000bp}{100.0000bp}
    \pgfpathqcurveto{125.0000bp}{79.2893bp}{141.7893bp}{62.5000bp}{162.5000bp}{62.5000bp}
    \pgfpathqcurveto{183.2107bp}{62.5000bp}{200.0000bp}{79.2893bp}{200.0000bp}{100.0000bp}
    \pgfpathclose
    \pgfusepathqfillstroke
  \end{pgfscope}
  \begin{pgfscope}
    \definecolor{fc}{rgb}{1.0000,1.0000,1.0000}
    \pgfsetfillcolor{fc}
    \pgfsetlinewidth{0.5000bp}
    \definecolor{sc}{rgb}{1.0000,1.0000,1.0000}
    \pgfsetstrokecolor{sc}
    \pgfsetmiterjoin
    \pgfsetbuttcap
    \pgfpathqmoveto{198.7500bp}{100.0000bp}
    \pgfpathqcurveto{198.7500bp}{120.0203bp}{182.5203bp}{136.2500bp}{162.5000bp}{136.2500bp}
    \pgfpathqcurveto{142.4797bp}{136.2500bp}{126.2500bp}{120.0203bp}{126.2500bp}{100.0000bp}
    \pgfpathqcurveto{126.2500bp}{79.9797bp}{142.4797bp}{63.7500bp}{162.5000bp}{63.7500bp}
    \pgfpathqcurveto{182.5203bp}{63.7500bp}{198.7500bp}{79.9797bp}{198.7500bp}{100.0000bp}
    \pgfpathclose
    \pgfusepathqfillstroke
  \end{pgfscope}
  \begin{pgfscope}
    \definecolor{fc}{rgb}{0.0000,0.0000,0.0000}
    \pgfsetfillcolor{fc}
    \pgfsetlinewidth{0.5000bp}
    \definecolor{sc}{rgb}{0.0000,0.0000,0.0000}
    \pgfsetstrokecolor{sc}
    \pgfsetmiterjoin
    \pgfsetbuttcap
    \pgfpathqmoveto{37.5000bp}{137.5000bp}
    \pgfpathqcurveto{16.7893bp}{137.5000bp}{0.0000bp}{120.7107bp}{0.0000bp}{100.0000bp}
    \pgfpathqcurveto{-0.0000bp}{79.2893bp}{16.7893bp}{62.5000bp}{37.5000bp}{62.5000bp}
    \pgfpathqlineto{100.0000bp}{62.5000bp}
    \pgfpathqcurveto{120.7107bp}{62.5000bp}{137.5000bp}{79.2893bp}{137.5000bp}{100.0000bp}
    \pgfpathqcurveto{137.5000bp}{120.7107bp}{120.7107bp}{137.5000bp}{100.0000bp}{137.5000bp}
    \pgfpathqlineto{37.5000bp}{137.5000bp}
    \pgfpathclose
    \pgfusepathqfillstroke
  \end{pgfscope}
  \begin{pgfscope}
    \definecolor{fc}{rgb}{1.0000,1.0000,1.0000}
    \pgfsetfillcolor{fc}
    \pgfsetlinewidth{0.5000bp}
    \definecolor{sc}{rgb}{1.0000,1.0000,1.0000}
    \pgfsetstrokecolor{sc}
    \pgfsetmiterjoin
    \pgfsetbuttcap
    \pgfpathqmoveto{37.5000bp}{136.2500bp}
    \pgfpathqcurveto{17.4797bp}{136.2500bp}{1.2500bp}{120.0203bp}{1.2500bp}{100.0000bp}
    \pgfpathqcurveto{1.2500bp}{79.9797bp}{17.4797bp}{63.7500bp}{37.5000bp}{63.7500bp}
    \pgfpathqlineto{100.0000bp}{63.7500bp}
    \pgfpathqcurveto{120.0203bp}{63.7500bp}{136.2500bp}{79.9797bp}{136.2500bp}{100.0000bp}
    \pgfpathqcurveto{136.2500bp}{120.0203bp}{120.0203bp}{136.2500bp}{100.0000bp}{136.2500bp}
    \pgfpathqlineto{37.5000bp}{136.2500bp}
    \pgfpathclose
    \pgfusepathqfillstroke
  \end{pgfscope}
  \begin{pgfscope}
    \definecolor{fc}{rgb}{0.0000,0.0000,0.0000}
    \pgfsetfillcolor{fc}
    \pgftransformshift{\pgfqpoint{168.7500bp}{106.2500bp}}
    \pgftransformscale{1.5625}
    \pgftext[base,left]{$p_2$}
  \end{pgfscope}
  \begin{pgfscope}
    \definecolor{fc}{rgb}{0.0000,0.0000,0.0000}
    \pgfsetfillcolor{fc}
    \pgfsetlinewidth{0.5000bp}
    \definecolor{sc}{rgb}{0.0000,0.0000,0.0000}
    \pgfsetstrokecolor{sc}
    \pgfsetmiterjoin
    \pgfsetbuttcap
    \pgfpathqmoveto{165.6250bp}{100.0000bp}
    \pgfpathqcurveto{165.6250bp}{101.7259bp}{164.2259bp}{103.1250bp}{162.5000bp}{103.1250bp}
    \pgfpathqcurveto{160.7741bp}{103.1250bp}{159.3750bp}{101.7259bp}{159.3750bp}{100.0000bp}
    \pgfpathqcurveto{159.3750bp}{98.2741bp}{160.7741bp}{96.8750bp}{162.5000bp}{96.8750bp}
    \pgfpathqcurveto{164.2259bp}{96.8750bp}{165.6250bp}{98.2741bp}{165.6250bp}{100.0000bp}
    \pgfpathclose
    \pgfusepathqfillstroke
  \end{pgfscope}
  \begin{pgfscope}
    \definecolor{fc}{rgb}{0.0000,0.0000,0.0000}
    \pgfsetfillcolor{fc}
    \pgftransformshift{\pgfqpoint{106.2500bp}{106.2500bp}}
    \pgftransformscale{1.5625}
    \pgftext[base,left]{$p_1$}
  \end{pgfscope}
  \begin{pgfscope}
    \definecolor{fc}{rgb}{0.0000,0.0000,0.0000}
    \pgfsetfillcolor{fc}
    \pgfsetlinewidth{0.5000bp}
    \definecolor{sc}{rgb}{0.0000,0.0000,0.0000}
    \pgfsetstrokecolor{sc}
    \pgfsetmiterjoin
    \pgfsetbuttcap
    \pgfpathqmoveto{103.1250bp}{100.0000bp}
    \pgfpathqcurveto{103.1250bp}{101.7259bp}{101.7259bp}{103.1250bp}{100.0000bp}{103.1250bp}
    \pgfpathqcurveto{98.2741bp}{103.1250bp}{96.8750bp}{101.7259bp}{96.8750bp}{100.0000bp}
    \pgfpathqcurveto{96.8750bp}{98.2741bp}{98.2741bp}{96.8750bp}{100.0000bp}{96.8750bp}
    \pgfpathqcurveto{101.7259bp}{96.8750bp}{103.1250bp}{98.2741bp}{103.1250bp}{100.0000bp}
    \pgfpathclose
    \pgfusepathqfillstroke
  \end{pgfscope}
  \begin{pgfscope}
    \pgfsetlinewidth{0.9186bp}
    \definecolor{sc}{rgb}{0.0000,0.0000,0.0000}
    \pgfsetstrokecolor{sc}
    \pgfsetmiterjoin
    \pgfsetbuttcap
    \pgfpathqmoveto{100.0000bp}{100.0000bp}
    \pgfpathqlineto{162.5000bp}{100.0000bp}
    \pgfusepathqstroke
  \end{pgfscope}
  \begin{pgfscope}
    \definecolor{fc}{rgb}{0.0000,0.0000,0.0000}
    \pgfsetfillcolor{fc}
    \pgfusepathqfill
  \end{pgfscope}
  \begin{pgfscope}
    \definecolor{fc}{rgb}{0.0000,0.0000,0.0000}
    \pgfsetfillcolor{fc}
    \pgfusepathqfill
  \end{pgfscope}
  \begin{pgfscope}
    \definecolor{fc}{rgb}{0.0000,0.0000,0.0000}
    \pgfsetfillcolor{fc}
    \pgfusepathqfill
  \end{pgfscope}
  \begin{pgfscope}
    \definecolor{fc}{rgb}{0.0000,0.0000,0.0000}
    \pgfsetfillcolor{fc}
    \pgfusepathqfill
  \end{pgfscope}
  \begin{pgfscope}
    \definecolor{fc}{rgb}{0.0000,0.0000,0.0000}
    \pgfsetfillcolor{fc}
    \pgftransformshift{\pgfqpoint{43.7500bp}{106.2500bp}}
    \pgftransformscale{1.5625}
    \pgftext[base,left]{$q_1$}
  \end{pgfscope}
  \begin{pgfscope}
    \definecolor{fc}{rgb}{0.0000,0.0000,0.0000}
    \pgfsetfillcolor{fc}
    \pgfsetlinewidth{0.5000bp}
    \definecolor{sc}{rgb}{0.0000,0.0000,0.0000}
    \pgfsetstrokecolor{sc}
    \pgfsetmiterjoin
    \pgfsetbuttcap
    \pgfpathqmoveto{40.6250bp}{100.0000bp}
    \pgfpathqcurveto{40.6250bp}{101.7259bp}{39.2259bp}{103.1250bp}{37.5000bp}{103.1250bp}
    \pgfpathqcurveto{35.7741bp}{103.1250bp}{34.3750bp}{101.7259bp}{34.3750bp}{100.0000bp}
    \pgfpathqcurveto{34.3750bp}{98.2741bp}{35.7741bp}{96.8750bp}{37.5000bp}{96.8750bp}
    \pgfpathqcurveto{39.2259bp}{96.8750bp}{40.6250bp}{98.2741bp}{40.6250bp}{100.0000bp}
    \pgfpathclose
    \pgfusepathqfillstroke
  \end{pgfscope}
\end{pgfpicture}
}
            \caption{}
            \label{fig:ex:ca:hgma:ex:isomorphic_reduced}
        \end{subfigure}
        \hfill
    \end{figure}
\end{example}

\begin{algorithm}
    \caption{Binding edge merging algorithm}
    \label{algo:bindingedgemerge}
    \begin{algorithmic}
        \Function {$\textsc{MergeBindingEdges}$} {$H_1, H_2, H', b_1, b_2$}
            \State $b' \gets b_1 \sqcap b_2$
            \State $H' \gets H' \cup b'$
            \ForAll {vertices $v \in b'$}
                \If {there is no predicate edge in $H'$ containing $v$}
                    \State $p_1 \gets$ predicate set in $H_1$ containing $f_1^{-1}(v)$
                    \State $p_2 \gets$ predicate set in $H_2$ containing $f_2^{-1}(v)$
                    \State $H' \gets \textsc{MergePredicateEdges}
                                        \left( H_1, H_2, H', p_1, p_2 \right)$
                \EndIf
            \EndFor
            \State \Return $H'$
        \EndFunction
    \end{algorithmic}
\end{algorithm}

\begin{algorithm}
    \caption{Predicate edge merging algorithm}
    \label{algo:prededgemerge}

    \begin{algorithmic}
        \Function {$\textsc{MergePredicateEdges}$} {$H_1, H_2, H', p_1, p_2$}
            \State $p' \gets p_1 \sqcap p_2$
            \State $H' \gets H' \cup p'$
            \ForAll {vertices $v \in p'$}
                \If {there is no binding edge in $H'$ containing $v$}
                    \State $b_1 \gets$ the binding edge in $H_1$ containing $f_1^{-1}(v)$
                    \State $b_2 \gets$ the binding edge in $H_2$ containing $f_2^{-1}(v)$
                    \State $H' \gets \textsc{MergeBindingEdges}
                        \left( H_1, H_2, H', b_1, b_2 \right)$
                \EndIf
            \EndFor
            \State \Return $H'$
        \EndFunction
    \end{algorithmic}
\end{algorithm}

\begin{algorithm}
    \caption{Hyper graph merging algorithm}
    \label{algo:hypergraphmerge}
    \begin{algorithmic}
        \Function {$\textsc{MergeHyperGraphs}$} {$H_1$, $H_2$}
            \State Let $q_1$ and $q_2$ denote the predicate hyper edges describing the effect in $H_1$ and $H_2$, respectively
            \State Let $H' = \emptyset$ be a new, empty hyper graph
            \State $H' \gets \textsc{MergePredicateEdges}(H_1,H_2,H',q_1,q_2)$
            \State $H' \gets \textsc{CollapseHyperGraph}(H')$
            \State \Return $H'$
        \EndFunction
    \end{algorithmic}
\end{algorithm}

\begin{example} \label{ex:ca:hgma:disconnected}
    \begin{equation*}
        H_1 = \forall x, y, z : q(x) \quad \text{when} \quad
            p(x,y) \land p(y, z)
    \end{equation*}

    \begin{equation*}
        H_2 = \forall \alpha, \beta, \gamma : q(\alpha) \quad \text{when} \quad
            p(\alpha, \beta) \land p(\gamma, \delta)
    \end{equation*}

    \begin{equation*}
        H_1 \sqcap H_2 = \forall x, y : q(x) \quad \text{when} \quad p(x, y)
    \end{equation*}

    \begin{figure}
        <placeholder>
        \caption{\label{fig:ex:ca:hgma:ex:disconnected} Figure of hypergraphs for example \ref{ex:ca:hgma:disconnected}.}
    \end{figure}

    Note that the precondition decribed by $H_1$ is more restrictive than that of $H_2$. Consequentially, if $H_1$ was the real precondition, the conditional effect would not have succeeded for $H_2$.
\end{example}

If, on the other hand, $v_j$ does exist in $e_2(q_i)$, then it cannot be discarded as a precondition. Furthermore, if there exists several nodes with name $v_j$ in $e_2(q_i)$, then it may be the case that

\begin{example} \label{ex:ca:hgma:generalization}
    \begin{equation*}
        H_1 = \forall x, y : q(x) \quad \text{when} \quad
            p(x,y) \land f(y) \land g(y)
    \end{equation*}

    \begin{equation*}
        H_2 = \forall x, y, z : q(x) \quad \text{when} \quad
            p(x, y) \land p(x,z) \land f(y) \land g(z)
    \end{equation*}

    \begin{equation*}
        H_1 \sqcap H_2 = H_2
    \end{equation*}

    \begin{figure}
        \centering
        \begin{subfigure}[b]{0.4\textwidth}
            \centering
            \resizebox{\linewidth}{!}{\begin{pgfpicture}
  \pgfpathrectangle{\pgfpointorigin}{\pgfqpoint{200.0000bp}{200.0000bp}}
  \pgfusepath{use as bounding box}
  \begin{pgfscope}
    \definecolor{fc}{rgb}{0.0000,0.0000,0.0000}
    \pgfsetfillcolor{fc}
    \pgfsetlinewidth{0.5506bp}
    \definecolor{sc}{rgb}{0.0000,0.0000,0.0000}
    \pgfsetstrokecolor{sc}
    \pgfsetmiterjoin
    \pgfsetbuttcap
    \pgfpathqmoveto{105.2632bp}{73.6842bp}
    \pgfpathqcurveto{105.2632bp}{62.0572bp}{114.6887bp}{52.6316bp}{126.3158bp}{52.6316bp}
    \pgfpathqcurveto{137.9428bp}{52.6316bp}{147.3684bp}{62.0572bp}{147.3684bp}{73.6842bp}
    \pgfpathqlineto{147.3684bp}{126.3158bp}
    \pgfpathqcurveto{147.3684bp}{137.9428bp}{137.9428bp}{147.3684bp}{126.3158bp}{147.3684bp}
    \pgfpathqcurveto{114.6887bp}{147.3684bp}{105.2632bp}{137.9428bp}{105.2632bp}{126.3158bp}
    \pgfpathqcurveto{105.2632bp}{114.6887bp}{114.6887bp}{105.2632bp}{126.3158bp}{105.2632bp}
    \pgfpathqlineto{178.9474bp}{105.2632bp}
    \pgfpathqcurveto{190.5744bp}{105.2632bp}{200.0000bp}{114.6887bp}{200.0000bp}{126.3158bp}
    \pgfpathqcurveto{200.0000bp}{137.9428bp}{190.5744bp}{147.3684bp}{178.9474bp}{147.3684bp}
    \pgfpathqlineto{126.3158bp}{147.3684bp}
    \pgfpathqcurveto{114.6887bp}{147.3684bp}{105.2632bp}{137.9428bp}{105.2632bp}{126.3158bp}
    \pgfpathqlineto{105.2632bp}{73.6842bp}
    \pgfpathclose
    \pgfusepathqfillstroke
  \end{pgfscope}
  \begin{pgfscope}
    \definecolor{fc}{rgb}{1.0000,1.0000,1.0000}
    \pgfsetfillcolor{fc}
    \pgfsetlinewidth{0.5506bp}
    \definecolor{sc}{rgb}{1.0000,1.0000,1.0000}
    \pgfsetstrokecolor{sc}
    \pgfsetmiterjoin
    \pgfsetbuttcap
    \pgfpathqmoveto{106.3158bp}{73.6842bp}
    \pgfpathqcurveto{106.3158bp}{62.6385bp}{115.2701bp}{53.6842bp}{126.3158bp}{53.6842bp}
    \pgfpathqcurveto{137.3615bp}{53.6842bp}{146.3158bp}{62.6385bp}{146.3158bp}{73.6842bp}
    \pgfpathqlineto{146.3158bp}{126.3158bp}
    \pgfpathqcurveto{146.3158bp}{137.3615bp}{137.3615bp}{146.3158bp}{126.3158bp}{146.3158bp}
    \pgfpathqcurveto{115.2701bp}{146.3158bp}{106.3158bp}{137.3615bp}{106.3158bp}{126.3158bp}
    \pgfpathqcurveto{106.3158bp}{115.2701bp}{115.2701bp}{106.3158bp}{126.3158bp}{106.3158bp}
    \pgfpathqlineto{178.9474bp}{106.3158bp}
    \pgfpathqcurveto{189.9931bp}{106.3158bp}{198.9474bp}{115.2701bp}{198.9474bp}{126.3158bp}
    \pgfpathqcurveto{198.9474bp}{137.3615bp}{189.9931bp}{146.3158bp}{178.9474bp}{146.3158bp}
    \pgfpathqlineto{126.3158bp}{146.3158bp}
    \pgfpathqcurveto{115.2701bp}{146.3158bp}{106.3158bp}{137.3615bp}{106.3158bp}{126.3158bp}
    \pgfpathqlineto{106.3158bp}{73.6842bp}
    \pgfpathclose
    \pgfusepathqfillstroke
  \end{pgfscope}
  \begin{pgfscope}
    \definecolor{fc}{rgb}{0.0000,0.0000,0.0000}
    \pgfsetfillcolor{fc}
    \pgfsetlinewidth{0.5506bp}
    \definecolor{sc}{rgb}{0.0000,0.0000,0.0000}
    \pgfsetstrokecolor{sc}
    \pgfsetmiterjoin
    \pgfsetbuttcap
    \pgfpathqmoveto{21.0526bp}{147.3684bp}
    \pgfpathqcurveto{9.4256bp}{147.3684bp}{0.0000bp}{137.9428bp}{0.0000bp}{126.3158bp}
    \pgfpathqcurveto{0.0000bp}{114.6887bp}{9.4256bp}{105.2632bp}{21.0526bp}{105.2632bp}
    \pgfpathqlineto{73.6842bp}{105.2632bp}
    \pgfpathqcurveto{85.3113bp}{105.2632bp}{94.7368bp}{114.6887bp}{94.7368bp}{126.3158bp}
    \pgfpathqcurveto{94.7368bp}{137.9428bp}{85.3113bp}{147.3684bp}{73.6842bp}{147.3684bp}
    \pgfpathqlineto{21.0526bp}{147.3684bp}
    \pgfpathclose
    \pgfusepathqfillstroke
  \end{pgfscope}
  \begin{pgfscope}
    \definecolor{fc}{rgb}{1.0000,1.0000,1.0000}
    \pgfsetfillcolor{fc}
    \pgfsetlinewidth{0.5506bp}
    \definecolor{sc}{rgb}{1.0000,1.0000,1.0000}
    \pgfsetstrokecolor{sc}
    \pgfsetmiterjoin
    \pgfsetbuttcap
    \pgfpathqmoveto{21.0526bp}{146.3158bp}
    \pgfpathqcurveto{10.0069bp}{146.3158bp}{1.0526bp}{137.3615bp}{1.0526bp}{126.3158bp}
    \pgfpathqcurveto{1.0526bp}{115.2701bp}{10.0069bp}{106.3158bp}{21.0526bp}{106.3158bp}
    \pgfpathqlineto{73.6842bp}{106.3158bp}
    \pgfpathqcurveto{84.7299bp}{106.3158bp}{93.6842bp}{115.2701bp}{93.6842bp}{126.3158bp}
    \pgfpathqcurveto{93.6842bp}{137.3615bp}{84.7299bp}{146.3158bp}{73.6842bp}{146.3158bp}
    \pgfpathqlineto{21.0526bp}{146.3158bp}
    \pgfpathclose
    \pgfusepathqfillstroke
  \end{pgfscope}
  \begin{pgfscope}
    \definecolor{fc}{rgb}{0.0000,0.0000,0.0000}
    \pgfsetfillcolor{fc}
    \pgftransformcm{1.0000}{0.0000}{0.0000}{1.0000}{\pgfqpoint{184.2105bp}{131.5789bp}}
    \pgftransformscale{1.3158}
    \pgftext[base,left]{$g_1$}
  \end{pgfscope}
  \begin{pgfscope}
    \definecolor{fc}{rgb}{0.0000,0.0000,0.0000}
    \pgfsetfillcolor{fc}
    \pgfsetlinewidth{0.5506bp}
    \definecolor{sc}{rgb}{0.0000,0.0000,0.0000}
    \pgfsetstrokecolor{sc}
    \pgfsetmiterjoin
    \pgfsetbuttcap
    \pgfpathqmoveto{181.5789bp}{126.3158bp}
    \pgfpathqcurveto{181.5789bp}{127.7692bp}{180.4007bp}{128.9474bp}{178.9474bp}{128.9474bp}
    \pgfpathqcurveto{177.4940bp}{128.9474bp}{176.3158bp}{127.7692bp}{176.3158bp}{126.3158bp}
    \pgfpathqcurveto{176.3158bp}{124.8624bp}{177.4940bp}{123.6842bp}{178.9474bp}{123.6842bp}
    \pgfpathqcurveto{180.4007bp}{123.6842bp}{181.5789bp}{124.8624bp}{181.5789bp}{126.3158bp}
    \pgfpathclose
    \pgfusepathqfillstroke
  \end{pgfscope}
  \begin{pgfscope}
    \definecolor{fc}{rgb}{0.0000,0.0000,0.0000}
    \pgfsetfillcolor{fc}
    \pgftransformcm{1.0000}{0.0000}{0.0000}{1.0000}{\pgfqpoint{131.5789bp}{78.9474bp}}
    \pgftransformscale{1.3158}
    \pgftext[base,left]{$f_1$}
  \end{pgfscope}
  \begin{pgfscope}
    \definecolor{fc}{rgb}{0.0000,0.0000,0.0000}
    \pgfsetfillcolor{fc}
    \pgfsetlinewidth{0.5506bp}
    \definecolor{sc}{rgb}{0.0000,0.0000,0.0000}
    \pgfsetstrokecolor{sc}
    \pgfsetmiterjoin
    \pgfsetbuttcap
    \pgfpathqmoveto{128.9474bp}{73.6842bp}
    \pgfpathqcurveto{128.9474bp}{75.1376bp}{127.7692bp}{76.3158bp}{126.3158bp}{76.3158bp}
    \pgfpathqcurveto{124.8624bp}{76.3158bp}{123.6842bp}{75.1376bp}{123.6842bp}{73.6842bp}
    \pgfpathqcurveto{123.6842bp}{72.2308bp}{124.8624bp}{71.0526bp}{126.3158bp}{71.0526bp}
    \pgfpathqcurveto{127.7692bp}{71.0526bp}{128.9474bp}{72.2308bp}{128.9474bp}{73.6842bp}
    \pgfpathclose
    \pgfusepathqfillstroke
  \end{pgfscope}
  \begin{pgfscope}
    \definecolor{fc}{rgb}{0.0000,0.0000,0.0000}
    \pgfsetfillcolor{fc}
    \pgftransformcm{1.0000}{0.0000}{0.0000}{1.0000}{\pgfqpoint{131.5789bp}{131.5789bp}}
    \pgftransformscale{1.3158}
    \pgftext[base,left]{$p_2$}
  \end{pgfscope}
  \begin{pgfscope}
    \definecolor{fc}{rgb}{0.0000,0.0000,0.0000}
    \pgfsetfillcolor{fc}
    \pgfsetlinewidth{0.5506bp}
    \definecolor{sc}{rgb}{0.0000,0.0000,0.0000}
    \pgfsetstrokecolor{sc}
    \pgfsetmiterjoin
    \pgfsetbuttcap
    \pgfpathqmoveto{128.9474bp}{126.3158bp}
    \pgfpathqcurveto{128.9474bp}{127.7692bp}{127.7692bp}{128.9474bp}{126.3158bp}{128.9474bp}
    \pgfpathqcurveto{124.8624bp}{128.9474bp}{123.6842bp}{127.7692bp}{123.6842bp}{126.3158bp}
    \pgfpathqcurveto{123.6842bp}{124.8624bp}{124.8624bp}{123.6842bp}{126.3158bp}{123.6842bp}
    \pgfpathqcurveto{127.7692bp}{123.6842bp}{128.9474bp}{124.8624bp}{128.9474bp}{126.3158bp}
    \pgfpathclose
    \pgfusepathqfillstroke
  \end{pgfscope}
  \begin{pgfscope}
    \definecolor{fc}{rgb}{0.0000,0.0000,0.0000}
    \pgfsetfillcolor{fc}
    \pgftransformcm{1.0000}{0.0000}{0.0000}{1.0000}{\pgfqpoint{78.9474bp}{131.5789bp}}
    \pgftransformscale{1.3158}
    \pgftext[base,left]{$p_1$}
  \end{pgfscope}
  \begin{pgfscope}
    \definecolor{fc}{rgb}{0.0000,0.0000,0.0000}
    \pgfsetfillcolor{fc}
    \pgfsetlinewidth{0.5506bp}
    \definecolor{sc}{rgb}{0.0000,0.0000,0.0000}
    \pgfsetstrokecolor{sc}
    \pgfsetmiterjoin
    \pgfsetbuttcap
    \pgfpathqmoveto{76.3158bp}{126.3158bp}
    \pgfpathqcurveto{76.3158bp}{127.7692bp}{75.1376bp}{128.9474bp}{73.6842bp}{128.9474bp}
    \pgfpathqcurveto{72.2308bp}{128.9474bp}{71.0526bp}{127.7692bp}{71.0526bp}{126.3158bp}
    \pgfpathqcurveto{71.0526bp}{124.8624bp}{72.2308bp}{123.6842bp}{73.6842bp}{123.6842bp}
    \pgfpathqcurveto{75.1376bp}{123.6842bp}{76.3158bp}{124.8624bp}{76.3158bp}{126.3158bp}
    \pgfpathclose
    \pgfusepathqfillstroke
  \end{pgfscope}
  \begin{pgfscope}
    \pgfsetlinewidth{1.0324bp}
    \definecolor{sc}{rgb}{0.0000,0.0000,0.0000}
    \pgfsetstrokecolor{sc}
    \pgfsetmiterjoin
    \pgfsetbuttcap
    \pgfpathqmoveto{73.6842bp}{126.3158bp}
    \pgfpathqlineto{126.3158bp}{126.3158bp}
    \pgfusepathqstroke
  \end{pgfscope}
  \begin{pgfscope}
    \definecolor{fc}{rgb}{0.0000,0.0000,0.0000}
    \pgfsetfillcolor{fc}
    \pgfusepathqfill
  \end{pgfscope}
  \begin{pgfscope}
    \definecolor{fc}{rgb}{0.0000,0.0000,0.0000}
    \pgfsetfillcolor{fc}
    \pgfusepathqfill
  \end{pgfscope}
  \begin{pgfscope}
    \definecolor{fc}{rgb}{0.0000,0.0000,0.0000}
    \pgfsetfillcolor{fc}
    \pgfusepathqfill
  \end{pgfscope}
  \begin{pgfscope}
    \definecolor{fc}{rgb}{0.0000,0.0000,0.0000}
    \pgfsetfillcolor{fc}
    \pgfusepathqfill
  \end{pgfscope}
  \begin{pgfscope}
    \definecolor{fc}{rgb}{0.0000,0.0000,0.0000}
    \pgfsetfillcolor{fc}
    \pgftransformcm{1.0000}{0.0000}{0.0000}{1.0000}{\pgfqpoint{26.3158bp}{131.5789bp}}
    \pgftransformscale{1.3158}
    \pgftext[base,left]{$q_1$}
  \end{pgfscope}
  \begin{pgfscope}
    \definecolor{fc}{rgb}{0.0000,0.0000,0.0000}
    \pgfsetfillcolor{fc}
    \pgfsetlinewidth{0.5506bp}
    \definecolor{sc}{rgb}{0.0000,0.0000,0.0000}
    \pgfsetstrokecolor{sc}
    \pgfsetmiterjoin
    \pgfsetbuttcap
    \pgfpathqmoveto{23.6842bp}{126.3158bp}
    \pgfpathqcurveto{23.6842bp}{127.7692bp}{22.5060bp}{128.9474bp}{21.0526bp}{128.9474bp}
    \pgfpathqcurveto{19.5993bp}{128.9474bp}{18.4211bp}{127.7692bp}{18.4211bp}{126.3158bp}
    \pgfpathqcurveto{18.4211bp}{124.8624bp}{19.5993bp}{123.6842bp}{21.0526bp}{123.6842bp}
    \pgfpathqcurveto{22.5060bp}{123.6842bp}{23.6842bp}{124.8624bp}{23.6842bp}{126.3158bp}
    \pgfpathclose
    \pgfusepathqfillstroke
  \end{pgfscope}
\end{pgfpicture}
}
            \caption{}
            \label{fig:ex:ca:hgma:ex:generalization1}
        \end{subfigure}%
        \hfill%
        \begin{subfigure}[b]{0.4\textwidth}
            \centering
            \resizebox{0.8\linewidth}{!}{\begin{pgfpicture}
  \pgfpathrectangle{\pgfpointorigin}{\pgfqpoint{200.0000bp}{200.0000bp}}
  \pgfusepath{use as bounding box}
  \begin{pgfscope}
    \definecolor{fc}{rgb}{0.0000,0.0000,0.0000}
    \pgfsetfillcolor{fc}
    \pgfsetlinewidth{0.8000bp}
    \definecolor{sc}{rgb}{0.0000,0.0000,0.0000}
    \pgfsetstrokecolor{sc}
    \pgfsetmiterjoin
    \pgfsetbuttcap
    \pgfpathqmoveto{28.5714bp}{57.1429bp}
    \pgfpathqcurveto{12.7919bp}{57.1429bp}{0.0000bp}{44.3510bp}{0.0000bp}{28.5714bp}
    \pgfpathqcurveto{0.0000bp}{12.7919bp}{12.7919bp}{0.0000bp}{28.5714bp}{0.0000bp}
    \pgfpathqlineto{100.0000bp}{0.0000bp}
    \pgfpathqcurveto{115.7796bp}{0.0000bp}{128.5714bp}{12.7919bp}{128.5714bp}{28.5714bp}
    \pgfpathqcurveto{128.5714bp}{44.3510bp}{115.7796bp}{57.1429bp}{100.0000bp}{57.1429bp}
    \pgfpathqlineto{28.5714bp}{57.1429bp}
    \pgfpathqcurveto{12.7919bp}{57.1429bp}{0.0000bp}{44.3510bp}{0.0000bp}{28.5714bp}
    \pgfpathqcurveto{0.0000bp}{12.7919bp}{12.7919bp}{0.0000bp}{28.5714bp}{0.0000bp}
    \pgfpathqlineto{100.0000bp}{0.0000bp}
    \pgfpathqcurveto{115.7796bp}{0.0000bp}{128.5714bp}{12.7919bp}{128.5714bp}{28.5714bp}
    \pgfpathqcurveto{128.5714bp}{44.3510bp}{115.7796bp}{57.1429bp}{100.0000bp}{57.1429bp}
    \pgfpathqlineto{28.5714bp}{57.1429bp}
    \pgfpathclose
    \pgfusepathqfillstroke
  \end{pgfscope}
  \begin{pgfscope}
    \definecolor{fc}{rgb}{1.0000,1.0000,1.0000}
    \pgfsetfillcolor{fc}
    \pgfsetlinewidth{0.8000bp}
    \definecolor{sc}{rgb}{1.0000,1.0000,1.0000}
    \pgfsetstrokecolor{sc}
    \pgfsetmiterjoin
    \pgfsetbuttcap
    \pgfpathqmoveto{28.5714bp}{55.7143bp}
    \pgfpathqcurveto{13.5808bp}{55.7143bp}{1.4286bp}{43.5620bp}{1.4286bp}{28.5714bp}
    \pgfpathqcurveto{1.4286bp}{13.5808bp}{13.5808bp}{1.4286bp}{28.5714bp}{1.4286bp}
    \pgfpathqlineto{100.0000bp}{1.4286bp}
    \pgfpathqcurveto{114.9906bp}{1.4286bp}{127.1429bp}{13.5808bp}{127.1429bp}{28.5714bp}
    \pgfpathqcurveto{127.1429bp}{43.5620bp}{114.9906bp}{55.7143bp}{100.0000bp}{55.7143bp}
    \pgfpathqlineto{28.5714bp}{55.7143bp}
    \pgfpathqcurveto{13.5808bp}{55.7143bp}{1.4286bp}{43.5620bp}{1.4286bp}{28.5714bp}
    \pgfpathqcurveto{1.4286bp}{13.5808bp}{13.5808bp}{1.4286bp}{28.5714bp}{1.4286bp}
    \pgfpathqlineto{100.0000bp}{1.4286bp}
    \pgfpathqcurveto{114.9906bp}{1.4286bp}{127.1429bp}{13.5808bp}{127.1429bp}{28.5714bp}
    \pgfpathqcurveto{127.1429bp}{43.5620bp}{114.9906bp}{55.7143bp}{100.0000bp}{55.7143bp}
    \pgfpathqlineto{28.5714bp}{55.7143bp}
    \pgfpathclose
    \pgfusepathqfillstroke
  \end{pgfscope}
  \begin{pgfscope}
    \definecolor{fc}{rgb}{0.0000,0.0000,0.0000}
    \pgfsetfillcolor{fc}
    \pgfsetlinewidth{0.8000bp}
    \definecolor{sc}{rgb}{0.0000,0.0000,0.0000}
    \pgfsetstrokecolor{sc}
    \pgfsetmiterjoin
    \pgfsetbuttcap
    \pgfpathqmoveto{200.0000bp}{171.4286bp}
    \pgfpathqcurveto{200.0000bp}{187.2081bp}{187.2081bp}{200.0000bp}{171.4286bp}{200.0000bp}
    \pgfpathqcurveto{155.6490bp}{200.0000bp}{142.8571bp}{187.2081bp}{142.8571bp}{171.4286bp}
    \pgfpathqlineto{142.8571bp}{100.0000bp}
    \pgfpathqcurveto{142.8571bp}{84.2204bp}{155.6490bp}{71.4286bp}{171.4286bp}{71.4286bp}
    \pgfpathqcurveto{187.2081bp}{71.4286bp}{200.0000bp}{84.2204bp}{200.0000bp}{100.0000bp}
    \pgfpathqlineto{200.0000bp}{171.4286bp}
    \pgfpathqcurveto{200.0000bp}{187.2081bp}{187.2081bp}{200.0000bp}{171.4286bp}{200.0000bp}
    \pgfpathqcurveto{155.6490bp}{200.0000bp}{142.8571bp}{187.2081bp}{142.8571bp}{171.4286bp}
    \pgfpathqlineto{142.8571bp}{100.0000bp}
    \pgfpathqcurveto{142.8571bp}{84.2204bp}{155.6490bp}{71.4286bp}{171.4286bp}{71.4286bp}
    \pgfpathqcurveto{187.2081bp}{71.4286bp}{200.0000bp}{84.2204bp}{200.0000bp}{100.0000bp}
    \pgfpathqlineto{200.0000bp}{171.4286bp}
    \pgfpathclose
    \pgfusepathqfillstroke
  \end{pgfscope}
  \begin{pgfscope}
    \definecolor{fc}{rgb}{1.0000,1.0000,1.0000}
    \pgfsetfillcolor{fc}
    \pgfsetlinewidth{0.8000bp}
    \definecolor{sc}{rgb}{1.0000,1.0000,1.0000}
    \pgfsetstrokecolor{sc}
    \pgfsetmiterjoin
    \pgfsetbuttcap
    \pgfpathqmoveto{198.5714bp}{171.4286bp}
    \pgfpathqcurveto{198.5714bp}{186.4192bp}{186.4192bp}{198.5714bp}{171.4286bp}{198.5714bp}
    \pgfpathqcurveto{156.4380bp}{198.5714bp}{144.2857bp}{186.4192bp}{144.2857bp}{171.4286bp}
    \pgfpathqlineto{144.2857bp}{100.0000bp}
    \pgfpathqcurveto{144.2857bp}{85.0094bp}{156.4380bp}{72.8571bp}{171.4286bp}{72.8571bp}
    \pgfpathqcurveto{186.4192bp}{72.8571bp}{198.5714bp}{85.0094bp}{198.5714bp}{100.0000bp}
    \pgfpathqlineto{198.5714bp}{171.4286bp}
    \pgfpathqcurveto{198.5714bp}{186.4192bp}{186.4192bp}{198.5714bp}{171.4286bp}{198.5714bp}
    \pgfpathqcurveto{156.4380bp}{198.5714bp}{144.2857bp}{186.4192bp}{144.2857bp}{171.4286bp}
    \pgfpathqlineto{144.2857bp}{100.0000bp}
    \pgfpathqcurveto{144.2857bp}{85.0094bp}{156.4380bp}{72.8571bp}{171.4286bp}{72.8571bp}
    \pgfpathqcurveto{186.4192bp}{72.8571bp}{198.5714bp}{85.0094bp}{198.5714bp}{100.0000bp}
    \pgfpathqlineto{198.5714bp}{171.4286bp}
    \pgfpathclose
    \pgfusepathqfillstroke
  \end{pgfscope}
  \begin{pgfscope}
    \definecolor{fc}{rgb}{0.0000,0.0000,0.0000}
    \pgfsetfillcolor{fc}
    \pgfsetlinewidth{0.8000bp}
    \definecolor{sc}{rgb}{0.0000,0.0000,0.0000}
    \pgfsetstrokecolor{sc}
    \pgfsetmiterjoin
    \pgfsetbuttcap
    \pgfpathqmoveto{8.3684bp}{120.2031bp}
    \pgfpathqcurveto{-2.7895bp}{109.0452bp}{-2.7895bp}{90.9548bp}{8.3684bp}{79.7969bp}
    \pgfpathqcurveto{19.5262bp}{68.6391bp}{37.6166bp}{68.6391bp}{48.7745bp}{79.7969bp}
    \pgfpathqlineto{120.2031bp}{151.2255bp}
    \pgfpathqcurveto{131.3609bp}{162.3834bp}{131.3609bp}{180.4738bp}{120.2031bp}{191.6316bp}
    \pgfpathqcurveto{114.8449bp}{196.9898bp}{107.5776bp}{200.0000bp}{100.0000bp}{200.0000bp}
    \pgfpathqlineto{28.5714bp}{200.0000bp}
    \pgfpathqcurveto{12.7919bp}{200.0000bp}{0.0000bp}{187.2081bp}{0.0000bp}{171.4286bp}
    \pgfpathqlineto{0.0000bp}{100.0000bp}
    \pgfpathqcurveto{-0.0000bp}{84.2204bp}{12.7919bp}{71.4286bp}{28.5714bp}{71.4286bp}
    \pgfpathqcurveto{44.3510bp}{71.4286bp}{57.1429bp}{84.2204bp}{57.1429bp}{100.0000bp}
    \pgfpathqlineto{57.1429bp}{171.4286bp}
    \pgfpathqcurveto{57.1429bp}{187.2081bp}{44.3510bp}{200.0000bp}{28.5714bp}{200.0000bp}
    \pgfpathqcurveto{12.7919bp}{200.0000bp}{0.0000bp}{187.2081bp}{0.0000bp}{171.4286bp}
    \pgfpathqcurveto{0.0000bp}{155.6490bp}{12.7919bp}{142.8571bp}{28.5714bp}{142.8571bp}
    \pgfpathqlineto{100.0000bp}{142.8571bp}
    \pgfpathqcurveto{115.7796bp}{142.8571bp}{128.5714bp}{155.6490bp}{128.5714bp}{171.4286bp}
    \pgfpathqcurveto{128.5714bp}{187.2081bp}{115.7796bp}{200.0000bp}{100.0000bp}{200.0000bp}
    \pgfpathqcurveto{92.4224bp}{200.0000bp}{85.1551bp}{196.9898bp}{79.7969bp}{191.6316bp}
    \pgfpathqlineto{8.3684bp}{120.2031bp}
    \pgfpathclose
    \pgfusepathqfillstroke
  \end{pgfscope}
  \begin{pgfscope}
    \definecolor{fc}{rgb}{1.0000,1.0000,1.0000}
    \pgfsetfillcolor{fc}
    \pgfsetlinewidth{0.8000bp}
    \definecolor{sc}{rgb}{1.0000,1.0000,1.0000}
    \pgfsetstrokecolor{sc}
    \pgfsetmiterjoin
    \pgfsetbuttcap
    \pgfpathqmoveto{9.3785bp}{119.1929bp}
    \pgfpathqcurveto{-1.2214bp}{108.5930bp}{-1.2214bp}{91.4070bp}{9.3785bp}{80.8071bp}
    \pgfpathqcurveto{19.9785bp}{70.2072bp}{37.1644bp}{70.2072bp}{47.7643bp}{80.8071bp}
    \pgfpathqlineto{119.1929bp}{152.2357bp}
    \pgfpathqcurveto{129.7928bp}{162.8356bp}{129.7928bp}{180.0215bp}{119.1929bp}{190.6215bp}
    \pgfpathqcurveto{114.1026bp}{195.7117bp}{107.1987bp}{198.5714bp}{100.0000bp}{198.5714bp}
    \pgfpathqlineto{28.5714bp}{198.5714bp}
    \pgfpathqcurveto{13.5808bp}{198.5714bp}{1.4286bp}{186.4192bp}{1.4286bp}{171.4286bp}
    \pgfpathqlineto{1.4286bp}{100.0000bp}
    \pgfpathqcurveto{1.4286bp}{85.0094bp}{13.5808bp}{72.8571bp}{28.5714bp}{72.8571bp}
    \pgfpathqcurveto{43.5620bp}{72.8571bp}{55.7143bp}{85.0094bp}{55.7143bp}{100.0000bp}
    \pgfpathqlineto{55.7143bp}{171.4286bp}
    \pgfpathqcurveto{55.7143bp}{186.4192bp}{43.5620bp}{198.5714bp}{28.5714bp}{198.5714bp}
    \pgfpathqcurveto{13.5808bp}{198.5714bp}{1.4286bp}{186.4192bp}{1.4286bp}{171.4286bp}
    \pgfpathqcurveto{1.4286bp}{156.4380bp}{13.5808bp}{144.2857bp}{28.5714bp}{144.2857bp}
    \pgfpathqlineto{100.0000bp}{144.2857bp}
    \pgfpathqcurveto{114.9906bp}{144.2857bp}{127.1429bp}{156.4380bp}{127.1429bp}{171.4286bp}
    \pgfpathqcurveto{127.1429bp}{186.4192bp}{114.9906bp}{198.5714bp}{100.0000bp}{198.5714bp}
    \pgfpathqcurveto{92.8013bp}{198.5714bp}{85.8974bp}{195.7117bp}{80.8071bp}{190.6215bp}
    \pgfpathqlineto{9.3785bp}{119.1929bp}
    \pgfpathclose
    \pgfusepathqfillstroke
  \end{pgfscope}
  \begin{pgfscope}
    \definecolor{fc}{rgb}{0.0000,0.0000,0.0000}
    \pgfsetfillcolor{fc}
    \pgftransformshift{\pgfqpoint{178.5714bp}{107.1429bp}}
    \pgftransformscale{1.7857}
    \pgftext[base,left]{$g_1$}
  \end{pgfscope}
  \begin{pgfscope}
    \definecolor{fc}{rgb}{0.0000,0.0000,0.0000}
    \pgfsetfillcolor{fc}
    \pgfsetlinewidth{0.8000bp}
    \definecolor{sc}{rgb}{0.0000,0.0000,0.0000}
    \pgfsetstrokecolor{sc}
    \pgfsetmiterjoin
    \pgfsetbuttcap
    \pgfpathqmoveto{175.0000bp}{100.0000bp}
    \pgfpathqcurveto{175.0000bp}{101.9724bp}{173.4010bp}{103.5714bp}{171.4286bp}{103.5714bp}
    \pgfpathqcurveto{169.4561bp}{103.5714bp}{167.8571bp}{101.9724bp}{167.8571bp}{100.0000bp}
    \pgfpathqcurveto{167.8571bp}{98.0276bp}{169.4561bp}{96.4286bp}{171.4286bp}{96.4286bp}
    \pgfpathqcurveto{173.4010bp}{96.4286bp}{175.0000bp}{98.0276bp}{175.0000bp}{100.0000bp}
    \pgfpathclose
    \pgfusepathqfillstroke
  \end{pgfscope}
  \begin{pgfscope}
    \definecolor{fc}{rgb}{0.0000,0.0000,0.0000}
    \pgfsetfillcolor{fc}
    \pgftransformshift{\pgfqpoint{107.1429bp}{35.7143bp}}
    \pgftransformscale{1.7857}
    \pgftext[base,left]{$f_1$}
  \end{pgfscope}
  \begin{pgfscope}
    \definecolor{fc}{rgb}{0.0000,0.0000,0.0000}
    \pgfsetfillcolor{fc}
    \pgfsetlinewidth{0.8000bp}
    \definecolor{sc}{rgb}{0.0000,0.0000,0.0000}
    \pgfsetstrokecolor{sc}
    \pgfsetmiterjoin
    \pgfsetbuttcap
    \pgfpathqmoveto{103.5714bp}{28.5714bp}
    \pgfpathqcurveto{103.5714bp}{30.5439bp}{101.9724bp}{32.1429bp}{100.0000bp}{32.1429bp}
    \pgfpathqcurveto{98.0276bp}{32.1429bp}{96.4286bp}{30.5439bp}{96.4286bp}{28.5714bp}
    \pgfpathqcurveto{96.4286bp}{26.5990bp}{98.0276bp}{25.0000bp}{100.0000bp}{25.0000bp}
    \pgfpathqcurveto{101.9724bp}{25.0000bp}{103.5714bp}{26.5990bp}{103.5714bp}{28.5714bp}
    \pgfpathclose
    \pgfusepathqfillstroke
  \end{pgfscope}
  \begin{pgfscope}
    \definecolor{fc}{rgb}{0.0000,0.0000,0.0000}
    \pgfsetfillcolor{fc}
    \pgftransformshift{\pgfqpoint{35.7143bp}{35.7143bp}}
    \pgftransformscale{1.7857}
    \pgftext[base,left]{$p_2$}
  \end{pgfscope}
  \begin{pgfscope}
    \definecolor{fc}{rgb}{0.0000,0.0000,0.0000}
    \pgfsetfillcolor{fc}
    \pgfsetlinewidth{0.8000bp}
    \definecolor{sc}{rgb}{0.0000,0.0000,0.0000}
    \pgfsetstrokecolor{sc}
    \pgfsetmiterjoin
    \pgfsetbuttcap
    \pgfpathqmoveto{32.1429bp}{28.5714bp}
    \pgfpathqcurveto{32.1429bp}{30.5439bp}{30.5439bp}{32.1429bp}{28.5714bp}{32.1429bp}
    \pgfpathqcurveto{26.5990bp}{32.1429bp}{25.0000bp}{30.5439bp}{25.0000bp}{28.5714bp}
    \pgfpathqcurveto{25.0000bp}{26.5990bp}{26.5990bp}{25.0000bp}{28.5714bp}{25.0000bp}
    \pgfpathqcurveto{30.5439bp}{25.0000bp}{32.1429bp}{26.5990bp}{32.1429bp}{28.5714bp}
    \pgfpathclose
    \pgfusepathqfillstroke
  \end{pgfscope}
  \begin{pgfscope}
    \definecolor{fc}{rgb}{0.0000,0.0000,0.0000}
    \pgfsetfillcolor{fc}
    \pgftransformshift{\pgfqpoint{35.7143bp}{107.1429bp}}
    \pgftransformscale{1.7857}
    \pgftext[base,left]{$p_1$}
  \end{pgfscope}
  \begin{pgfscope}
    \definecolor{fc}{rgb}{0.0000,0.0000,0.0000}
    \pgfsetfillcolor{fc}
    \pgfsetlinewidth{0.8000bp}
    \definecolor{sc}{rgb}{0.0000,0.0000,0.0000}
    \pgfsetstrokecolor{sc}
    \pgfsetmiterjoin
    \pgfsetbuttcap
    \pgfpathqmoveto{32.1429bp}{100.0000bp}
    \pgfpathqcurveto{32.1429bp}{101.9724bp}{30.5439bp}{103.5714bp}{28.5714bp}{103.5714bp}
    \pgfpathqcurveto{26.5990bp}{103.5714bp}{25.0000bp}{101.9724bp}{25.0000bp}{100.0000bp}
    \pgfpathqcurveto{25.0000bp}{98.0276bp}{26.5990bp}{96.4286bp}{28.5714bp}{96.4286bp}
    \pgfpathqcurveto{30.5439bp}{96.4286bp}{32.1429bp}{98.0276bp}{32.1429bp}{100.0000bp}
    \pgfpathclose
    \pgfusepathqfillstroke
  \end{pgfscope}
  \begin{pgfscope}
    \pgfsetlinewidth{1.5000bp}
    \definecolor{sc}{rgb}{0.0000,0.0000,0.0000}
    \pgfsetstrokecolor{sc}
    \pgfsetmiterjoin
    \pgfsetbuttcap
    \pgfpathqmoveto{28.5714bp}{100.0000bp}
    \pgfpathqlineto{28.5714bp}{28.5714bp}
    \pgfusepathqstroke
  \end{pgfscope}
  \begin{pgfscope}
    \definecolor{fc}{rgb}{0.0000,0.0000,0.0000}
    \pgfsetfillcolor{fc}
    \pgfusepathqfill
  \end{pgfscope}
  \begin{pgfscope}
    \definecolor{fc}{rgb}{0.0000,0.0000,0.0000}
    \pgfsetfillcolor{fc}
    \pgfusepathqfill
  \end{pgfscope}
  \begin{pgfscope}
    \definecolor{fc}{rgb}{0.0000,0.0000,0.0000}
    \pgfsetfillcolor{fc}
    \pgfusepathqfill
  \end{pgfscope}
  \begin{pgfscope}
    \definecolor{fc}{rgb}{0.0000,0.0000,0.0000}
    \pgfsetfillcolor{fc}
    \pgfusepathqfill
  \end{pgfscope}
  \begin{pgfscope}
    \definecolor{fc}{rgb}{0.0000,0.0000,0.0000}
    \pgfsetfillcolor{fc}
    \pgftransformshift{\pgfqpoint{178.5714bp}{178.5714bp}}
    \pgftransformscale{1.7857}
    \pgftext[base,left]{$p_2$}
  \end{pgfscope}
  \begin{pgfscope}
    \definecolor{fc}{rgb}{0.0000,0.0000,0.0000}
    \pgfsetfillcolor{fc}
    \pgfsetlinewidth{0.8000bp}
    \definecolor{sc}{rgb}{0.0000,0.0000,0.0000}
    \pgfsetstrokecolor{sc}
    \pgfsetmiterjoin
    \pgfsetbuttcap
    \pgfpathqmoveto{175.0000bp}{171.4286bp}
    \pgfpathqcurveto{175.0000bp}{173.4010bp}{173.4010bp}{175.0000bp}{171.4286bp}{175.0000bp}
    \pgfpathqcurveto{169.4561bp}{175.0000bp}{167.8571bp}{173.4010bp}{167.8571bp}{171.4286bp}
    \pgfpathqcurveto{167.8571bp}{169.4561bp}{169.4561bp}{167.8571bp}{171.4286bp}{167.8571bp}
    \pgfpathqcurveto{173.4010bp}{167.8571bp}{175.0000bp}{169.4561bp}{175.0000bp}{171.4286bp}
    \pgfpathclose
    \pgfusepathqfillstroke
  \end{pgfscope}
  \begin{pgfscope}
    \definecolor{fc}{rgb}{0.0000,0.0000,0.0000}
    \pgfsetfillcolor{fc}
    \pgftransformshift{\pgfqpoint{107.1429bp}{178.5714bp}}
    \pgftransformscale{1.7857}
    \pgftext[base,left]{$p_1$}
  \end{pgfscope}
  \begin{pgfscope}
    \definecolor{fc}{rgb}{0.0000,0.0000,0.0000}
    \pgfsetfillcolor{fc}
    \pgfsetlinewidth{0.8000bp}
    \definecolor{sc}{rgb}{0.0000,0.0000,0.0000}
    \pgfsetstrokecolor{sc}
    \pgfsetmiterjoin
    \pgfsetbuttcap
    \pgfpathqmoveto{103.5714bp}{171.4286bp}
    \pgfpathqcurveto{103.5714bp}{173.4010bp}{101.9724bp}{175.0000bp}{100.0000bp}{175.0000bp}
    \pgfpathqcurveto{98.0276bp}{175.0000bp}{96.4286bp}{173.4010bp}{96.4286bp}{171.4286bp}
    \pgfpathqcurveto{96.4286bp}{169.4561bp}{98.0276bp}{167.8571bp}{100.0000bp}{167.8571bp}
    \pgfpathqcurveto{101.9724bp}{167.8571bp}{103.5714bp}{169.4561bp}{103.5714bp}{171.4286bp}
    \pgfpathclose
    \pgfusepathqfillstroke
  \end{pgfscope}
  \begin{pgfscope}
    \pgfsetlinewidth{1.5000bp}
    \definecolor{sc}{rgb}{0.0000,0.0000,0.0000}
    \pgfsetstrokecolor{sc}
    \pgfsetmiterjoin
    \pgfsetbuttcap
    \pgfpathqmoveto{100.0000bp}{171.4286bp}
    \pgfpathqlineto{171.4286bp}{171.4286bp}
    \pgfusepathqstroke
  \end{pgfscope}
  \begin{pgfscope}
    \definecolor{fc}{rgb}{0.0000,0.0000,0.0000}
    \pgfsetfillcolor{fc}
    \pgfusepathqfill
  \end{pgfscope}
  \begin{pgfscope}
    \definecolor{fc}{rgb}{0.0000,0.0000,0.0000}
    \pgfsetfillcolor{fc}
    \pgfusepathqfill
  \end{pgfscope}
  \begin{pgfscope}
    \definecolor{fc}{rgb}{0.0000,0.0000,0.0000}
    \pgfsetfillcolor{fc}
    \pgfusepathqfill
  \end{pgfscope}
  \begin{pgfscope}
    \definecolor{fc}{rgb}{0.0000,0.0000,0.0000}
    \pgfsetfillcolor{fc}
    \pgfusepathqfill
  \end{pgfscope}
  \begin{pgfscope}
    \definecolor{fc}{rgb}{0.0000,0.0000,0.0000}
    \pgfsetfillcolor{fc}
    \pgftransformshift{\pgfqpoint{35.7143bp}{178.5714bp}}
    \pgftransformscale{1.7857}
    \pgftext[base,left]{$q_1$}
  \end{pgfscope}
  \begin{pgfscope}
    \definecolor{fc}{rgb}{0.0000,0.0000,0.0000}
    \pgfsetfillcolor{fc}
    \pgfsetlinewidth{0.8000bp}
    \definecolor{sc}{rgb}{0.0000,0.0000,0.0000}
    \pgfsetstrokecolor{sc}
    \pgfsetmiterjoin
    \pgfsetbuttcap
    \pgfpathqmoveto{32.1429bp}{171.4286bp}
    \pgfpathqcurveto{32.1429bp}{173.4010bp}{30.5439bp}{175.0000bp}{28.5714bp}{175.0000bp}
    \pgfpathqcurveto{26.5990bp}{175.0000bp}{25.0000bp}{173.4010bp}{25.0000bp}{171.4286bp}
    \pgfpathqcurveto{25.0000bp}{169.4561bp}{26.5990bp}{167.8571bp}{28.5714bp}{167.8571bp}
    \pgfpathqcurveto{30.5439bp}{167.8571bp}{32.1429bp}{169.4561bp}{32.1429bp}{171.4286bp}
    \pgfpathclose
    \pgfusepathqfillstroke
  \end{pgfscope}
\end{pgfpicture}
}
            \caption{}
            \label{fig:ex:ca:hgma:ex:generalization1}
        \end{subfigure}
        \caption{} Figure of hypergraphs for example
    \end{figure}

\end{example}

\end{document}
