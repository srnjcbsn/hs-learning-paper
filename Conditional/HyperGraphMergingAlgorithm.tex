\documentclass[../Master.tex]{subfiles}
\providecommand{\master}{..}

\begin{document}

Given two hypergraphs $H_1$ and $H_2$ describing the same conditional effect $q$:

Starting with the hyperedge describing the effect $q$, ie. $\left\{ q_1, \dots, q_{|q|}  \right\}$:

Each argument $q_i$ of $q$ is now member of exactly one \emph{arg}-hyperedge in  both $H_1$ and $H_2$, denoted $e_1(q_i)$ and $e_2(q_i)$. For all nodes $v_j \in e_1(q_i)$, it holds that  if $v_j$ does not exist in $e_2(q_i)$, then it is not a precondition that the $i$'th argument of $q$ is the same as the $j$'th argument of $v$.

\begin{proposition}
    Given two binding edges $e_1$ and $e_2$ from hypergraphs $H_1$ and $H_2$, respectively; any node $v$ that only exists in \emph{either} $e_1$ or $e_2$ is discarded as a precondition, formally:
    \begin{equation*}
        v \notin e_1 \lor v \notin e_2 \rightarrow v \notin e_1 \sqcap e_2
    \end{equation*}
\end{proposition}

\begin{proposition}
    Given two binding edges $e_1$ and $e_2$,
\end{proposition}

\begin{equation}
    e \sqcap f =
\end{equation}

\begin{example} \label{ex:ca:hgma:disconnected}
    \begin{equation*}
        H_1 = \forall x, y, z : q(x) \quad \text{when} \quad
            p(x,y) \land p(y, z)
    \end{equation*}

    \begin{equation*}
        H_2 = \forall \alpha, \beta, \gamma : q(\alpha) \quad \text{when} \quad
            p(\alpha, \beta) \land p(\gamma, \delta)
    \end{equation*}

    \begin{equation*}
        H_1 \sqcap H_2 = \forall x, y : q(x) \quad \text{when} \quad p(x, y)
    \end{equation*}

    \begin{figure}
        <placeholder>
        \caption{\label{fig:ex:ca:hgma:ex:disconnected} Figure of hypergraphs for example \ref{ex:ca:hgma:disconnected}.}
    \end{figure}

    Note that the precondition decribed by $H_1$ is more restrictive than that of $H_2$. Consequentially, if $H_1$ was the real precondition, the conditional effect would not have succeeded for $H_2$.
\end{example}

If, on the other hand, $v_j$ does exist in $e_2(q_i)$, then it cannot be discarded as a precondition. Furthermore, if there exists several nodes with name $v_j$ in $e_2(q_i)$, then it may be the case that

\begin{example} \label{ex:ca:hgma:generalization}
    \begin{equation*}
        H_1 = \forall x, y : q(x) \quad \text{when} \quad
            p(x,y) \land f(y) \land g(y)
    \end{equation*}

    \begin{equation*}
        H_2 = \forall x, y, z : q(x) \quad \text{when} \quad
            p(x, y) \land p(x,z) \land f(y) \land g(z)
    \end{equation*}

    \begin{equation*}
        H_1 \sqcap H_2 = H_2
    \end{equation*}

    \begin{figure}
        \centering
        \begin{subfigure}[b]{0.3\textwidth}
            \centering
            \scalebox{1}{\begin{pgfpicture}
  \pgfpathrectangle{\pgfpointorigin}{\pgfqpoint{200.0000bp}{200.0000bp}}
  \pgfusepath{use as bounding box}
  \begin{pgfscope}
    \definecolor{fc}{rgb}{0.0000,0.0000,0.0000}
    \pgfsetfillcolor{fc}
    \pgfsetlinewidth{0.5790bp}
    \definecolor{sc}{rgb}{0.0000,0.0000,0.0000}
    \pgfsetstrokecolor{sc}
    \pgfsetmiterjoin
    \pgfsetbuttcap
    \pgfpathqmoveto{95.2381bp}{76.1905bp}
    \pgfpathqcurveto{95.2381bp}{60.4109bp}{108.0300bp}{47.6190bp}{123.8095bp}{47.6190bp}
    \pgfpathqcurveto{139.5891bp}{47.6190bp}{152.3810bp}{60.4109bp}{152.3810bp}{76.1905bp}
    \pgfpathqlineto{152.3810bp}{123.8095bp}
    \pgfpathqcurveto{152.3810bp}{139.5891bp}{139.5891bp}{152.3810bp}{123.8095bp}{152.3810bp}
    \pgfpathqcurveto{108.0300bp}{152.3810bp}{95.2381bp}{139.5891bp}{95.2381bp}{123.8095bp}
    \pgfpathqcurveto{95.2381bp}{108.0300bp}{108.0300bp}{95.2381bp}{123.8095bp}{95.2381bp}
    \pgfpathqlineto{171.4286bp}{95.2381bp}
    \pgfpathqcurveto{187.2081bp}{95.2381bp}{200.0000bp}{108.0300bp}{200.0000bp}{123.8095bp}
    \pgfpathqcurveto{200.0000bp}{139.5891bp}{187.2081bp}{152.3810bp}{171.4286bp}{152.3810bp}
    \pgfpathqlineto{123.8095bp}{152.3810bp}
    \pgfpathqcurveto{108.0300bp}{152.3810bp}{95.2381bp}{139.5891bp}{95.2381bp}{123.8095bp}
    \pgfpathqlineto{95.2381bp}{76.1905bp}
    \pgfpathclose
    \pgfusepathqfillstroke
  \end{pgfscope}
  \begin{pgfscope}
    \definecolor{fc}{rgb}{1.0000,1.0000,1.0000}
    \pgfsetfillcolor{fc}
    \pgfsetlinewidth{0.5790bp}
    \definecolor{sc}{rgb}{1.0000,1.0000,1.0000}
    \pgfsetstrokecolor{sc}
    \pgfsetmiterjoin
    \pgfsetbuttcap
    \pgfpathqmoveto{96.1905bp}{76.1905bp}
    \pgfpathqcurveto{96.1905bp}{60.9369bp}{108.5559bp}{48.5714bp}{123.8095bp}{48.5714bp}
    \pgfpathqcurveto{139.0631bp}{48.5714bp}{151.4286bp}{60.9369bp}{151.4286bp}{76.1905bp}
    \pgfpathqlineto{151.4286bp}{123.8095bp}
    \pgfpathqcurveto{151.4286bp}{139.0631bp}{139.0631bp}{151.4286bp}{123.8095bp}{151.4286bp}
    \pgfpathqcurveto{108.5559bp}{151.4286bp}{96.1905bp}{139.0631bp}{96.1905bp}{123.8095bp}
    \pgfpathqcurveto{96.1905bp}{108.5559bp}{108.5559bp}{96.1905bp}{123.8095bp}{96.1905bp}
    \pgfpathqlineto{171.4286bp}{96.1905bp}
    \pgfpathqcurveto{186.6822bp}{96.1905bp}{199.0476bp}{108.5559bp}{199.0476bp}{123.8095bp}
    \pgfpathqcurveto{199.0476bp}{139.0631bp}{186.6822bp}{151.4286bp}{171.4286bp}{151.4286bp}
    \pgfpathqlineto{123.8095bp}{151.4286bp}
    \pgfpathqcurveto{108.5559bp}{151.4286bp}{96.1905bp}{139.0631bp}{96.1905bp}{123.8095bp}
    \pgfpathqlineto{96.1905bp}{76.1905bp}
    \pgfpathclose
    \pgfusepathqfillstroke
  \end{pgfscope}
  \begin{pgfscope}
    \definecolor{fc}{rgb}{0.0000,0.0000,0.0000}
    \pgfsetfillcolor{fc}
    \pgfsetlinewidth{0.5790bp}
    \definecolor{sc}{rgb}{0.0000,0.0000,0.0000}
    \pgfsetstrokecolor{sc}
    \pgfsetmiterjoin
    \pgfsetbuttcap
    \pgfpathqmoveto{28.5714bp}{152.3810bp}
    \pgfpathqcurveto{12.7919bp}{152.3810bp}{-0.0000bp}{139.5891bp}{-0.0000bp}{123.8095bp}
    \pgfpathqcurveto{-0.0000bp}{108.0300bp}{12.7919bp}{95.2381bp}{28.5714bp}{95.2381bp}
    \pgfpathqlineto{76.1905bp}{95.2381bp}
    \pgfpathqcurveto{91.9700bp}{95.2381bp}{104.7619bp}{108.0300bp}{104.7619bp}{123.8095bp}
    \pgfpathqcurveto{104.7619bp}{139.5891bp}{91.9700bp}{152.3810bp}{76.1905bp}{152.3810bp}
    \pgfpathqlineto{28.5714bp}{152.3810bp}
    \pgfpathclose
    \pgfusepathqfillstroke
  \end{pgfscope}
  \begin{pgfscope}
    \definecolor{fc}{rgb}{1.0000,1.0000,1.0000}
    \pgfsetfillcolor{fc}
    \pgfsetlinewidth{0.5790bp}
    \definecolor{sc}{rgb}{1.0000,1.0000,1.0000}
    \pgfsetstrokecolor{sc}
    \pgfsetmiterjoin
    \pgfsetbuttcap
    \pgfpathqmoveto{28.5714bp}{151.4286bp}
    \pgfpathqcurveto{13.3178bp}{151.4286bp}{0.9524bp}{139.0631bp}{0.9524bp}{123.8095bp}
    \pgfpathqcurveto{0.9524bp}{108.5559bp}{13.3178bp}{96.1905bp}{28.5714bp}{96.1905bp}
    \pgfpathqlineto{76.1905bp}{96.1905bp}
    \pgfpathqcurveto{91.4441bp}{96.1905bp}{103.8095bp}{108.5559bp}{103.8095bp}{123.8095bp}
    \pgfpathqcurveto{103.8095bp}{139.0631bp}{91.4441bp}{151.4286bp}{76.1905bp}{151.4286bp}
    \pgfpathqlineto{28.5714bp}{151.4286bp}
    \pgfpathclose
    \pgfusepathqfillstroke
  \end{pgfscope}
  \begin{pgfscope}
    \definecolor{fc}{rgb}{0.0000,0.0000,0.0000}
    \pgfsetfillcolor{fc}
    \pgftransformshift{\pgfqpoint{176.1905bp}{128.5714bp}}
    \pgftransformscale{1.1905}
    \pgftext[base,left]{$g_1$}
  \end{pgfscope}
  \begin{pgfscope}
    \definecolor{fc}{rgb}{0.0000,0.0000,0.0000}
    \pgfsetfillcolor{fc}
    \pgfsetlinewidth{0.5790bp}
    \definecolor{sc}{rgb}{0.0000,0.0000,0.0000}
    \pgfsetstrokecolor{sc}
    \pgfsetmiterjoin
    \pgfsetbuttcap
    \pgfpathqmoveto{173.8095bp}{123.8095bp}
    \pgfpathqcurveto{173.8095bp}{125.1245bp}{172.7435bp}{126.1905bp}{171.4286bp}{126.1905bp}
    \pgfpathqcurveto{170.1136bp}{126.1905bp}{169.0476bp}{125.1245bp}{169.0476bp}{123.8095bp}
    \pgfpathqcurveto{169.0476bp}{122.4946bp}{170.1136bp}{121.4286bp}{171.4286bp}{121.4286bp}
    \pgfpathqcurveto{172.7435bp}{121.4286bp}{173.8095bp}{122.4946bp}{173.8095bp}{123.8095bp}
    \pgfpathclose
    \pgfusepathqfillstroke
  \end{pgfscope}
  \begin{pgfscope}
    \definecolor{fc}{rgb}{0.0000,0.0000,0.0000}
    \pgfsetfillcolor{fc}
    \pgftransformshift{\pgfqpoint{128.5714bp}{80.9524bp}}
    \pgftransformscale{1.1905}
    \pgftext[base,left]{$f_1$}
  \end{pgfscope}
  \begin{pgfscope}
    \definecolor{fc}{rgb}{0.0000,0.0000,0.0000}
    \pgfsetfillcolor{fc}
    \pgfsetlinewidth{0.5790bp}
    \definecolor{sc}{rgb}{0.0000,0.0000,0.0000}
    \pgfsetstrokecolor{sc}
    \pgfsetmiterjoin
    \pgfsetbuttcap
    \pgfpathqmoveto{126.1905bp}{76.1905bp}
    \pgfpathqcurveto{126.1905bp}{77.5054bp}{125.1245bp}{78.5714bp}{123.8095bp}{78.5714bp}
    \pgfpathqcurveto{122.4946bp}{78.5714bp}{121.4286bp}{77.5054bp}{121.4286bp}{76.1905bp}
    \pgfpathqcurveto{121.4286bp}{74.8755bp}{122.4946bp}{73.8095bp}{123.8095bp}{73.8095bp}
    \pgfpathqcurveto{125.1245bp}{73.8095bp}{126.1905bp}{74.8755bp}{126.1905bp}{76.1905bp}
    \pgfpathclose
    \pgfusepathqfillstroke
  \end{pgfscope}
  \begin{pgfscope}
    \definecolor{fc}{rgb}{0.0000,0.0000,0.0000}
    \pgfsetfillcolor{fc}
    \pgftransformshift{\pgfqpoint{128.5714bp}{128.5714bp}}
    \pgftransformscale{1.1905}
    \pgftext[base,left]{$p_2$}
  \end{pgfscope}
  \begin{pgfscope}
    \definecolor{fc}{rgb}{0.0000,0.0000,0.0000}
    \pgfsetfillcolor{fc}
    \pgfsetlinewidth{0.5790bp}
    \definecolor{sc}{rgb}{0.0000,0.0000,0.0000}
    \pgfsetstrokecolor{sc}
    \pgfsetmiterjoin
    \pgfsetbuttcap
    \pgfpathqmoveto{126.1905bp}{123.8095bp}
    \pgfpathqcurveto{126.1905bp}{125.1245bp}{125.1245bp}{126.1905bp}{123.8095bp}{126.1905bp}
    \pgfpathqcurveto{122.4946bp}{126.1905bp}{121.4286bp}{125.1245bp}{121.4286bp}{123.8095bp}
    \pgfpathqcurveto{121.4286bp}{122.4946bp}{122.4946bp}{121.4286bp}{123.8095bp}{121.4286bp}
    \pgfpathqcurveto{125.1245bp}{121.4286bp}{126.1905bp}{122.4946bp}{126.1905bp}{123.8095bp}
    \pgfpathclose
    \pgfusepathqfillstroke
  \end{pgfscope}
  \begin{pgfscope}
    \definecolor{fc}{rgb}{0.0000,0.0000,0.0000}
    \pgfsetfillcolor{fc}
    \pgftransformshift{\pgfqpoint{80.9524bp}{128.5714bp}}
    \pgftransformscale{1.1905}
    \pgftext[base,left]{$p_1$}
  \end{pgfscope}
  \begin{pgfscope}
    \definecolor{fc}{rgb}{0.0000,0.0000,0.0000}
    \pgfsetfillcolor{fc}
    \pgfsetlinewidth{0.5790bp}
    \definecolor{sc}{rgb}{0.0000,0.0000,0.0000}
    \pgfsetstrokecolor{sc}
    \pgfsetmiterjoin
    \pgfsetbuttcap
    \pgfpathqmoveto{78.5714bp}{123.8095bp}
    \pgfpathqcurveto{78.5714bp}{125.1245bp}{77.5054bp}{126.1905bp}{76.1905bp}{126.1905bp}
    \pgfpathqcurveto{74.8755bp}{126.1905bp}{73.8095bp}{125.1245bp}{73.8095bp}{123.8095bp}
    \pgfpathqcurveto{73.8095bp}{122.4946bp}{74.8755bp}{121.4286bp}{76.1905bp}{121.4286bp}
    \pgfpathqcurveto{77.5054bp}{121.4286bp}{78.5714bp}{122.4946bp}{78.5714bp}{123.8095bp}
    \pgfpathclose
    \pgfusepathqfillstroke
  \end{pgfscope}
  \begin{pgfscope}
    \pgfsetlinewidth{1.0856bp}
    \definecolor{sc}{rgb}{0.0000,0.0000,0.0000}
    \pgfsetstrokecolor{sc}
    \pgfsetmiterjoin
    \pgfsetbuttcap
    \pgfpathqmoveto{76.1905bp}{123.8095bp}
    \pgfpathqlineto{123.8095bp}{123.8095bp}
    \pgfusepathqstroke
  \end{pgfscope}
  \begin{pgfscope}
    \definecolor{fc}{rgb}{0.0000,0.0000,0.0000}
    \pgfsetfillcolor{fc}
    \pgfusepathqfill
  \end{pgfscope}
  \begin{pgfscope}
    \definecolor{fc}{rgb}{0.0000,0.0000,0.0000}
    \pgfsetfillcolor{fc}
    \pgfusepathqfill
  \end{pgfscope}
  \begin{pgfscope}
    \definecolor{fc}{rgb}{0.0000,0.0000,0.0000}
    \pgfsetfillcolor{fc}
    \pgfusepathqfill
  \end{pgfscope}
  \begin{pgfscope}
    \definecolor{fc}{rgb}{0.0000,0.0000,0.0000}
    \pgfsetfillcolor{fc}
    \pgfusepathqfill
  \end{pgfscope}
  \begin{pgfscope}
    \definecolor{fc}{rgb}{0.0000,0.0000,0.0000}
    \pgfsetfillcolor{fc}
    \pgftransformshift{\pgfqpoint{33.3333bp}{128.5714bp}}
    \pgftransformscale{1.1905}
    \pgftext[base,left]{$q_1$}
  \end{pgfscope}
  \begin{pgfscope}
    \definecolor{fc}{rgb}{0.0000,0.0000,0.0000}
    \pgfsetfillcolor{fc}
    \pgfsetlinewidth{0.5790bp}
    \definecolor{sc}{rgb}{0.0000,0.0000,0.0000}
    \pgfsetstrokecolor{sc}
    \pgfsetmiterjoin
    \pgfsetbuttcap
    \pgfpathqmoveto{30.9524bp}{123.8095bp}
    \pgfpathqcurveto{30.9524bp}{125.1245bp}{29.8864bp}{126.1905bp}{28.5714bp}{126.1905bp}
    \pgfpathqcurveto{27.2565bp}{126.1905bp}{26.1905bp}{125.1245bp}{26.1905bp}{123.8095bp}
    \pgfpathqcurveto{26.1905bp}{122.4946bp}{27.2565bp}{121.4286bp}{28.5714bp}{121.4286bp}
    \pgfpathqcurveto{29.8864bp}{121.4286bp}{30.9524bp}{122.4946bp}{30.9524bp}{123.8095bp}
    \pgfpathclose
    \pgfusepathqfillstroke
  \end{pgfscope}
\end{pgfpicture}
}
            \caption{}
            \label{fig:ex:ca:hgma:ex:generalization1}
        \end{subfigure}%
        \hfill%
        \begin{subfigure}[b]{0.3\textwidth}
            \centering
            \scalebox{0.75}{\input{\master/Graphics/hgEx2_2.pgf}}
            \caption{}
            \label{fig:ex:ca:hgma:ex:generalization1}
        \end{subfigure}
        \caption{} Figure of hypergraphs for example
    \end{figure}

\end{example}



it now holds that if there does not exist a node with the same name in $e_2(q_i)$, then it is

Define hyperedge intersection ($e_1 \sqcap e_2$) as follows:
\begin{equation}
    \left\{
        p \; | \; v \in e_1 \land t \in e_2 \land name(v) = name(t)
    \right\}
\end{equation}

For each predicate $p$ in $H_1$ whose $k$'th argument is identical to the $i$'th argument of $q$

\begin{easylist}[itemize]
    \ListProperties(Progressive=3ex)
    & For each $p(x_1, x_2, \dots, x_{|p|}) \in H_1$:
    && For each $x \in (x_1, x_2, \dots, x_|p|)$:
    &&& Collect all predicates $q$ in $H_1$ which has $x$ as argument.
    &&& For each $q$:
    &&&& Let $k$ be the argument places in $q$ which equals $x$
    &&&& For each $k$:
    &&&&& For each predicate $f$ with the same name as $q$ in $H_2$ for which $args(f)[k] = x$:
    &&&&&&
\end{easylist}

\begin{equation}
    c(x,k) =
    \left\{
        p \; | \; p \in H_2 \land name(p) \land args(p)[k] = x
    \right\}
\end{equation}

\begin{equation}
    \left\{
        q \; | \; args(p) \cap args(q) \neq \emptyset
    \right\}
\end{equation}

\end{document}
