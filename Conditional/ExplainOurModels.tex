\providecommand{\master}{..}
\documentclass[\master/Master.tex]{subfiles}
\begin{document}
In the model --- Connecting paths--- we have presented for conditional effect learning, there is the obvious flaw that all paths from each possible precondition, has an intractable size complexity. 
As all possible paths is NP-Hard. Which we require to represent all  the unproven connecting paths. 
In this section we will explore a new model for reduction of the size complexity, and show how the model can be utilized for our purpose. 

Using connecting paths to represent knowledge requires a large amount of redundancy as many paths share a common subpath to the effect. To remedy this, a set of paths can be joined to a forest of trees (where each tree is rooted in an argument of the effect atom), where each branch is labelled as either $\bc$ (binding edge) or $\pc$ (atom edge).

\begin{example}[Path trees from connecting paths]\label{ex:ca:pathTrees}
    Consider a state transition $\left( S, a, S' \right)$, where 
    \begin{equation*}
        S = \left\{ p(x, y), m(y), g(y) \right\}
    \end{equation*}
    and $q(x) \in \Delta S$. Then, two connecting paths for effect $q(x)$ and $S$ are the following (ignoring negative preconditions):
    \begin{align*}
        & g_1 \bc m_1 \bc p_2 \pc p_1 \bc q_1 \\
        & m_1 \bc g_1 \bc p_2 \pc p_1 \bc q_1
    \end{align*}
    All other paths for this effect are subpaths of the above. As can be seen, $p_2 \pc p_1 \bc q_1$ is a common tail of both paths, and therefore they can be merged into the path tree depicted in~\figref{fig:ca:graphExTree}.
\end{example}

While joining paths as described above removes redundant tails, there are still duplicate vertices in the resulting trees. Specifically, if two subtrees $T_1$ and $T_2$ are rooted in the same vertex (originating from the same argument of the same atom), then these two subtrees can be merged into one inheriting the children of both $T_1$ and $T_2$. If $T_1$'s parent is also $T_2$'s child, then this merging forms a cycle, and results in a cyclic, undirected \emph{path graph} $G = (V, E)$. 

The edges in a path graph can be divided into two sets: those labelled $\bc$, and those labelled $\pc$, denoted $E_B$ and $E_A$, respectively.

\begin{example}[Path graphs from path trees]\label{ex:ca:pathGraph}
    For the path tree constructed in Example~\ref{ex:ca:pathTrees}, the two subtrees of $p_2$ ($g_1 \bc m_1$ and $m_1 \bc g_1$) are mutually redundant, as both denote that $g_1$, $m_1$ and $p_2$ are $\bc$-connected. Therefore, they can be replaced with the cycle depicted in~\figref{fig:ca:graphExGraph}. 
\end{example}

Notably, constructing a graph from a set of paths causes vertices to be transitively connected by binding edges:

\begin{proposition}\label{prop:ca:transitiveConn}
    For a graph constructed from a complete set of connecting paths $B$, it holds that if a vertex $v$ is reachable from a vertex $u$ by only following binding edges, then there is a binding edge between $u$ and $v$. Similarly for atom edges.
\end{proposition}

\begin{proof}
   If $v$ is reachable from $u$ by only following a sequence of binding edges $b$, then $b$ is a subpath of some path in $B$. Since $b$ does not contain atom connections, all the atom arguments contained in it have the same variable in the atom set $B$ originated from. Hence, $B$ will also contain a path with subpath $b \bc v$, which will be present as a branch in the path tree, and therefore merged into the path graph as an edge.
\end{proof}

As a consequence of Proposition~\ref{prop:ca:transitiveConn}, a path graph $G = (V,E)$ consists of subgraphs that are either completely connected by binding edges or completely connected by atom edges. That is, the subgraphs $G_B = \left(V, E_B \right)$ and $G_P = \left( V, E_P \right)$ consists of complete, disconnected subgraphs.

Consequently, a path graph consists of \textit{binding} and \textit{atom} cliques, i.e.\ vertices in a clique are either $\bc$-connected (making it a binding clique) or they are $\pc$-connected (making it a atom clique). Note that a clique can be both $\bc$- and $\pc$-connected, in case a atom has several identical arguments. In the following, we interpret such a clique as two different cliques.

As the graph can be meaningfully partitioned into complete subgraphs, the vertices of each subgraph can simply be seen as a set, where each set overlaps with the others. Thus, a path graph $G$ becomes $HG = \left(V, E_H\right)$, where each member of $E_H$ is a set of vertices comprising either a binding or atom clique in $G$ (labelled $\bc$ or $\pc$ respectively). Then, $HG$ describes a \emph{conformal hypergraph} (see e.g.\ \cite{bretto}), where $E_H$ is a set of hyperedges.

A hypergraph constructed from a set of connecting paths contains $O\left(|\mathcal{O}|^m\right)$ vertices and $O\left( |\mathcal{O}|^m\right)$ hyperedges, where $m$ is the maximum arity of predicates. 

\subsubsection*{Learning with hypergraphs}
A thorough explanation of learning conditional effects using hypergraphs is given in Appendix~\ref{sec:app:hypergraph}, and we will recount the most important findings here:

Proven knowledge and candidates can be represented as pairs of vertices rather than sets of complete paths. As a vertex pair can represent several connecting paths, candidates can more easily be reduced to singleton sets. However, candidate sets can also be expanded, and as such knowledge is never decidedly proven to hold (although we still operate with singleton candidates as ``proven'' knowledge, for the sake of planning).

We have provided a merging algorithm (See \Cref{sec:app:hypergraph}), which can combine the knowledge from two hypergraphs into one, disproving connections to be preconditions. However, this algorithm adds redundant, mutually isomorphic subhypergraphs to the resulting hypergraph, possibly expanding it instead of shrinking it. Removing these redundancies seems to be an NP-complete problem, although approximation algorithms do exist. As these redundancies is a problem of efficiency rather than correctness, an approximation algorithm removing a \emph{most} of them could be desirable.

\begin{figure}
    \centering
    \begin{subfigure}[t]{0.30\textwidth}
        \centering
        \raisebox{-0.5\height}{\resizebox{\linewidth}{!}{\begin{pgfpicture}
  \pgfpathrectangle{\pgfpointorigin}{\pgfqpoint{200.0000bp}{200.0000bp}}
  \pgfusepath{use as bounding box}
  \begin{pgfscope}
    \definecolor{fc}{rgb}{0.0000,0.0000,0.0000}
    \pgfsetfillcolor{fc}
    \pgftransformshift{\pgfqpoint{139.1304bp}{12.6087bp}}
    \pgftransformscale{1.0870}
    \pgftext[base,left]{$g_1$}
  \end{pgfscope}
  \begin{pgfscope}
    \definecolor{fc}{rgb}{0.0000,0.0000,0.0000}
    \pgfsetfillcolor{fc}
    \pgfsetfillopacity{0.0000}
    \pgfsetlinewidth{0.6014bp}
    \definecolor{sc}{rgb}{0.0000,0.0000,0.0000}
    \pgfsetstrokecolor{sc}
    \pgfsetstrokeopacity{0.0000}
    \pgfsetmiterjoin
    \pgfsetbuttcap
    \pgfpathqmoveto{156.5217bp}{13.0435bp}
    \pgfpathqcurveto{156.5217bp}{20.2472bp}{150.6820bp}{26.0870bp}{143.4783bp}{26.0870bp}
    \pgfpathqcurveto{136.2745bp}{26.0870bp}{130.4348bp}{20.2472bp}{130.4348bp}{13.0435bp}
    \pgfpathqcurveto{130.4348bp}{5.8398bp}{136.2745bp}{0.0000bp}{143.4783bp}{0.0000bp}
    \pgfpathqcurveto{150.6820bp}{-0.0000bp}{156.5217bp}{5.8398bp}{156.5217bp}{13.0435bp}
    \pgfpathclose
    \pgfusepathqfillstroke
  \end{pgfscope}
  \begin{pgfscope}
    \definecolor{fc}{rgb}{0.0000,0.0000,0.0000}
    \pgfsetfillcolor{fc}
    \pgftransformshift{\pgfqpoint{139.1304bp}{56.0870bp}}
    \pgftransformscale{1.0870}
    \pgftext[base,left]{$m_1$}
  \end{pgfscope}
  \begin{pgfscope}
    \definecolor{fc}{rgb}{0.0000,0.0000,0.0000}
    \pgfsetfillcolor{fc}
    \pgfsetfillopacity{0.0000}
    \pgfsetlinewidth{0.6014bp}
    \definecolor{sc}{rgb}{0.0000,0.0000,0.0000}
    \pgfsetstrokecolor{sc}
    \pgfsetstrokeopacity{0.0000}
    \pgfsetmiterjoin
    \pgfsetbuttcap
    \pgfpathqmoveto{156.5217bp}{56.5217bp}
    \pgfpathqcurveto{156.5217bp}{63.7255bp}{150.6820bp}{69.5652bp}{143.4783bp}{69.5652bp}
    \pgfpathqcurveto{136.2745bp}{69.5652bp}{130.4348bp}{63.7255bp}{130.4348bp}{56.5217bp}
    \pgfpathqcurveto{130.4348bp}{49.3180bp}{136.2745bp}{43.4783bp}{143.4783bp}{43.4783bp}
    \pgfpathqcurveto{150.6820bp}{43.4783bp}{156.5217bp}{49.3180bp}{156.5217bp}{56.5217bp}
    \pgfpathclose
    \pgfusepathqfillstroke
  \end{pgfscope}
  \begin{pgfscope}
    \definecolor{fc}{rgb}{0.0000,0.0000,0.0000}
    \pgfsetfillcolor{fc}
    \pgftransformshift{\pgfqpoint{52.1739bp}{12.6087bp}}
    \pgftransformscale{1.0870}
    \pgftext[base,left]{$m_1$}
  \end{pgfscope}
  \begin{pgfscope}
    \definecolor{fc}{rgb}{0.0000,0.0000,0.0000}
    \pgfsetfillcolor{fc}
    \pgfsetfillopacity{0.0000}
    \pgfsetlinewidth{0.6014bp}
    \definecolor{sc}{rgb}{0.0000,0.0000,0.0000}
    \pgfsetstrokecolor{sc}
    \pgfsetstrokeopacity{0.0000}
    \pgfsetmiterjoin
    \pgfsetbuttcap
    \pgfpathqmoveto{69.5652bp}{13.0435bp}
    \pgfpathqcurveto{69.5652bp}{20.2472bp}{63.7255bp}{26.0870bp}{56.5217bp}{26.0870bp}
    \pgfpathqcurveto{49.3180bp}{26.0870bp}{43.4783bp}{20.2472bp}{43.4783bp}{13.0435bp}
    \pgfpathqcurveto{43.4783bp}{5.8398bp}{49.3180bp}{0.0000bp}{56.5217bp}{0.0000bp}
    \pgfpathqcurveto{63.7255bp}{-0.0000bp}{69.5652bp}{5.8398bp}{69.5652bp}{13.0435bp}
    \pgfpathclose
    \pgfusepathqfillstroke
  \end{pgfscope}
  \begin{pgfscope}
    \definecolor{fc}{rgb}{0.0000,0.0000,0.0000}
    \pgfsetfillcolor{fc}
    \pgftransformshift{\pgfqpoint{52.1739bp}{56.0870bp}}
    \pgftransformscale{1.0870}
    \pgftext[base,left]{$g_1$}
  \end{pgfscope}
  \begin{pgfscope}
    \definecolor{fc}{rgb}{0.0000,0.0000,0.0000}
    \pgfsetfillcolor{fc}
    \pgfsetfillopacity{0.0000}
    \pgfsetlinewidth{0.6014bp}
    \definecolor{sc}{rgb}{0.0000,0.0000,0.0000}
    \pgfsetstrokecolor{sc}
    \pgfsetstrokeopacity{0.0000}
    \pgfsetmiterjoin
    \pgfsetbuttcap
    \pgfpathqmoveto{69.5652bp}{56.5217bp}
    \pgfpathqcurveto{69.5652bp}{63.7255bp}{63.7255bp}{69.5652bp}{56.5217bp}{69.5652bp}
    \pgfpathqcurveto{49.3180bp}{69.5652bp}{43.4783bp}{63.7255bp}{43.4783bp}{56.5217bp}
    \pgfpathqcurveto{43.4783bp}{49.3180bp}{49.3180bp}{43.4783bp}{56.5217bp}{43.4783bp}
    \pgfpathqcurveto{63.7255bp}{43.4783bp}{69.5652bp}{49.3180bp}{69.5652bp}{56.5217bp}
    \pgfpathclose
    \pgfusepathqfillstroke
  \end{pgfscope}
  \begin{pgfscope}
    \definecolor{fc}{rgb}{0.0000,0.0000,0.0000}
    \pgfsetfillcolor{fc}
    \pgftransformshift{\pgfqpoint{95.6522bp}{99.5652bp}}
    \pgftransformscale{1.0870}
    \pgftext[base,left]{$p_2$}
  \end{pgfscope}
  \begin{pgfscope}
    \definecolor{fc}{rgb}{0.0000,0.0000,0.0000}
    \pgfsetfillcolor{fc}
    \pgfsetfillopacity{0.0000}
    \pgfsetlinewidth{0.6014bp}
    \definecolor{sc}{rgb}{0.0000,0.0000,0.0000}
    \pgfsetstrokecolor{sc}
    \pgfsetstrokeopacity{0.0000}
    \pgfsetmiterjoin
    \pgfsetbuttcap
    \pgfpathqmoveto{113.0435bp}{100.0000bp}
    \pgfpathqcurveto{113.0435bp}{107.2037bp}{107.2037bp}{113.0435bp}{100.0000bp}{113.0435bp}
    \pgfpathqcurveto{92.7963bp}{113.0435bp}{86.9565bp}{107.2037bp}{86.9565bp}{100.0000bp}
    \pgfpathqcurveto{86.9565bp}{92.7963bp}{92.7963bp}{86.9565bp}{100.0000bp}{86.9565bp}
    \pgfpathqcurveto{107.2037bp}{86.9565bp}{113.0435bp}{92.7963bp}{113.0435bp}{100.0000bp}
    \pgfpathclose
    \pgfusepathqfillstroke
  \end{pgfscope}
  \begin{pgfscope}
    \definecolor{fc}{rgb}{0.0000,0.0000,0.0000}
    \pgfsetfillcolor{fc}
    \pgftransformshift{\pgfqpoint{95.6522bp}{143.0435bp}}
    \pgftransformscale{1.0870}
    \pgftext[base,left]{$p_1$}
  \end{pgfscope}
  \begin{pgfscope}
    \definecolor{fc}{rgb}{0.0000,0.0000,0.0000}
    \pgfsetfillcolor{fc}
    \pgfsetfillopacity{0.0000}
    \pgfsetlinewidth{0.6014bp}
    \definecolor{sc}{rgb}{0.0000,0.0000,0.0000}
    \pgfsetstrokecolor{sc}
    \pgfsetstrokeopacity{0.0000}
    \pgfsetmiterjoin
    \pgfsetbuttcap
    \pgfpathqmoveto{113.0435bp}{143.4783bp}
    \pgfpathqcurveto{113.0435bp}{150.6820bp}{107.2037bp}{156.5217bp}{100.0000bp}{156.5217bp}
    \pgfpathqcurveto{92.7963bp}{156.5217bp}{86.9565bp}{150.6820bp}{86.9565bp}{143.4783bp}
    \pgfpathqcurveto{86.9565bp}{136.2745bp}{92.7963bp}{130.4348bp}{100.0000bp}{130.4348bp}
    \pgfpathqcurveto{107.2037bp}{130.4348bp}{113.0435bp}{136.2745bp}{113.0435bp}{143.4783bp}
    \pgfpathclose
    \pgfusepathqfillstroke
  \end{pgfscope}
  \begin{pgfscope}
    \definecolor{fc}{rgb}{0.0000,0.0000,0.0000}
    \pgfsetfillcolor{fc}
    \pgftransformshift{\pgfqpoint{95.6522bp}{186.5217bp}}
    \pgftransformscale{1.0870}
    \pgftext[base,left]{$q_1$}
  \end{pgfscope}
  \begin{pgfscope}
    \definecolor{fc}{rgb}{0.0000,0.0000,0.0000}
    \pgfsetfillcolor{fc}
    \pgfsetfillopacity{0.0000}
    \pgfsetlinewidth{0.6014bp}
    \definecolor{sc}{rgb}{0.0000,0.0000,0.0000}
    \pgfsetstrokecolor{sc}
    \pgfsetstrokeopacity{0.0000}
    \pgfsetmiterjoin
    \pgfsetbuttcap
    \pgfpathqmoveto{113.0435bp}{186.9565bp}
    \pgfpathqcurveto{113.0435bp}{194.1602bp}{107.2037bp}{200.0000bp}{100.0000bp}{200.0000bp}
    \pgfpathqcurveto{92.7963bp}{200.0000bp}{86.9565bp}{194.1602bp}{86.9565bp}{186.9565bp}
    \pgfpathqcurveto{86.9565bp}{179.7528bp}{92.7963bp}{173.9130bp}{100.0000bp}{173.9130bp}
    \pgfpathqcurveto{107.2037bp}{173.9130bp}{113.0435bp}{179.7528bp}{113.0435bp}{186.9565bp}
    \pgfpathclose
    \pgfusepathqfillstroke
  \end{pgfscope}
  \begin{pgfscope}
    \definecolor{fc}{rgb}{0.0000,0.0000,0.0000}
    \pgfsetfillcolor{fc}
    \pgftransformshift{\pgfqpoint{95.6522bp}{164.7826bp}}
    \pgftransformscale{1.0870}
    \pgftext[base,left]{$\bullet$}
  \end{pgfscope}
  \begin{pgfscope}
    \definecolor{fc}{rgb}{0.0000,0.0000,0.0000}
    \pgfsetfillcolor{fc}
    \pgftransformshift{\pgfqpoint{95.6522bp}{121.3043bp}}
    \pgftransformscale{1.0870}
    \pgftext[base,left]{$\times$}
  \end{pgfscope}
  \begin{pgfscope}
    \definecolor{fc}{rgb}{0.0000,0.0000,0.0000}
    \pgfsetfillcolor{fc}
    \pgftransformshift{\pgfqpoint{73.9130bp}{77.8261bp}}
    \pgftransformscale{1.0870}
    \pgftext[base,left]{$\bullet$}
  \end{pgfscope}
  \begin{pgfscope}
    \definecolor{fc}{rgb}{0.0000,0.0000,0.0000}
    \pgfsetfillcolor{fc}
    \pgftransformshift{\pgfqpoint{52.1739bp}{34.3478bp}}
    \pgftransformscale{1.0870}
    \pgftext[base,left]{$\bullet$}
  \end{pgfscope}
  \begin{pgfscope}
    \definecolor{fc}{rgb}{0.0000,0.0000,0.0000}
    \pgfsetfillcolor{fc}
    \pgftransformshift{\pgfqpoint{117.3913bp}{77.8261bp}}
    \pgftransformscale{1.0870}
    \pgftext[base,left]{$\bullet$}
  \end{pgfscope}
  \begin{pgfscope}
    \definecolor{fc}{rgb}{0.0000,0.0000,0.0000}
    \pgfsetfillcolor{fc}
    \pgftransformshift{\pgfqpoint{139.1304bp}{34.3478bp}}
    \pgftransformscale{1.0870}
    \pgftext[base,left]{$\bullet$}
  \end{pgfscope}
\end{pgfpicture}
}}
        \caption{Paths joined to a tree.}
        \label{fig:ca:graphExTree}
    \end{subfigure}%
    \hfill%
    \begin{subfigure}[t]{0.30\textwidth}
        \centering
        \resizebox{1\linewidth}{!}{\raisebox{-0.5\height}{\begin{pgfpicture}
  \pgfpathrectangle{\pgfpointorigin}{\pgfqpoint{200.0000bp}{200.0000bp}}
  \pgfusepath{use as bounding box}
  \begin{pgfscope}
    \definecolor{fc}{rgb}{0.0000,0.0000,0.0000}
    \pgfsetfillcolor{fc}
    \pgftransformshift{\pgfqpoint{150.0000bp}{16.1111bp}}
    \pgftransformscale{1.3889}
    \pgftext[base,left]{$m_1$}
  \end{pgfscope}
  \begin{pgfscope}
    \definecolor{fc}{rgb}{0.0000,0.0000,0.0000}
    \pgfsetfillcolor{fc}
    \pgfsetfillopacity{0.0000}
    \pgfsetlinewidth{0.6799bp}
    \definecolor{sc}{rgb}{0.0000,0.0000,0.0000}
    \pgfsetstrokecolor{sc}
    \pgfsetstrokeopacity{0.0000}
    \pgfsetmiterjoin
    \pgfsetbuttcap
    \pgfpathqmoveto{172.2222bp}{16.6667bp}
    \pgfpathqcurveto{172.2222bp}{25.8714bp}{164.7603bp}{33.3333bp}{155.5556bp}{33.3333bp}
    \pgfpathqcurveto{146.3508bp}{33.3333bp}{138.8889bp}{25.8714bp}{138.8889bp}{16.6667bp}
    \pgfpathqcurveto{138.8889bp}{7.4619bp}{146.3508bp}{-0.0000bp}{155.5556bp}{-0.0000bp}
    \pgfpathqcurveto{164.7603bp}{-0.0000bp}{172.2222bp}{7.4619bp}{172.2222bp}{16.6667bp}
    \pgfpathclose
    \pgfusepathqfillstroke
  \end{pgfscope}
  \begin{pgfscope}
    \definecolor{fc}{rgb}{0.0000,0.0000,0.0000}
    \pgfsetfillcolor{fc}
    \pgftransformshift{\pgfqpoint{38.8889bp}{16.1111bp}}
    \pgftransformscale{1.3889}
    \pgftext[base,left]{$g_1$}
  \end{pgfscope}
  \begin{pgfscope}
    \definecolor{fc}{rgb}{0.0000,0.0000,0.0000}
    \pgfsetfillcolor{fc}
    \pgfsetfillopacity{0.0000}
    \pgfsetlinewidth{0.6799bp}
    \definecolor{sc}{rgb}{0.0000,0.0000,0.0000}
    \pgfsetstrokecolor{sc}
    \pgfsetstrokeopacity{0.0000}
    \pgfsetmiterjoin
    \pgfsetbuttcap
    \pgfpathqmoveto{61.1111bp}{16.6667bp}
    \pgfpathqcurveto{61.1111bp}{25.8714bp}{53.6492bp}{33.3333bp}{44.4444bp}{33.3333bp}
    \pgfpathqcurveto{35.2397bp}{33.3333bp}{27.7778bp}{25.8714bp}{27.7778bp}{16.6667bp}
    \pgfpathqcurveto{27.7778bp}{7.4619bp}{35.2397bp}{-0.0000bp}{44.4444bp}{-0.0000bp}
    \pgfpathqcurveto{53.6492bp}{-0.0000bp}{61.1111bp}{7.4619bp}{61.1111bp}{16.6667bp}
    \pgfpathclose
    \pgfusepathqfillstroke
  \end{pgfscope}
  \begin{pgfscope}
    \definecolor{fc}{rgb}{0.0000,0.0000,0.0000}
    \pgfsetfillcolor{fc}
    \pgftransformshift{\pgfqpoint{94.4444bp}{71.6667bp}}
    \pgftransformscale{1.3889}
    \pgftext[base,left]{$p_2$}
  \end{pgfscope}
  \begin{pgfscope}
    \definecolor{fc}{rgb}{0.0000,0.0000,0.0000}
    \pgfsetfillcolor{fc}
    \pgfsetfillopacity{0.0000}
    \pgfsetlinewidth{0.6799bp}
    \definecolor{sc}{rgb}{0.0000,0.0000,0.0000}
    \pgfsetstrokecolor{sc}
    \pgfsetstrokeopacity{0.0000}
    \pgfsetmiterjoin
    \pgfsetbuttcap
    \pgfpathqmoveto{116.6667bp}{72.2222bp}
    \pgfpathqcurveto{116.6667bp}{81.4270bp}{109.2047bp}{88.8889bp}{100.0000bp}{88.8889bp}
    \pgfpathqcurveto{90.7953bp}{88.8889bp}{83.3333bp}{81.4270bp}{83.3333bp}{72.2222bp}
    \pgfpathqcurveto{83.3333bp}{63.0175bp}{90.7953bp}{55.5556bp}{100.0000bp}{55.5556bp}
    \pgfpathqcurveto{109.2047bp}{55.5556bp}{116.6667bp}{63.0175bp}{116.6667bp}{72.2222bp}
    \pgfpathclose
    \pgfusepathqfillstroke
  \end{pgfscope}
  \begin{pgfscope}
    \definecolor{fc}{rgb}{0.0000,0.0000,0.0000}
    \pgfsetfillcolor{fc}
    \pgftransformshift{\pgfqpoint{94.4444bp}{127.2222bp}}
    \pgftransformscale{1.3889}
    \pgftext[base,left]{$p_1$}
  \end{pgfscope}
  \begin{pgfscope}
    \definecolor{fc}{rgb}{0.0000,0.0000,0.0000}
    \pgfsetfillcolor{fc}
    \pgfsetfillopacity{0.0000}
    \pgfsetlinewidth{0.6799bp}
    \definecolor{sc}{rgb}{0.0000,0.0000,0.0000}
    \pgfsetstrokecolor{sc}
    \pgfsetstrokeopacity{0.0000}
    \pgfsetmiterjoin
    \pgfsetbuttcap
    \pgfpathqmoveto{116.6667bp}{127.7778bp}
    \pgfpathqcurveto{116.6667bp}{136.9825bp}{109.2047bp}{144.4444bp}{100.0000bp}{144.4444bp}
    \pgfpathqcurveto{90.7953bp}{144.4444bp}{83.3333bp}{136.9825bp}{83.3333bp}{127.7778bp}
    \pgfpathqcurveto{83.3333bp}{118.5730bp}{90.7953bp}{111.1111bp}{100.0000bp}{111.1111bp}
    \pgfpathqcurveto{109.2047bp}{111.1111bp}{116.6667bp}{118.5730bp}{116.6667bp}{127.7778bp}
    \pgfpathclose
    \pgfusepathqfillstroke
  \end{pgfscope}
  \begin{pgfscope}
    \definecolor{fc}{rgb}{0.0000,0.0000,0.0000}
    \pgfsetfillcolor{fc}
    \pgftransformshift{\pgfqpoint{94.4444bp}{182.7778bp}}
    \pgftransformscale{1.3889}
    \pgftext[base,left]{$q_1$}
  \end{pgfscope}
  \begin{pgfscope}
    \definecolor{fc}{rgb}{0.0000,0.0000,0.0000}
    \pgfsetfillcolor{fc}
    \pgfsetfillopacity{0.0000}
    \pgfsetlinewidth{0.6799bp}
    \definecolor{sc}{rgb}{0.0000,0.0000,0.0000}
    \pgfsetstrokecolor{sc}
    \pgfsetstrokeopacity{0.0000}
    \pgfsetmiterjoin
    \pgfsetbuttcap
    \pgfpathqmoveto{116.6667bp}{183.3333bp}
    \pgfpathqcurveto{116.6667bp}{192.5381bp}{109.2047bp}{200.0000bp}{100.0000bp}{200.0000bp}
    \pgfpathqcurveto{90.7953bp}{200.0000bp}{83.3333bp}{192.5381bp}{83.3333bp}{183.3333bp}
    \pgfpathqcurveto{83.3333bp}{174.1286bp}{90.7953bp}{166.6667bp}{100.0000bp}{166.6667bp}
    \pgfpathqcurveto{109.2047bp}{166.6667bp}{116.6667bp}{174.1286bp}{116.6667bp}{183.3333bp}
    \pgfpathclose
    \pgfusepathqfillstroke
  \end{pgfscope}
  \begin{pgfscope}
    \definecolor{fc}{rgb}{0.0000,0.0000,0.0000}
    \pgfsetfillcolor{fc}
    \pgftransformshift{\pgfqpoint{94.4444bp}{155.0000bp}}
    \pgftransformscale{1.3889}
    \pgftext[base,left]{$\bullet$}
  \end{pgfscope}
  \begin{pgfscope}
    \definecolor{fc}{rgb}{0.0000,0.0000,0.0000}
    \pgfsetfillcolor{fc}
    \pgftransformshift{\pgfqpoint{94.4444bp}{99.4444bp}}
    \pgftransformscale{1.3889}
    \pgftext[base,left]{$\times$}
  \end{pgfscope}
  \begin{pgfscope}
    \definecolor{fc}{rgb}{0.0000,0.0000,0.0000}
    \pgfsetfillcolor{fc}
    \pgftransformshift{\pgfqpoint{66.6667bp}{43.8889bp}}
    \pgftransformscale{1.3889}
    \pgftext[base,left]{$\bullet$}
  \end{pgfscope}
  \begin{pgfscope}
    \definecolor{fc}{rgb}{0.0000,0.0000,0.0000}
    \pgfsetfillcolor{fc}
    \pgftransformshift{\pgfqpoint{122.2222bp}{43.8889bp}}
    \pgftransformscale{1.3889}
    \pgftext[base,left]{$\bullet$}
  \end{pgfscope}
  \begin{pgfscope}
    \definecolor{fc}{rgb}{0.0000,0.0000,0.0000}
    \pgfsetfillcolor{fc}
    \pgftransformshift{\pgfqpoint{94.4444bp}{16.1111bp}}
    \pgftransformscale{1.3889}
    \pgftext[base,left]{$\bullet$}
  \end{pgfscope}
  \begin{pgfscope}
    \definecolor{fc}{rgb}{0.0000,0.0000,0.0000}
    \pgfsetfillcolor{fc}
    \pgftransformshift{\pgfqpoint{94.4444bp}{16.1111bp}}
    \pgftransformscale{1.3889}
    \pgftext[base,left]{$\bullet$}
  \end{pgfscope}
\end{pgfpicture}
}}
        \caption{Path sets as a graph by joining redundant subtrees in path tree.}
        \label{fig:ca:graphExGraph}
    \end{subfigure}
    \hfill%
    \begin{subfigure}[t]{0.30\textwidth}
        \centering
        \resizebox{1\linewidth}{!}{\raisebox{-0.5\height}{\begin{pgfpicture}
  \pgfpathrectangle{\pgfpointorigin}{\pgfqpoint{200.0000bp}{200.0000bp}}
  \pgfusepath{use as bounding box}
  \begin{pgfscope}
    \definecolor{fc}{rgb}{0.0000,0.0000,0.0000}
    \pgfsetfillcolor{fc}
    \pgfsetlinewidth{0.6867bp}
    \definecolor{sc}{rgb}{0.0000,0.0000,0.0000}
    \pgfsetstrokecolor{sc}
    \pgfsetmiterjoin
    \pgfsetbuttcap
    \pgfpathqmoveto{114.8865bp}{88.5707bp}
    \pgfpathqcurveto{106.6649bp}{96.7922bp}{93.3351bp}{96.7922bp}{85.1135bp}{88.5707bp}
    \pgfpathqcurveto{76.8920bp}{80.3491bp}{76.8920bp}{67.0193bp}{85.1135bp}{58.7978bp}
    \pgfpathqlineto{137.7451bp}{6.1662bp}
    \pgfpathqcurveto{145.9667bp}{-2.0554bp}{159.2965bp}{-2.0554bp}{167.5180bp}{6.1662bp}
    \pgfpathqcurveto{175.7396bp}{14.3877bp}{175.7396bp}{27.7175bp}{167.5180bp}{35.9391bp}
    \pgfpathqcurveto{163.5699bp}{39.8872bp}{158.2151bp}{42.1053bp}{152.6316bp}{42.1053bp}
    \pgfpathqlineto{47.3684bp}{42.1053bp}
    \pgfpathqcurveto{35.7414bp}{42.1053bp}{26.3158bp}{32.6797bp}{26.3158bp}{21.0526bp}
    \pgfpathqcurveto{26.3158bp}{9.4256bp}{35.7414bp}{0.0000bp}{47.3684bp}{0.0000bp}
    \pgfpathqcurveto{52.9519bp}{0.0000bp}{58.3067bp}{2.2180bp}{62.2549bp}{6.1662bp}
    \pgfpathqlineto{114.8865bp}{58.7978bp}
    \pgfpathqcurveto{123.1080bp}{67.0193bp}{123.1080bp}{80.3491bp}{114.8865bp}{88.5707bp}
    \pgfpathqcurveto{106.6649bp}{96.7922bp}{93.3351bp}{96.7922bp}{85.1135bp}{88.5707bp}
    \pgfpathqlineto{32.4820bp}{35.9391bp}
    \pgfpathqcurveto{24.2604bp}{27.7175bp}{24.2604bp}{14.3877bp}{32.4820bp}{6.1662bp}
    \pgfpathqcurveto{36.4301bp}{2.2180bp}{41.7849bp}{0.0000bp}{47.3684bp}{0.0000bp}
    \pgfpathqlineto{152.6316bp}{0.0000bp}
    \pgfpathqcurveto{164.2586bp}{0.0000bp}{173.6842bp}{9.4256bp}{173.6842bp}{21.0526bp}
    \pgfpathqcurveto{173.6842bp}{26.6361bp}{171.4662bp}{31.9910bp}{167.5180bp}{35.9391bp}
    \pgfpathqlineto{114.8865bp}{88.5707bp}
    \pgfpathclose
    \pgfusepathqfillstroke
  \end{pgfscope}
  \begin{pgfscope}
    \definecolor{fc}{rgb}{1.0000,1.0000,1.0000}
    \pgfsetfillcolor{fc}
    \pgfsetlinewidth{0.6867bp}
    \definecolor{sc}{rgb}{1.0000,1.0000,1.0000}
    \pgfsetstrokecolor{sc}
    \pgfsetmiterjoin
    \pgfsetbuttcap
    \pgfpathqmoveto{114.1421bp}{87.8263bp}
    \pgfpathqcurveto{106.3316bp}{95.6368bp}{93.6684bp}{95.6368bp}{85.8579bp}{87.8263bp}
    \pgfpathqcurveto{78.0474bp}{80.0159bp}{78.0474bp}{67.3526bp}{85.8579bp}{59.5421bp}
    \pgfpathqlineto{138.4894bp}{6.9105bp}
    \pgfpathqcurveto{146.2999bp}{-0.9000bp}{158.9632bp}{-0.9000bp}{166.7737bp}{6.9105bp}
    \pgfpathqcurveto{174.5842bp}{14.7210bp}{174.5842bp}{27.3843bp}{166.7737bp}{35.1948bp}
    \pgfpathqcurveto{163.0230bp}{38.9455bp}{157.9359bp}{41.0526bp}{152.6316bp}{41.0526bp}
    \pgfpathqlineto{47.3684bp}{41.0526bp}
    \pgfpathqcurveto{36.3227bp}{41.0526bp}{27.3684bp}{32.0983bp}{27.3684bp}{21.0526bp}
    \pgfpathqcurveto{27.3684bp}{10.0069bp}{36.3227bp}{1.0526bp}{47.3684bp}{1.0526bp}
    \pgfpathqcurveto{52.6728bp}{1.0526bp}{57.7598bp}{3.1598bp}{61.5106bp}{6.9105bp}
    \pgfpathqlineto{114.1421bp}{59.5421bp}
    \pgfpathqcurveto{121.9526bp}{67.3526bp}{121.9526bp}{80.0159bp}{114.1421bp}{87.8263bp}
    \pgfpathqcurveto{106.3316bp}{95.6368bp}{93.6684bp}{95.6368bp}{85.8579bp}{87.8263bp}
    \pgfpathqlineto{33.2263bp}{35.1948bp}
    \pgfpathqcurveto{25.4158bp}{27.3843bp}{25.4158bp}{14.7210bp}{33.2263bp}{6.9105bp}
    \pgfpathqcurveto{36.9770bp}{3.1598bp}{42.0641bp}{1.0526bp}{47.3684bp}{1.0526bp}
    \pgfpathqlineto{152.6316bp}{1.0526bp}
    \pgfpathqcurveto{163.6773bp}{1.0526bp}{172.6316bp}{10.0069bp}{172.6316bp}{21.0526bp}
    \pgfpathqcurveto{172.6316bp}{26.3570bp}{170.5244bp}{31.4440bp}{166.7737bp}{35.1948bp}
    \pgfpathqlineto{114.1421bp}{87.8263bp}
    \pgfpathclose
    \pgfusepathqfillstroke
  \end{pgfscope}
  \begin{pgfscope}
    \definecolor{fc}{rgb}{0.0000,0.0000,0.0000}
    \pgfsetfillcolor{fc}
    \pgfsetlinewidth{0.6867bp}
    \definecolor{sc}{rgb}{0.0000,0.0000,0.0000}
    \pgfsetstrokecolor{sc}
    \pgfsetmiterjoin
    \pgfsetbuttcap
    \pgfpathqmoveto{121.0526bp}{178.9474bp}
    \pgfpathqcurveto{121.0526bp}{190.5744bp}{111.6270bp}{200.0000bp}{100.0000bp}{200.0000bp}
    \pgfpathqcurveto{88.3730bp}{200.0000bp}{78.9474bp}{190.5744bp}{78.9474bp}{178.9474bp}
    \pgfpathqlineto{78.9474bp}{126.3158bp}
    \pgfpathqcurveto{78.9474bp}{114.6887bp}{88.3730bp}{105.2632bp}{100.0000bp}{105.2632bp}
    \pgfpathqcurveto{111.6270bp}{105.2632bp}{121.0526bp}{114.6887bp}{121.0526bp}{126.3158bp}
    \pgfpathqlineto{121.0526bp}{178.9474bp}
    \pgfpathqcurveto{121.0526bp}{190.5744bp}{111.6270bp}{200.0000bp}{100.0000bp}{200.0000bp}
    \pgfpathqcurveto{88.3730bp}{200.0000bp}{78.9474bp}{190.5744bp}{78.9474bp}{178.9474bp}
    \pgfpathqlineto{78.9474bp}{126.3158bp}
    \pgfpathqcurveto{78.9474bp}{114.6887bp}{88.3730bp}{105.2632bp}{100.0000bp}{105.2632bp}
    \pgfpathqcurveto{111.6270bp}{105.2632bp}{121.0526bp}{114.6887bp}{121.0526bp}{126.3158bp}
    \pgfpathqlineto{121.0526bp}{178.9474bp}
    \pgfpathclose
    \pgfusepathqfillstroke
  \end{pgfscope}
  \begin{pgfscope}
    \definecolor{fc}{rgb}{1.0000,1.0000,1.0000}
    \pgfsetfillcolor{fc}
    \pgfsetlinewidth{0.6867bp}
    \definecolor{sc}{rgb}{1.0000,1.0000,1.0000}
    \pgfsetstrokecolor{sc}
    \pgfsetmiterjoin
    \pgfsetbuttcap
    \pgfpathqmoveto{120.0000bp}{178.9474bp}
    \pgfpathqcurveto{120.0000bp}{189.9931bp}{111.0457bp}{198.9474bp}{100.0000bp}{198.9474bp}
    \pgfpathqcurveto{88.9543bp}{198.9474bp}{80.0000bp}{189.9931bp}{80.0000bp}{178.9474bp}
    \pgfpathqlineto{80.0000bp}{126.3158bp}
    \pgfpathqcurveto{80.0000bp}{115.2701bp}{88.9543bp}{106.3158bp}{100.0000bp}{106.3158bp}
    \pgfpathqcurveto{111.0457bp}{106.3158bp}{120.0000bp}{115.2701bp}{120.0000bp}{126.3158bp}
    \pgfpathqlineto{120.0000bp}{178.9474bp}
    \pgfpathqcurveto{120.0000bp}{189.9931bp}{111.0457bp}{198.9474bp}{100.0000bp}{198.9474bp}
    \pgfpathqcurveto{88.9543bp}{198.9474bp}{80.0000bp}{189.9931bp}{80.0000bp}{178.9474bp}
    \pgfpathqlineto{80.0000bp}{126.3158bp}
    \pgfpathqcurveto{80.0000bp}{115.2701bp}{88.9543bp}{106.3158bp}{100.0000bp}{106.3158bp}
    \pgfpathqcurveto{111.0457bp}{106.3158bp}{120.0000bp}{115.2701bp}{120.0000bp}{126.3158bp}
    \pgfpathqlineto{120.0000bp}{178.9474bp}
    \pgfpathclose
    \pgfusepathqfillstroke
  \end{pgfscope}
  \begin{pgfscope}
    \definecolor{fc}{rgb}{0.0000,0.0000,0.0000}
    \pgfsetfillcolor{fc}
    \pgftransformcm{1.0000}{0.0000}{0.0000}{1.0000}{\pgfqpoint{157.8947bp}{26.3158bp}}
    \pgftransformscale{1.3158}
    \pgftext[base,left]{$m_1$}
  \end{pgfscope}
  \begin{pgfscope}
    \definecolor{fc}{rgb}{0.0000,0.0000,0.0000}
    \pgfsetfillcolor{fc}
    \pgfsetlinewidth{0.6867bp}
    \definecolor{sc}{rgb}{0.0000,0.0000,0.0000}
    \pgfsetstrokecolor{sc}
    \pgfsetmiterjoin
    \pgfsetbuttcap
    \pgfpathqmoveto{155.2632bp}{21.0526bp}
    \pgfpathqcurveto{155.2632bp}{22.5060bp}{154.0850bp}{23.6842bp}{152.6316bp}{23.6842bp}
    \pgfpathqcurveto{151.1782bp}{23.6842bp}{150.0000bp}{22.5060bp}{150.0000bp}{21.0526bp}
    \pgfpathqcurveto{150.0000bp}{19.5993bp}{151.1782bp}{18.4211bp}{152.6316bp}{18.4211bp}
    \pgfpathqcurveto{154.0850bp}{18.4211bp}{155.2632bp}{19.5993bp}{155.2632bp}{21.0526bp}
    \pgfpathclose
    \pgfusepathqfillstroke
  \end{pgfscope}
  \begin{pgfscope}
    \definecolor{fc}{rgb}{0.0000,0.0000,0.0000}
    \pgfsetfillcolor{fc}
    \pgftransformcm{1.0000}{0.0000}{0.0000}{1.0000}{\pgfqpoint{52.6316bp}{26.3158bp}}
    \pgftransformscale{1.3158}
    \pgftext[base,left]{$g_1$}
  \end{pgfscope}
  \begin{pgfscope}
    \definecolor{fc}{rgb}{0.0000,0.0000,0.0000}
    \pgfsetfillcolor{fc}
    \pgfsetlinewidth{0.6867bp}
    \definecolor{sc}{rgb}{0.0000,0.0000,0.0000}
    \pgfsetstrokecolor{sc}
    \pgfsetmiterjoin
    \pgfsetbuttcap
    \pgfpathqmoveto{50.0000bp}{21.0526bp}
    \pgfpathqcurveto{50.0000bp}{22.5060bp}{48.8218bp}{23.6842bp}{47.3684bp}{23.6842bp}
    \pgfpathqcurveto{45.9150bp}{23.6842bp}{44.7368bp}{22.5060bp}{44.7368bp}{21.0526bp}
    \pgfpathqcurveto{44.7368bp}{19.5993bp}{45.9150bp}{18.4211bp}{47.3684bp}{18.4211bp}
    \pgfpathqcurveto{48.8218bp}{18.4211bp}{50.0000bp}{19.5993bp}{50.0000bp}{21.0526bp}
    \pgfpathclose
    \pgfusepathqfillstroke
  \end{pgfscope}
  \begin{pgfscope}
    \definecolor{fc}{rgb}{0.0000,0.0000,0.0000}
    \pgfsetfillcolor{fc}
    \pgftransformcm{1.0000}{0.0000}{0.0000}{1.0000}{\pgfqpoint{105.2632bp}{78.9474bp}}
    \pgftransformscale{1.3158}
    \pgftext[base,left]{$p_2$}
  \end{pgfscope}
  \begin{pgfscope}
    \definecolor{fc}{rgb}{0.0000,0.0000,0.0000}
    \pgfsetfillcolor{fc}
    \pgfsetlinewidth{0.6867bp}
    \definecolor{sc}{rgb}{0.0000,0.0000,0.0000}
    \pgfsetstrokecolor{sc}
    \pgfsetmiterjoin
    \pgfsetbuttcap
    \pgfpathqmoveto{102.6316bp}{73.6842bp}
    \pgfpathqcurveto{102.6316bp}{75.1376bp}{101.4534bp}{76.3158bp}{100.0000bp}{76.3158bp}
    \pgfpathqcurveto{98.5466bp}{76.3158bp}{97.3684bp}{75.1376bp}{97.3684bp}{73.6842bp}
    \pgfpathqcurveto{97.3684bp}{72.2308bp}{98.5466bp}{71.0526bp}{100.0000bp}{71.0526bp}
    \pgfpathqcurveto{101.4534bp}{71.0526bp}{102.6316bp}{72.2308bp}{102.6316bp}{73.6842bp}
    \pgfpathclose
    \pgfusepathqfillstroke
  \end{pgfscope}
  \begin{pgfscope}
    \definecolor{fc}{rgb}{0.0000,0.0000,0.0000}
    \pgfsetfillcolor{fc}
    \pgftransformcm{1.0000}{0.0000}{0.0000}{1.0000}{\pgfqpoint{105.2632bp}{131.5789bp}}
    \pgftransformscale{1.3158}
    \pgftext[base,left]{$p_1$}
  \end{pgfscope}
  \begin{pgfscope}
    \definecolor{fc}{rgb}{0.0000,0.0000,0.0000}
    \pgfsetfillcolor{fc}
    \pgfsetlinewidth{0.6867bp}
    \definecolor{sc}{rgb}{0.0000,0.0000,0.0000}
    \pgfsetstrokecolor{sc}
    \pgfsetmiterjoin
    \pgfsetbuttcap
    \pgfpathqmoveto{102.6316bp}{126.3158bp}
    \pgfpathqcurveto{102.6316bp}{127.7692bp}{101.4534bp}{128.9474bp}{100.0000bp}{128.9474bp}
    \pgfpathqcurveto{98.5466bp}{128.9474bp}{97.3684bp}{127.7692bp}{97.3684bp}{126.3158bp}
    \pgfpathqcurveto{97.3684bp}{124.8624bp}{98.5466bp}{123.6842bp}{100.0000bp}{123.6842bp}
    \pgfpathqcurveto{101.4534bp}{123.6842bp}{102.6316bp}{124.8624bp}{102.6316bp}{126.3158bp}
    \pgfpathclose
    \pgfusepathqfillstroke
  \end{pgfscope}
  \begin{pgfscope}
    \pgfsetlinewidth{1.2876bp}
    \definecolor{sc}{rgb}{0.0000,0.0000,0.0000}
    \pgfsetstrokecolor{sc}
    \pgfsetmiterjoin
    \pgfsetbuttcap
    \pgfpathqmoveto{100.0000bp}{126.3158bp}
    \pgfpathqlineto{100.0000bp}{73.6842bp}
    \pgfusepathqstroke
  \end{pgfscope}
  \begin{pgfscope}
    \definecolor{fc}{rgb}{0.0000,0.0000,0.0000}
    \pgfsetfillcolor{fc}
    \pgfusepathqfill
  \end{pgfscope}
  \begin{pgfscope}
    \definecolor{fc}{rgb}{0.0000,0.0000,0.0000}
    \pgfsetfillcolor{fc}
    \pgfusepathqfill
  \end{pgfscope}
  \begin{pgfscope}
    \definecolor{fc}{rgb}{0.0000,0.0000,0.0000}
    \pgfsetfillcolor{fc}
    \pgfusepathqfill
  \end{pgfscope}
  \begin{pgfscope}
    \definecolor{fc}{rgb}{0.0000,0.0000,0.0000}
    \pgfsetfillcolor{fc}
    \pgfusepathqfill
  \end{pgfscope}
  \begin{pgfscope}
    \definecolor{fc}{rgb}{0.0000,0.0000,0.0000}
    \pgfsetfillcolor{fc}
    \pgftransformcm{1.0000}{0.0000}{0.0000}{1.0000}{\pgfqpoint{105.2632bp}{184.2105bp}}
    \pgftransformscale{1.3158}
    \pgftext[base,left]{$q_1$}
  \end{pgfscope}
  \begin{pgfscope}
    \definecolor{fc}{rgb}{0.0000,0.0000,0.0000}
    \pgfsetfillcolor{fc}
    \pgfsetlinewidth{0.6867bp}
    \definecolor{sc}{rgb}{0.0000,0.0000,0.0000}
    \pgfsetstrokecolor{sc}
    \pgfsetmiterjoin
    \pgfsetbuttcap
    \pgfpathqmoveto{102.6316bp}{178.9474bp}
    \pgfpathqcurveto{102.6316bp}{180.4007bp}{101.4534bp}{181.5789bp}{100.0000bp}{181.5789bp}
    \pgfpathqcurveto{98.5466bp}{181.5789bp}{97.3684bp}{180.4007bp}{97.3684bp}{178.9474bp}
    \pgfpathqcurveto{97.3684bp}{177.4940bp}{98.5466bp}{176.3158bp}{100.0000bp}{176.3158bp}
    \pgfpathqcurveto{101.4534bp}{176.3158bp}{102.6316bp}{177.4940bp}{102.6316bp}{178.9474bp}
    \pgfpathclose
    \pgfusepathqfillstroke
  \end{pgfscope}
\end{pgfpicture}
}}
        \caption{Path graph seen as a hypergraph.}
        \label{fig:ca:graphExHG}
    \end{subfigure}
    \caption{Figure of hypergraphs for example~\ref{ex:ca:pathTrees}.}\label{fig:ex:ca:hgi}
\end{figure}
%
%To summarize the our prior model for preconditions and effects we have the four following sets.
%\begin{description}
%	\item [{Unproven~$B_u$}] The set of all bindings which has
%	neither been proven nor disproven.
%	\item [{Proven~$B_k$}] The set of all bindings which has been
%	proven.
%	\item [{Disproven~$B_d$}] Is the set of bindings and atoms which have
%	been disproven, it can be derived from $B_d={\left( B_u \cup B_k \right)}^c$.
%	\item [{Candidates~$B_c$}] Is a set of bindings where at least one of the bindings is an actual binding.
%	This set like for non-conditional actions are used to prove bindings,
%	which is accomplished through disproving the other candidates in the set.
%\end{description}
%
%
%	\subsection*{Hypergraph}
%	As we have shown we need a model that represent different bindings over any number of atoms.
%	Furthermore, it must provide support for multiple of the same atoms.
%
%	For these reason we have designed a model for atom bindings using hypergraphs;
%	in this	model a vertex is a single atom's variable position.
%	For instance, if we modelled the following atom $p(x,x)$ its vertices would be $p1$ and $p2$ because it contains two variables.
%	The variables position is enumerated in sequence $1\ldots |p|$.
%
%	%<\texttt{TODO:} introduction to hypergraphs, edges are sets of vertices>
%
%	In hypergraphs edges can be between more than 2 vertices, we use this to our advantage when modelling bindings.
%	We can define two kinds of edges: one to combine atom vertices into a single atom and one to represent bindings.
%
%	The two kinds of edges are defined as follows:
%
%
%	\begin{description}
%		\item [{Predicate~edge}] As  single atoms are represented as several vertices one for each variable
%		it has; a number is used to indicate the order of the variable's
%		positions. This number is appended to the label of the vertex. And an edge is used to connect these vertices together into a single atom.
%		We choose to visualize this type of edge with a line.\\
%		\begin{tabular}{c  c}
%
%			$p(x, y)$ becomes &
%            \raisebox{-.5\height}{\resizebox{0.16\linewidth}{!}{\begin{pgfpicture}
  \pgfpathrectangle{\pgfpointorigin}{\pgfqpoint{200.0000bp}{200.0000bp}}
  \pgfusepath{use as bounding box}
  \begin{pgfscope}
    \definecolor{fc}{rgb}{0.0000,0.0000,0.0000}
    \pgfsetfillcolor{fc}
    \pgftransformshift{\pgfqpoint{200.0000bp}{113.0435bp}}
    \pgftransformscale{4.3478}
    \pgftext[base,left]{$p_2$}
  \end{pgfscope}
  \begin{pgfscope}
    \definecolor{fc}{rgb}{0.0000,0.0000,0.0000}
    \pgfsetfillcolor{fc}
    \pgfsetlinewidth{0.5000bp}
    \definecolor{sc}{rgb}{0.0000,0.0000,0.0000}
    \pgfsetstrokecolor{sc}
    \pgfsetmiterjoin
    \pgfsetbuttcap
    \pgfpathqmoveto{191.3043bp}{95.6522bp}
    \pgfpathqcurveto{191.3043bp}{100.4546bp}{187.4112bp}{104.3478bp}{182.6087bp}{104.3478bp}
    \pgfpathqcurveto{177.8062bp}{104.3478bp}{173.9130bp}{100.4546bp}{173.9130bp}{95.6522bp}
    \pgfpathqcurveto{173.9130bp}{90.8497bp}{177.8062bp}{86.9565bp}{182.6087bp}{86.9565bp}
    \pgfpathqcurveto{187.4112bp}{86.9565bp}{191.3043bp}{90.8497bp}{191.3043bp}{95.6522bp}
    \pgfpathclose
    \pgfusepathqfillstroke
  \end{pgfscope}
  \begin{pgfscope}
    \definecolor{fc}{rgb}{0.0000,0.0000,0.0000}
    \pgfsetfillcolor{fc}
    \pgftransformshift{\pgfqpoint{26.0870bp}{113.0435bp}}
    \pgftransformscale{4.3478}
    \pgftext[base,left]{$p_1$}
  \end{pgfscope}
  \begin{pgfscope}
    \definecolor{fc}{rgb}{0.0000,0.0000,0.0000}
    \pgfsetfillcolor{fc}
    \pgfsetlinewidth{0.5000bp}
    \definecolor{sc}{rgb}{0.0000,0.0000,0.0000}
    \pgfsetstrokecolor{sc}
    \pgfsetmiterjoin
    \pgfsetbuttcap
    \pgfpathqmoveto{17.3913bp}{95.6522bp}
    \pgfpathqcurveto{17.3913bp}{100.4546bp}{13.4981bp}{104.3478bp}{8.6957bp}{104.3478bp}
    \pgfpathqcurveto{3.8932bp}{104.3478bp}{0.0000bp}{100.4546bp}{0.0000bp}{95.6522bp}
    \pgfpathqcurveto{0.0000bp}{90.8497bp}{3.8932bp}{86.9565bp}{8.6957bp}{86.9565bp}
    \pgfpathqcurveto{13.4981bp}{86.9565bp}{17.3913bp}{90.8497bp}{17.3913bp}{95.6522bp}
    \pgfpathclose
    \pgfusepathqfillstroke
  \end{pgfscope}
  \begin{pgfscope}
    \pgfsetlinewidth{0.5417bp}
    \definecolor{sc}{rgb}{0.0000,0.0000,0.0000}
    \pgfsetstrokecolor{sc}
    \pgfsetmiterjoin
    \pgfsetbuttcap
    \pgfpathqmoveto{8.6957bp}{95.6522bp}
    \pgfpathqlineto{182.6087bp}{95.6522bp}
    \pgfusepathqstroke
  \end{pgfscope}
  \begin{pgfscope}
    \definecolor{fc}{rgb}{0.0000,0.0000,0.0000}
    \pgfsetfillcolor{fc}
    \pgfusepathqfill
  \end{pgfscope}
  \begin{pgfscope}
    \definecolor{fc}{rgb}{0.0000,0.0000,0.0000}
    \pgfsetfillcolor{fc}
    \pgfusepathqfill
  \end{pgfscope}
  \begin{pgfscope}
    \definecolor{fc}{rgb}{0.0000,0.0000,0.0000}
    \pgfsetfillcolor{fc}
    \pgfusepathqfill
  \end{pgfscope}
  \begin{pgfscope}
    \definecolor{fc}{rgb}{0.0000,0.0000,0.0000}
    \pgfsetfillcolor{fc}
    \pgfusepathqfill
  \end{pgfscope}
\end{pgfpicture}
}}
%		\end{tabular}
%		\item[{Binding edge}] To represent variables bound together through a common variable (Bindings) we
%		choose to visualize this using normal cloud notation around them.\\
%		\begin{tabular}{c  c}
%
%			$p(x) \land q(x)$ becomes &
%            \raisebox{-.5\height}{\resizebox{0.2\linewidth}{!}{\begin{pgfpicture}
  \pgfpathrectangle{\pgfpointorigin}{\pgfqpoint{200.0000bp}{200.0000bp}}
  \pgfusepath{use as bounding box}
  \begin{pgfscope}
    \definecolor{fc}{rgb}{0.0000,0.0000,0.0000}
    \pgfsetfillcolor{fc}
    \pgfsetlinewidth{0.5908bp}
    \definecolor{sc}{rgb}{0.0000,0.0000,0.0000}
    \pgfsetstrokecolor{sc}
    \pgfsetmiterjoin
    \pgfsetbuttcap
    \pgfpathqmoveto{54.5455bp}{154.5455bp}
    \pgfpathqcurveto{24.4208bp}{154.5455bp}{0.0000bp}{130.1246bp}{0.0000bp}{100.0000bp}
    \pgfpathqcurveto{-0.0000bp}{69.8754bp}{24.4208bp}{45.4545bp}{54.5455bp}{45.4545bp}
    \pgfpathqlineto{145.4545bp}{45.4545bp}
    \pgfpathqcurveto{175.5792bp}{45.4545bp}{200.0000bp}{69.8754bp}{200.0000bp}{100.0000bp}
    \pgfpathqcurveto{200.0000bp}{130.1246bp}{175.5792bp}{154.5455bp}{145.4545bp}{154.5455bp}
    \pgfpathqlineto{54.5455bp}{154.5455bp}
    \pgfpathclose
    \pgfusepathqfillstroke
  \end{pgfscope}
  \begin{pgfscope}
    \definecolor{fc}{rgb}{1.0000,1.0000,1.0000}
    \pgfsetfillcolor{fc}
    \pgfsetlinewidth{0.5908bp}
    \definecolor{sc}{rgb}{1.0000,1.0000,1.0000}
    \pgfsetstrokecolor{sc}
    \pgfsetmiterjoin
    \pgfsetbuttcap
    \pgfpathqmoveto{54.5455bp}{152.7273bp}
    \pgfpathqcurveto{25.4250bp}{152.7273bp}{1.8182bp}{129.1205bp}{1.8182bp}{100.0000bp}
    \pgfpathqcurveto{1.8182bp}{70.8795bp}{25.4250bp}{47.2727bp}{54.5455bp}{47.2727bp}
    \pgfpathqlineto{145.4545bp}{47.2727bp}
    \pgfpathqcurveto{174.5750bp}{47.2727bp}{198.1818bp}{70.8795bp}{198.1818bp}{100.0000bp}
    \pgfpathqcurveto{198.1818bp}{129.1205bp}{174.5750bp}{152.7273bp}{145.4545bp}{152.7273bp}
    \pgfpathqlineto{54.5455bp}{152.7273bp}
    \pgfpathclose
    \pgfusepathqfillstroke
  \end{pgfscope}
  \begin{pgfscope}
    \definecolor{fc}{rgb}{0.0000,0.0000,0.0000}
    \pgfsetfillcolor{fc}
    \pgftransformshift{\pgfqpoint{154.5455bp}{109.0909bp}}
    \pgftransformscale{2.2727}
    \pgftext[base,left]{$q_1$}
  \end{pgfscope}
  \begin{pgfscope}
    \definecolor{fc}{rgb}{0.0000,0.0000,0.0000}
    \pgfsetfillcolor{fc}
    \pgfsetlinewidth{0.5908bp}
    \definecolor{sc}{rgb}{0.0000,0.0000,0.0000}
    \pgfsetstrokecolor{sc}
    \pgfsetmiterjoin
    \pgfsetbuttcap
    \pgfpathqmoveto{150.0000bp}{100.0000bp}
    \pgfpathqcurveto{150.0000bp}{102.5104bp}{147.9649bp}{104.5455bp}{145.4545bp}{104.5455bp}
    \pgfpathqcurveto{142.9442bp}{104.5455bp}{140.9091bp}{102.5104bp}{140.9091bp}{100.0000bp}
    \pgfpathqcurveto{140.9091bp}{97.4896bp}{142.9442bp}{95.4545bp}{145.4545bp}{95.4545bp}
    \pgfpathqcurveto{147.9649bp}{95.4545bp}{150.0000bp}{97.4896bp}{150.0000bp}{100.0000bp}
    \pgfpathclose
    \pgfusepathqfillstroke
  \end{pgfscope}
  \begin{pgfscope}
    \definecolor{fc}{rgb}{0.0000,0.0000,0.0000}
    \pgfsetfillcolor{fc}
    \pgftransformshift{\pgfqpoint{63.6364bp}{109.0909bp}}
    \pgftransformscale{2.2727}
    \pgftext[base,left]{$p_1$}
  \end{pgfscope}
  \begin{pgfscope}
    \definecolor{fc}{rgb}{0.0000,0.0000,0.0000}
    \pgfsetfillcolor{fc}
    \pgfsetlinewidth{0.5908bp}
    \definecolor{sc}{rgb}{0.0000,0.0000,0.0000}
    \pgfsetstrokecolor{sc}
    \pgfsetmiterjoin
    \pgfsetbuttcap
    \pgfpathqmoveto{59.0909bp}{100.0000bp}
    \pgfpathqcurveto{59.0909bp}{102.5104bp}{57.0558bp}{104.5455bp}{54.5455bp}{104.5455bp}
    \pgfpathqcurveto{52.0351bp}{104.5455bp}{50.0000bp}{102.5104bp}{50.0000bp}{100.0000bp}
    \pgfpathqcurveto{50.0000bp}{97.4896bp}{52.0351bp}{95.4545bp}{54.5455bp}{95.4545bp}
    \pgfpathqcurveto{57.0558bp}{95.4545bp}{59.0909bp}{97.4896bp}{59.0909bp}{100.0000bp}
    \pgfpathclose
    \pgfusepathqfillstroke
  \end{pgfscope}
\end{pgfpicture}
}}
%		\end{tabular}
%	\end{description}
%
%    A state can thus be represented as a hypergraph by converting each literal to a set of vertices as described above, and collecting all vertices stemming from variables with the same name in binding edges. A detailed algorithm for this conversion is given in Section~\ref{sec:Impl:HGConstruction}, and an example is provided below.
%
%% In addition to the above, a hypergraph also contains exactly one predicate edge which is labelled as the effect of the pattern the hypergraph describes (as a conditional effect can only have one effect, see~\ref{rst:ca:no-multiple-effect}).
%%
%% Note that a predicate edge will always contain a number of vertices equal to the arity of the predicate, while a binding edge can contain any number of vertices. Furthermore, a binding set can contain several vertices describing the same argument place of the same predicate. To discern vertices, we define them as a tuple containing a name denoting the predicate and the argument place it represents (such as $p_3$), as well as a unique identifier. 
%%
%% \begin{proposition}
%%     All vertices in a hypergraph $H$ is member of exactly one predicate edge and one binding edge.
%% \end{proposition}
%
%\begin{figure}
%    \centering
%        \begin{pgfpicture}
  \pgfpathrectangle{\pgfpointorigin}{\pgfqpoint{200.0000bp}{200.0000bp}}
  \pgfusepath{use as bounding box}
  \begin{pgfscope}
    \definecolor{fc}{rgb}{0.0000,0.0000,0.0000}
    \pgfsetfillcolor{fc}
    \pgfsetlinewidth{0.6867bp}
    \definecolor{sc}{rgb}{0.0000,0.0000,0.0000}
    \pgfsetstrokecolor{sc}
    \pgfsetmiterjoin
    \pgfsetbuttcap
    \pgfpathqmoveto{73.6842bp}{173.6842bp}
    \pgfpathqcurveto{62.0572bp}{173.6842bp}{52.6316bp}{164.2586bp}{52.6316bp}{152.6316bp}
    \pgfpathqcurveto{52.6316bp}{141.0045bp}{62.0572bp}{131.5789bp}{73.6842bp}{131.5789bp}
    \pgfpathqlineto{178.9474bp}{131.5789bp}
    \pgfpathqcurveto{190.5744bp}{131.5789bp}{200.0000bp}{141.0045bp}{200.0000bp}{152.6316bp}
    \pgfpathqcurveto{200.0000bp}{164.2586bp}{190.5744bp}{173.6842bp}{178.9474bp}{173.6842bp}
    \pgfpathqlineto{73.6842bp}{173.6842bp}
    \pgfpathqcurveto{62.0572bp}{173.6842bp}{52.6316bp}{164.2586bp}{52.6316bp}{152.6316bp}
    \pgfpathqcurveto{52.6316bp}{141.0045bp}{62.0572bp}{131.5789bp}{73.6842bp}{131.5789bp}
    \pgfpathqlineto{178.9474bp}{131.5789bp}
    \pgfpathqcurveto{190.5744bp}{131.5789bp}{200.0000bp}{141.0045bp}{200.0000bp}{152.6316bp}
    \pgfpathqcurveto{200.0000bp}{164.2586bp}{190.5744bp}{173.6842bp}{178.9474bp}{173.6842bp}
    \pgfpathqlineto{73.6842bp}{173.6842bp}
    \pgfpathclose
    \pgfusepathqfillstroke
  \end{pgfscope}
  \begin{pgfscope}
    \definecolor{fc}{rgb}{1.0000,1.0000,1.0000}
    \pgfsetfillcolor{fc}
    \pgfsetlinewidth{0.6867bp}
    \definecolor{sc}{rgb}{1.0000,1.0000,1.0000}
    \pgfsetstrokecolor{sc}
    \pgfsetmiterjoin
    \pgfsetbuttcap
    \pgfpathqmoveto{73.6842bp}{172.6316bp}
    \pgfpathqcurveto{62.6385bp}{172.6316bp}{53.6842bp}{163.6773bp}{53.6842bp}{152.6316bp}
    \pgfpathqcurveto{53.6842bp}{141.5859bp}{62.6385bp}{132.6316bp}{73.6842bp}{132.6316bp}
    \pgfpathqlineto{178.9474bp}{132.6316bp}
    \pgfpathqcurveto{189.9931bp}{132.6316bp}{198.9474bp}{141.5859bp}{198.9474bp}{152.6316bp}
    \pgfpathqcurveto{198.9474bp}{163.6773bp}{189.9931bp}{172.6316bp}{178.9474bp}{172.6316bp}
    \pgfpathqlineto{73.6842bp}{172.6316bp}
    \pgfpathqcurveto{62.6385bp}{172.6316bp}{53.6842bp}{163.6773bp}{53.6842bp}{152.6316bp}
    \pgfpathqcurveto{53.6842bp}{141.5859bp}{62.6385bp}{132.6316bp}{73.6842bp}{132.6316bp}
    \pgfpathqlineto{178.9474bp}{132.6316bp}
    \pgfpathqcurveto{189.9931bp}{132.6316bp}{198.9474bp}{141.5859bp}{198.9474bp}{152.6316bp}
    \pgfpathqcurveto{198.9474bp}{163.6773bp}{189.9931bp}{172.6316bp}{178.9474bp}{172.6316bp}
    \pgfpathqlineto{73.6842bp}{172.6316bp}
    \pgfpathclose
    \pgfusepathqfillstroke
  \end{pgfscope}
  \begin{pgfscope}
    \definecolor{fc}{rgb}{0.0000,0.0000,0.0000}
    \pgfsetfillcolor{fc}
    \pgfsetlinewidth{0.6867bp}
    \definecolor{sc}{rgb}{0.0000,0.0000,0.0000}
    \pgfsetstrokecolor{sc}
    \pgfsetmiterjoin
    \pgfsetbuttcap
    \pgfpathqmoveto{200.0000bp}{100.0000bp}
    \pgfpathqcurveto{200.0000bp}{111.6270bp}{190.5744bp}{121.0526bp}{178.9474bp}{121.0526bp}
    \pgfpathqcurveto{167.3203bp}{121.0526bp}{157.8947bp}{111.6270bp}{157.8947bp}{100.0000bp}
    \pgfpathqcurveto{157.8947bp}{88.3730bp}{167.3203bp}{78.9474bp}{178.9474bp}{78.9474bp}
    \pgfpathqcurveto{190.5744bp}{78.9474bp}{200.0000bp}{88.3730bp}{200.0000bp}{100.0000bp}
    \pgfpathclose
    \pgfusepathqfillstroke
  \end{pgfscope}
  \begin{pgfscope}
    \definecolor{fc}{rgb}{1.0000,1.0000,1.0000}
    \pgfsetfillcolor{fc}
    \pgfsetlinewidth{0.6867bp}
    \definecolor{sc}{rgb}{1.0000,1.0000,1.0000}
    \pgfsetstrokecolor{sc}
    \pgfsetmiterjoin
    \pgfsetbuttcap
    \pgfpathqmoveto{198.9474bp}{100.0000bp}
    \pgfpathqcurveto{198.9474bp}{111.0457bp}{189.9931bp}{120.0000bp}{178.9474bp}{120.0000bp}
    \pgfpathqcurveto{167.9017bp}{120.0000bp}{158.9474bp}{111.0457bp}{158.9474bp}{100.0000bp}
    \pgfpathqcurveto{158.9474bp}{88.9543bp}{167.9017bp}{80.0000bp}{178.9474bp}{80.0000bp}
    \pgfpathqcurveto{189.9931bp}{80.0000bp}{198.9474bp}{88.9543bp}{198.9474bp}{100.0000bp}
    \pgfpathclose
    \pgfusepathqfillstroke
  \end{pgfscope}
  \begin{pgfscope}
    \definecolor{fc}{rgb}{0.0000,0.0000,0.0000}
    \pgfsetfillcolor{fc}
    \pgfsetlinewidth{0.6867bp}
    \definecolor{sc}{rgb}{0.0000,0.0000,0.0000}
    \pgfsetstrokecolor{sc}
    \pgfsetmiterjoin
    \pgfsetbuttcap
    \pgfpathqmoveto{21.0526bp}{121.0526bp}
    \pgfpathqcurveto{9.4256bp}{121.0526bp}{0.0000bp}{111.6270bp}{0.0000bp}{100.0000bp}
    \pgfpathqcurveto{0.0000bp}{88.3730bp}{9.4256bp}{78.9474bp}{21.0526bp}{78.9474bp}
    \pgfpathqlineto{126.3158bp}{78.9474bp}
    \pgfpathqcurveto{137.9428bp}{78.9474bp}{147.3684bp}{88.3730bp}{147.3684bp}{100.0000bp}
    \pgfpathqcurveto{147.3684bp}{111.6270bp}{137.9428bp}{121.0526bp}{126.3158bp}{121.0526bp}
    \pgfpathqlineto{73.6842bp}{121.0526bp}
    \pgfpathqlineto{21.0526bp}{121.0526bp}
    \pgfpathqcurveto{9.4256bp}{121.0526bp}{-0.0000bp}{111.6270bp}{-0.0000bp}{100.0000bp}
    \pgfpathqcurveto{-0.0000bp}{88.3730bp}{9.4256bp}{78.9474bp}{21.0526bp}{78.9474bp}
    \pgfpathqlineto{73.6842bp}{78.9474bp}
    \pgfpathqlineto{126.3158bp}{78.9474bp}
    \pgfpathqcurveto{137.9428bp}{78.9474bp}{147.3684bp}{88.3730bp}{147.3684bp}{100.0000bp}
    \pgfpathqcurveto{147.3684bp}{111.6270bp}{137.9428bp}{121.0526bp}{126.3158bp}{121.0526bp}
    \pgfpathqlineto{21.0526bp}{121.0526bp}
    \pgfpathclose
    \pgfusepathqfillstroke
  \end{pgfscope}
  \begin{pgfscope}
    \definecolor{fc}{rgb}{1.0000,1.0000,1.0000}
    \pgfsetfillcolor{fc}
    \pgfsetlinewidth{0.6867bp}
    \definecolor{sc}{rgb}{1.0000,1.0000,1.0000}
    \pgfsetstrokecolor{sc}
    \pgfsetmiterjoin
    \pgfsetbuttcap
    \pgfpathqmoveto{21.0526bp}{120.0000bp}
    \pgfpathqcurveto{10.0069bp}{120.0000bp}{1.0526bp}{111.0457bp}{1.0526bp}{100.0000bp}
    \pgfpathqcurveto{1.0526bp}{88.9543bp}{10.0069bp}{80.0000bp}{21.0526bp}{80.0000bp}
    \pgfpathqlineto{126.3158bp}{80.0000bp}
    \pgfpathqcurveto{137.3615bp}{80.0000bp}{146.3158bp}{88.9543bp}{146.3158bp}{100.0000bp}
    \pgfpathqcurveto{146.3158bp}{111.0457bp}{137.3615bp}{120.0000bp}{126.3158bp}{120.0000bp}
    \pgfpathqlineto{73.6842bp}{120.0000bp}
    \pgfpathqlineto{21.0526bp}{120.0000bp}
    \pgfpathqcurveto{10.0069bp}{120.0000bp}{1.0526bp}{111.0457bp}{1.0526bp}{100.0000bp}
    \pgfpathqcurveto{1.0526bp}{88.9543bp}{10.0069bp}{80.0000bp}{21.0526bp}{80.0000bp}
    \pgfpathqlineto{73.6842bp}{80.0000bp}
    \pgfpathqlineto{126.3158bp}{80.0000bp}
    \pgfpathqcurveto{137.3615bp}{80.0000bp}{146.3158bp}{88.9543bp}{146.3158bp}{100.0000bp}
    \pgfpathqcurveto{146.3158bp}{111.0457bp}{137.3615bp}{120.0000bp}{126.3158bp}{120.0000bp}
    \pgfpathqlineto{21.0526bp}{120.0000bp}
    \pgfpathclose
    \pgfusepathqfillstroke
  \end{pgfscope}
  \begin{pgfscope}
    \definecolor{fc}{rgb}{0.0000,0.0000,0.0000}
    \pgfsetfillcolor{fc}
    \pgfsetlinewidth{0.6867bp}
    \definecolor{sc}{rgb}{0.0000,0.0000,0.0000}
    \pgfsetstrokecolor{sc}
    \pgfsetmiterjoin
    \pgfsetbuttcap
    \pgfpathqmoveto{42.1053bp}{47.3684bp}
    \pgfpathqcurveto{42.1053bp}{58.9955bp}{32.6797bp}{68.4211bp}{21.0526bp}{68.4211bp}
    \pgfpathqcurveto{9.4256bp}{68.4211bp}{0.0000bp}{58.9955bp}{0.0000bp}{47.3684bp}
    \pgfpathqcurveto{0.0000bp}{35.7414bp}{9.4256bp}{26.3158bp}{21.0526bp}{26.3158bp}
    \pgfpathqcurveto{32.6797bp}{26.3158bp}{42.1053bp}{35.7414bp}{42.1053bp}{47.3684bp}
    \pgfpathclose
    \pgfusepathqfillstroke
  \end{pgfscope}
  \begin{pgfscope}
    \definecolor{fc}{rgb}{1.0000,1.0000,1.0000}
    \pgfsetfillcolor{fc}
    \pgfsetlinewidth{0.6867bp}
    \definecolor{sc}{rgb}{1.0000,1.0000,1.0000}
    \pgfsetstrokecolor{sc}
    \pgfsetmiterjoin
    \pgfsetbuttcap
    \pgfpathqmoveto{41.0526bp}{47.3684bp}
    \pgfpathqcurveto{41.0526bp}{58.4141bp}{32.0983bp}{67.3684bp}{21.0526bp}{67.3684bp}
    \pgfpathqcurveto{10.0069bp}{67.3684bp}{1.0526bp}{58.4141bp}{1.0526bp}{47.3684bp}
    \pgfpathqcurveto{1.0526bp}{36.3227bp}{10.0069bp}{27.3684bp}{21.0526bp}{27.3684bp}
    \pgfpathqcurveto{32.0983bp}{27.3684bp}{41.0526bp}{36.3227bp}{41.0526bp}{47.3684bp}
    \pgfpathclose
    \pgfusepathqfillstroke
  \end{pgfscope}
  \begin{pgfscope}
    \definecolor{fc}{rgb}{0.0000,0.0000,0.0000}
    \pgfsetfillcolor{fc}
    \pgftransformcm{1.0000}{0.0000}{0.0000}{1.0000}{\pgfqpoint{184.2105bp}{157.8947bp}}
    \pgftransformscale{1.3158}
    \pgftext[base,left]{$s_1$}
  \end{pgfscope}
  \begin{pgfscope}
    \definecolor{fc}{rgb}{0.0000,0.0000,0.0000}
    \pgfsetfillcolor{fc}
    \pgfsetlinewidth{0.6867bp}
    \definecolor{sc}{rgb}{0.0000,0.0000,0.0000}
    \pgfsetstrokecolor{sc}
    \pgfsetmiterjoin
    \pgfsetbuttcap
    \pgfpathqmoveto{181.5789bp}{152.6316bp}
    \pgfpathqcurveto{181.5789bp}{154.0850bp}{180.4007bp}{155.2632bp}{178.9474bp}{155.2632bp}
    \pgfpathqcurveto{177.4940bp}{155.2632bp}{176.3158bp}{154.0850bp}{176.3158bp}{152.6316bp}
    \pgfpathqcurveto{176.3158bp}{151.1782bp}{177.4940bp}{150.0000bp}{178.9474bp}{150.0000bp}
    \pgfpathqcurveto{180.4007bp}{150.0000bp}{181.5789bp}{151.1782bp}{181.5789bp}{152.6316bp}
    \pgfpathclose
    \pgfusepathqfillstroke
  \end{pgfscope}
  \begin{pgfscope}
    \definecolor{fc}{rgb}{0.0000,0.0000,0.0000}
    \pgfsetfillcolor{fc}
    \pgftransformcm{1.0000}{0.0000}{0.0000}{1.0000}{\pgfqpoint{131.5789bp}{105.2632bp}}
    \pgftransformscale{1.3158}
    \pgftext[base,left]{$at_2$}
  \end{pgfscope}
  \begin{pgfscope}
    \definecolor{fc}{rgb}{0.0000,0.0000,0.0000}
    \pgfsetfillcolor{fc}
    \pgfsetlinewidth{0.6867bp}
    \definecolor{sc}{rgb}{0.0000,0.0000,0.0000}
    \pgfsetstrokecolor{sc}
    \pgfsetmiterjoin
    \pgfsetbuttcap
    \pgfpathqmoveto{128.9474bp}{100.0000bp}
    \pgfpathqcurveto{128.9474bp}{101.4534bp}{127.7692bp}{102.6316bp}{126.3158bp}{102.6316bp}
    \pgfpathqcurveto{124.8624bp}{102.6316bp}{123.6842bp}{101.4534bp}{123.6842bp}{100.0000bp}
    \pgfpathqcurveto{123.6842bp}{98.5466bp}{124.8624bp}{97.3684bp}{126.3158bp}{97.3684bp}
    \pgfpathqcurveto{127.7692bp}{97.3684bp}{128.9474bp}{98.5466bp}{128.9474bp}{100.0000bp}
    \pgfpathclose
    \pgfusepathqfillstroke
  \end{pgfscope}
  \begin{pgfscope}
    \definecolor{fc}{rgb}{0.0000,0.0000,0.0000}
    \pgfsetfillcolor{fc}
    \pgftransformcm{1.0000}{0.0000}{0.0000}{1.0000}{\pgfqpoint{184.2105bp}{105.2632bp}}
    \pgftransformscale{1.3158}
    \pgftext[base,left]{$at_1$}
  \end{pgfscope}
  \begin{pgfscope}
    \definecolor{fc}{rgb}{0.0000,0.0000,0.0000}
    \pgfsetfillcolor{fc}
    \pgfsetlinewidth{0.6867bp}
    \definecolor{sc}{rgb}{0.0000,0.0000,0.0000}
    \pgfsetstrokecolor{sc}
    \pgfsetmiterjoin
    \pgfsetbuttcap
    \pgfpathqmoveto{181.5789bp}{100.0000bp}
    \pgfpathqcurveto{181.5789bp}{101.4534bp}{180.4007bp}{102.6316bp}{178.9474bp}{102.6316bp}
    \pgfpathqcurveto{177.4940bp}{102.6316bp}{176.3158bp}{101.4534bp}{176.3158bp}{100.0000bp}
    \pgfpathqcurveto{176.3158bp}{98.5466bp}{177.4940bp}{97.3684bp}{178.9474bp}{97.3684bp}
    \pgfpathqcurveto{180.4007bp}{97.3684bp}{181.5789bp}{98.5466bp}{181.5789bp}{100.0000bp}
    \pgfpathclose
    \pgfusepathqfillstroke
  \end{pgfscope}
  \begin{pgfscope}
    \pgfsetlinewidth{1.2876bp}
    \definecolor{sc}{rgb}{0.0000,0.0000,0.0000}
    \pgfsetstrokecolor{sc}
    \pgfsetmiterjoin
    \pgfsetbuttcap
    \pgfpathqmoveto{178.9474bp}{100.0000bp}
    \pgfpathqlineto{126.3158bp}{100.0000bp}
    \pgfusepathqstroke
  \end{pgfscope}
  \begin{pgfscope}
    \definecolor{fc}{rgb}{0.0000,0.0000,0.0000}
    \pgfsetfillcolor{fc}
    \pgfusepathqfill
  \end{pgfscope}
  \begin{pgfscope}
    \definecolor{fc}{rgb}{0.0000,0.0000,0.0000}
    \pgfsetfillcolor{fc}
    \pgfusepathqfill
  \end{pgfscope}
  \begin{pgfscope}
    \definecolor{fc}{rgb}{0.0000,0.0000,0.0000}
    \pgfsetfillcolor{fc}
    \pgfusepathqfill
  \end{pgfscope}
  \begin{pgfscope}
    \definecolor{fc}{rgb}{0.0000,0.0000,0.0000}
    \pgfsetfillcolor{fc}
    \pgfusepathqfill
  \end{pgfscope}
  \begin{pgfscope}
    \definecolor{fc}{rgb}{0.0000,0.0000,0.0000}
    \pgfsetfillcolor{fc}
    \pgftransformcm{1.0000}{0.0000}{0.0000}{1.0000}{\pgfqpoint{78.9474bp}{157.8947bp}}
    \pgftransformscale{1.3158}
    \pgftext[base,left]{$adj_2$}
  \end{pgfscope}
  \begin{pgfscope}
    \definecolor{fc}{rgb}{0.0000,0.0000,0.0000}
    \pgfsetfillcolor{fc}
    \pgfsetlinewidth{0.6867bp}
    \definecolor{sc}{rgb}{0.0000,0.0000,0.0000}
    \pgfsetstrokecolor{sc}
    \pgfsetmiterjoin
    \pgfsetbuttcap
    \pgfpathqmoveto{76.3158bp}{152.6316bp}
    \pgfpathqcurveto{76.3158bp}{154.0850bp}{75.1376bp}{155.2632bp}{73.6842bp}{155.2632bp}
    \pgfpathqcurveto{72.2308bp}{155.2632bp}{71.0526bp}{154.0850bp}{71.0526bp}{152.6316bp}
    \pgfpathqcurveto{71.0526bp}{151.1782bp}{72.2308bp}{150.0000bp}{73.6842bp}{150.0000bp}
    \pgfpathqcurveto{75.1376bp}{150.0000bp}{76.3158bp}{151.1782bp}{76.3158bp}{152.6316bp}
    \pgfpathclose
    \pgfusepathqfillstroke
  \end{pgfscope}
  \begin{pgfscope}
    \definecolor{fc}{rgb}{0.0000,0.0000,0.0000}
    \pgfsetfillcolor{fc}
    \pgftransformcm{1.0000}{0.0000}{0.0000}{1.0000}{\pgfqpoint{78.9474bp}{105.2632bp}}
    \pgftransformscale{1.3158}
    \pgftext[base,left]{$adj_1$}
  \end{pgfscope}
  \begin{pgfscope}
    \definecolor{fc}{rgb}{0.0000,0.0000,0.0000}
    \pgfsetfillcolor{fc}
    \pgfsetlinewidth{0.6867bp}
    \definecolor{sc}{rgb}{0.0000,0.0000,0.0000}
    \pgfsetstrokecolor{sc}
    \pgfsetmiterjoin
    \pgfsetbuttcap
    \pgfpathqmoveto{76.3158bp}{100.0000bp}
    \pgfpathqcurveto{76.3158bp}{101.4534bp}{75.1376bp}{102.6316bp}{73.6842bp}{102.6316bp}
    \pgfpathqcurveto{72.2308bp}{102.6316bp}{71.0526bp}{101.4534bp}{71.0526bp}{100.0000bp}
    \pgfpathqcurveto{71.0526bp}{98.5466bp}{72.2308bp}{97.3684bp}{73.6842bp}{97.3684bp}
    \pgfpathqcurveto{75.1376bp}{97.3684bp}{76.3158bp}{98.5466bp}{76.3158bp}{100.0000bp}
    \pgfpathclose
    \pgfusepathqfillstroke
  \end{pgfscope}
  \begin{pgfscope}
    \pgfsetlinewidth{1.2876bp}
    \definecolor{sc}{rgb}{0.0000,0.0000,0.0000}
    \pgfsetstrokecolor{sc}
    \pgfsetmiterjoin
    \pgfsetbuttcap
    \pgfpathqmoveto{73.6842bp}{100.0000bp}
    \pgfpathqlineto{73.6842bp}{152.6316bp}
    \pgfusepathqstroke
  \end{pgfscope}
  \begin{pgfscope}
    \definecolor{fc}{rgb}{0.0000,0.0000,0.0000}
    \pgfsetfillcolor{fc}
    \pgfusepathqfill
  \end{pgfscope}
  \begin{pgfscope}
    \definecolor{fc}{rgb}{0.0000,0.0000,0.0000}
    \pgfsetfillcolor{fc}
    \pgfusepathqfill
  \end{pgfscope}
  \begin{pgfscope}
    \definecolor{fc}{rgb}{0.0000,0.0000,0.0000}
    \pgfsetfillcolor{fc}
    \pgfusepathqfill
  \end{pgfscope}
  \begin{pgfscope}
    \definecolor{fc}{rgb}{0.0000,0.0000,0.0000}
    \pgfsetfillcolor{fc}
    \pgfusepathqfill
  \end{pgfscope}
  \begin{pgfscope}
    \definecolor{fc}{rgb}{0.0000,0.0000,0.0000}
    \pgfsetfillcolor{fc}
    \pgftransformcm{1.0000}{0.0000}{0.0000}{1.0000}{\pgfqpoint{26.3158bp}{105.2632bp}}
    \pgftransformscale{1.3158}
    \pgftext[base,left]{$adj_2$}
  \end{pgfscope}
  \begin{pgfscope}
    \definecolor{fc}{rgb}{0.0000,0.0000,0.0000}
    \pgfsetfillcolor{fc}
    \pgfsetlinewidth{0.6867bp}
    \definecolor{sc}{rgb}{0.0000,0.0000,0.0000}
    \pgfsetstrokecolor{sc}
    \pgfsetmiterjoin
    \pgfsetbuttcap
    \pgfpathqmoveto{23.6842bp}{100.0000bp}
    \pgfpathqcurveto{23.6842bp}{101.4534bp}{22.5060bp}{102.6316bp}{21.0526bp}{102.6316bp}
    \pgfpathqcurveto{19.5993bp}{102.6316bp}{18.4211bp}{101.4534bp}{18.4211bp}{100.0000bp}
    \pgfpathqcurveto{18.4211bp}{98.5466bp}{19.5993bp}{97.3684bp}{21.0526bp}{97.3684bp}
    \pgfpathqcurveto{22.5060bp}{97.3684bp}{23.6842bp}{98.5466bp}{23.6842bp}{100.0000bp}
    \pgfpathclose
    \pgfusepathqfillstroke
  \end{pgfscope}
  \begin{pgfscope}
    \definecolor{fc}{rgb}{0.0000,0.0000,0.0000}
    \pgfsetfillcolor{fc}
    \pgftransformcm{1.0000}{0.0000}{0.0000}{1.0000}{\pgfqpoint{26.3158bp}{52.6316bp}}
    \pgftransformscale{1.3158}
    \pgftext[base,left]{$adj_1$}
  \end{pgfscope}
  \begin{pgfscope}
    \definecolor{fc}{rgb}{0.0000,0.0000,0.0000}
    \pgfsetfillcolor{fc}
    \pgfsetlinewidth{0.6867bp}
    \definecolor{sc}{rgb}{0.0000,0.0000,0.0000}
    \pgfsetstrokecolor{sc}
    \pgfsetmiterjoin
    \pgfsetbuttcap
    \pgfpathqmoveto{23.6842bp}{47.3684bp}
    \pgfpathqcurveto{23.6842bp}{48.8218bp}{22.5060bp}{50.0000bp}{21.0526bp}{50.0000bp}
    \pgfpathqcurveto{19.5993bp}{50.0000bp}{18.4211bp}{48.8218bp}{18.4211bp}{47.3684bp}
    \pgfpathqcurveto{18.4211bp}{45.9150bp}{19.5993bp}{44.7368bp}{21.0526bp}{44.7368bp}
    \pgfpathqcurveto{22.5060bp}{44.7368bp}{23.6842bp}{45.9150bp}{23.6842bp}{47.3684bp}
    \pgfpathclose
    \pgfusepathqfillstroke
  \end{pgfscope}
  \begin{pgfscope}
    \pgfsetlinewidth{1.2876bp}
    \definecolor{sc}{rgb}{0.0000,0.0000,0.0000}
    \pgfsetstrokecolor{sc}
    \pgfsetmiterjoin
    \pgfsetbuttcap
    \pgfpathqmoveto{21.0526bp}{47.3684bp}
    \pgfpathqlineto{21.0526bp}{100.0000bp}
    \pgfusepathqstroke
  \end{pgfscope}
  \begin{pgfscope}
    \definecolor{fc}{rgb}{0.0000,0.0000,0.0000}
    \pgfsetfillcolor{fc}
    \pgfusepathqfill
  \end{pgfscope}
  \begin{pgfscope}
    \definecolor{fc}{rgb}{0.0000,0.0000,0.0000}
    \pgfsetfillcolor{fc}
    \pgfusepathqfill
  \end{pgfscope}
  \begin{pgfscope}
    \definecolor{fc}{rgb}{0.0000,0.0000,0.0000}
    \pgfsetfillcolor{fc}
    \pgfusepathqfill
  \end{pgfscope}
  \begin{pgfscope}
    \definecolor{fc}{rgb}{0.0000,0.0000,0.0000}
    \pgfsetfillcolor{fc}
    \pgfusepathqfill
  \end{pgfscope}
\end{pgfpicture}

%        \caption{\label{fig:ca:sokoban-hypergraph}Sokoban state from Example~\ref{ex:ca:sokoban-hypergraph} represented as a hypergraph. Here, $s = \texttt{sokobanAt}$, $adj = \texttt{hAdj}$ and $at = \texttt{at}$}
%\end{figure}
%
%\begin{example}\label{ex:ca:sokoban-hypergraph}
%	Consider Example~\ref{ex:ca:sokoban-moveleft-action}. In this we had
%	\begin{align*}
%		\sigma(S_{before}, \delta) =
%			\left\{
%			\begin{gathered}
%				\texttt{adj-h}(x, y), \\
%				\texttt{adj-h}(y, z), \\
%				\texttt{at}(y), \\
%				\texttt{sokobanAt}(z)
%			\end{gathered}
%			\right\}
%	\end{align*}
%	When represented as a hypergraph (see \figref{fig:ca:sokoban-hypergraph}),
%	we see that each binding is clearly visualized and that two of the same atom (\texttt{adj-h}) can be in the same graph.
%	Also by modelling this as a graph the idea of disproving bindings also
%	becomes explicit as it literally means to remove a vertex from a binding edge.
%	Furthermore, in this model the notion of disjoint atoms (Restriction~\ref{rst:ca:no-disjuntive-conditionals}) is shown by the vertices not being connected through any type of edge.
%\end{example}


\end{document}
