\documentclass[../Master.tex]{subfiles}
\begin{document}


Conditional actions are actions with effects that are applied based on the state it is applied, that means individual effects can have its own set of preconditions.
For instance an effect in an action schema can have the following form:
\begin{align*}
Effect&:\\
&\forall x, y, z, \dots \left[ q(x, y), \dots \right] \quad when \quad 
\left[ g(z, x), \dots  \right] \\
& \land \\
& \forall x, y, z, \dots \left[ p(x, y), \dots \right] \quad when \quad 
\left[ f(z, x), \dots  \right] \\
&\land \\ 
&\dots
\end{align*}
With this addition to the action schema, we are able to express condition actions (see \Cref{ex:ca:sokoban-moveleft-action} for an applied example). 
It is learning of these types of actions that we will explore in this section. 
We will begin by show casing the problems with use of models from non-conditional actions, and we will show necessary changes to the models, to achieve a solution.


\end{document}
