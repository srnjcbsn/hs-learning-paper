\documentclass[../Master.tex]{subfiles}
\begin{document}

We will now turn our attention to the task of learning action schemas with conditional effects. While we will briefly introduce the core concepts here, a more detailed explanation can be found in~\oldcref{sec:app:pddl}. 

What separates conditional action schemas from their non-conditional counterparts is that instead of relying solely on the arguments the action was invoked with, each effect is determined based on the state the action was applied to. That is, an effect can contain \emph{conditionals} of the form:
\begin{equation*}
    \cond = \forall \underbrace{x, y, z, \dots}_{\varC \left(\cond\right)}\in \objs : 
    \underbrace{\left[ q(x, y), \dots \right]}_{\textit{effects}} \quad when \quad 
    \underbrace{\left[ g(z, x), \dots  \right]}_{\textit{preconditions}}
\end{equation*}
For any interpretation of the variables $\varC\left(\cond\right)$, the \textit{effects} are only applied if the \textit{preconditions} are satisfied. Note that --- in this sense --- a conditional is similar to a non-conditional action with as many parameters as there are objects in the world. In the following sections, we assume that each conditionals only have a single effect, and that this effect is unique from effects of other conditionals in the same action schema. This allows us to discern between different conditionals, and learn the preconditions of each separately.

See \Cref{ex:ca:sokoban-moveleft-action} for an example of an action with conditional effects. 

In the following sections, we will discuss what problems arise when learning conditional action schemas, analyse the applicability of the methods explored in Chapter~\ref{chp:nca}, and finally discuss new models that are better suited.

\end{document}
