\documentclass[../Master.tex]{subfiles}
\providecommand{\master}{..}
\begin{document}

Whenever a single grounded predicate $q$ is observed to occur in $\Delta s$ as a consequence of a state transition $\left(s, a, s'\right)$, 

We will now explore how knowledge about conditional effects for an action schema $A$ can be derived from a  state transition $\left( s, a ,s' \right)$. As explained previously, each grounded predicate $q$ in $\Delta s$ is the outcome of a conditional effect for which an interpretation of its preconditions was satisfied by $s$. By Restriction~\ref{rst:ca:no-multiple-effect}, each conditional effect can have only one effect, and that effect must be unique. 

\noindent\rule{\textwidth}{1pt}

By Restrictions~\ref{rst:ca:no-disjuntive-conditionals} and~\ref{rst:ca:no-multiple-effect}, each predicate $p \in \Delta s$ is spawned by exactly one conditional. As such, it can be deduced that if a predicate $p\left(u_1, \dots, u_{|p|}\right)$ (where $\left\{u_1, \dots, u_{|p|}\right\} \subseteq \mathcal{O}$) occurs in $\Delta s$ after application of $a$, then $A$ contains a conditional of the form

\begin{equation*}
    \forall x_1, \dots, x_n 
        \left[ p\left(\delta\left(u_1\right), \dots, \delta \left(u_{|p|}\right) \right) \right] \quad when \quad 
        \left[ \left<unknown\right> \right]
\end{equation*}
where $\left\{\delta \left( u_1 \right), \dots, \delta \left(u_{|p|} \right) \right\} \subseteq \left\{ x_1, \dots, x_n \right\} \subseteq \delta [\mathcal{O}]$ and $\delta$ is a substitution function as explained in <ref>. Observe that once this conditional has been proven to exist, the problem of finding its preconditions is similar to finding the preconditions of a non-conditional action schema $A'$ accepting as many parameters as the number of objects in the world, ie. 

\begin{align*}
    \begin{split}
        \textsc{Action} &\; \textit{A'}\left(\delta \left( o_1 \right), \ldots, \delta \left( o_{|\mathcal{O}|}\right)\right): \\
        \textsc{Pre}: \; & \left< unknown \right> \\
        \textsc{Eff}: \; & p\left( \delta \left(u_1\right), \dots, \delta \left( u_{|p|} \right)\right)
    \end{split}
\end{align*}

in which case the grounding function becomes $\delta^{-1}$. Thus, the function $d(s)$ from Section~\ref{sec:NC:Preconditions} can be used to disprove preconditions, and the set of predicates neither proven nor disproven to be preconditions becomes $\mathbb{CP} \setminus d(s)$. Note that since $\mathbb{P}_{A'} = \mathbb{CP}_A$, $|d(s)| = |\mathbb{P}_{A'}|$, as each predicate is either present or absent from the state. 

As all predicates in $\Delta s$ with the same name are effects of the same conditional, it is tempting to consider them different applications of $A'$ and prove knowledge about their preconditions as in Section~\ref{sec:NC:Preconditions}. However, the above only holds for a single grounded predicate observed to be the effect of the conditional; if a predicate with the same name but different arguments occurs in $\Delta s$, then it is the effect of another interpretation of the universal quantifier in the conditional, and the nonconditional action schema constructed for it will not equal $A'$.

Consequently, the preconditions disproven by $d(s)$ for $A'$ are only applicable to the grounded predicate $A'$ was constructed for. To find the preconditions of the conditional, a method of combining the different disproven predicates is needed.

\noindent\rule{\textwidth}{1pt}



% \begin{algorithm}
%     \caption{Constructing hypergraphs from a state}\label{algo:HypGraphCtor}
%     \begin{algorithmic}
%         \Function{$\textsc{ConstructHyperGraph}$} {$s$, $s'$, \texttt{HyperGraph} $U$, $K$}
%             \ForAll{grounded predicates $q \left( o_1, \dots, o_{|q|} \right) \in s'$}
%                 \ForAll{object $o \in o_1,\dots,o_{|q|}$}
%                     \State{$v \gets Vertex 
%                 \EndFor%
%             \EndFor%
%         \EndFunction%
%     \end{algorithmic}
% \end{algorithm}

\begin{algorithm}
    \caption{Algorithm for learning conditional effects using hypergraphs}\label{algo:CondEffLearn}
    \begin{algorithmic}
        \Function{$\textsc{Conditional-Effect-Learn}$} {$\left( s, a, s'\right)$, \texttt{HyperGraph} $U$, $K$}
            \ForAll{predicates $p \in \Delta s$}
                    
            \EndFor
        \EndFunction%
    \end{algorithmic}
\end{algorithm}

\textsc{\texttt{Work in progress}}

<TODO:\@ Include explanation of how proven effects are combined with the hypergraphs of the preconditions>

<TODO:\@ Explain how there are many hypergraphs in one state, I.E. each effect contains its own hypergraph of preconditions>

Given a state transition $(s,a,s')$:

(TO Thomas this is a very condensed form of what will be written)
find all predicates in $\Delta s = s \setminus s'$. For each effect predicate, find predicates that are connected to it, and build a hypergraph from this information (after applying a substitution to reduce it to variable form). 

For each effect with the same predicate form (ie. $p(x,y)$ has the same form as $p(y,x)$, but not the same form as $p(x,x)$ or $q(x,y)$), merge their hyper graphs as described in the next section. 

Then merge the resulting hypergraphs with prior knowledge (which is also a hypergraph).

When a conditional action fails:

As for non-conditional actions, we will introduce the notion of a candidate set for proving preconditions. In contrast to non-conditional actions, the preconditions we wish to prove are bindings between variables. 

Whenever a conditional effect fails, the 

\end{document}
