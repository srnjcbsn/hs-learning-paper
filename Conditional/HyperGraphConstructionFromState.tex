\documentclass[../Master.tex]{subfiles}
\providecommand{\master}{..}
\begin{document}

\textsc{\texttt{Work in progress}}

<TODO:\@ Include explanation of how proven effects are combined with the hypergraphs of the preconditions>

<TODO:\@ Explain how there are many hypergraphs in one state, I.E. each effect contains its own hypergraph of preconditions>

Given a state transition $(s,a,s')$:

(TO Thomas this is a very condensed form of what will be written)
find all predicates in $\Delta s = s \setminus s'$. For each effect predicate, find predicates that are connected to it, and build a hypergraph from this information (after applying a substitution to reduce it to variable form). 

For each effect with the same predicate form (ie. $p(x,y)$ has the same form as $p(y,x)$, but not the same form as $p(x,x)$ or $q(x,y)$), merge their hyper graphs as described in the next section. 

Then merge the resulting hypergraphs with prior knowledge (which is also a hypergraph).

When a conditional action fails:

As for non-conditional actions, we will introduce the notion of a candidate set for proving preconditions. In contrast to non-conditional actions, the preconditions we wish to prove are bindings between variables. 

Whenever a conditional effect fails, the 

\end{document}
