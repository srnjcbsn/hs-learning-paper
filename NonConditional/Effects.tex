\documentclass[../Master.tex]{subfiles}
\begin{document}


In~\cite{Walsh2008}, they presented algorithm on how to learn effects of action schemes limited to non-conditional effects. We will expand on that and generalize some of their concepts.

If ignore the concept of planning with constants in effects, then we know that all fluent literals in our effects must have variables which are permutations of the action's variables. The set of all those fluent literals we define as $\mathbb{F}$, another way understand $\mathbb{F}$ is that it is all the possible fluent literals which an action schema can have for its effects.

\begin{definition} 
$\mathbb{F}_A$ is the set of all fluent literals available to action schema $A$. For ease of use $A$ can be omitted if it is implied through context.
	\begin{equation*}
		\mathbb{F}_A = \left\{ 
				\begin{gathered}
					p(x_1,\dots,x_n), \\
					\neg p(x_1,\dots,x_n)
				\end{gathered}
					\left|
				\begin{gathered} p \in \mathbb{P}~\land \\
								n = arity(p)~\land  \\
								(x_1,\dots,x_n) = permutation(Vars(A))
				\end{gathered}				
							\right.\right\}
	\end{equation*}
\end{definition}


\begin{figure}
	\def\firstcircle{(0,0) circle (1.5cm)}
	\def\secondcircle{(0:2cm) circle (1.5cm)}
	\def\thirdcircle{(0:3.8cm) circle (2.9cm)}
	\centering
% Now we can draw the sets:
\begin{tikzpicture}


\draw \firstcircle node[above] {$S$};
\draw \secondcircle node [above] {$S'$};

% Now we want to highlight the intersection of the first and the
% second circle:

\begin{scope}
\clip \firstcircle;
\fill[red] \secondcircle;
\end{scope}

% Next, we want the highlight the intersection of all three circles:

\begin{scope}
\clip \firstcircle;
\clip \secondcircle;
\fill[green] \thirdcircle;
\end{scope}


\end{tikzpicture}
\caption{\label{fig:nca:venn-of-effects} Venn diagram of a state transition. The red area shows predicates which was unaffected, and the Green show effects that was added to S but masked as unaffected because they were already present in S.}

\end{figure}

For any action schema schema it is common that only a small subset of $\mathbb{F}$ are actually present in its effects, 
therefore we differentiate between them. 
We define whether we have proved or disproved them. 
Initially when a fluent has yet to proved or disproved we say that it is unproven. 
It then follows that if no fluents are considered unproven then we have complete knowledge for the action schemas effects.


\begin{definition} 
	The proved set $K$ contains which fluents $\mathbb{F}$ we know are in the action schemas actual effects.
	Conversely the disproved set $D$ define which fluents are for certain not present.
	Those fluents that are neither in $K$ nor in $D$ are said to be in the unproven set $U$.
	
\end{definition}

For $K$, $D$ and $U$ two important invariant holds.

\begin{invariant}[Mutual exclusivity]
	\begin{equation*}
		\begin{split}
		K \cap D & = \emptyset  \\ 
		K \cap U &= \emptyset \\
		D \cap U &= \emptyset
		\end{split}
	\end{equation*}
\end{invariant}

\begin{invariant}[Completeness]
	\begin{equation*}
		U \cup K \cup D = \mathbb{F}
	\end{equation*}
\end{invariant}

In order to prove or disprove the effects through a transition $(S,a,S')$ we must understand what occurs in a state transition.
In a state transition effects are added directly to $S$ to form $S'$, however for predicates in $S \cap S'$ this means there are two distinct possibilities either they were part of the effects of the action or they were not (see Figure \ref{fig:nca:venn-of-effects}).


\begin{theorem}
	Positive fluents that when grounded are not in $S'$ are considered disproved. Conversely negative fluents that when grounded are not in $S$ are also considered disproved. This gives us:
	
	\begin{equation*}
		D = \left\{
			f \; | \; f \in \mathbb{F} \land g(f) \notin s'
			\right\} 
			\cup
			 \left\{
			\neg f \; | \; \neg f \in \mathbb{F} \land g(f) \notin s
			\right\}
	\end{equation*}
	\begin{proof} If a positive grounded  fluent   \qedhere
	\end{proof}
\end{theorem}

 

\begin{equation}
    u^+ = \left\{
        p \; | \; p \in \mathbb{P} \land g(p) \in s \cap s'
    \right\}
\end{equation}

\begin{equation}
    u^- = \left\{
        p \; | \; p \in \mathbb{P} \land g(p) \notin s \cup s'
    \right\}
\end{equation}

\begin{equation}
    d^+ = \left\{
        p \; | \; p \in \mathbb{P} \land g(p) \notin s'
    \right\}
\end{equation}

\begin{equation}
    d^- = \left\{
        p \; | \; p \in \mathbb{P} \land g(p) \in s'
    \right\}
\end{equation}

\begin{equation}
    k^+ = \left\{
        p \; | \; q \in a^+ \land ug(q) = \{ p \}
    \right\}
\end{equation}

\begin{equation}
    k^- = \left\{
        p \; | \; q \in a^- \land ug(q) = \{ p \}
    \right\}
\end{equation}

    Consider a state transition $\left(s,a,s'\right)$ where $s$ is a
state consisting of grounded predicates, $a$ is a $k$-ary action
application with arguments $a_{1},\dots,a_{k}$, and $s'$ is the
state resulting from the application of $a$ in $s$.

Trivially, $a^{+}=s'\setminus s$ resp. $a^{-}=s\setminus s'$ are
the grounded predicates added to resp. removed from the world by application
of $a$. Consequentially, for each predicate $p\in a^{+}$ there is
a predicate $q\in E^{+}$ such that $g\left(q\right)=p$ (dually for
$a^{-}$ and $E^{-}$). However, since the grounding function $g$
is non-injective, several different fluent predicates may be grounded
to the same $p$, hence the ungrounding function $u$ yields a set
of predicates:
\[
ug:p\mapsto\left\{ q:g\left(q\right)=p\right\}
\]


For a predicate $p\in s^{+}$, the set $u\left(p\right)$ is contains
at least one element that is a member of $E^{+}$. If $u\left(p\right)$
is the singleton set, its one element $q$ can unambiguously be determined
to be a member of $E^{+}$. A similar argument can be made for $a^{-}$
and $E^{-}$. Let $uamb$ denote a function which, given a set of
ungrounded predicate sets, yields the union of the singleton sets,
formally:

\[
uamb:X\mapsto\bigcup\left\{ x:x\in X\land\left|x\right|=1\right\}
\]


Then, the sets of fluent predicates $uamb\left[ug\left[a^{+}\right]\right]$
and $uamb\left[u\left[a^{-}\right]\right]$ are proven to be in $E^{+}$
and $E^{-}$, respectively.

The approach described above does not generally enumerate the effects
of $A$, for two reasons:
\begin{description}
\item [{Hidden\ effects:}] As $s'$ contains the subset of $s$ not explicitly
removed by application of $a$, it is unknown whether any predicate
$p\in s\cap s'$ was produced by $a$ or inherited from $s$. Similarly,
it is unknon whether a predicate $q\notin s\cup s'$ would be deleted
by $a$ if it had existed in $s$.
\item [{Ambiguous\ predicates:}] If for any two arguments $\psi_{i}$
and $\psi_{j}$ of $a$ it holds that $\psi_{i}=\psi_{j}$, then the
predicate $p\left(\psi_{i}\right)$ could be produced by grounding
of either $p\left(\pi_{i}\right)$ or $p\left(\pi_{j}\right)$. Hence,
it is unknown which predicate in the set $ug\left[p\left(\psi\right)\right]$
is in $A$'s add or delete list, unless it is the singleton set.
\end{description}
Consequentially, an ungrounded predicate $p$ can only be established
as being part of $A$'s (positive or negative) effects if $\exists q\in a^{+}\cup a^{-}\left[g\left(q\right)=\left\{ p\right\} \right]$.
To lift this restriction, we propose a scheme for transition learning
taking into account prior knowledge, based on {[}Walsh{]}:

Consider a state transition $\left(s,a,s'\right)$ as well as a set
of grounded predicates $K^{+}$ that have previously been proven to
exist in $E^{+}$ and a set of ungrounded predicates $U^{+}$ that
have neither been proved nor disproved to be in $E^{+}$. Note that
$K^{+}\cap U^{+}=\emptyset$ , and that if a predicate $p$ is not
in either set, it must have been proven to be absent from $E^{+}$.
The possible hidden effects $s\cap s'$ can now be reduced to the
set
\[
u^{+}=s\cap s'\cap g\left[U^{+}\cup K^{+}\right]
\]
where $g\left(U^{+}\cup K^{+}\right)$ is the the set of all positive
grounded predicates that have not been disproven to exist in $E^{+}$,
ie. $g\left(E^{+}\right)\subseteq g\left(U^{+}\cup K^{+}\right)$.
When ungrounding the positive effects $a^{+}$, each of the resulting
sets are intersected with $U^{+}\cup K^{+}$, effectively removing
from the sets predicates which have previously been proven to be absent
from $E^{+}$.
\[
\alpha=\left\{ x\cap\left(U^{+}\cup K^{+}\right):x\in ug\left[a^{+}\right]\right\}
\]


The sets in $\alpha$ that are unambiguous can now be added to the
set of known positive predicates of $E$, while the set of unkowns
is redefined as the union of all ungroundings of $u^{+}$ that were
not previously proven included in or absent from $E^{+}$, as well
as all ambiguous predicates in $\alpha$.
\[
K^{+}=K^{+}\cup uamb\left[\alpha\right]
\]


\[
U^{+}=\left(\bigcup ug\left[u^{+}\right]\cap U^{+}\right)\cup amb\left[\alpha\right]
\]


Similarly to the positive effects described above, the sets $K^{-}$
and $U^{-}$ is maintained for the negative effects. Effect candidtes
are the sets resulting from ungrounding $a^{-}$, each intersected
with $U^{-}\cap K^{-}$ to remove predicates proven to be false:
\[
\beta=\left\{ x\cap\left(U^{-}\cup K^{-}\right):x\in ug\left[a^{-}\right]\right\}
\]


As above, the unambiguous predicates in $\beta$ have now been proven
to be false, and can be added to $K^{-}$.

\[
K^{-}=K^{-}\cup uamb\left[\beta\right]
\]


As we do not allow spurious effects, all ungrounded predicates that
would produce a predicate $p\in s'$ when grounded are considered
proven to be absent from $E^{-}$. This is exactly the set $\bigcup ug\left[s'\right]$.
Additionally, the ambiguous negative effects in $\beta$ are added
to the set $U^{-}$:

\[
U^{-}=\left(U^{-}\setminus\bigcup ug\left[s'\right]\right)\cup amb\left[\beta\right]
\]
where the ambiguous actions are found as follows:
\[
amb:X\mapsto\bigcup\left\{ x:x\in X\land|x|>1\right\}
\]
\end{document}
