\providecommand{\master}{..}
\documentclass[../Master.tex]{subfiles}

\begin{document}



\cite{Walsh2008} presents an algorithm for learning effects of action schemas limited to non-conditional effects. In this section we will expand on that and generalize some of their concepts.
For any action schema it is common that only a small subset of $\lits$ are actually present in its effects, 
therefore we differentiate between them by defining whether we have proved or disproved them. 
Initially when a literal has yet to proved or disproved, we say that it is unproven. 
It then follows that if no literals are considered unproven then we have complete knowledge for the action schema's effects.


\begin{definition} 
	For an action schema $A$, we define the following sets of literals:
	\begin{itemize}
		\item The set $\Pro$ contains the literals  $l \in \lits$ proven to be effects $A$.
		\item Conversely, $\Dsp$ is the set of literals disproven to be effects of $A$.
		\item Those literals that are neither in $\Pro$ nor in $\Dsp$ are said to be in the unproven set $\Up$.
	\end{itemize}
	For instance, if an action's effect can have the literals $\lits = \{p(x), \neg p(x)\}$, 
	then initially $\Up = \lits$, $\Pro = \emptyset$ and $\Dsp = \emptyset$. 
	If we then learn that $p(x)$ is an effect and $\neg p(x)$ is not, 
	our sets should by this definition be: $\Up = \emptyset$, $\Pro = \{p(x)\}$ and $\Dsp = \{\neg p(x)\}$. 
	
\end{definition}

For $\Pro$, $\Dsp$ and $\Up$ two important invariant holds.

\begin{invariant}[Mutual exclusivity]\label{inv:nca:mutual-ex}
	A literal $l \in \lits$ is either proven or disproven to be a precondition, or it is neither, i.e.\
	\begin{equation*}
	\begin{split}
	\Pro \cap \Dsp & = \emptyset  \\ 
	\Pro \cap \Up &= \emptyset \\
	\Dsp \cap \Up &= \emptyset
	\end{split}
	\end{equation*}
\end{invariant}

\begin{invariant}[Completeness]\label{inv:nca:completeness}
	The three sets completely describe $\lits$, i.e.\ 
	\begin{equation*}
	\Up \cup \Pro \cup \Dsp = \lits
	\end{equation*}
\end{invariant}



\begin{figure}
	\def\firstcircle{(0,0) circle (1.5cm)}
	\def\secondcircle{(0:2cm) circle (1.5cm)}
	\def\thirdcircle{(0:3.8cm) circle (2.9cm)}
	\centering
    % Now we can draw the sets:
    \begin{tikzpicture}
        \draw \firstcircle node[above] {$S$};
        \draw \secondcircle node [above] {$S'$};

        % Now we want to highlight the intersection of the first and the
        % second circle:

        \begin{scope}
        \clip \firstcircle;
        \fill[red] \secondcircle;
        \end{scope}

        % Next, we want the highlight the intersection of all three circles:

        \begin{scope}
        \clip \firstcircle;
        \clip \secondcircle;
        \fill[green] \thirdcircle;
        \end{scope}
    \end{tikzpicture}
    \caption{\label{fig:nca:venn-of-effects} Venn diagram of a state transition. The red area shows unaffected atoms, and the green show effects that was added to $S$ but masked as unaffected because they were already present in $S$.}
\end{figure}


\subsection{Disproving effects}
In order to prove or disprove the effects through a transition $(S,a,S')$ we must understand what occurs in a state transition.

When an action is successfully applied in a state $S$, the literals comprising its positive effects are grounded and added to $S$ to form $S'$. Similarly, its negative effects are grounded and removed from $S$. Therefore, we should be able to observe the effects of the action as the differences between $S$ to $S'$ i.e \geffects. For the special case that an atom is both in $S$ and $S'$ there are two distinct possibilities: either it is part of the effects of the action but was already part of $S$ or it was not an effect (see \Cref{fig:nca:venn-of-effects}). The same holds for atoms that are in neither $S$ nor $S'$.

To disprove atoms for a single transition we use the idea that grounded atoms which are not in $S'$ cannot be positive effects, and grounded atoms which are in $S'$ cannot be negative effects. 
This is logically sound and gives us the following theorem:

\begin{theorem}[Disproving effects]\label{thm:nca:disprove-effects}
	 Let $\left(S, a, S'\right)$ be a state transition. Then any grounded literal $l$ is disproven to be a positive effect if its grounded version is not in $S'$. Conversely, a negative literal is disproven to be an effect if its grounded version is in $S'$. This gives us the following function to get all disproved effects for a state transition result $S'$:	
	\begin{equation*}
		 \dsp(S') = \left\{
			p \mid p \in \lits \land \ground (p) \notin S'
			\right\} 
			\cup
			 \left\{
			\neg p \mid \neg p \in \lits \land \ground(p) \in S'
			\right\}
	\end{equation*}
\end{theorem}
	\begin{proof}[Proof by definition]
		Recall that $S'$ is the result of a union which include the grounded atom of an action $A$'s effects. Hence, it follows trivially that any atom $\ground(p)\footnote{Ground takes takes a literal and outputs its grounded literal counterpart, see \Cref{def:pddl:ground-func}.} \notin S'$ can not have been an effect of action $A$ or it would by definition be in $S$.
	Likewise if a potential negative effect $\neg p$ is $\ground(p) \in S'$ then that could not have been a negative effect.
	\end{proof}
	
	To summarize this theorem think of it as: 
	\begin{enumerate}
		\item Effects present in the state could by definition could not have been negative effects.
		\item  Effects not present in the state could not have been positive effects (As they would then be in the state).
	\end{enumerate}
	
	
	
	
\begin{example}[disproving effects in \texttt{move-h}] \label{ex:nca:moveSucceeded-disproving-effects}
	In case the sokoban agent applies the action $\texttt{move-h}(t_1, t_2)$ in state $S_0$, the resulting state $S_1$ will have the visible effect that the sokoban relocated from $t_1$ to $t_2$. 
	For the definition of action $\texttt{move-h}$ see \Cref{sec:SokobanPDDL}.
	
	\begin{equation*}
		\begin{split}
		S_1 &=
		\left\{
		\begin{gathered}
			\texttt{sokobanAt}(t_2), \\
			\texttt{clear}(t_1), \\
			\texttt{hAdj}(t_1, t_2)			
		\end{gathered}
		\right\} \\				
		\lits_{move-h} &= \left\{
		\begin{gathered}
			\texttt{sokobanAt}(to), \texttt{clear}(from), \\
			\texttt{vAdj}(from, to), \texttt{vAdj}(to, from), \\
			\texttt{vAdj}(from, from), \texttt{vAdj}(to, to), \\
			\texttt{hAdj}(from, from), \texttt{hAdj}(to, to), \\
			\texttt{at}(from, to), \texttt{at}(to, from), \\
			\texttt{goal}(from), \texttt{goal}(to)  \\
			\dots
		\end{gathered}
		\right\} \\
		\ground[\lits_{move-h}]&= \left\{
		\begin{gathered}
			\texttt{sokobanAt}(t_2), \texttt{clear}(t_1), \\
			\texttt{vAdj}(t_1, t_2), \texttt{vAdj}(t_2, t_1), \\
			\texttt{vAdj}(t_1, t_1), \texttt{vAdj}(t_2, t_2), \\
			\texttt{hAdj}(t_1, t_1), \texttt{hAdj}(t_2, t_2), \\
			\texttt{at}(t_1, t_2), \texttt{at}(t_2, t_1), \\
			\texttt{goal}(t_1), \texttt{goal}(t_2)  \\
			\dots
		\end{gathered}
		\right\} \\		
		\dsp(S_1) &= \left\{
		\begin{gathered}
			\neg \texttt{sokobanAt}(to), \\
			\neg \texttt{clear}(from), \\
			\neg \texttt{hAdj}(from, to), \\
			\texttt{vAdj}(from, to), \texttt{vAdj}(to, from), \\
			\texttt{vAdj}(from, from), \texttt{vAdj}(to, to), \\
			\texttt{hAdj}(to, to), \\
			\texttt{at}(from, to), \texttt{at}(to, from), \\
			\texttt{goal}(from), \texttt{goal}(to)  \\	
			\dots (\textsc{many disproven positive atoms})			
		\end{gathered}
		\right\}
		\end{split}
	\end{equation*}
We see that incorrect effects are correctly disproved, such as moving the sokoban back where it came from $\texttt{sokobanAt}(from)$.
\end{example}


\begin{figure}
	\centering
	\begin{subfigure}{.5\textwidth}
		\centering
\begin{tikzpicture}[ele/.style={fill=black,circle,minimum width=.8pt,inner sep=1pt},every fit/.style={ellipse,draw,inner sep=-2pt}]
\node[ele,label=below:{$p(x,y)$}] (a1) at (3,4) {};    
\node[ele,label=below:{$p(y,x)$}] (a2) at (3,3) {};    
\node[ele,label=below:{$p(x,x)$}] (a3) at (3,2) {};
\node[ele,label=below:{$p(y,y)$}] (a4) at (3,1) {};
\node[] (a5) at (3,0.5) {};

\node[ele,,label=below:{$p(O_1,O_2)$}] (b1) at (0,4) {};
\node[ele,,label=below:{$p(O_2,O_1)$}] (b2) at (0,3) {};
\node[ele,,label=below:{$p(O_1,O_1)$}] (b3) at (0,2) {};
\node[ele,,label=below:{$p(O_2,O_2)$}] (b4) at (0,1) {};
\node[] (b5) at (0,0.5) {};

\node[draw,fit= (a1) (a2) (a3) (a4) (a5),minimum width=2cm] {} ;
\node[draw,fit= (b1) (b2) (b3) (b4) (b5),minimum width=2cm] {} ;  
\draw[<-,thick,shorten <=2pt,shorten >=2pt] (a1) -- (b1);
\draw[<-,thick,shorten <=2pt,shorten >=2] (a2) -- (b2);
\draw[<-,thick,shorten <=2pt,shorten >=2] (a3) -- (b3);
\draw[<-,thick,shorten <=2pt,shorten >=2] (a4) -- (b4);
\end{tikzpicture}
		\caption{Unique $A(O_1,O_2)$, \newline $\invground$ is single-valued}
	\end{subfigure}%
	\begin{subfigure}{.5\textwidth}
		\centering

\begin{tikzpicture}[ele/.style={fill=black,circle,minimum width=.8pt,inner sep=1pt},every fit/.style={ellipse,draw,inner sep=-2pt}]
\node[ele,label=below:{$p(x,y)$}] (a1) at (3,4) {};    
\node[ele,label=below:{$p(y,x)$}] (a2) at (3,3) {};    
\node[ele,label=below:{$p(x,x)$}] (a3) at (3,2) {};
\node[ele,label=below:{$p(y,y)$}] (a4) at (3,1) {};
\node[] (a5) at (3,0.5) {};

\node[] (b2) at (0,4) {};
\node[ele,,label=below:{$p(O_1,O_1)$}] (b1) at (0,2.5) {};
\node[] (b5) at (0,0.5) {};

\node[draw,fit= (a1) (a2) (a3) (a4) (a5),minimum width=2cm] {} ;
\node[draw,fit= (b1) (b2) (b5),minimum width=2cm] {} ;  
\draw[<-,thick,shorten <=2pt,shorten >=2pt] (a1) -- (b1);
\draw[<-,thick,shorten <=2pt,shorten >=2] (a2) -- (b1);
\draw[<-,thick,shorten <=2pt,shorten >=2] (a3) -- (b1);
\draw[<-,thick,shorten <=2pt,shorten >=2] (a4) -- (b1);
\end{tikzpicture}
		
		\caption{Identical $A(O_1,O_1)$, \newline 
			$\invground$ is multivalued.}
	\end{subfigure}
	\caption{Shows how $\invground$ maps to the domain $\lits$ from the grounded atoms, for unique and identical action arguments. }
	\label{fig:nca:ground-injectivity}
\end{figure}

\subsection{Proving effects}
In order to understand proving effects, we must understand how effects are generated based on the action schema.
The $\ground$ function is used to ground atoms from an action's effect. This is important since if we know what atom a grounded atom originated from, then that is a proof for that atom. This means that if a grounded atom $p \in \Delta S$ returns only one atom when $\invground$ is applied to it, i.e. $\invground(p(O_1,O_2)) = {p(x,y)}$ then that atom $p(x,y)$ has been proved. In other words if $\ground$ is injective then $\invground$ is single-valued and it can thus be applied to all grounded atoms in $\Delta S$ and that will prove all of their respective atoms.
Whether or not $\ground$ is injective depends on the arguments of the action (see \Cref{fig:nca:ground-injectivity}).
For instance, if for the action $A(x,y)$ an application of it is $A(O_1,O_2)$ then $\ground$ is injective, meaning that no two different atom can be outputted as the same grounded atom. However if the action is applied with $A(O_1,O_1)$, then it is obvious to see that \ground is non-injective as $\ground(p(x,x))=\ground(p(y,y))=p(O_1,O_1)$.
This gives rise to the following lemma:
\begin{lemma}
If the arguments given to an action $A$ are all unique, then $\ground$ is injective, thereby making $\invground$ single-valued.
\end{lemma}

\begin{proof}
    Consider an action schema $A\left(a_1, \dots, a_n\right)$ and two atoms 
    \begin{equation*}
        atom_1 = p\left(x_1, \dots, x_{|p|}\right)
    \end{equation*}
    and
    \begin{equation*}
       atom_2 = q\left(y_1, \dots, y_{|q|}\right)
    \end{equation*}
    where $\left\{ x_1, \dots, x_{|p|}\right\} \cup \left\{ y_1, \dots, y_{|q|}\right\} \subseteq \left\{a_1, \dots, a_n\right\}$. If $A$ is applied with unique objects $O_1, \dots, O_n$, 
    then $atom_1$ and $atom_2$ are grounded as $ground(atom_1, \sigma) = atom_1'$ and $ground(atom_2,\sigma) = atom_2'$, where $\sigma = \left[ a_1 \mapsto O_1, \dots, a_n \mapsto O_n \right]$.

    % Two atoms are ent when either their variables or predicates are different. Consider two ungrounded atoms $p\left(x_1, \dots, x_{|p|}\right)$ and $q\left(y_1, \dots, y_{|q|}\right)$ where all parameters of the two atoms  and their groundings $p' = ground\left(p \right)$ and $q' = ground\left( q \right)$. 
	\begin{itemize}
		\item If the two atoms have different predicates $p \neq q$ then (by definition) $atom_1' \neq atom_2'$
		
        \item If the two atoms have the same predicate $p = q$, then at least some of the parameters of $p$ and $q$ differ. I.e.\ there exists at least one $i$ such that $x_i \neq y_i$. Since all action arguments are different, $\sigma\left( x_i \right) \neq \sigma\left( y_i \right)$. Consequently, $atom_1' \neq atom_2'$.
		% \item If the $p_1(v_1,\dots,v_n)$ and  $p_2(v_1,\dots,v_n)$ variables are different
		% 	\begin{equation*}
		% 		v_{1,1} = v_{2,1} \land 
		% 		v_{1,2} = v_{2,2} \dots
		% 		v_{1,n-1} = v_{2,n-1} \land
		% 		v_{1,n} = v_{2,n}
		% 	\end{equation*} 		
		% and arguments to the action $A(arg_1,\dots,arg_n)$ are unique
		% 	\begin{equation*}
		% 	arg_i \neq arg_j ~ where~ i,j \in \{ 1 \dots n \}
		% 	\end{equation*} 
		%  
		% Then the result of grounding the atom will remain different even when each variable is mapped to an object, 
		% as all those objects are unique. 
		% Since $\ground$ is the function that performs mapping.
	\end{itemize}	
	Given that we only consider cases where action arguments are unique. 
	Then we have proved that $\ground$ can never output to the same grounded atom, for two different atoms. Therefore it must follow that $\ground$ is an injective function by definition when action arguments are unique.
\end{proof}

\textsc{An observation}, it follows from this proof that it is preferable for an agent to try actions, where the arguments to the action are all unique. This could be interesting when defining a heuristic for the agent's planner. 

While \invground is single-valued when the arguments to the action are all unique. 
There is no guarantee that the arguments to an action are always unique, but we should still be able to prove some of the literals.
We can achieve this by removing outputs from \invground that has previously been disproved.
As we maintain a disproved set \Dsp, built from earlier transitions, we can discard incorrect literals that we are guaranteed are not actual effects. 
This gives us the following theorem: 

\begin{theorem}[Proving effects]\label{thm:nca:prove-effects} Consider a state transition $(S,a,S')$ where
	\begin{equation*}
	\Delta S = \left(S' \setminus S\right) \cup \{\neg p \mid  p \in S \setminus S' \}
	\end{equation*}
	
	Then, for all $p \in \geffects$, the following holds:
	
    \begin{itemize}
        \item If \invground is single-valued, then $\invground(p)$ is an effect of $A$.
        \item If \invground is multivalued but all other literals outputted from $\invground(p)$ have been disproven to be an effect, 
        such that there is only a single literal remaining, then that literal is an effect.
    \end{itemize}

	% If $\invground$ is single-valued or it is multivalued but all except one output are disproven, then that the literal outputted by $\invground$ is proven to be an effect.
	We can write this as a function:
	
	\begin{equation*}
	\pro(\geffects,\Dsp) = 
	\left\{
		p' \mid 
				p \in \geffects \land 
				\invground(p) \setminus \Dsp = \{p'\}
		\right\}
	\end{equation*}
	where $\Dsp$ is the set of disproven literals that the agent already knows about.
\end{theorem}
	
\begin{proof}
    We prove the two cases described above:
    \begin{itemize}
        \item If $\invground$ is single-valued, then $\invground (p)$ is the only literal in $A$ that could have caused $p$ to occur in $S'$.
        \item If $\invground$ is multivalued, then it is generally not known which literal(s) caused $p$ to occur in $S'$. However, if all literals except one in $\invground\left( p \right)$ have previously been disproven to be effect, then that single literal must be an effect. 
    \end{itemize}
\end{proof}

To understand this theorem think of it as, atoms in the state that we know only one effect could produce/remove, is a proof for that effect.

By defining these theorems we can determine whether effects occurring in the state is proven to always occur, or at least 
know what effects will definitely not occur. Furthermore, as we remove all ambiguous effects, the agent should be able to make the best possible predictions based on the information it has gathered.

\begin{example}[Proving effects in \texttt{move-h}] \label{ex:nca:moveSucceeded-proving-effects}
	Consider the move action from \Cref{ex:nca:moveSucceeded-disproving-effects}.	
	
	\begin{align*}
		S_0 &=
		\left\{
		\begin{gathered}
			\texttt{sokobanAt}(t_1), \\
			\texttt{clear}(t_2), \\
			\texttt{hAdj}(t_1, t_2)			
		\end{gathered}
		\right\} \\	
		S_1 &=
		\left\{
		\begin{gathered}
			\texttt{sokobanAt}(t_2), \\
			\texttt{clear}(t_1), \\
			\texttt{hAdj}(t_1, t_2)			
		\end{gathered}
		\right\} \\				
	\Delta S &= \left\{
        \begin{gathered}
            \texttt{sokobanAt}(t_2), \\
            \neg  \texttt{sokobanAt}(t_1), \\
            \texttt{clear}(t_1), \\
            \neg \texttt{clear}(t_2)		
        \end{gathered}
        \right\}
	\end{align*}
And the known knowledge can be extracted as
    \begin{equation*}
        \pro(\Delta S) = \left\{
            \begin{gathered}
               \texttt{sokobanAt}(to), \\
               \neg  \texttt{sokobanAt}(from), \\
               \texttt{clear}(from), \\
               \neg \texttt{clear}(to)		
            \end{gathered}
        \right\}
    \end{equation*}
	We see that all effects are correctly proven. As the action used contain only unique arguments.
\end{example}



\end{document}
