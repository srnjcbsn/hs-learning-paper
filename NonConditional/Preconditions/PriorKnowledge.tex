\providecommand{\master}{../..}
\documentclass[\master/Master.tex]{subfiles}
\begin{document}
\begin{algorithm}
	\caption{Algorithm for learning non-conditional actions}\label{algo:precondLearn}
	\begin{algorithmic}
		\Function {$\textsc{NC-Action-learn}$} {$\left(S, a, S'\right)$, ($\Dsp_E$, $\Pro_E$), ($\Dsp_P$, $\Pro_P$, \cset)}
		\If {$S' = S$}
		\State $\cset \gets \cset \cup \{ \cand(S) \}$
		\ElsIf {$S' \neq  S$}
		\State Disprove grounded effects not present in $S'$:
		\State $\Dsp_E \gets \Dsp_E \cup d_E(S')$
		\State Prove unambiguous effects in $S'$:
		\State $\Pro_E \gets \Pro_E \cup d_E(\geffects,\Dsp_E)$
		\State Disprove grounded precondition present in $S$:
		\State $\Dsp_P \gets \Dsp_P \cup d_P(S)$
		\EndIf		
		
		\State Reduce candidate sets by removing literals disproven to preconditions:
		\State $\cset \gets \left\{ c \setminus \Dsp_P \; | \; c \in \cset \right\}$
		
		\State Add singleton sets from $\cset$ to $\Pro_P$:
		\State $\Pro_P \gets \Pro_P \cup \left\{ p \; | \; \{ p \} \in \cset \right\}$
		
		\State Remove candidate sets containing literals proven to be preconditions:
		\State $\cset \gets \left\{ c \; | \; c \in \cset \land c \cap \Pro_P \neq \emptyset \right\}$
		% \State Remove disproven literals from candidates
		% \Comment \emph{see \eqref{reduceCands}}
		% \State $K^\pm \gets \left\{ \textnormal{singleton sets from } C\right\}$ $\cup K^\pm$
		% \Comment \emph{see \eqref{eq:extractKnown}}
		% \State Remove candidate sets from $C$ which contain a literal in $K^\pm$
		% \Comment \emph{see \eqref{eq:removeKnown}}
		\State \Return $((\Dsp_E, \Pro_E), (\Dsp_P, \Pro_P, \cset))$
		\EndFunction%
	\end{algorithmic}
\end{algorithm}

We will now present an algorithm (see \Cref{algo:precondLearn}) for obtaining knowledge about preconditions / effects from a series of state transitions. 
The algorithm takes as input a state transition $(S, a, S')$ as well as prior knowledge about the action schema for $a$, and outputs an updated knowledge.
The prior knowledge is consisting of the following:

For effect knowledge:
\begin{itemize}
	\item A set $\Pro_E$ containing literals proven to be effects.
	\item A set $\Dsp_E$ containing literals \emph{disproven} to be effects.
\end{itemize}

For precondition knowledge:
\begin{itemize}
    \item A set $\Pro_P$ containing literals proven to be prconditions.
	\item A set $\Dsp_P$ containing literals \emph{disproven} to be preconditions.
    \item A set \cset containing candidate sets, each satisfying the invariant in~\eqref{eq:inv}.
\end{itemize}

\subsection*{On action success}
An action is said to be successful when $S \neq S'$. When this occurs the following knowledge acquired.

For preconditions the sets of literals that can be disproven to be preconditions based on this transition is calculated using~\eqref{eq:DPlusTrans} and~\eqref{eq:DMinusTrans}, and added to the existing knowledge:
\begin{equation*}
    \Dsp_P = \Dsp_P \cup \dsp_P(S)
\end{equation*}

For effects the sets of literals that can be disproven and proven to be an effect based on this transition is calculated using~\Cref{thm:nca:prove-effects} and \Cref{thm:nca:disprove-effects}, and added to the existing knowledge:
\begin{align*}
\Dsp_E &= \Dsp_E \cup \dsp_E(s') \\
\Pro_E &= \Pro_E \cup \pro_E(\geffects, \Dsp_E)
\end{align*}


\subsection*{On action failure}

When an action fails, i.e. if $S = S'$, the candidate sets are determined using~\eqref{eq:cPlus} and~\eqref{eq:cMinus}, and are added to the set $\cset$.

It now holds that either more literals have been disproven, or one more candidate set have been added to $\cset$. According to Theorem~\ref{th:invHolds}, each candidate set in $\cset$ can now be reduced by removing disproven literals, while maintaining Invariant~\eqref{eq:inv}:

\begin{equation} \label{reduceCands}
    \cset = \left\{ c \setminus \Dsp_P \; | \; c \in \Cand \right\}
\end{equation}

By Proposition~\ref{prop:ncp:precond-holds}, members of any singleton candidate set can now be added to the proven knowledge:

\begin{equation} \label{eq:extractKnown}
    \Pro_P = \Pro_P \cup \left\{ p \; | \; \{ p \} \in \cset \right\}
\end{equation}

As discussed previously, any candidate set containing a literal which have already been proven to be a precondition can be safely removed from $\cset$, as no more knowledge can be obtained from it, formally:


\begin{equation} \label{eq:removeKnown}
    \cset = \left\{ c \; | \; c \in \Cand \land c \cap \Pro_P \neq \emptyset \right\}
\end{equation}



\end{document}
