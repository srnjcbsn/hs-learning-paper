\documentclass[../Master.tex]{subfiles}
\begin{document}

\graphicspath{{.../Graphics/}}

\subsubsection*{Domain specification}
When designing a virtual or physical robot capable of solving a sokoban puzzle, it would be sufficient to equip it with a $move$ action which, given a direction or an adjacent tile, would relocate the sokoban appropriately, displacing any crates in the way. However, as STRIPS-style action schemas can only contain a single effect, it does not allow an action to, for example, push a crate \textit{if}  it is on the tile the sokoban is moving to. Instead, this requires different actions, which are outlined below:

The effect of a movement action from tile $t_1$ to tile $t_2$ is that the sokoban is no longer present at $t_1$, but appears on $t_2$. Since there are no notion of variables in STRIPS-style action specifications, both $t_1$ and $t_2$ are given as arguments to the action, so that it can be asserted whether the sokoban is actually located at $t_1$ and whether it is possible to relocate it to $t_2$. Since pushing a crate has a different effect, and must therefore be implemented as another action, the destination tile must be empty for the movement action to succeed. Since STRIPS-style action schemas do not allow existentially or universally quantified formulae, the $at$ predicate can not be used, and the $clear$ predicate is used instead.

As per the sokoban rules, movement from $t_1$ to $t_2$ is allowed only if the two tiles are either horizontally \textit{or} vertically adjacent. As preconditions can not contain disjunctions, and there can only be one precondition per action schema, two schemas are required for movement: One (\textit{move-h}) containing the precondition $hAdj(t_1,t_2)$ and the other (\textit{move-v}) the precondition $vAdj(t_1, t_2)$.
In \eqref{act:moveh} the \textit{move-h} action is listed. The \textit{move-v} action is similar, except for having $vAdj$ as a precondition instead of $hAdj$.

\begin{align}
\begin{split} \label{act:moveh}
    \textsc{Action} &\; \textit{move-h}(from, to): \\
    \textsc{Pre}: \; & sokobanAt(from) \land hAdj(from, to) \land clear(to) \\
    \textsc{Eff}: \; & \neg sokobanAt(from) \land sokobanAt(to) \land \\
                     & clear(from) \land clear(to)
\end{split}
\end{align}

The action schema for pushing a crate depends on four parameters: the crate to push, its location, the location, its destination, and the location of the sokoban. As for the movement actions, two different push actions are required; one for pushing horizontally (depicted in \eqref{act:pushh}) and one for pushing vertically. Its preconditions assert that the sokoban is horizontally adjacent to the crate, and that the crate is horizontally adjacent to the destination. It also ensures that the destination is an empty tile, which precludes it from being the tile inhabited by the sokoban, ensuring that the sokoban can not just swap places with the crate, which is not allowed.

As the effect of moving a crate to a goal tile is different from that of moving it onto a non-goal tile, the former requires its own \textit{push-h-goal}  and \textit{push-v-goal} actions (listed in \eqref{act:pushhgoal}). Hence, the destination tile is required to be a non-goal tile for the push action, and the effect is that the crate is not occupying a goal tile.

\begin{align}
\begin{split} \label{act:pushh}
    \textsc{Action} &\; \textit{push-h}(crate, sokoban, from, to): \\
    \textsc{Pre}: \; & sokobanAt(sokoban) \land
                       at(crate, from) \land
                       \neg goal(to) \land \\
                     & hAdj(sokoban, from) \land
                       hAdj(from, to) \land
                       clear(to)
                       \\
    \textsc{Eff}: \; & \neg sokobanAt(sokoban) \land
                       sokobanAt(from) \land
                       clear(sokoban) \land \\
                     & \neg at(crate, from) \land
                       at(crate, to) \land
                       \neg clear(to) \land
                       \neg atGoal(crate)
\end{split}
\end{align}

\begin{align}
\begin{split} \label{act:pushhgoal}
    \textsc{Action} &\; \textit{push-h-goal}(crate, sokoban, from, to): \\
    \textsc{Pre}: \; & sokobanAt(sokoban) \land
                       at(crate, from) \land
                       goal(to) \land \\
                     & hAdj(sokoban, from) \land
                       hAdj(from, to) \land
                       clear(to) \land
                       \\
    \textsc{Eff}: \; & \neg sokobanAt(sokoban) \land
                       sokobanAt(from) \land
                       clear(sokoban) \land \\
                     & \neg at(crate, from) \land
                       at(crate, to) \land
                       \neg clear(to) \land
                       atGoal(crate)
\end{split}
\end{align}

In total, six different action schemas are required for the sokoban domain, and more would be required if the domain was to be made reversible. <\texttt{TODO:} conclude something interesting?>

\end{document}
