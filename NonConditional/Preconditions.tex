\documentclass[../Master.tex]{subfiles}
\begin{document}

\begin{itemize}
    \item If the action
    application is unsuccessful, ie. if $s=s'$, then at least one of $A$'s
    preconditions were not satisfied by $s$. From this information, however, it
    is unclear which predicates should have been absent from or present in $s$
    for the application to be successful: any ungrounded variant of a predicate
    $p\notin s$ resp. $q\in s$ would cause the failure if part of $P^{+}$ resp.
    $P^{-}$.

    \item When $a$ was successfully applied, ie. when $s\neq s'$, all
    preconditions of $A$ were satisfied, such that $g\left[P^{+}\right]\subseteq
    s$ and $g\left[P^{-}\right]\cap s=\emptyset$. It follows logically that no
    ungrounded version of a predicate $p\in s$ resp. $q\notin s$ can be part of
    $P^{-}$ resp. $P^{+}$, as that would have caused the action to fail.
    However, it is not clear which predicates in $s$ and $s^{C}$ satisfied the
    preconditions, as some of them may have been redundant.
\end{itemize}
Thus,     only when action applications are successful can predicates be
disproven to     be preconditions, but no predicate can be proven to be a
precondition based     solely on a state transition.

Consider the addition of prior knowledge $\left(K^{+},K^{-},U^{+},U^{-}\right)$,
and assume that $s$ does not violate any precondition that is already known to
be true (ie. $g\left[K^{+}\right]\subseteq s$ and $g\left[K^{-}\right]\cap
s=\emptyset$). Then, in the case of action failure, any violated precondition
$p$ must exist in either $U^{+}$ or $U^{-}$. The ungrounded predicates that
would individually cause the failure if they were present in $P^{+}$ (resp.
$P^{-}$) are denoted $c^{+}$ (resp. $c^{-}$), and are determined as follows:

\[ c^{+}=U^{+}\setminus\bigcup ug\left[s\right] \] \[ c^{-}=U^{-}\cap\bigcup
ug\left[s\right] \]

For any \emph{candidate set} $c=\left(c^{+},c^{-}\right)$ constructed in this
manner, it holds that at least one of the candidates is a precondition of $A$,
ie. \begin{equation} c^{+}\cap P^{+}\neq\emptyset\lor c^{-}\cap
P^{-}\neq\emptyset\label{eq:cP} \end{equation}

\begin{cor} \label{candThe}Trivially, if (\ref{eq:cP}) holds and $c^{+}=\left\{
p\right\} \land c^{-}=\emptyset$, then $p\in P^{+}$. Conversely, if
$c^{+}=\emptyset\land c^{-}=\{q\}$, then $q\in P^{-}$. \end{cor} Notably, if
several members of $c^{+}\cup c^{-}$ are preconditions of $A$, it can never be
recuced to a singleton set, and no information can be gained from it.

If, in a later state transition for the same action schema $A$, a predicate $p$
is proven to be absent from $P^{\pm}$, then it is no longer a valid candidate,
and can be removed from $c^{\pm}$ while maintaining the validity of
(\ref{eq:cP}).

If, however, a predicate $q\in c^{+}\cup c^{-}$ is proven to be a precondition,
then removal of $q$ from $c$ might violate (\ref{eq:cP}). If $q$ is retained in
$c$, then one of the following cases apply: \begin{itemize} \item
$\left(c^{+}\cap P^{+}\right)\cup\left(c^{-}\cap P^{-}\right)=\{q\}$, ie. $q$ is
the only predicate in $c$ that is a precondition. Then $c$ could eventually be
reduced to the singleton set $\{q\}$, which does not produce any new knowledge,
since $q$ is already known to be a precondition. \item $\left|\left(c^{+}\cap
P^{+}\right)\cup\left(c^{-}\cap P^{-}\right)\right|>1$, ie. more than two
predicates in $c$ are preconditions of $A$. As mentioned above, $c$ can never be
reduced to the singleton set, and no knowledge can be gained from it.
\end{itemize} In both cases, retaining the set $c$ as it is will not provide any
new knowledge. Thus, a candidate set can be discarded when one of its predicates
are proven to be present in $P^{\pm}$.

We can now define a precondition hypothesis as follows: \begin{defn} A
precondition hypothesis $H_{P}$ for an action schema $A$ is a five-tuple
$\left(K^{+},K^{-},U^{+},U^{-},C\right)$, where $C$ is a set of candidate sets.
\end{defn} Whenever new knowledge about a predicate is obtained, $C$ is updated
using the following rules, ensuring that (\ref{eq:cP}) holds for all candidate
sets in $C$: \begin{enumerate} \item If a predicate $p$ is proven to be absent
from $P^{+}$, it is removed from all positive parts of candidate sets in $C$
(similarly for $P^{-}$ and negative parts): \[ C'=\left\{ \left(x^{+}\cap
U^{+},x^{-}\cap U^{-}\right):\left(x^{+},x^{-}\right)\in C\right\} \]

\item If a predicate $p$ is proven to be present in $P^{+}$ or $P^{-}$, any
candidate set containing $p$ is removed from $C$: \[ C^{\prime}=\left\{
\left(x^{+},x^{-}\right):\left(x^{+},x^{-}\right)\in C\land\left(x^{+}\cap
K^{+}\neq\emptyset\lor x^{-}\cap K^{-}\neq\emptyset\right)\right\} \]

\end{enumerate} Given a state transition $\left(s,a,s'\right)$ and a
precondition hypothesis $H_{P}$, we can now produce an updated hypothesis by the
following two cases: \begin{description} \item [{Action\ failure:}] The
candidate set $c$ is determined as described above. If $c$ is the singleton set,
the contained predicate is added to $K^{\pm}$ and removed from $U^{\pm}$, and
$C$ is updated using rules 1 and 2. Otherwise, $c$ is added to $C$. \item
[{Action\ success:}] Since all preconditions are satisfied, all ungrounded
versions of predicates in $s$ are disproven to be negative predicates, and all
ungrounded versions of predicates $q\notin s$ are disproven to be positive
predicates, formally: \[ U^{+\prime}=U^{+}\cap\bigcup u\left(s\right) \] \[
U^{-\prime}=U^{-}\setminus\bigcup u\left(s\right) \] Now, $C$ can be updated
using rule 1, and theorem \ref{candThe} can be used on each candidate set in $C$
to add singleton sets to $K^{+}$ and $K^{-}$: \[ K^{\pm\prime}=\left\{
x^{\pm}:\left(x^{+},x^{-}\right)\in
C^{\prime}\land\left|x^{\pm}\right|=1\land\left|x^{+}\right|+\left|x^{-}\right|=1\right\}
\cup K^{\pm} \] As this may prove the presence of predicates in $P^{\pm}$, $C$
must be updated using rule 2. \end{description}

\subsection{Undiscoverable preconditions}

To see this machinery in action, we once again apply it to the sokoban domain.

\[
U_0^+ = U_0^- =
\left\{
    \begin{gathered}
        sokobanAt(to), sokobanAt(from), \\
        clear(to), clear(from), \\
        goal(to), goal(from), \\
        vAdj(from), vAdj(to), \\
        hAdj(from), hAdj(to), \\
        at(from, to), at(to, from)
    \end{gathered}
\right\}
\]

% GXSO
%
% sokoban tries to move left (move-h(t3, t2)), but fails

\begin{align}
(c^+, c^-) =
\left(
    \left\{
        \begin{gathered}
            sokobanAt(to), \\
            clear(to), clear(from), \\
            goal(to), goal(from), \\
            vAdj(from), vAdj(to), \\
            at(from, to), at(to, from)
        \end{gathered}
    \right\}
    ,
    \left\{
        \begin{gathered}
            sokobanAt(from), \\
            hAdj(from), hAdj(to)
        \end{gathered}
    \right\}
\right)
\end{align}

% sokoban successfully executes action move-h(t3, t4)

\[
    U_2^+ = U_0^+ \cap
\]

\end{document}
