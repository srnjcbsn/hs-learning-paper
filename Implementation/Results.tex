\documentclass[../Master.tex]{subfiles}
\providecommand{\master}{..}
\begin{document}

When the scientific learning algorithm is applied through the main program, a statistics file is produced. This file contains, in order:
\begin{itemize}
    \item A header containing the name of each action schema in the domain, along with the action's $\mathbb{F}_A$.
    \item The string \verb|``-- RUNNING --''|, signifying the end of the header.
    \item For each experiment (plan) the agent conducted:
        \begin{itemize}
            \item The action which did not produce the expected outcome when applied (as explained in Section~\ref{sec:PDDLAlgo})
            \item The amount of problems the agent had solved before execution of this action.
            \item Eight integers, signifying how many predicates the agent has proven to be positive and negative effects and preconditions for the action, and how many are neither proven nor disproven to be positive and negative effects and preconditions for the action. Lastly, the number of candidate sets for the action is listed.
        \end{itemize}
\end{itemize}

In the following, we will show the results of applying the optimistic non-conditional learning algorithm to the sokoban domain described previously. In order to ease the agent's understanding of the  six actions (see Section~\ref{sec:SokobanPDDL}), we have provided two simple sokoban problems, depicted in Figures~\ref{fig:results:train1} and~\ref{fig:results:train2}. 

These problems serve as a training ground for the agent; in order to solve the first problem, the agent must utilize all three actions that operate on the horizontal axis, and to solve the second it must use the ones operating on the vertical axis. It then reuses the knowledge obtained from the training problems to solve the one in Figure~\ref{fig:results:train3}, which requires use of at least five of the six sokoban actions (it can be solved without use of $\texttt{move-h}$ or without use of $\texttt{move-v}$, but not without both).

From inspecting the generated statistics file, it can be seen that the agent conducts a total of 1793 experiments (i.e.\ failed plans) in order to solve the the large problem without any prior knowledge.

In contrast, solving the two training problems followed by the large one requires a total of 710 experiments (516 for the first problem, 151 for the second, and 43 for the third).

\begin{figure}
    \begin{subfigure}{0.3\textwidth}
        \resizebox{\linewidth}{!}{\begin{pgfpicture}
  \pgfpathrectangle{\pgfpointorigin}{\pgfqpoint{200.0000bp}{200.0000bp}}
  \pgfusepath{use as bounding box}
  \begin{pgfscope}
    \definecolor{fc}{rgb}{0.0000,0.0000,0.0000}
    \pgfsetfillcolor{fc}
    \pgfsetfillopacity{0.0000}
    \pgfsetlinewidth{2.0000bp}
    \definecolor{sc}{rgb}{0.0000,0.0000,0.0000}
    \pgfsetstrokecolor{sc}
    \pgfsetmiterjoin
    \pgfsetbuttcap
    \pgfpathqmoveto{192.0000bp}{100.0000bp}
    \pgfpathqcurveto{192.0000bp}{106.6274bp}{186.6274bp}{112.0000bp}{180.0000bp}{112.0000bp}
    \pgfpathqcurveto{173.3726bp}{112.0000bp}{168.0000bp}{106.6274bp}{168.0000bp}{100.0000bp}
    \pgfpathqcurveto{168.0000bp}{93.3726bp}{173.3726bp}{88.0000bp}{180.0000bp}{88.0000bp}
    \pgfpathqcurveto{186.6274bp}{88.0000bp}{192.0000bp}{93.3726bp}{192.0000bp}{100.0000bp}
    \pgfpathclose
    \pgfusepathqfillstroke
  \end{pgfscope}
  \begin{pgfscope}
    \definecolor{fc}{rgb}{0.0000,0.0000,0.0000}
    \pgfsetfillcolor{fc}
    \pgfsetfillopacity{0.0000}
    \pgfsetlinewidth{2.0000bp}
    \definecolor{sc}{rgb}{0.0000,0.0000,0.0000}
    \pgfsetstrokecolor{sc}
    \pgfsetmiterjoin
    \pgfsetbuttcap
    \pgfpathqmoveto{200.0000bp}{80.0000bp}
    \pgfpathqlineto{200.0000bp}{120.0000bp}
    \pgfpathqlineto{160.0000bp}{120.0000bp}
    \pgfpathqlineto{160.0000bp}{80.0000bp}
    \pgfpathqlineto{200.0000bp}{80.0000bp}
    \pgfpathclose
    \pgfusepathqfillstroke
  \end{pgfscope}
  \begin{pgfscope}
    \definecolor{fc}{rgb}{0.0000,0.0000,0.0000}
    \pgfsetfillcolor{fc}
    \pgfsetfillopacity{0.0000}
    \pgfsetlinewidth{2.0000bp}
    \definecolor{sc}{rgb}{0.0000,0.0000,0.0000}
    \pgfsetstrokecolor{sc}
    \pgfsetmiterjoin
    \pgfsetbuttcap
    \pgfpathqmoveto{160.0000bp}{80.0000bp}
    \pgfpathqlineto{160.0000bp}{120.0000bp}
    \pgfpathqlineto{120.0000bp}{120.0000bp}
    \pgfpathqlineto{120.0000bp}{80.0000bp}
    \pgfpathqlineto{160.0000bp}{80.0000bp}
    \pgfpathclose
    \pgfusepathqfillstroke
  \end{pgfscope}
  \begin{pgfscope}
    \definecolor{fc}{rgb}{0.0000,0.0000,0.0000}
    \pgfsetfillcolor{fc}
    \pgftransformshift{\pgfqpoint{100.0000bp}{100.0000bp}}
    \pgftransformscale{1.0000}
    \pgftext[base,left]{$c$}
  \end{pgfscope}
  \begin{pgfscope}
    \definecolor{fc}{rgb}{0.0000,0.0000,0.0000}
    \pgfsetfillcolor{fc}
    \pgfsetfillopacity{0.0000}
    \pgfsetlinewidth{2.0000bp}
    \definecolor{sc}{rgb}{0.0000,0.0000,0.0000}
    \pgfsetstrokecolor{sc}
    \pgfsetmiterjoin
    \pgfsetbuttcap
    \pgfpathqmoveto{112.0000bp}{88.0000bp}
    \pgfpathqlineto{112.0000bp}{112.0000bp}
    \pgfpathqlineto{88.0000bp}{112.0000bp}
    \pgfpathqlineto{88.0000bp}{88.0000bp}
    \pgfpathqlineto{112.0000bp}{88.0000bp}
    \pgfpathclose
    \pgfusepathqfillstroke
  \end{pgfscope}
  \begin{pgfscope}
    \definecolor{fc}{rgb}{0.0000,0.0000,0.0000}
    \pgfsetfillcolor{fc}
    \pgfsetfillopacity{0.0000}
    \pgfsetlinewidth{2.0000bp}
    \definecolor{sc}{rgb}{0.0000,0.0000,0.0000}
    \pgfsetstrokecolor{sc}
    \pgfsetmiterjoin
    \pgfsetbuttcap
    \pgfpathqmoveto{120.0000bp}{80.0000bp}
    \pgfpathqlineto{120.0000bp}{120.0000bp}
    \pgfpathqlineto{80.0000bp}{120.0000bp}
    \pgfpathqlineto{80.0000bp}{80.0000bp}
    \pgfpathqlineto{120.0000bp}{80.0000bp}
    \pgfpathclose
    \pgfusepathqfillstroke
  \end{pgfscope}
  \begin{pgfscope}
    \definecolor{fc}{rgb}{0.0000,0.0000,0.0000}
    \pgfsetfillcolor{fc}
    \pgfsetfillopacity{0.0000}
    \pgfsetlinewidth{2.0000bp}
    \definecolor{sc}{rgb}{0.0000,0.0000,0.0000}
    \pgfsetstrokecolor{sc}
    \pgfsetmiterjoin
    \pgfsetbuttcap
    \pgfpathqmoveto{80.0000bp}{80.0000bp}
    \pgfpathqlineto{80.0000bp}{120.0000bp}
    \pgfpathqlineto{40.0000bp}{120.0000bp}
    \pgfpathqlineto{40.0000bp}{80.0000bp}
    \pgfpathqlineto{80.0000bp}{80.0000bp}
    \pgfpathclose
    \pgfusepathqfillstroke
  \end{pgfscope}
  \begin{pgfscope}
    \definecolor{fc}{rgb}{0.0000,0.0000,0.0000}
    \pgfsetfillcolor{fc}
    \pgfsetfillopacity{0.0000}
    \pgfsetlinewidth{2.0000bp}
    \definecolor{sc}{rgb}{0.0000,0.0000,0.0000}
    \pgfsetstrokecolor{sc}
    \pgfsetmiterjoin
    \pgfsetbuttcap
    \pgfsetdash{{2.6833bp}{2.6833bp}}{0.0000bp}
    \pgfpathqmoveto{34.0000bp}{86.0000bp}
    \pgfpathqlineto{34.0000bp}{114.0000bp}
    \pgfpathqlineto{6.0000bp}{114.0000bp}
    \pgfpathqlineto{6.0000bp}{86.0000bp}
    \pgfpathqlineto{34.0000bp}{86.0000bp}
    \pgfpathclose
    \pgfusepathqfillstroke
  \end{pgfscope}
  \begin{pgfscope}
    \definecolor{fc}{rgb}{0.0000,0.0000,0.0000}
    \pgfsetfillcolor{fc}
    \pgfsetfillopacity{0.0000}
    \pgfsetlinewidth{2.0000bp}
    \definecolor{sc}{rgb}{0.0000,0.0000,0.0000}
    \pgfsetstrokecolor{sc}
    \pgfsetmiterjoin
    \pgfsetbuttcap
    \pgfpathqmoveto{40.0000bp}{80.0000bp}
    \pgfpathqlineto{40.0000bp}{120.0000bp}
    \pgfpathqlineto{-0.0000bp}{120.0000bp}
    \pgfpathqlineto{-0.0000bp}{80.0000bp}
    \pgfpathqlineto{40.0000bp}{80.0000bp}
    \pgfpathclose
    \pgfusepathqfillstroke
  \end{pgfscope}
\end{pgfpicture}
}
        \caption{Horizontal training problem for the sokoban agent.}\label{fig:results:train1}
    \end{subfigure}
    \begin{subfigure}{0.3\textwidth}
        \resizebox{\linewidth}{!}{\begin{pgfpicture}
  \pgfpathrectangle{\pgfpointorigin}{\pgfqpoint{200.0000bp}{200.0000bp}}
  \pgfusepath{use as bounding box}
  \begin{pgfscope}
    \definecolor{fc}{rgb}{0.0000,0.0000,0.0000}
    \pgfsetfillcolor{fc}
    \pgfsetfillopacity{0.0000}
    \pgfsetlinewidth{2.0000bp}
    \definecolor{sc}{rgb}{0.0000,0.0000,0.0000}
    \pgfsetstrokecolor{sc}
    \pgfsetmiterjoin
    \pgfsetbuttcap
    \pgfpathqmoveto{112.0000bp}{20.0000bp}
    \pgfpathqcurveto{112.0000bp}{26.6274bp}{106.6274bp}{32.0000bp}{100.0000bp}{32.0000bp}
    \pgfpathqcurveto{93.3726bp}{32.0000bp}{88.0000bp}{26.6274bp}{88.0000bp}{20.0000bp}
    \pgfpathqcurveto{88.0000bp}{13.3726bp}{93.3726bp}{8.0000bp}{100.0000bp}{8.0000bp}
    \pgfpathqcurveto{106.6274bp}{8.0000bp}{112.0000bp}{13.3726bp}{112.0000bp}{20.0000bp}
    \pgfpathclose
    \pgfusepathqfillstroke
  \end{pgfscope}
  \begin{pgfscope}
    \definecolor{fc}{rgb}{0.0000,0.0000,0.0000}
    \pgfsetfillcolor{fc}
    \pgfsetfillopacity{0.0000}
    \pgfsetlinewidth{2.0000bp}
    \definecolor{sc}{rgb}{0.0000,0.0000,0.0000}
    \pgfsetstrokecolor{sc}
    \pgfsetmiterjoin
    \pgfsetbuttcap
    \pgfpathqmoveto{120.0000bp}{0.0000bp}
    \pgfpathqlineto{120.0000bp}{40.0000bp}
    \pgfpathqlineto{80.0000bp}{40.0000bp}
    \pgfpathqlineto{80.0000bp}{0.0000bp}
    \pgfpathqlineto{120.0000bp}{0.0000bp}
    \pgfpathclose
    \pgfusepathqfillstroke
  \end{pgfscope}
  \begin{pgfscope}
    \definecolor{fc}{rgb}{0.0000,0.0000,0.0000}
    \pgfsetfillcolor{fc}
    \pgfsetfillopacity{0.0000}
    \pgfsetlinewidth{2.0000bp}
    \definecolor{sc}{rgb}{0.0000,0.0000,0.0000}
    \pgfsetstrokecolor{sc}
    \pgfsetmiterjoin
    \pgfsetbuttcap
    \pgfpathqmoveto{120.0000bp}{40.0000bp}
    \pgfpathqlineto{120.0000bp}{80.0000bp}
    \pgfpathqlineto{80.0000bp}{80.0000bp}
    \pgfpathqlineto{80.0000bp}{40.0000bp}
    \pgfpathqlineto{120.0000bp}{40.0000bp}
    \pgfpathclose
    \pgfusepathqfillstroke
  \end{pgfscope}
  \begin{pgfscope}
    \definecolor{fc}{rgb}{0.0000,0.0000,0.0000}
    \pgfsetfillcolor{fc}
    \pgftransformshift{\pgfqpoint{100.0000bp}{100.0000bp}}
    \pgftransformscale{1.0000}
    \pgftext[base,left]{$c$}
  \end{pgfscope}
  \begin{pgfscope}
    \definecolor{fc}{rgb}{0.0000,0.0000,0.0000}
    \pgfsetfillcolor{fc}
    \pgfsetfillopacity{0.0000}
    \pgfsetlinewidth{2.0000bp}
    \definecolor{sc}{rgb}{0.0000,0.0000,0.0000}
    \pgfsetstrokecolor{sc}
    \pgfsetmiterjoin
    \pgfsetbuttcap
    \pgfpathqmoveto{112.0000bp}{88.0000bp}
    \pgfpathqlineto{112.0000bp}{112.0000bp}
    \pgfpathqlineto{88.0000bp}{112.0000bp}
    \pgfpathqlineto{88.0000bp}{88.0000bp}
    \pgfpathqlineto{112.0000bp}{88.0000bp}
    \pgfpathclose
    \pgfusepathqfillstroke
  \end{pgfscope}
  \begin{pgfscope}
    \definecolor{fc}{rgb}{0.0000,0.0000,0.0000}
    \pgfsetfillcolor{fc}
    \pgfsetfillopacity{0.0000}
    \pgfsetlinewidth{2.0000bp}
    \definecolor{sc}{rgb}{0.0000,0.0000,0.0000}
    \pgfsetstrokecolor{sc}
    \pgfsetmiterjoin
    \pgfsetbuttcap
    \pgfpathqmoveto{120.0000bp}{80.0000bp}
    \pgfpathqlineto{120.0000bp}{120.0000bp}
    \pgfpathqlineto{80.0000bp}{120.0000bp}
    \pgfpathqlineto{80.0000bp}{80.0000bp}
    \pgfpathqlineto{120.0000bp}{80.0000bp}
    \pgfpathclose
    \pgfusepathqfillstroke
  \end{pgfscope}
  \begin{pgfscope}
    \definecolor{fc}{rgb}{0.0000,0.0000,0.0000}
    \pgfsetfillcolor{fc}
    \pgfsetfillopacity{0.0000}
    \pgfsetlinewidth{2.0000bp}
    \definecolor{sc}{rgb}{0.0000,0.0000,0.0000}
    \pgfsetstrokecolor{sc}
    \pgfsetmiterjoin
    \pgfsetbuttcap
    \pgfpathqmoveto{120.0000bp}{120.0000bp}
    \pgfpathqlineto{120.0000bp}{160.0000bp}
    \pgfpathqlineto{80.0000bp}{160.0000bp}
    \pgfpathqlineto{80.0000bp}{120.0000bp}
    \pgfpathqlineto{120.0000bp}{120.0000bp}
    \pgfpathclose
    \pgfusepathqfillstroke
  \end{pgfscope}
  \begin{pgfscope}
    \definecolor{fc}{rgb}{0.0000,0.0000,0.0000}
    \pgfsetfillcolor{fc}
    \pgfsetfillopacity{0.0000}
    \pgfsetlinewidth{2.0000bp}
    \definecolor{sc}{rgb}{0.0000,0.0000,0.0000}
    \pgfsetstrokecolor{sc}
    \pgfsetmiterjoin
    \pgfsetbuttcap
    \pgfsetdash{{2.6833bp}{2.6833bp}}{0.0000bp}
    \pgfpathqmoveto{114.0000bp}{166.0000bp}
    \pgfpathqlineto{114.0000bp}{194.0000bp}
    \pgfpathqlineto{86.0000bp}{194.0000bp}
    \pgfpathqlineto{86.0000bp}{166.0000bp}
    \pgfpathqlineto{114.0000bp}{166.0000bp}
    \pgfpathclose
    \pgfusepathqfillstroke
  \end{pgfscope}
  \begin{pgfscope}
    \definecolor{fc}{rgb}{0.0000,0.0000,0.0000}
    \pgfsetfillcolor{fc}
    \pgfsetfillopacity{0.0000}
    \pgfsetlinewidth{2.0000bp}
    \definecolor{sc}{rgb}{0.0000,0.0000,0.0000}
    \pgfsetstrokecolor{sc}
    \pgfsetmiterjoin
    \pgfsetbuttcap
    \pgfpathqmoveto{120.0000bp}{160.0000bp}
    \pgfpathqlineto{120.0000bp}{200.0000bp}
    \pgfpathqlineto{80.0000bp}{200.0000bp}
    \pgfpathqlineto{80.0000bp}{160.0000bp}
    \pgfpathqlineto{120.0000bp}{160.0000bp}
    \pgfpathclose
    \pgfusepathqfillstroke
  \end{pgfscope}
\end{pgfpicture}
}
        \caption{Vertical training problem for the sokoban agent.}\label{fig:results:train2}
    \end{subfigure}
    \begin{subfigure}{0.3\textwidth}
        \resizebox{\linewidth}{!}{\begin{pgfpicture}
  \pgfpathrectangle{\pgfpointorigin}{\pgfqpoint{200.0000bp}{200.0000bp}}
  \pgfusepath{use as bounding box}
  \begin{pgfscope}
    \definecolor{fc}{rgb}{0.0000,0.0000,0.0000}
    \pgfsetfillcolor{fc}
    \pgfsetfillopacity{0.0000}
    \pgfsetlinewidth{2.0000bp}
    \definecolor{sc}{rgb}{0.0000,0.0000,0.0000}
    \pgfsetstrokecolor{sc}
    \pgfsetmiterjoin
    \pgfsetbuttcap
    \pgfsetdash{{6.0000bp}{6.0000bp}}{0.0000bp}
    \pgfpathqmoveto{42.5000bp}{7.5000bp}
    \pgfpathqlineto{42.5000bp}{42.5000bp}
    \pgfpathqlineto{7.5000bp}{42.5000bp}
    \pgfpathqlineto{7.5000bp}{7.5000bp}
    \pgfpathqlineto{42.5000bp}{7.5000bp}
    \pgfpathclose
    \pgfusepathqfillstroke
  \end{pgfscope}
  \begin{pgfscope}
    \definecolor{fc}{rgb}{0.0000,0.0000,0.0000}
    \pgfsetfillcolor{fc}
    \pgfsetfillopacity{0.0000}
    \pgfsetlinewidth{2.0000bp}
    \definecolor{sc}{rgb}{0.0000,0.0000,0.0000}
    \pgfsetstrokecolor{sc}
    \pgfsetmiterjoin
    \pgfsetbuttcap
    \pgfpathqmoveto{50.0000bp}{0.0000bp}
    \pgfpathqlineto{50.0000bp}{50.0000bp}
    \pgfpathqlineto{0.0000bp}{50.0000bp}
    \pgfpathqlineto{-0.0000bp}{0.0000bp}
    \pgfpathqlineto{50.0000bp}{0.0000bp}
    \pgfpathclose
    \pgfusepathqfillstroke
  \end{pgfscope}
  \begin{pgfscope}
    \definecolor{fc}{rgb}{0.0000,0.0000,0.0000}
    \pgfsetfillcolor{fc}
    \pgfsetfillopacity{0.0000}
    \pgfsetlinewidth{2.0000bp}
    \definecolor{sc}{rgb}{0.0000,0.0000,0.0000}
    \pgfsetstrokecolor{sc}
    \pgfsetmiterjoin
    \pgfsetbuttcap
    \pgfpathqmoveto{50.0000bp}{50.0000bp}
    \pgfpathqlineto{50.0000bp}{100.0000bp}
    \pgfpathqlineto{0.0000bp}{100.0000bp}
    \pgfpathqlineto{-0.0000bp}{50.0000bp}
    \pgfpathqlineto{50.0000bp}{50.0000bp}
    \pgfpathclose
    \pgfusepathqfillstroke
  \end{pgfscope}
  \begin{pgfscope}
    \definecolor{fc}{rgb}{0.0000,0.0000,0.0000}
    \pgfsetfillcolor{fc}
    \pgftransformshift{\pgfqpoint{25.0000bp}{125.0000bp}}
    \pgftransformscale{1.2500}
    \pgftext[base,left]{$c_2$}
  \end{pgfscope}
  \begin{pgfscope}
    \definecolor{fc}{rgb}{0.0000,0.0000,0.0000}
    \pgfsetfillcolor{fc}
    \pgfsetfillopacity{0.0000}
    \pgfsetlinewidth{2.0000bp}
    \definecolor{sc}{rgb}{0.0000,0.0000,0.0000}
    \pgfsetstrokecolor{sc}
    \pgfsetmiterjoin
    \pgfsetbuttcap
    \pgfpathqmoveto{40.0000bp}{110.0000bp}
    \pgfpathqlineto{40.0000bp}{140.0000bp}
    \pgfpathqlineto{10.0000bp}{140.0000bp}
    \pgfpathqlineto{10.0000bp}{110.0000bp}
    \pgfpathqlineto{40.0000bp}{110.0000bp}
    \pgfpathclose
    \pgfusepathqfillstroke
  \end{pgfscope}
  \begin{pgfscope}
    \definecolor{fc}{rgb}{0.0000,0.0000,0.0000}
    \pgfsetfillcolor{fc}
    \pgfsetfillopacity{0.0000}
    \pgfsetlinewidth{2.0000bp}
    \definecolor{sc}{rgb}{0.0000,0.0000,0.0000}
    \pgfsetstrokecolor{sc}
    \pgfsetmiterjoin
    \pgfsetbuttcap
    \pgfpathqmoveto{50.0000bp}{100.0000bp}
    \pgfpathqlineto{50.0000bp}{150.0000bp}
    \pgfpathqlineto{0.0000bp}{150.0000bp}
    \pgfpathqlineto{-0.0000bp}{100.0000bp}
    \pgfpathqlineto{50.0000bp}{100.0000bp}
    \pgfpathclose
    \pgfusepathqfillstroke
  \end{pgfscope}
  \begin{pgfscope}
    \definecolor{fc}{rgb}{0.0000,0.0000,0.0000}
    \pgfsetfillcolor{fc}
    \pgfsetfillopacity{0.0000}
    \pgfsetlinewidth{2.0000bp}
    \definecolor{sc}{rgb}{0.0000,0.0000,0.0000}
    \pgfsetstrokecolor{sc}
    \pgfsetmiterjoin
    \pgfsetbuttcap
    \pgfsetdash{{6.0000bp}{6.0000bp}}{0.0000bp}
    \pgfpathqmoveto{192.5000bp}{157.5000bp}
    \pgfpathqlineto{192.5000bp}{192.5000bp}
    \pgfpathqlineto{157.5000bp}{192.5000bp}
    \pgfpathqlineto{157.5000bp}{157.5000bp}
    \pgfpathqlineto{192.5000bp}{157.5000bp}
    \pgfpathclose
    \pgfusepathqfillstroke
  \end{pgfscope}
  \begin{pgfscope}
    \definecolor{fc}{rgb}{0.0000,0.0000,0.0000}
    \pgfsetfillcolor{fc}
    \pgfsetfillopacity{0.0000}
    \pgfsetlinewidth{2.0000bp}
    \definecolor{sc}{rgb}{0.0000,0.0000,0.0000}
    \pgfsetstrokecolor{sc}
    \pgfsetmiterjoin
    \pgfsetbuttcap
    \pgfpathqmoveto{200.0000bp}{150.0000bp}
    \pgfpathqlineto{200.0000bp}{200.0000bp}
    \pgfpathqlineto{150.0000bp}{200.0000bp}
    \pgfpathqlineto{150.0000bp}{150.0000bp}
    \pgfpathqlineto{200.0000bp}{150.0000bp}
    \pgfpathclose
    \pgfusepathqfillstroke
  \end{pgfscope}
  \begin{pgfscope}
    \definecolor{fc}{rgb}{0.0000,0.0000,0.0000}
    \pgfsetfillcolor{fc}
    \pgfsetfillopacity{0.0000}
    \pgfsetlinewidth{2.0000bp}
    \definecolor{sc}{rgb}{0.0000,0.0000,0.0000}
    \pgfsetstrokecolor{sc}
    \pgfsetmiterjoin
    \pgfsetbuttcap
    \pgfpathqmoveto{150.0000bp}{150.0000bp}
    \pgfpathqlineto{150.0000bp}{200.0000bp}
    \pgfpathqlineto{100.0000bp}{200.0000bp}
    \pgfpathqlineto{100.0000bp}{150.0000bp}
    \pgfpathqlineto{150.0000bp}{150.0000bp}
    \pgfpathclose
    \pgfusepathqfillstroke
  \end{pgfscope}
  \begin{pgfscope}
    \definecolor{fc}{rgb}{0.0000,0.0000,0.0000}
    \pgfsetfillcolor{fc}
    \pgftransformshift{\pgfqpoint{75.0000bp}{175.0000bp}}
    \pgftransformscale{1.2500}
    \pgftext[base,left]{$c_1$}
  \end{pgfscope}
  \begin{pgfscope}
    \definecolor{fc}{rgb}{0.0000,0.0000,0.0000}
    \pgfsetfillcolor{fc}
    \pgfsetfillopacity{0.0000}
    \pgfsetlinewidth{2.0000bp}
    \definecolor{sc}{rgb}{0.0000,0.0000,0.0000}
    \pgfsetstrokecolor{sc}
    \pgfsetmiterjoin
    \pgfsetbuttcap
    \pgfpathqmoveto{90.0000bp}{160.0000bp}
    \pgfpathqlineto{90.0000bp}{190.0000bp}
    \pgfpathqlineto{60.0000bp}{190.0000bp}
    \pgfpathqlineto{60.0000bp}{160.0000bp}
    \pgfpathqlineto{90.0000bp}{160.0000bp}
    \pgfpathclose
    \pgfusepathqfillstroke
  \end{pgfscope}
  \begin{pgfscope}
    \definecolor{fc}{rgb}{0.0000,0.0000,0.0000}
    \pgfsetfillcolor{fc}
    \pgfsetfillopacity{0.0000}
    \pgfsetlinewidth{2.0000bp}
    \definecolor{sc}{rgb}{0.0000,0.0000,0.0000}
    \pgfsetstrokecolor{sc}
    \pgfsetmiterjoin
    \pgfsetbuttcap
    \pgfpathqmoveto{100.0000bp}{150.0000bp}
    \pgfpathqlineto{100.0000bp}{200.0000bp}
    \pgfpathqlineto{50.0000bp}{200.0000bp}
    \pgfpathqlineto{50.0000bp}{150.0000bp}
    \pgfpathqlineto{100.0000bp}{150.0000bp}
    \pgfpathclose
    \pgfusepathqfillstroke
  \end{pgfscope}
  \begin{pgfscope}
    \definecolor{fc}{rgb}{0.0000,0.0000,0.0000}
    \pgfsetfillcolor{fc}
    \pgfsetfillopacity{0.0000}
    \pgfsetlinewidth{2.0000bp}
    \definecolor{sc}{rgb}{0.0000,0.0000,0.0000}
    \pgfsetstrokecolor{sc}
    \pgfsetmiterjoin
    \pgfsetbuttcap
    \pgfpathqmoveto{40.0000bp}{175.0000bp}
    \pgfpathqcurveto{40.0000bp}{183.2843bp}{33.2843bp}{190.0000bp}{25.0000bp}{190.0000bp}
    \pgfpathqcurveto{16.7157bp}{190.0000bp}{10.0000bp}{183.2843bp}{10.0000bp}{175.0000bp}
    \pgfpathqcurveto{10.0000bp}{166.7157bp}{16.7157bp}{160.0000bp}{25.0000bp}{160.0000bp}
    \pgfpathqcurveto{33.2843bp}{160.0000bp}{40.0000bp}{166.7157bp}{40.0000bp}{175.0000bp}
    \pgfpathclose
    \pgfusepathqfillstroke
  \end{pgfscope}
  \begin{pgfscope}
    \definecolor{fc}{rgb}{0.0000,0.0000,0.0000}
    \pgfsetfillcolor{fc}
    \pgfsetfillopacity{0.0000}
    \pgfsetlinewidth{2.0000bp}
    \definecolor{sc}{rgb}{0.0000,0.0000,0.0000}
    \pgfsetstrokecolor{sc}
    \pgfsetmiterjoin
    \pgfsetbuttcap
    \pgfpathqmoveto{50.0000bp}{150.0000bp}
    \pgfpathqlineto{50.0000bp}{200.0000bp}
    \pgfpathqlineto{0.0000bp}{200.0000bp}
    \pgfpathqlineto{-0.0000bp}{150.0000bp}
    \pgfpathqlineto{50.0000bp}{150.0000bp}
    \pgfpathclose
    \pgfusepathqfillstroke
  \end{pgfscope}
\end{pgfpicture}
}
        \caption{A problem that requires use of all six actions in the sokoban domain.}\label{fig:results:train3}
    \end{subfigure}
    \caption{Sokoban worlds used for training.}\label{fig:results:sokoTraining}
\end{figure}

In the source code bundled with this thesis, the program \texttt{Plotting/Statistics.hs} for interpreting and visually representing statistics files are included  (along with source code for generating most diagrams used in this thesis). For each action schema mentioned in the statistics file, the program will output two line plots: one showing the effects the agent learned for the action, and one showing the corresponding preconditions.

The generated plots for the \texttt{move-*} actions are presented in Figures~\ref{fig:res:ekmove} and~\ref{fig:res:pkmove}, with an explanation of how to read them. Plots for the other four sokoban actions are listed in Appendix~\ref{sec:app:results}.

As can be seen in~\figref{fig:res:ekmoveh}, the agent will execute arbitrary \texttt{move-h} actions, untill it finds a valid one after 20 trials. When that happens, it immediately proves its two positive and negative effects, and disproved all others. In~\figref{fig:res:pkmoveh}, it can be seen that it builds up the set of candidates for each failing action application. Upon action success in step 21, it is able to disprove many positive preconditions and some negative ones, which in turn reduces two candidate sets to singletons, thus proving them.

This pattern is very similar for the \texttt{move-v} action (see Figures~\ref{fig:res:ekmovev} and~\ref{fig:res:pkmovev}), except that its proves are delayed; in the first training problem, this action is never applicable, but that is unknown to the agent. It will attempt to apply the \texttt{move-v} action, and only succeed in finding cadidates. These candidates are not reduced untill the action succeeds in the second training problem. However, once the agent starts solving the second problem, the \texttt{move-v} action only requires three failed trials before success, as the condidates are used to limit the applicability of the action.

\begin{figure}
    \centering
    \begin{subfigure}{0.45\linewidth}
        \resizebox{\linewidth}{!}{\begin{pgfpicture}
  \pgfpathrectangle{\pgfpointorigin}{\pgfqpoint{200.0000bp}{200.0000bp}}
  \pgfusepath{use as bounding box}
  \begin{pgfscope}
    \definecolor{fc}{rgb}{0.0000,0.0000,0.0000}
    \pgfsetfillcolor{fc}
    \pgftransformshift{\pgfqpoint{28.3333bp}{25.8333bp}}
    \pgftransformscale{1.0417}
    \pgftext[base,left]{candidates}
  \end{pgfscope}
  \begin{pgfscope}
    \definecolor{fc}{rgb}{0.0000,0.0000,0.0000}
    \pgfsetfillcolor{fc}
    \pgfsetlinewidth{0.6928bp}
    \definecolor{sc}{rgb}{0.0000,0.0000,0.0000}
    \pgfsetstrokecolor{sc}
    \pgfsetmiterjoin
    \pgfsetbuttcap
    \pgfpathqmoveto{20.0000bp}{28.3333bp}
    \pgfpathqcurveto{20.0000bp}{30.1743bp}{18.5076bp}{31.6667bp}{16.6667bp}{31.6667bp}
    \pgfpathqcurveto{14.8257bp}{31.6667bp}{13.3333bp}{30.1743bp}{13.3333bp}{28.3333bp}
    \pgfpathqcurveto{13.3333bp}{26.4924bp}{14.8257bp}{25.0000bp}{16.6667bp}{25.0000bp}
    \pgfpathqcurveto{18.5076bp}{25.0000bp}{20.0000bp}{26.4924bp}{20.0000bp}{28.3333bp}
    \pgfpathclose
    \pgfusepathqfillstroke
  \end{pgfscope}
  \begin{pgfscope}
    \definecolor{fc}{rgb}{0.0000,0.0000,0.0000}
    \pgfsetfillcolor{fc}
    \pgftransformshift{\pgfqpoint{28.3333bp}{36.6667bp}}
    \pgftransformscale{1.0417}
    \pgftext[base,left]{negative unproven}
  \end{pgfscope}
  \begin{pgfscope}
    \definecolor{fc}{rgb}{1.0000,1.0000,0.0000}
    \pgfsetfillcolor{fc}
    \pgfsetlinewidth{0.6928bp}
    \definecolor{sc}{rgb}{1.0000,1.0000,0.0000}
    \pgfsetstrokecolor{sc}
    \pgfsetmiterjoin
    \pgfsetbuttcap
    \pgfpathqmoveto{20.0000bp}{39.1667bp}
    \pgfpathqcurveto{20.0000bp}{41.0076bp}{18.5076bp}{42.5000bp}{16.6667bp}{42.5000bp}
    \pgfpathqcurveto{14.8257bp}{42.5000bp}{13.3333bp}{41.0076bp}{13.3333bp}{39.1667bp}
    \pgfpathqcurveto{13.3333bp}{37.3257bp}{14.8257bp}{35.8333bp}{16.6667bp}{35.8333bp}
    \pgfpathqcurveto{18.5076bp}{35.8333bp}{20.0000bp}{37.3257bp}{20.0000bp}{39.1667bp}
    \pgfpathclose
    \pgfusepathqfillstroke
  \end{pgfscope}
  \begin{pgfscope}
    \definecolor{fc}{rgb}{0.0000,0.0000,0.0000}
    \pgfsetfillcolor{fc}
    \pgftransformshift{\pgfqpoint{28.3333bp}{47.5000bp}}
    \pgftransformscale{1.0417}
    \pgftext[base,left]{negative proven}
  \end{pgfscope}
  \begin{pgfscope}
    \definecolor{fc}{rgb}{0.0000,0.5020,0.0000}
    \pgfsetfillcolor{fc}
    \pgfsetlinewidth{0.6928bp}
    \definecolor{sc}{rgb}{0.0000,0.5020,0.0000}
    \pgfsetstrokecolor{sc}
    \pgfsetmiterjoin
    \pgfsetbuttcap
    \pgfpathqmoveto{20.0000bp}{50.0000bp}
    \pgfpathqcurveto{20.0000bp}{51.8409bp}{18.5076bp}{53.3333bp}{16.6667bp}{53.3333bp}
    \pgfpathqcurveto{14.8257bp}{53.3333bp}{13.3333bp}{51.8409bp}{13.3333bp}{50.0000bp}
    \pgfpathqcurveto{13.3333bp}{48.1591bp}{14.8257bp}{46.6667bp}{16.6667bp}{46.6667bp}
    \pgfpathqcurveto{18.5076bp}{46.6667bp}{20.0000bp}{48.1591bp}{20.0000bp}{50.0000bp}
    \pgfpathclose
    \pgfusepathqfillstroke
  \end{pgfscope}
  \begin{pgfscope}
    \definecolor{fc}{rgb}{0.0000,0.0000,0.0000}
    \pgfsetfillcolor{fc}
    \pgftransformshift{\pgfqpoint{28.3333bp}{58.3333bp}}
    \pgftransformscale{1.0417}
    \pgftext[base,left]{positive unproven}
  \end{pgfscope}
  \begin{pgfscope}
    \definecolor{fc}{rgb}{1.0000,0.0000,0.0000}
    \pgfsetfillcolor{fc}
    \pgfsetlinewidth{0.6928bp}
    \definecolor{sc}{rgb}{1.0000,0.0000,0.0000}
    \pgfsetstrokecolor{sc}
    \pgfsetmiterjoin
    \pgfsetbuttcap
    \pgfpathqmoveto{20.0000bp}{60.8333bp}
    \pgfpathqcurveto{20.0000bp}{62.6743bp}{18.5076bp}{64.1667bp}{16.6667bp}{64.1667bp}
    \pgfpathqcurveto{14.8257bp}{64.1667bp}{13.3333bp}{62.6743bp}{13.3333bp}{60.8333bp}
    \pgfpathqcurveto{13.3333bp}{58.9924bp}{14.8257bp}{57.5000bp}{16.6667bp}{57.5000bp}
    \pgfpathqcurveto{18.5076bp}{57.5000bp}{20.0000bp}{58.9924bp}{20.0000bp}{60.8333bp}
    \pgfpathclose
    \pgfusepathqfillstroke
  \end{pgfscope}
  \begin{pgfscope}
    \definecolor{fc}{rgb}{0.0000,0.0000,0.0000}
    \pgfsetfillcolor{fc}
    \pgftransformshift{\pgfqpoint{28.3333bp}{69.1667bp}}
    \pgftransformscale{1.0417}
    \pgftext[base,left]{positive proven}
  \end{pgfscope}
  \begin{pgfscope}
    \definecolor{fc}{rgb}{0.0000,0.0000,1.0000}
    \pgfsetfillcolor{fc}
    \pgfsetlinewidth{0.6928bp}
    \definecolor{sc}{rgb}{0.0000,0.0000,1.0000}
    \pgfsetstrokecolor{sc}
    \pgfsetmiterjoin
    \pgfsetbuttcap
    \pgfpathqmoveto{20.0000bp}{71.6667bp}
    \pgfpathqcurveto{20.0000bp}{73.5076bp}{18.5076bp}{75.0000bp}{16.6667bp}{75.0000bp}
    \pgfpathqcurveto{14.8257bp}{75.0000bp}{13.3333bp}{73.5076bp}{13.3333bp}{71.6667bp}
    \pgfpathqcurveto{13.3333bp}{69.8257bp}{14.8257bp}{68.3333bp}{16.6667bp}{68.3333bp}
    \pgfpathqcurveto{18.5076bp}{68.3333bp}{20.0000bp}{69.8257bp}{20.0000bp}{71.6667bp}
    \pgfpathclose
    \pgfusepathqfillstroke
  \end{pgfscope}
  \begin{pgfscope}
    \pgfsetlinewidth{0.6928bp}
    \definecolor{sc}{rgb}{1.0000,1.0000,0.0000}
    \pgfsetstrokecolor{sc}
    \pgfsetmiterjoin
    \pgfsetbuttcap
    \pgfpathqmoveto{25.0000bp}{175.0000bp}
    \pgfpathqlineto{33.3333bp}{175.0000bp}
    \pgfpathqlineto{41.6667bp}{175.0000bp}
    \pgfpathqlineto{50.0000bp}{175.0000bp}
    \pgfpathqlineto{58.3333bp}{175.0000bp}
    \pgfpathqlineto{66.6667bp}{175.0000bp}
    \pgfpathqlineto{75.0000bp}{175.0000bp}
    \pgfpathqlineto{83.3333bp}{175.0000bp}
    \pgfpathqlineto{91.6667bp}{175.0000bp}
    \pgfpathqlineto{100.0000bp}{175.0000bp}
    \pgfpathqlineto{108.3333bp}{175.0000bp}
    \pgfpathqlineto{116.6667bp}{175.0000bp}
    \pgfpathqlineto{125.0000bp}{175.0000bp}
    \pgfpathqlineto{133.3333bp}{175.0000bp}
    \pgfpathqlineto{141.6667bp}{175.0000bp}
    \pgfpathqlineto{150.0000bp}{175.0000bp}
    \pgfpathqlineto{158.3333bp}{175.0000bp}
    \pgfpathqlineto{166.6667bp}{91.6667bp}
    \pgfpathqlineto{175.0000bp}{91.6667bp}
    \pgfpathqlineto{183.3333bp}{91.6667bp}
    \pgfpathqlineto{191.6667bp}{91.6667bp}
    \pgfpathqlineto{200.0000bp}{91.6667bp}
    \pgfusepathqstroke
  \end{pgfscope}
  \begin{pgfscope}
    \pgfsetlinewidth{0.6928bp}
    \definecolor{sc}{rgb}{0.0000,0.5020,0.0000}
    \pgfsetstrokecolor{sc}
    \pgfsetmiterjoin
    \pgfsetbuttcap
    \pgfpathqmoveto{25.0000bp}{91.6667bp}
    \pgfpathqlineto{33.3333bp}{91.6667bp}
    \pgfpathqlineto{41.6667bp}{91.6667bp}
    \pgfpathqlineto{50.0000bp}{91.6667bp}
    \pgfpathqlineto{58.3333bp}{91.6667bp}
    \pgfpathqlineto{66.6667bp}{91.6667bp}
    \pgfpathqlineto{75.0000bp}{91.6667bp}
    \pgfpathqlineto{83.3333bp}{91.6667bp}
    \pgfpathqlineto{91.6667bp}{91.6667bp}
    \pgfpathqlineto{100.0000bp}{91.6667bp}
    \pgfpathqlineto{108.3333bp}{91.6667bp}
    \pgfpathqlineto{116.6667bp}{91.6667bp}
    \pgfpathqlineto{125.0000bp}{91.6667bp}
    \pgfpathqlineto{133.3333bp}{91.6667bp}
    \pgfpathqlineto{141.6667bp}{91.6667bp}
    \pgfpathqlineto{150.0000bp}{91.6667bp}
    \pgfpathqlineto{158.3333bp}{91.6667bp}
    \pgfpathqlineto{166.6667bp}{100.0000bp}
    \pgfpathqlineto{175.0000bp}{100.0000bp}
    \pgfpathqlineto{183.3333bp}{100.0000bp}
    \pgfpathqlineto{191.6667bp}{100.0000bp}
    \pgfpathqlineto{200.0000bp}{100.0000bp}
    \pgfusepathqstroke
  \end{pgfscope}
  \begin{pgfscope}
    \pgfsetlinewidth{0.6928bp}
    \definecolor{sc}{rgb}{1.0000,0.0000,0.0000}
    \pgfsetstrokecolor{sc}
    \pgfsetmiterjoin
    \pgfsetbuttcap
    \pgfpathqmoveto{25.0000bp}{175.0000bp}
    \pgfpathqlineto{33.3333bp}{175.0000bp}
    \pgfpathqlineto{41.6667bp}{175.0000bp}
    \pgfpathqlineto{50.0000bp}{175.0000bp}
    \pgfpathqlineto{58.3333bp}{175.0000bp}
    \pgfpathqlineto{66.6667bp}{175.0000bp}
    \pgfpathqlineto{75.0000bp}{175.0000bp}
    \pgfpathqlineto{83.3333bp}{175.0000bp}
    \pgfpathqlineto{91.6667bp}{175.0000bp}
    \pgfpathqlineto{100.0000bp}{175.0000bp}
    \pgfpathqlineto{108.3333bp}{175.0000bp}
    \pgfpathqlineto{116.6667bp}{175.0000bp}
    \pgfpathqlineto{125.0000bp}{175.0000bp}
    \pgfpathqlineto{133.3333bp}{175.0000bp}
    \pgfpathqlineto{141.6667bp}{175.0000bp}
    \pgfpathqlineto{150.0000bp}{175.0000bp}
    \pgfpathqlineto{158.3333bp}{175.0000bp}
    \pgfpathqlineto{166.6667bp}{91.6667bp}
    \pgfpathqlineto{175.0000bp}{91.6667bp}
    \pgfpathqlineto{183.3333bp}{91.6667bp}
    \pgfpathqlineto{191.6667bp}{91.6667bp}
    \pgfpathqlineto{200.0000bp}{91.6667bp}
    \pgfusepathqstroke
  \end{pgfscope}
  \begin{pgfscope}
    \pgfsetlinewidth{0.6928bp}
    \definecolor{sc}{rgb}{0.0000,0.0000,1.0000}
    \pgfsetstrokecolor{sc}
    \pgfsetmiterjoin
    \pgfsetbuttcap
    \pgfpathqmoveto{25.0000bp}{91.6667bp}
    \pgfpathqlineto{33.3333bp}{91.6667bp}
    \pgfpathqlineto{41.6667bp}{91.6667bp}
    \pgfpathqlineto{50.0000bp}{91.6667bp}
    \pgfpathqlineto{58.3333bp}{91.6667bp}
    \pgfpathqlineto{66.6667bp}{91.6667bp}
    \pgfpathqlineto{75.0000bp}{91.6667bp}
    \pgfpathqlineto{83.3333bp}{91.6667bp}
    \pgfpathqlineto{91.6667bp}{91.6667bp}
    \pgfpathqlineto{100.0000bp}{91.6667bp}
    \pgfpathqlineto{108.3333bp}{91.6667bp}
    \pgfpathqlineto{116.6667bp}{91.6667bp}
    \pgfpathqlineto{125.0000bp}{91.6667bp}
    \pgfpathqlineto{133.3333bp}{91.6667bp}
    \pgfpathqlineto{141.6667bp}{91.6667bp}
    \pgfpathqlineto{150.0000bp}{91.6667bp}
    \pgfpathqlineto{158.3333bp}{91.6667bp}
    \pgfpathqlineto{166.6667bp}{100.0000bp}
    \pgfpathqlineto{175.0000bp}{100.0000bp}
    \pgfpathqlineto{183.3333bp}{100.0000bp}
    \pgfpathqlineto{191.6667bp}{100.0000bp}
    \pgfpathqlineto{200.0000bp}{100.0000bp}
    \pgfusepathqstroke
  \end{pgfscope}
  \begin{pgfscope}
    \pgfsetlinewidth{0.6928bp}
    \definecolor{sc}{rgb}{1.0000,0.0000,0.0000}
    \pgfsetstrokecolor{sc}
    \pgfsetmiterjoin
    \pgfsetbuttcap
    \pgfpathqmoveto{50.0000bp}{91.6667bp}
    \pgfpathqlineto{50.0000bp}{83.3333bp}
    \pgfusepathqstroke
  \end{pgfscope}
  \begin{pgfscope}
    \pgfsetlinewidth{0.6928bp}
    \definecolor{sc}{rgb}{1.0000,0.0000,0.0000}
    \pgfsetstrokecolor{sc}
    \pgfsetmiterjoin
    \pgfsetbuttcap
    \pgfpathqmoveto{166.6667bp}{91.6667bp}
    \pgfpathqlineto{166.6667bp}{83.3333bp}
    \pgfusepathqstroke
  \end{pgfscope}
  \begin{pgfscope}
    \pgfsetlinewidth{0.6928bp}
    \definecolor{sc}{rgb}{0.0000,0.0000,0.0000}
    \pgfsetstrokecolor{sc}
    \pgfsetmiterjoin
    \pgfsetbuttcap
    \pgfpathqmoveto{183.3333bp}{91.6667bp}
    \pgfpathqlineto{183.3333bp}{87.5000bp}
    \pgfusepathqstroke
  \end{pgfscope}
  \begin{pgfscope}
    \pgfsetlinewidth{0.6928bp}
    \definecolor{sc}{rgb}{0.0000,0.0000,0.0000}
    \pgfsetstrokecolor{sc}
    \pgfsetmiterjoin
    \pgfsetbuttcap
    \pgfpathqmoveto{141.6667bp}{91.6667bp}
    \pgfpathqlineto{141.6667bp}{87.5000bp}
    \pgfusepathqstroke
  \end{pgfscope}
  \begin{pgfscope}
    \pgfsetlinewidth{0.6928bp}
    \definecolor{sc}{rgb}{0.0000,0.0000,0.0000}
    \pgfsetstrokecolor{sc}
    \pgfsetmiterjoin
    \pgfsetbuttcap
    \pgfpathqmoveto{100.0000bp}{91.6667bp}
    \pgfpathqlineto{100.0000bp}{87.5000bp}
    \pgfusepathqstroke
  \end{pgfscope}
  \begin{pgfscope}
    \pgfsetlinewidth{0.6928bp}
    \definecolor{sc}{rgb}{0.0000,0.0000,0.0000}
    \pgfsetstrokecolor{sc}
    \pgfsetmiterjoin
    \pgfsetbuttcap
    \pgfpathqmoveto{58.3333bp}{91.6667bp}
    \pgfpathqlineto{58.3333bp}{87.5000bp}
    \pgfusepathqstroke
  \end{pgfscope}
  \begin{pgfscope}
    \definecolor{fc}{rgb}{0.0000,0.0000,0.0000}
    \pgfsetfillcolor{fc}
    \pgftransformshift{\pgfqpoint{-0.0000bp}{172.5000bp}}
    \pgftransformscale{1.0417}
    \pgftext[base,left]{$\mathbb{F}_A$}
  \end{pgfscope}
  \begin{pgfscope}
    \pgfsetlinewidth{0.6928bp}
    \definecolor{sc}{rgb}{0.0000,0.0000,0.0000}
    \pgfsetstrokecolor{sc}
    \pgfsetmiterjoin
    \pgfsetbuttcap
    \pgfpathqmoveto{16.6667bp}{175.0000bp}
    \pgfpathqlineto{15.0000bp}{175.0000bp}
    \pgfusepathqstroke
  \end{pgfscope}
  \begin{pgfscope}
    \pgfsetlinewidth{0.6928bp}
    \definecolor{sc}{rgb}{0.0000,0.0000,0.0000}
    \pgfsetstrokecolor{sc}
    \pgfsetmiterjoin
    \pgfsetbuttcap
    \pgfpathqmoveto{16.6667bp}{91.6667bp}
    \pgfpathqlineto{16.6667bp}{175.0000bp}
    \pgfusepathqstroke
  \end{pgfscope}
  \begin{pgfscope}
    \pgfsetlinewidth{0.6928bp}
    \definecolor{sc}{rgb}{0.0000,0.0000,0.0000}
    \pgfsetstrokecolor{sc}
    \pgfsetmiterjoin
    \pgfsetbuttcap
    \pgfpathqmoveto{16.6667bp}{91.6667bp}
    \pgfpathqlineto{200.0000bp}{91.6667bp}
    \pgfusepathqstroke
  \end{pgfscope}
\end{pgfpicture}
}
        \caption{\texttt{move-h} effects}\label{fig:res:ekmoveh}
    \end{subfigure}
    \begin{subfigure}{0.45\linewidth}
        \resizebox{\linewidth}{!}{\begin{pgfpicture}
  \pgfpathrectangle{\pgfpointorigin}{\pgfqpoint{200.0000bp}{200.0000bp}}
  \pgfusepath{use as bounding box}
  \begin{pgfscope}
    \definecolor{fc}{rgb}{0.0000,0.0000,0.0000}
    \pgfsetfillcolor{fc}
    \pgftransformshift{\pgfqpoint{27.2000bp}{28.8000bp}}
    \pgftransformscale{1.0000}
    \pgftext[base,left]{candidates}
  \end{pgfscope}
  \begin{pgfscope}
    \definecolor{fc}{rgb}{0.0000,0.0000,0.0000}
    \pgfsetfillcolor{fc}
    \pgfsetlinewidth{0.6788bp}
    \definecolor{sc}{rgb}{0.0000,0.0000,0.0000}
    \pgfsetstrokecolor{sc}
    \pgfsetmiterjoin
    \pgfsetbuttcap
    \pgfpathqmoveto{19.2000bp}{31.2000bp}
    \pgfpathqcurveto{19.2000bp}{32.9673bp}{17.7673bp}{34.4000bp}{16.0000bp}{34.4000bp}
    \pgfpathqcurveto{14.2327bp}{34.4000bp}{12.8000bp}{32.9673bp}{12.8000bp}{31.2000bp}
    \pgfpathqcurveto{12.8000bp}{29.4327bp}{14.2327bp}{28.0000bp}{16.0000bp}{28.0000bp}
    \pgfpathqcurveto{17.7673bp}{28.0000bp}{19.2000bp}{29.4327bp}{19.2000bp}{31.2000bp}
    \pgfpathclose
    \pgfusepathqfillstroke
  \end{pgfscope}
  \begin{pgfscope}
    \definecolor{fc}{rgb}{0.0000,0.0000,0.0000}
    \pgfsetfillcolor{fc}
    \pgftransformshift{\pgfqpoint{27.2000bp}{39.2000bp}}
    \pgftransformscale{1.0000}
    \pgftext[base,left]{negative unproven}
  \end{pgfscope}
  \begin{pgfscope}
    \definecolor{fc}{rgb}{1.0000,1.0000,0.0000}
    \pgfsetfillcolor{fc}
    \pgfsetlinewidth{0.6788bp}
    \definecolor{sc}{rgb}{1.0000,1.0000,0.0000}
    \pgfsetstrokecolor{sc}
    \pgfsetmiterjoin
    \pgfsetbuttcap
    \pgfpathqmoveto{19.2000bp}{41.6000bp}
    \pgfpathqcurveto{19.2000bp}{43.3673bp}{17.7673bp}{44.8000bp}{16.0000bp}{44.8000bp}
    \pgfpathqcurveto{14.2327bp}{44.8000bp}{12.8000bp}{43.3673bp}{12.8000bp}{41.6000bp}
    \pgfpathqcurveto{12.8000bp}{39.8327bp}{14.2327bp}{38.4000bp}{16.0000bp}{38.4000bp}
    \pgfpathqcurveto{17.7673bp}{38.4000bp}{19.2000bp}{39.8327bp}{19.2000bp}{41.6000bp}
    \pgfpathclose
    \pgfusepathqfillstroke
  \end{pgfscope}
  \begin{pgfscope}
    \definecolor{fc}{rgb}{0.0000,0.0000,0.0000}
    \pgfsetfillcolor{fc}
    \pgftransformshift{\pgfqpoint{27.2000bp}{49.6000bp}}
    \pgftransformscale{1.0000}
    \pgftext[base,left]{negative proven}
  \end{pgfscope}
  \begin{pgfscope}
    \definecolor{fc}{rgb}{0.0000,0.5020,0.0000}
    \pgfsetfillcolor{fc}
    \pgfsetlinewidth{0.6788bp}
    \definecolor{sc}{rgb}{0.0000,0.5020,0.0000}
    \pgfsetstrokecolor{sc}
    \pgfsetmiterjoin
    \pgfsetbuttcap
    \pgfpathqmoveto{19.2000bp}{52.0000bp}
    \pgfpathqcurveto{19.2000bp}{53.7673bp}{17.7673bp}{55.2000bp}{16.0000bp}{55.2000bp}
    \pgfpathqcurveto{14.2327bp}{55.2000bp}{12.8000bp}{53.7673bp}{12.8000bp}{52.0000bp}
    \pgfpathqcurveto{12.8000bp}{50.2327bp}{14.2327bp}{48.8000bp}{16.0000bp}{48.8000bp}
    \pgfpathqcurveto{17.7673bp}{48.8000bp}{19.2000bp}{50.2327bp}{19.2000bp}{52.0000bp}
    \pgfpathclose
    \pgfusepathqfillstroke
  \end{pgfscope}
  \begin{pgfscope}
    \definecolor{fc}{rgb}{0.0000,0.0000,0.0000}
    \pgfsetfillcolor{fc}
    \pgftransformshift{\pgfqpoint{27.2000bp}{60.0000bp}}
    \pgftransformscale{1.0000}
    \pgftext[base,left]{positive unproven}
  \end{pgfscope}
  \begin{pgfscope}
    \definecolor{fc}{rgb}{1.0000,0.0000,0.0000}
    \pgfsetfillcolor{fc}
    \pgfsetlinewidth{0.6788bp}
    \definecolor{sc}{rgb}{1.0000,0.0000,0.0000}
    \pgfsetstrokecolor{sc}
    \pgfsetmiterjoin
    \pgfsetbuttcap
    \pgfpathqmoveto{19.2000bp}{62.4000bp}
    \pgfpathqcurveto{19.2000bp}{64.1673bp}{17.7673bp}{65.6000bp}{16.0000bp}{65.6000bp}
    \pgfpathqcurveto{14.2327bp}{65.6000bp}{12.8000bp}{64.1673bp}{12.8000bp}{62.4000bp}
    \pgfpathqcurveto{12.8000bp}{60.6327bp}{14.2327bp}{59.2000bp}{16.0000bp}{59.2000bp}
    \pgfpathqcurveto{17.7673bp}{59.2000bp}{19.2000bp}{60.6327bp}{19.2000bp}{62.4000bp}
    \pgfpathclose
    \pgfusepathqfillstroke
  \end{pgfscope}
  \begin{pgfscope}
    \definecolor{fc}{rgb}{0.0000,0.0000,0.0000}
    \pgfsetfillcolor{fc}
    \pgftransformshift{\pgfqpoint{27.2000bp}{70.4000bp}}
    \pgftransformscale{1.0000}
    \pgftext[base,left]{positive proven}
  \end{pgfscope}
  \begin{pgfscope}
    \definecolor{fc}{rgb}{0.0000,0.0000,1.0000}
    \pgfsetfillcolor{fc}
    \pgfsetlinewidth{0.6788bp}
    \definecolor{sc}{rgb}{0.0000,0.0000,1.0000}
    \pgfsetstrokecolor{sc}
    \pgfsetmiterjoin
    \pgfsetbuttcap
    \pgfpathqmoveto{19.2000bp}{72.8000bp}
    \pgfpathqcurveto{19.2000bp}{74.5673bp}{17.7673bp}{76.0000bp}{16.0000bp}{76.0000bp}
    \pgfpathqcurveto{14.2327bp}{76.0000bp}{12.8000bp}{74.5673bp}{12.8000bp}{72.8000bp}
    \pgfpathqcurveto{12.8000bp}{71.0327bp}{14.2327bp}{69.6000bp}{16.0000bp}{69.6000bp}
    \pgfpathqcurveto{17.7673bp}{69.6000bp}{19.2000bp}{71.0327bp}{19.2000bp}{72.8000bp}
    \pgfpathclose
    \pgfusepathqfillstroke
  \end{pgfscope}
  \begin{pgfscope}
    \pgfsetlinewidth{0.6788bp}
    \definecolor{sc}{rgb}{1.0000,1.0000,0.0000}
    \pgfsetstrokecolor{sc}
    \pgfsetmiterjoin
    \pgfsetbuttcap
    \pgfpathqmoveto{16.0000bp}{172.0000bp}
    \pgfpathqlineto{24.0000bp}{172.0000bp}
    \pgfpathqlineto{32.0000bp}{172.0000bp}
    \pgfpathqlineto{40.0000bp}{172.0000bp}
    \pgfpathqlineto{48.0000bp}{172.0000bp}
    \pgfpathqlineto{56.0000bp}{172.0000bp}
    \pgfpathqlineto{64.0000bp}{172.0000bp}
    \pgfpathqlineto{72.0000bp}{172.0000bp}
    \pgfpathqlineto{80.0000bp}{172.0000bp}
    \pgfpathqlineto{88.0000bp}{172.0000bp}
    \pgfpathqlineto{96.0000bp}{172.0000bp}
    \pgfpathqlineto{104.0000bp}{172.0000bp}
    \pgfpathqlineto{112.0000bp}{172.0000bp}
    \pgfpathqlineto{120.0000bp}{172.0000bp}
    \pgfpathqlineto{128.0000bp}{172.0000bp}
    \pgfpathqlineto{136.0000bp}{172.0000bp}
    \pgfpathqlineto{144.0000bp}{172.0000bp}
    \pgfpathqlineto{152.0000bp}{172.0000bp}
    \pgfpathqlineto{160.0000bp}{172.0000bp}
    \pgfpathqlineto{168.0000bp}{172.0000bp}
    \pgfpathqlineto{176.0000bp}{172.0000bp}
    \pgfpathqlineto{184.0000bp}{92.0000bp}
    \pgfpathqlineto{192.0000bp}{92.0000bp}
    \pgfusepathqstroke
  \end{pgfscope}
  \begin{pgfscope}
    \pgfsetlinewidth{0.6788bp}
    \definecolor{sc}{rgb}{0.0000,0.5020,0.0000}
    \pgfsetstrokecolor{sc}
    \pgfsetmiterjoin
    \pgfsetbuttcap
    \pgfpathqmoveto{16.0000bp}{92.0000bp}
    \pgfpathqlineto{24.0000bp}{92.0000bp}
    \pgfpathqlineto{32.0000bp}{92.0000bp}
    \pgfpathqlineto{40.0000bp}{92.0000bp}
    \pgfpathqlineto{48.0000bp}{92.0000bp}
    \pgfpathqlineto{56.0000bp}{92.0000bp}
    \pgfpathqlineto{64.0000bp}{92.0000bp}
    \pgfpathqlineto{72.0000bp}{92.0000bp}
    \pgfpathqlineto{80.0000bp}{92.0000bp}
    \pgfpathqlineto{88.0000bp}{92.0000bp}
    \pgfpathqlineto{96.0000bp}{92.0000bp}
    \pgfpathqlineto{104.0000bp}{92.0000bp}
    \pgfpathqlineto{112.0000bp}{92.0000bp}
    \pgfpathqlineto{120.0000bp}{92.0000bp}
    \pgfpathqlineto{128.0000bp}{92.0000bp}
    \pgfpathqlineto{136.0000bp}{92.0000bp}
    \pgfpathqlineto{144.0000bp}{92.0000bp}
    \pgfpathqlineto{152.0000bp}{92.0000bp}
    \pgfpathqlineto{160.0000bp}{92.0000bp}
    \pgfpathqlineto{168.0000bp}{92.0000bp}
    \pgfpathqlineto{176.0000bp}{92.0000bp}
    \pgfpathqlineto{184.0000bp}{100.0000bp}
    \pgfpathqlineto{192.0000bp}{100.0000bp}
    \pgfusepathqstroke
  \end{pgfscope}
  \begin{pgfscope}
    \pgfsetlinewidth{0.6788bp}
    \definecolor{sc}{rgb}{1.0000,0.0000,0.0000}
    \pgfsetstrokecolor{sc}
    \pgfsetmiterjoin
    \pgfsetbuttcap
    \pgfpathqmoveto{16.0000bp}{172.0000bp}
    \pgfpathqlineto{24.0000bp}{172.0000bp}
    \pgfpathqlineto{32.0000bp}{172.0000bp}
    \pgfpathqlineto{40.0000bp}{172.0000bp}
    \pgfpathqlineto{48.0000bp}{172.0000bp}
    \pgfpathqlineto{56.0000bp}{172.0000bp}
    \pgfpathqlineto{64.0000bp}{172.0000bp}
    \pgfpathqlineto{72.0000bp}{172.0000bp}
    \pgfpathqlineto{80.0000bp}{172.0000bp}
    \pgfpathqlineto{88.0000bp}{172.0000bp}
    \pgfpathqlineto{96.0000bp}{172.0000bp}
    \pgfpathqlineto{104.0000bp}{172.0000bp}
    \pgfpathqlineto{112.0000bp}{172.0000bp}
    \pgfpathqlineto{120.0000bp}{172.0000bp}
    \pgfpathqlineto{128.0000bp}{172.0000bp}
    \pgfpathqlineto{136.0000bp}{172.0000bp}
    \pgfpathqlineto{144.0000bp}{172.0000bp}
    \pgfpathqlineto{152.0000bp}{172.0000bp}
    \pgfpathqlineto{160.0000bp}{172.0000bp}
    \pgfpathqlineto{168.0000bp}{172.0000bp}
    \pgfpathqlineto{176.0000bp}{172.0000bp}
    \pgfpathqlineto{184.0000bp}{92.0000bp}
    \pgfpathqlineto{192.0000bp}{92.0000bp}
    \pgfusepathqstroke
  \end{pgfscope}
  \begin{pgfscope}
    \pgfsetlinewidth{0.6788bp}
    \definecolor{sc}{rgb}{0.0000,0.0000,1.0000}
    \pgfsetstrokecolor{sc}
    \pgfsetmiterjoin
    \pgfsetbuttcap
    \pgfpathqmoveto{16.0000bp}{92.0000bp}
    \pgfpathqlineto{24.0000bp}{92.0000bp}
    \pgfpathqlineto{32.0000bp}{92.0000bp}
    \pgfpathqlineto{40.0000bp}{92.0000bp}
    \pgfpathqlineto{48.0000bp}{92.0000bp}
    \pgfpathqlineto{56.0000bp}{92.0000bp}
    \pgfpathqlineto{64.0000bp}{92.0000bp}
    \pgfpathqlineto{72.0000bp}{92.0000bp}
    \pgfpathqlineto{80.0000bp}{92.0000bp}
    \pgfpathqlineto{88.0000bp}{92.0000bp}
    \pgfpathqlineto{96.0000bp}{92.0000bp}
    \pgfpathqlineto{104.0000bp}{92.0000bp}
    \pgfpathqlineto{112.0000bp}{92.0000bp}
    \pgfpathqlineto{120.0000bp}{92.0000bp}
    \pgfpathqlineto{128.0000bp}{92.0000bp}
    \pgfpathqlineto{136.0000bp}{92.0000bp}
    \pgfpathqlineto{144.0000bp}{92.0000bp}
    \pgfpathqlineto{152.0000bp}{92.0000bp}
    \pgfpathqlineto{160.0000bp}{92.0000bp}
    \pgfpathqlineto{168.0000bp}{92.0000bp}
    \pgfpathqlineto{176.0000bp}{92.0000bp}
    \pgfpathqlineto{184.0000bp}{100.0000bp}
    \pgfpathqlineto{192.0000bp}{100.0000bp}
    \pgfusepathqstroke
  \end{pgfscope}
  \begin{pgfscope}
    \pgfsetlinewidth{0.6788bp}
    \definecolor{sc}{rgb}{1.0000,0.0000,0.0000}
    \pgfsetstrokecolor{sc}
    \pgfsetmiterjoin
    \pgfsetbuttcap
    \pgfpathqmoveto{200.0000bp}{92.0000bp}
    \pgfpathqlineto{200.0000bp}{84.0000bp}
    \pgfusepathqstroke
  \end{pgfscope}
  \begin{pgfscope}
    \pgfsetlinewidth{0.6788bp}
    \definecolor{sc}{rgb}{1.0000,0.0000,0.0000}
    \pgfsetstrokecolor{sc}
    \pgfsetmiterjoin
    \pgfsetbuttcap
    \pgfpathqmoveto{192.0000bp}{92.0000bp}
    \pgfpathqlineto{192.0000bp}{84.0000bp}
    \pgfusepathqstroke
  \end{pgfscope}
  \begin{pgfscope}
    \pgfsetlinewidth{0.6788bp}
    \definecolor{sc}{rgb}{1.0000,0.0000,0.0000}
    \pgfsetstrokecolor{sc}
    \pgfsetmiterjoin
    \pgfsetbuttcap
    \pgfpathqmoveto{152.0000bp}{92.0000bp}
    \pgfpathqlineto{152.0000bp}{84.0000bp}
    \pgfusepathqstroke
  \end{pgfscope}
  \begin{pgfscope}
    \pgfsetlinewidth{0.6788bp}
    \definecolor{sc}{rgb}{1.0000,0.0000,0.0000}
    \pgfsetstrokecolor{sc}
    \pgfsetmiterjoin
    \pgfsetbuttcap
    \pgfpathqmoveto{16.0000bp}{92.0000bp}
    \pgfpathqlineto{16.0000bp}{84.0000bp}
    \pgfusepathqstroke
  \end{pgfscope}
  \begin{pgfscope}
    \pgfsetlinewidth{0.6788bp}
    \definecolor{sc}{rgb}{0.0000,0.0000,0.0000}
    \pgfsetstrokecolor{sc}
    \pgfsetmiterjoin
    \pgfsetbuttcap
    \pgfpathqmoveto{176.0000bp}{92.0000bp}
    \pgfpathqlineto{176.0000bp}{88.0000bp}
    \pgfusepathqstroke
  \end{pgfscope}
  \begin{pgfscope}
    \pgfsetlinewidth{0.6788bp}
    \definecolor{sc}{rgb}{0.0000,0.0000,0.0000}
    \pgfsetstrokecolor{sc}
    \pgfsetmiterjoin
    \pgfsetbuttcap
    \pgfpathqmoveto{136.0000bp}{92.0000bp}
    \pgfpathqlineto{136.0000bp}{88.0000bp}
    \pgfusepathqstroke
  \end{pgfscope}
  \begin{pgfscope}
    \pgfsetlinewidth{0.6788bp}
    \definecolor{sc}{rgb}{0.0000,0.0000,0.0000}
    \pgfsetstrokecolor{sc}
    \pgfsetmiterjoin
    \pgfsetbuttcap
    \pgfpathqmoveto{96.0000bp}{92.0000bp}
    \pgfpathqlineto{96.0000bp}{88.0000bp}
    \pgfusepathqstroke
  \end{pgfscope}
  \begin{pgfscope}
    \pgfsetlinewidth{0.6788bp}
    \definecolor{sc}{rgb}{0.0000,0.0000,0.0000}
    \pgfsetstrokecolor{sc}
    \pgfsetmiterjoin
    \pgfsetbuttcap
    \pgfpathqmoveto{56.0000bp}{92.0000bp}
    \pgfpathqlineto{56.0000bp}{88.0000bp}
    \pgfusepathqstroke
  \end{pgfscope}
  \begin{pgfscope}
    \definecolor{fc}{rgb}{0.0000,0.0000,0.0000}
    \pgfsetfillcolor{fc}
    \pgftransformshift{\pgfqpoint{0.0000bp}{169.6000bp}}
    \pgftransformscale{1.0000}
    \pgftext[base,left]{$\mathbb{L}_A$}
  \end{pgfscope}
  \begin{pgfscope}
    \pgfsetlinewidth{0.6788bp}
    \definecolor{sc}{rgb}{0.0000,0.0000,0.0000}
    \pgfsetstrokecolor{sc}
    \pgfsetmiterjoin
    \pgfsetbuttcap
    \pgfpathqmoveto{16.0000bp}{172.0000bp}
    \pgfpathqlineto{14.4000bp}{172.0000bp}
    \pgfusepathqstroke
  \end{pgfscope}
  \begin{pgfscope}
    \pgfsetlinewidth{0.6788bp}
    \definecolor{sc}{rgb}{0.0000,0.0000,0.0000}
    \pgfsetstrokecolor{sc}
    \pgfsetmiterjoin
    \pgfsetbuttcap
    \pgfpathqmoveto{16.0000bp}{92.0000bp}
    \pgfpathqlineto{16.0000bp}{172.0000bp}
    \pgfusepathqstroke
  \end{pgfscope}
  \begin{pgfscope}
    \pgfsetlinewidth{0.6788bp}
    \definecolor{sc}{rgb}{0.0000,0.0000,0.0000}
    \pgfsetstrokecolor{sc}
    \pgfsetmiterjoin
    \pgfsetbuttcap
    \pgfpathqmoveto{16.0000bp}{92.0000bp}
    \pgfpathqlineto{200.0000bp}{92.0000bp}
    \pgfusepathqstroke
  \end{pgfscope}
\end{pgfpicture}
}
        \caption{\texttt{move-v} effects}\label{fig:res:ekmovev}
    \end{subfigure}
    \caption{Knowledge obtained about \emph{effects} of the \texttt{move-*} actions from solving the problems presented in~\figref{fig:results:sokoTraining}. The unit of the $y$-axis is number of ungrounded predicates, and the marker $\mathbb{F}_A$ denotes the maximum value. The $x$-axis is measured in experiments. For readability, the experiments where the relevant action was not the cause of failure are removed, as nothing is learned. The small black ticks denote a duration of 5 experiments, and the larger red ticks denote that a problem has been solved, and that a new one has started.}\label{fig:res:ekmove}
\end{figure}

\begin{figure}
    \centering
    \begin{subfigure}{0.42\linewidth}
        \resizebox{\linewidth}{!}{\begin{pgfpicture}
  \pgfpathrectangle{\pgfpointorigin}{\pgfqpoint{200.0000bp}{200.0000bp}}
  \pgfusepath{use as bounding box}
  \begin{pgfscope}
    \definecolor{fc}{rgb}{0.0000,0.0000,0.0000}
    \pgfsetfillcolor{fc}
    \pgftransformshift{\pgfqpoint{28.3333bp}{27.9167bp}}
    \pgftransformscale{1.0417}
    \pgftext[base,left]{candidates}
  \end{pgfscope}
  \begin{pgfscope}
    \definecolor{fc}{rgb}{0.0000,0.0000,0.0000}
    \pgfsetfillcolor{fc}
    \pgfsetlinewidth{0.6831bp}
    \definecolor{sc}{rgb}{0.0000,0.0000,0.0000}
    \pgfsetstrokecolor{sc}
    \pgfsetmiterjoin
    \pgfsetbuttcap
    \pgfpathqmoveto{20.0000bp}{30.4167bp}
    \pgfpathqcurveto{20.0000bp}{32.2576bp}{18.5076bp}{33.7500bp}{16.6667bp}{33.7500bp}
    \pgfpathqcurveto{14.8257bp}{33.7500bp}{13.3333bp}{32.2576bp}{13.3333bp}{30.4167bp}
    \pgfpathqcurveto{13.3333bp}{28.5757bp}{14.8257bp}{27.0833bp}{16.6667bp}{27.0833bp}
    \pgfpathqcurveto{18.5076bp}{27.0833bp}{20.0000bp}{28.5757bp}{20.0000bp}{30.4167bp}
    \pgfpathclose
    \pgfusepathqfillstroke
  \end{pgfscope}
  \begin{pgfscope}
    \definecolor{fc}{rgb}{0.0000,0.0000,0.0000}
    \pgfsetfillcolor{fc}
    \pgftransformshift{\pgfqpoint{28.3333bp}{38.7500bp}}
    \pgftransformscale{1.0417}
    \pgftext[base,left]{negative unproven}
  \end{pgfscope}
  \begin{pgfscope}
    \definecolor{fc}{rgb}{1.0000,1.0000,0.0000}
    \pgfsetfillcolor{fc}
    \pgfsetlinewidth{0.6831bp}
    \definecolor{sc}{rgb}{1.0000,1.0000,0.0000}
    \pgfsetstrokecolor{sc}
    \pgfsetmiterjoin
    \pgfsetbuttcap
    \pgfpathqmoveto{20.0000bp}{41.2500bp}
    \pgfpathqcurveto{20.0000bp}{43.0909bp}{18.5076bp}{44.5833bp}{16.6667bp}{44.5833bp}
    \pgfpathqcurveto{14.8257bp}{44.5833bp}{13.3333bp}{43.0909bp}{13.3333bp}{41.2500bp}
    \pgfpathqcurveto{13.3333bp}{39.4091bp}{14.8257bp}{37.9167bp}{16.6667bp}{37.9167bp}
    \pgfpathqcurveto{18.5076bp}{37.9167bp}{20.0000bp}{39.4091bp}{20.0000bp}{41.2500bp}
    \pgfpathclose
    \pgfusepathqfillstroke
  \end{pgfscope}
  \begin{pgfscope}
    \definecolor{fc}{rgb}{0.0000,0.0000,0.0000}
    \pgfsetfillcolor{fc}
    \pgftransformshift{\pgfqpoint{28.3333bp}{49.5833bp}}
    \pgftransformscale{1.0417}
    \pgftext[base,left]{negative proven}
  \end{pgfscope}
  \begin{pgfscope}
    \definecolor{fc}{rgb}{0.0000,0.5020,0.0000}
    \pgfsetfillcolor{fc}
    \pgfsetlinewidth{0.6831bp}
    \definecolor{sc}{rgb}{0.0000,0.5020,0.0000}
    \pgfsetstrokecolor{sc}
    \pgfsetmiterjoin
    \pgfsetbuttcap
    \pgfpathqmoveto{20.0000bp}{52.0833bp}
    \pgfpathqcurveto{20.0000bp}{53.9243bp}{18.5076bp}{55.4167bp}{16.6667bp}{55.4167bp}
    \pgfpathqcurveto{14.8257bp}{55.4167bp}{13.3333bp}{53.9243bp}{13.3333bp}{52.0833bp}
    \pgfpathqcurveto{13.3333bp}{50.2424bp}{14.8257bp}{48.7500bp}{16.6667bp}{48.7500bp}
    \pgfpathqcurveto{18.5076bp}{48.7500bp}{20.0000bp}{50.2424bp}{20.0000bp}{52.0833bp}
    \pgfpathclose
    \pgfusepathqfillstroke
  \end{pgfscope}
  \begin{pgfscope}
    \definecolor{fc}{rgb}{0.0000,0.0000,0.0000}
    \pgfsetfillcolor{fc}
    \pgftransformshift{\pgfqpoint{28.3333bp}{60.4167bp}}
    \pgftransformscale{1.0417}
    \pgftext[base,left]{positive unproven}
  \end{pgfscope}
  \begin{pgfscope}
    \definecolor{fc}{rgb}{1.0000,0.0000,0.0000}
    \pgfsetfillcolor{fc}
    \pgfsetlinewidth{0.6831bp}
    \definecolor{sc}{rgb}{1.0000,0.0000,0.0000}
    \pgfsetstrokecolor{sc}
    \pgfsetmiterjoin
    \pgfsetbuttcap
    \pgfpathqmoveto{20.0000bp}{62.9167bp}
    \pgfpathqcurveto{20.0000bp}{64.7576bp}{18.5076bp}{66.2500bp}{16.6667bp}{66.2500bp}
    \pgfpathqcurveto{14.8257bp}{66.2500bp}{13.3333bp}{64.7576bp}{13.3333bp}{62.9167bp}
    \pgfpathqcurveto{13.3333bp}{61.0757bp}{14.8257bp}{59.5833bp}{16.6667bp}{59.5833bp}
    \pgfpathqcurveto{18.5076bp}{59.5833bp}{20.0000bp}{61.0757bp}{20.0000bp}{62.9167bp}
    \pgfpathclose
    \pgfusepathqfillstroke
  \end{pgfscope}
  \begin{pgfscope}
    \definecolor{fc}{rgb}{0.0000,0.0000,0.0000}
    \pgfsetfillcolor{fc}
    \pgftransformshift{\pgfqpoint{28.3333bp}{71.2500bp}}
    \pgftransformscale{1.0417}
    \pgftext[base,left]{positive proven}
  \end{pgfscope}
  \begin{pgfscope}
    \definecolor{fc}{rgb}{0.0000,0.0000,1.0000}
    \pgfsetfillcolor{fc}
    \pgfsetlinewidth{0.6831bp}
    \definecolor{sc}{rgb}{0.0000,0.0000,1.0000}
    \pgfsetstrokecolor{sc}
    \pgfsetmiterjoin
    \pgfsetbuttcap
    \pgfpathqmoveto{20.0000bp}{73.7500bp}
    \pgfpathqcurveto{20.0000bp}{75.5909bp}{18.5076bp}{77.0833bp}{16.6667bp}{77.0833bp}
    \pgfpathqcurveto{14.8257bp}{77.0833bp}{13.3333bp}{75.5909bp}{13.3333bp}{73.7500bp}
    \pgfpathqcurveto{13.3333bp}{71.9091bp}{14.8257bp}{70.4167bp}{16.6667bp}{70.4167bp}
    \pgfpathqcurveto{18.5076bp}{70.4167bp}{20.0000bp}{71.9091bp}{20.0000bp}{73.7500bp}
    \pgfpathclose
    \pgfusepathqfillstroke
  \end{pgfscope}
  \begin{pgfscope}
    \pgfsetlinewidth{0.6831bp}
    \definecolor{sc}{rgb}{0.0000,0.0000,0.0000}
    \pgfsetstrokecolor{sc}
    \pgfsetmiterjoin
    \pgfsetbuttcap
    \pgfpathqmoveto{25.0000bp}{93.7500bp}
    \pgfpathqlineto{33.3333bp}{97.9167bp}
    \pgfpathqlineto{41.6667bp}{102.0833bp}
    \pgfpathqlineto{50.0000bp}{106.2500bp}
    \pgfpathqlineto{58.3333bp}{110.4167bp}
    \pgfpathqlineto{66.6667bp}{114.5833bp}
    \pgfpathqlineto{75.0000bp}{118.7500bp}
    \pgfpathqlineto{83.3333bp}{122.9167bp}
    \pgfpathqlineto{91.6667bp}{127.0833bp}
    \pgfpathqlineto{100.0000bp}{131.2500bp}
    \pgfpathqlineto{108.3333bp}{135.4167bp}
    \pgfpathqlineto{116.6667bp}{139.5833bp}
    \pgfpathqlineto{125.0000bp}{143.7500bp}
    \pgfpathqlineto{133.3333bp}{147.9167bp}
    \pgfpathqlineto{141.6667bp}{152.0833bp}
    \pgfpathqlineto{150.0000bp}{156.2500bp}
    \pgfpathqlineto{158.3333bp}{160.4167bp}
    \pgfpathqlineto{166.6667bp}{102.0833bp}
    \pgfpathqlineto{175.0000bp}{97.9167bp}
    \pgfpathqlineto{183.3333bp}{97.9167bp}
    \pgfpathqlineto{191.6667bp}{102.0833bp}
    \pgfpathqlineto{200.0000bp}{102.0833bp}
    \pgfusepathqstroke
  \end{pgfscope}
  \begin{pgfscope}
    \pgfsetlinewidth{0.6831bp}
    \definecolor{sc}{rgb}{1.0000,1.0000,0.0000}
    \pgfsetstrokecolor{sc}
    \pgfsetmiterjoin
    \pgfsetbuttcap
    \pgfpathqmoveto{25.0000bp}{172.9167bp}
    \pgfpathqlineto{33.3333bp}{172.9167bp}
    \pgfpathqlineto{41.6667bp}{172.9167bp}
    \pgfpathqlineto{50.0000bp}{172.9167bp}
    \pgfpathqlineto{58.3333bp}{172.9167bp}
    \pgfpathqlineto{66.6667bp}{172.9167bp}
    \pgfpathqlineto{75.0000bp}{172.9167bp}
    \pgfpathqlineto{83.3333bp}{172.9167bp}
    \pgfpathqlineto{91.6667bp}{172.9167bp}
    \pgfpathqlineto{100.0000bp}{172.9167bp}
    \pgfpathqlineto{108.3333bp}{172.9167bp}
    \pgfpathqlineto{116.6667bp}{172.9167bp}
    \pgfpathqlineto{125.0000bp}{172.9167bp}
    \pgfpathqlineto{133.3333bp}{172.9167bp}
    \pgfpathqlineto{141.6667bp}{172.9167bp}
    \pgfpathqlineto{150.0000bp}{172.9167bp}
    \pgfpathqlineto{158.3333bp}{172.9167bp}
    \pgfpathqlineto{166.6667bp}{156.2500bp}
    \pgfpathqlineto{175.0000bp}{156.2500bp}
    \pgfpathqlineto{183.3333bp}{156.2500bp}
    \pgfpathqlineto{191.6667bp}{156.2500bp}
    \pgfpathqlineto{200.0000bp}{156.2500bp}
    \pgfusepathqstroke
  \end{pgfscope}
  \begin{pgfscope}
    \pgfsetlinewidth{0.6831bp}
    \definecolor{sc}{rgb}{0.0000,0.5020,0.0000}
    \pgfsetstrokecolor{sc}
    \pgfsetmiterjoin
    \pgfsetbuttcap
    \pgfpathqmoveto{25.0000bp}{89.5833bp}
    \pgfpathqlineto{33.3333bp}{89.5833bp}
    \pgfpathqlineto{41.6667bp}{89.5833bp}
    \pgfpathqlineto{50.0000bp}{89.5833bp}
    \pgfpathqlineto{58.3333bp}{89.5833bp}
    \pgfpathqlineto{66.6667bp}{89.5833bp}
    \pgfpathqlineto{75.0000bp}{89.5833bp}
    \pgfpathqlineto{83.3333bp}{89.5833bp}
    \pgfpathqlineto{91.6667bp}{89.5833bp}
    \pgfpathqlineto{100.0000bp}{89.5833bp}
    \pgfpathqlineto{108.3333bp}{89.5833bp}
    \pgfpathqlineto{116.6667bp}{89.5833bp}
    \pgfpathqlineto{125.0000bp}{89.5833bp}
    \pgfpathqlineto{133.3333bp}{89.5833bp}
    \pgfpathqlineto{141.6667bp}{89.5833bp}
    \pgfpathqlineto{150.0000bp}{89.5833bp}
    \pgfpathqlineto{158.3333bp}{89.5833bp}
    \pgfpathqlineto{166.6667bp}{89.5833bp}
    \pgfpathqlineto{175.0000bp}{89.5833bp}
    \pgfpathqlineto{183.3333bp}{89.5833bp}
    \pgfpathqlineto{191.6667bp}{89.5833bp}
    \pgfpathqlineto{200.0000bp}{89.5833bp}
    \pgfusepathqstroke
  \end{pgfscope}
  \begin{pgfscope}
    \pgfsetlinewidth{0.6831bp}
    \definecolor{sc}{rgb}{1.0000,0.0000,0.0000}
    \pgfsetstrokecolor{sc}
    \pgfsetmiterjoin
    \pgfsetbuttcap
    \pgfpathqmoveto{25.0000bp}{172.9167bp}
    \pgfpathqlineto{33.3333bp}{172.9167bp}
    \pgfpathqlineto{41.6667bp}{172.9167bp}
    \pgfpathqlineto{50.0000bp}{172.9167bp}
    \pgfpathqlineto{58.3333bp}{172.9167bp}
    \pgfpathqlineto{66.6667bp}{172.9167bp}
    \pgfpathqlineto{75.0000bp}{172.9167bp}
    \pgfpathqlineto{83.3333bp}{172.9167bp}
    \pgfpathqlineto{91.6667bp}{172.9167bp}
    \pgfpathqlineto{100.0000bp}{172.9167bp}
    \pgfpathqlineto{108.3333bp}{172.9167bp}
    \pgfpathqlineto{116.6667bp}{172.9167bp}
    \pgfpathqlineto{125.0000bp}{172.9167bp}
    \pgfpathqlineto{133.3333bp}{172.9167bp}
    \pgfpathqlineto{141.6667bp}{172.9167bp}
    \pgfpathqlineto{150.0000bp}{172.9167bp}
    \pgfpathqlineto{158.3333bp}{172.9167bp}
    \pgfpathqlineto{166.6667bp}{102.0833bp}
    \pgfpathqlineto{175.0000bp}{97.9167bp}
    \pgfpathqlineto{183.3333bp}{97.9167bp}
    \pgfpathqlineto{191.6667bp}{97.9167bp}
    \pgfpathqlineto{200.0000bp}{97.9167bp}
    \pgfusepathqstroke
  \end{pgfscope}
  \begin{pgfscope}
    \pgfsetlinewidth{0.6831bp}
    \definecolor{sc}{rgb}{0.0000,0.0000,1.0000}
    \pgfsetstrokecolor{sc}
    \pgfsetmiterjoin
    \pgfsetbuttcap
    \pgfpathqmoveto{25.0000bp}{89.5833bp}
    \pgfpathqlineto{33.3333bp}{89.5833bp}
    \pgfpathqlineto{41.6667bp}{89.5833bp}
    \pgfpathqlineto{50.0000bp}{89.5833bp}
    \pgfpathqlineto{58.3333bp}{89.5833bp}
    \pgfpathqlineto{66.6667bp}{89.5833bp}
    \pgfpathqlineto{75.0000bp}{89.5833bp}
    \pgfpathqlineto{83.3333bp}{89.5833bp}
    \pgfpathqlineto{91.6667bp}{89.5833bp}
    \pgfpathqlineto{100.0000bp}{89.5833bp}
    \pgfpathqlineto{108.3333bp}{89.5833bp}
    \pgfpathqlineto{116.6667bp}{89.5833bp}
    \pgfpathqlineto{125.0000bp}{89.5833bp}
    \pgfpathqlineto{133.3333bp}{89.5833bp}
    \pgfpathqlineto{141.6667bp}{89.5833bp}
    \pgfpathqlineto{150.0000bp}{89.5833bp}
    \pgfpathqlineto{158.3333bp}{89.5833bp}
    \pgfpathqlineto{166.6667bp}{93.7500bp}
    \pgfpathqlineto{175.0000bp}{97.9167bp}
    \pgfpathqlineto{183.3333bp}{97.9167bp}
    \pgfpathqlineto{191.6667bp}{97.9167bp}
    \pgfpathqlineto{200.0000bp}{97.9167bp}
    \pgfusepathqstroke
  \end{pgfscope}
  \begin{pgfscope}
    \pgfsetlinewidth{0.6831bp}
    \definecolor{sc}{rgb}{1.0000,0.0000,0.0000}
    \pgfsetstrokecolor{sc}
    \pgfsetmiterjoin
    \pgfsetbuttcap
    \pgfpathqmoveto{50.0000bp}{89.5833bp}
    \pgfpathqlineto{50.0000bp}{85.4167bp}
    \pgfusepathqstroke
  \end{pgfscope}
  \begin{pgfscope}
    \pgfsetlinewidth{0.6831bp}
    \definecolor{sc}{rgb}{1.0000,0.0000,0.0000}
    \pgfsetstrokecolor{sc}
    \pgfsetmiterjoin
    \pgfsetbuttcap
    \pgfpathqmoveto{166.6667bp}{89.5833bp}
    \pgfpathqlineto{166.6667bp}{85.4167bp}
    \pgfusepathqstroke
  \end{pgfscope}
  \begin{pgfscope}
    \pgfsetlinewidth{0.6831bp}
    \definecolor{sc}{rgb}{0.0000,0.0000,0.0000}
    \pgfsetstrokecolor{sc}
    \pgfsetmiterjoin
    \pgfsetbuttcap
    \pgfpathqmoveto{183.3333bp}{89.5833bp}
    \pgfpathqlineto{183.3333bp}{85.4167bp}
    \pgfusepathqstroke
  \end{pgfscope}
  \begin{pgfscope}
    \pgfsetlinewidth{0.6831bp}
    \definecolor{sc}{rgb}{0.0000,0.0000,0.0000}
    \pgfsetstrokecolor{sc}
    \pgfsetmiterjoin
    \pgfsetbuttcap
    \pgfpathqmoveto{141.6667bp}{89.5833bp}
    \pgfpathqlineto{141.6667bp}{85.4167bp}
    \pgfusepathqstroke
  \end{pgfscope}
  \begin{pgfscope}
    \pgfsetlinewidth{0.6831bp}
    \definecolor{sc}{rgb}{0.0000,0.0000,0.0000}
    \pgfsetstrokecolor{sc}
    \pgfsetmiterjoin
    \pgfsetbuttcap
    \pgfpathqmoveto{100.0000bp}{89.5833bp}
    \pgfpathqlineto{100.0000bp}{85.4167bp}
    \pgfusepathqstroke
  \end{pgfscope}
  \begin{pgfscope}
    \pgfsetlinewidth{0.6831bp}
    \definecolor{sc}{rgb}{0.0000,0.0000,0.0000}
    \pgfsetstrokecolor{sc}
    \pgfsetmiterjoin
    \pgfsetbuttcap
    \pgfpathqmoveto{58.3333bp}{89.5833bp}
    \pgfpathqlineto{58.3333bp}{85.4167bp}
    \pgfusepathqstroke
  \end{pgfscope}
  \begin{pgfscope}
    \definecolor{fc}{rgb}{0.0000,0.0000,0.0000}
    \pgfsetfillcolor{fc}
    \pgftransformshift{\pgfqpoint{-0.0000bp}{170.4167bp}}
    \pgftransformscale{1.0417}
    \pgftext[base,left]{$\mathbb{F}_A$}
  \end{pgfscope}
  \begin{pgfscope}
    \pgfsetlinewidth{0.6831bp}
    \definecolor{sc}{rgb}{0.0000,0.0000,0.0000}
    \pgfsetstrokecolor{sc}
    \pgfsetmiterjoin
    \pgfsetbuttcap
    \pgfpathqmoveto{16.6667bp}{172.9167bp}
    \pgfpathqlineto{15.0000bp}{172.9167bp}
    \pgfusepathqstroke
  \end{pgfscope}
  \begin{pgfscope}
    \pgfsetlinewidth{0.6831bp}
    \definecolor{sc}{rgb}{0.0000,0.0000,0.0000}
    \pgfsetstrokecolor{sc}
    \pgfsetmiterjoin
    \pgfsetbuttcap
    \pgfpathqmoveto{16.6667bp}{89.5833bp}
    \pgfpathqlineto{16.6667bp}{172.9167bp}
    \pgfusepathqstroke
  \end{pgfscope}
  \begin{pgfscope}
    \pgfsetlinewidth{0.6831bp}
    \definecolor{sc}{rgb}{0.0000,0.0000,0.0000}
    \pgfsetstrokecolor{sc}
    \pgfsetmiterjoin
    \pgfsetbuttcap
    \pgfpathqmoveto{16.6667bp}{89.5833bp}
    \pgfpathqlineto{200.0000bp}{89.5833bp}
    \pgfusepathqstroke
  \end{pgfscope}
\end{pgfpicture}
}
        \caption{\texttt{move-h} preconditions}\label{fig:res:pkmoveh}
    \end{subfigure}
    \begin{subfigure}{0.42\linewidth}
        \resizebox{\linewidth}{!}{\begin{pgfpicture}
  \pgfpathrectangle{\pgfpointorigin}{\pgfqpoint{200.0000bp}{200.0000bp}}
  \pgfusepath{use as bounding box}
  \begin{pgfscope}
    \definecolor{fc}{rgb}{0.0000,0.0000,0.0000}
    \pgfsetfillcolor{fc}
    \pgftransformshift{\pgfqpoint{27.2000bp}{28.8000bp}}
    \pgftransformscale{1.0000}
    \pgftext[base,left]{candidates}
  \end{pgfscope}
  \begin{pgfscope}
    \definecolor{fc}{rgb}{0.0000,0.0000,0.0000}
    \pgfsetfillcolor{fc}
    \pgfsetlinewidth{0.6788bp}
    \definecolor{sc}{rgb}{0.0000,0.0000,0.0000}
    \pgfsetstrokecolor{sc}
    \pgfsetmiterjoin
    \pgfsetbuttcap
    \pgfpathqmoveto{19.2000bp}{31.2000bp}
    \pgfpathqcurveto{19.2000bp}{32.9673bp}{17.7673bp}{34.4000bp}{16.0000bp}{34.4000bp}
    \pgfpathqcurveto{14.2327bp}{34.4000bp}{12.8000bp}{32.9673bp}{12.8000bp}{31.2000bp}
    \pgfpathqcurveto{12.8000bp}{29.4327bp}{14.2327bp}{28.0000bp}{16.0000bp}{28.0000bp}
    \pgfpathqcurveto{17.7673bp}{28.0000bp}{19.2000bp}{29.4327bp}{19.2000bp}{31.2000bp}
    \pgfpathclose
    \pgfusepathqfillstroke
  \end{pgfscope}
  \begin{pgfscope}
    \definecolor{fc}{rgb}{0.0000,0.0000,0.0000}
    \pgfsetfillcolor{fc}
    \pgftransformshift{\pgfqpoint{27.2000bp}{39.2000bp}}
    \pgftransformscale{1.0000}
    \pgftext[base,left]{negative unproven}
  \end{pgfscope}
  \begin{pgfscope}
    \definecolor{fc}{rgb}{1.0000,1.0000,0.0000}
    \pgfsetfillcolor{fc}
    \pgfsetlinewidth{0.6788bp}
    \definecolor{sc}{rgb}{1.0000,1.0000,0.0000}
    \pgfsetstrokecolor{sc}
    \pgfsetmiterjoin
    \pgfsetbuttcap
    \pgfpathqmoveto{19.2000bp}{41.6000bp}
    \pgfpathqcurveto{19.2000bp}{43.3673bp}{17.7673bp}{44.8000bp}{16.0000bp}{44.8000bp}
    \pgfpathqcurveto{14.2327bp}{44.8000bp}{12.8000bp}{43.3673bp}{12.8000bp}{41.6000bp}
    \pgfpathqcurveto{12.8000bp}{39.8327bp}{14.2327bp}{38.4000bp}{16.0000bp}{38.4000bp}
    \pgfpathqcurveto{17.7673bp}{38.4000bp}{19.2000bp}{39.8327bp}{19.2000bp}{41.6000bp}
    \pgfpathclose
    \pgfusepathqfillstroke
  \end{pgfscope}
  \begin{pgfscope}
    \definecolor{fc}{rgb}{0.0000,0.0000,0.0000}
    \pgfsetfillcolor{fc}
    \pgftransformshift{\pgfqpoint{27.2000bp}{49.6000bp}}
    \pgftransformscale{1.0000}
    \pgftext[base,left]{negative proven}
  \end{pgfscope}
  \begin{pgfscope}
    \definecolor{fc}{rgb}{0.0000,0.5020,0.0000}
    \pgfsetfillcolor{fc}
    \pgfsetlinewidth{0.6788bp}
    \definecolor{sc}{rgb}{0.0000,0.5020,0.0000}
    \pgfsetstrokecolor{sc}
    \pgfsetmiterjoin
    \pgfsetbuttcap
    \pgfpathqmoveto{19.2000bp}{52.0000bp}
    \pgfpathqcurveto{19.2000bp}{53.7673bp}{17.7673bp}{55.2000bp}{16.0000bp}{55.2000bp}
    \pgfpathqcurveto{14.2327bp}{55.2000bp}{12.8000bp}{53.7673bp}{12.8000bp}{52.0000bp}
    \pgfpathqcurveto{12.8000bp}{50.2327bp}{14.2327bp}{48.8000bp}{16.0000bp}{48.8000bp}
    \pgfpathqcurveto{17.7673bp}{48.8000bp}{19.2000bp}{50.2327bp}{19.2000bp}{52.0000bp}
    \pgfpathclose
    \pgfusepathqfillstroke
  \end{pgfscope}
  \begin{pgfscope}
    \definecolor{fc}{rgb}{0.0000,0.0000,0.0000}
    \pgfsetfillcolor{fc}
    \pgftransformshift{\pgfqpoint{27.2000bp}{60.0000bp}}
    \pgftransformscale{1.0000}
    \pgftext[base,left]{positive unproven}
  \end{pgfscope}
  \begin{pgfscope}
    \definecolor{fc}{rgb}{1.0000,0.0000,0.0000}
    \pgfsetfillcolor{fc}
    \pgfsetlinewidth{0.6788bp}
    \definecolor{sc}{rgb}{1.0000,0.0000,0.0000}
    \pgfsetstrokecolor{sc}
    \pgfsetmiterjoin
    \pgfsetbuttcap
    \pgfpathqmoveto{19.2000bp}{62.4000bp}
    \pgfpathqcurveto{19.2000bp}{64.1673bp}{17.7673bp}{65.6000bp}{16.0000bp}{65.6000bp}
    \pgfpathqcurveto{14.2327bp}{65.6000bp}{12.8000bp}{64.1673bp}{12.8000bp}{62.4000bp}
    \pgfpathqcurveto{12.8000bp}{60.6327bp}{14.2327bp}{59.2000bp}{16.0000bp}{59.2000bp}
    \pgfpathqcurveto{17.7673bp}{59.2000bp}{19.2000bp}{60.6327bp}{19.2000bp}{62.4000bp}
    \pgfpathclose
    \pgfusepathqfillstroke
  \end{pgfscope}
  \begin{pgfscope}
    \definecolor{fc}{rgb}{0.0000,0.0000,0.0000}
    \pgfsetfillcolor{fc}
    \pgftransformshift{\pgfqpoint{27.2000bp}{70.4000bp}}
    \pgftransformscale{1.0000}
    \pgftext[base,left]{positive proven}
  \end{pgfscope}
  \begin{pgfscope}
    \definecolor{fc}{rgb}{0.0000,0.0000,1.0000}
    \pgfsetfillcolor{fc}
    \pgfsetlinewidth{0.6788bp}
    \definecolor{sc}{rgb}{0.0000,0.0000,1.0000}
    \pgfsetstrokecolor{sc}
    \pgfsetmiterjoin
    \pgfsetbuttcap
    \pgfpathqmoveto{19.2000bp}{72.8000bp}
    \pgfpathqcurveto{19.2000bp}{74.5673bp}{17.7673bp}{76.0000bp}{16.0000bp}{76.0000bp}
    \pgfpathqcurveto{14.2327bp}{76.0000bp}{12.8000bp}{74.5673bp}{12.8000bp}{72.8000bp}
    \pgfpathqcurveto{12.8000bp}{71.0327bp}{14.2327bp}{69.6000bp}{16.0000bp}{69.6000bp}
    \pgfpathqcurveto{17.7673bp}{69.6000bp}{19.2000bp}{71.0327bp}{19.2000bp}{72.8000bp}
    \pgfpathclose
    \pgfusepathqfillstroke
  \end{pgfscope}
  \begin{pgfscope}
    \pgfsetlinewidth{0.6788bp}
    \definecolor{sc}{rgb}{0.0000,0.0000,0.0000}
    \pgfsetstrokecolor{sc}
    \pgfsetmiterjoin
    \pgfsetbuttcap
    \pgfpathqmoveto{16.0000bp}{92.0000bp}
    \pgfpathqlineto{24.0000bp}{96.0000bp}
    \pgfpathqlineto{32.0000bp}{100.0000bp}
    \pgfpathqlineto{40.0000bp}{104.0000bp}
    \pgfpathqlineto{48.0000bp}{108.0000bp}
    \pgfpathqlineto{56.0000bp}{112.0000bp}
    \pgfpathqlineto{64.0000bp}{116.0000bp}
    \pgfpathqlineto{72.0000bp}{120.0000bp}
    \pgfpathqlineto{80.0000bp}{124.0000bp}
    \pgfpathqlineto{88.0000bp}{128.0000bp}
    \pgfpathqlineto{96.0000bp}{132.0000bp}
    \pgfpathqlineto{104.0000bp}{136.0000bp}
    \pgfpathqlineto{112.0000bp}{140.0000bp}
    \pgfpathqlineto{120.0000bp}{144.0000bp}
    \pgfpathqlineto{128.0000bp}{148.0000bp}
    \pgfpathqlineto{136.0000bp}{152.0000bp}
    \pgfpathqlineto{144.0000bp}{156.0000bp}
    \pgfpathqlineto{152.0000bp}{160.0000bp}
    \pgfpathqlineto{160.0000bp}{164.0000bp}
    \pgfpathqlineto{168.0000bp}{168.0000bp}
    \pgfpathqlineto{176.0000bp}{172.0000bp}
    \pgfpathqlineto{184.0000bp}{100.0000bp}
    \pgfpathqlineto{192.0000bp}{96.0000bp}
    \pgfusepathqstroke
  \end{pgfscope}
  \begin{pgfscope}
    \pgfsetlinewidth{0.6788bp}
    \definecolor{sc}{rgb}{1.0000,1.0000,0.0000}
    \pgfsetstrokecolor{sc}
    \pgfsetmiterjoin
    \pgfsetbuttcap
    \pgfpathqmoveto{16.0000bp}{168.0000bp}
    \pgfpathqlineto{24.0000bp}{168.0000bp}
    \pgfpathqlineto{32.0000bp}{168.0000bp}
    \pgfpathqlineto{40.0000bp}{168.0000bp}
    \pgfpathqlineto{48.0000bp}{168.0000bp}
    \pgfpathqlineto{56.0000bp}{168.0000bp}
    \pgfpathqlineto{64.0000bp}{168.0000bp}
    \pgfpathqlineto{72.0000bp}{168.0000bp}
    \pgfpathqlineto{80.0000bp}{168.0000bp}
    \pgfpathqlineto{88.0000bp}{168.0000bp}
    \pgfpathqlineto{96.0000bp}{168.0000bp}
    \pgfpathqlineto{104.0000bp}{168.0000bp}
    \pgfpathqlineto{112.0000bp}{168.0000bp}
    \pgfpathqlineto{120.0000bp}{168.0000bp}
    \pgfpathqlineto{128.0000bp}{168.0000bp}
    \pgfpathqlineto{136.0000bp}{168.0000bp}
    \pgfpathqlineto{144.0000bp}{168.0000bp}
    \pgfpathqlineto{152.0000bp}{168.0000bp}
    \pgfpathqlineto{160.0000bp}{168.0000bp}
    \pgfpathqlineto{168.0000bp}{168.0000bp}
    \pgfpathqlineto{176.0000bp}{168.0000bp}
    \pgfpathqlineto{184.0000bp}{152.0000bp}
    \pgfpathqlineto{192.0000bp}{152.0000bp}
    \pgfusepathqstroke
  \end{pgfscope}
  \begin{pgfscope}
    \pgfsetlinewidth{0.6788bp}
    \definecolor{sc}{rgb}{0.0000,0.5020,0.0000}
    \pgfsetstrokecolor{sc}
    \pgfsetmiterjoin
    \pgfsetbuttcap
    \pgfpathqmoveto{16.0000bp}{88.0000bp}
    \pgfpathqlineto{24.0000bp}{88.0000bp}
    \pgfpathqlineto{32.0000bp}{88.0000bp}
    \pgfpathqlineto{40.0000bp}{88.0000bp}
    \pgfpathqlineto{48.0000bp}{88.0000bp}
    \pgfpathqlineto{56.0000bp}{88.0000bp}
    \pgfpathqlineto{64.0000bp}{88.0000bp}
    \pgfpathqlineto{72.0000bp}{88.0000bp}
    \pgfpathqlineto{80.0000bp}{88.0000bp}
    \pgfpathqlineto{88.0000bp}{88.0000bp}
    \pgfpathqlineto{96.0000bp}{88.0000bp}
    \pgfpathqlineto{104.0000bp}{88.0000bp}
    \pgfpathqlineto{112.0000bp}{88.0000bp}
    \pgfpathqlineto{120.0000bp}{88.0000bp}
    \pgfpathqlineto{128.0000bp}{88.0000bp}
    \pgfpathqlineto{136.0000bp}{88.0000bp}
    \pgfpathqlineto{144.0000bp}{88.0000bp}
    \pgfpathqlineto{152.0000bp}{88.0000bp}
    \pgfpathqlineto{160.0000bp}{88.0000bp}
    \pgfpathqlineto{168.0000bp}{88.0000bp}
    \pgfpathqlineto{176.0000bp}{88.0000bp}
    \pgfpathqlineto{184.0000bp}{88.0000bp}
    \pgfpathqlineto{192.0000bp}{88.0000bp}
    \pgfusepathqstroke
  \end{pgfscope}
  \begin{pgfscope}
    \pgfsetlinewidth{0.6788bp}
    \definecolor{sc}{rgb}{1.0000,0.0000,0.0000}
    \pgfsetstrokecolor{sc}
    \pgfsetmiterjoin
    \pgfsetbuttcap
    \pgfpathqmoveto{16.0000bp}{168.0000bp}
    \pgfpathqlineto{24.0000bp}{168.0000bp}
    \pgfpathqlineto{32.0000bp}{168.0000bp}
    \pgfpathqlineto{40.0000bp}{168.0000bp}
    \pgfpathqlineto{48.0000bp}{168.0000bp}
    \pgfpathqlineto{56.0000bp}{168.0000bp}
    \pgfpathqlineto{64.0000bp}{168.0000bp}
    \pgfpathqlineto{72.0000bp}{168.0000bp}
    \pgfpathqlineto{80.0000bp}{168.0000bp}
    \pgfpathqlineto{88.0000bp}{168.0000bp}
    \pgfpathqlineto{96.0000bp}{168.0000bp}
    \pgfpathqlineto{104.0000bp}{168.0000bp}
    \pgfpathqlineto{112.0000bp}{168.0000bp}
    \pgfpathqlineto{120.0000bp}{168.0000bp}
    \pgfpathqlineto{128.0000bp}{168.0000bp}
    \pgfpathqlineto{136.0000bp}{168.0000bp}
    \pgfpathqlineto{144.0000bp}{168.0000bp}
    \pgfpathqlineto{152.0000bp}{168.0000bp}
    \pgfpathqlineto{160.0000bp}{168.0000bp}
    \pgfpathqlineto{168.0000bp}{168.0000bp}
    \pgfpathqlineto{176.0000bp}{168.0000bp}
    \pgfpathqlineto{184.0000bp}{100.0000bp}
    \pgfpathqlineto{192.0000bp}{96.0000bp}
    \pgfusepathqstroke
  \end{pgfscope}
  \begin{pgfscope}
    \pgfsetlinewidth{0.6788bp}
    \definecolor{sc}{rgb}{0.0000,0.0000,1.0000}
    \pgfsetstrokecolor{sc}
    \pgfsetmiterjoin
    \pgfsetbuttcap
    \pgfpathqmoveto{16.0000bp}{88.0000bp}
    \pgfpathqlineto{24.0000bp}{88.0000bp}
    \pgfpathqlineto{32.0000bp}{88.0000bp}
    \pgfpathqlineto{40.0000bp}{88.0000bp}
    \pgfpathqlineto{48.0000bp}{88.0000bp}
    \pgfpathqlineto{56.0000bp}{88.0000bp}
    \pgfpathqlineto{64.0000bp}{88.0000bp}
    \pgfpathqlineto{72.0000bp}{88.0000bp}
    \pgfpathqlineto{80.0000bp}{88.0000bp}
    \pgfpathqlineto{88.0000bp}{88.0000bp}
    \pgfpathqlineto{96.0000bp}{88.0000bp}
    \pgfpathqlineto{104.0000bp}{88.0000bp}
    \pgfpathqlineto{112.0000bp}{88.0000bp}
    \pgfpathqlineto{120.0000bp}{88.0000bp}
    \pgfpathqlineto{128.0000bp}{88.0000bp}
    \pgfpathqlineto{136.0000bp}{88.0000bp}
    \pgfpathqlineto{144.0000bp}{88.0000bp}
    \pgfpathqlineto{152.0000bp}{88.0000bp}
    \pgfpathqlineto{160.0000bp}{88.0000bp}
    \pgfpathqlineto{168.0000bp}{88.0000bp}
    \pgfpathqlineto{176.0000bp}{88.0000bp}
    \pgfpathqlineto{184.0000bp}{92.0000bp}
    \pgfpathqlineto{192.0000bp}{96.0000bp}
    \pgfusepathqstroke
  \end{pgfscope}
  \begin{pgfscope}
    \pgfsetlinewidth{0.6788bp}
    \definecolor{sc}{rgb}{1.0000,0.0000,0.0000}
    \pgfsetstrokecolor{sc}
    \pgfsetmiterjoin
    \pgfsetbuttcap
    \pgfpathqmoveto{200.0000bp}{88.0000bp}
    \pgfpathqlineto{200.0000bp}{84.0000bp}
    \pgfusepathqstroke
  \end{pgfscope}
  \begin{pgfscope}
    \pgfsetlinewidth{0.6788bp}
    \definecolor{sc}{rgb}{1.0000,0.0000,0.0000}
    \pgfsetstrokecolor{sc}
    \pgfsetmiterjoin
    \pgfsetbuttcap
    \pgfpathqmoveto{192.0000bp}{88.0000bp}
    \pgfpathqlineto{192.0000bp}{84.0000bp}
    \pgfusepathqstroke
  \end{pgfscope}
  \begin{pgfscope}
    \pgfsetlinewidth{0.6788bp}
    \definecolor{sc}{rgb}{1.0000,0.0000,0.0000}
    \pgfsetstrokecolor{sc}
    \pgfsetmiterjoin
    \pgfsetbuttcap
    \pgfpathqmoveto{152.0000bp}{88.0000bp}
    \pgfpathqlineto{152.0000bp}{84.0000bp}
    \pgfusepathqstroke
  \end{pgfscope}
  \begin{pgfscope}
    \pgfsetlinewidth{0.6788bp}
    \definecolor{sc}{rgb}{1.0000,0.0000,0.0000}
    \pgfsetstrokecolor{sc}
    \pgfsetmiterjoin
    \pgfsetbuttcap
    \pgfpathqmoveto{16.0000bp}{88.0000bp}
    \pgfpathqlineto{16.0000bp}{84.0000bp}
    \pgfusepathqstroke
  \end{pgfscope}
  \begin{pgfscope}
    \pgfsetlinewidth{0.6788bp}
    \definecolor{sc}{rgb}{0.0000,0.0000,0.0000}
    \pgfsetstrokecolor{sc}
    \pgfsetmiterjoin
    \pgfsetbuttcap
    \pgfpathqmoveto{176.0000bp}{88.0000bp}
    \pgfpathqlineto{176.0000bp}{84.0000bp}
    \pgfusepathqstroke
  \end{pgfscope}
  \begin{pgfscope}
    \pgfsetlinewidth{0.6788bp}
    \definecolor{sc}{rgb}{0.0000,0.0000,0.0000}
    \pgfsetstrokecolor{sc}
    \pgfsetmiterjoin
    \pgfsetbuttcap
    \pgfpathqmoveto{136.0000bp}{88.0000bp}
    \pgfpathqlineto{136.0000bp}{84.0000bp}
    \pgfusepathqstroke
  \end{pgfscope}
  \begin{pgfscope}
    \pgfsetlinewidth{0.6788bp}
    \definecolor{sc}{rgb}{0.0000,0.0000,0.0000}
    \pgfsetstrokecolor{sc}
    \pgfsetmiterjoin
    \pgfsetbuttcap
    \pgfpathqmoveto{96.0000bp}{88.0000bp}
    \pgfpathqlineto{96.0000bp}{84.0000bp}
    \pgfusepathqstroke
  \end{pgfscope}
  \begin{pgfscope}
    \pgfsetlinewidth{0.6788bp}
    \definecolor{sc}{rgb}{0.0000,0.0000,0.0000}
    \pgfsetstrokecolor{sc}
    \pgfsetmiterjoin
    \pgfsetbuttcap
    \pgfpathqmoveto{56.0000bp}{88.0000bp}
    \pgfpathqlineto{56.0000bp}{84.0000bp}
    \pgfusepathqstroke
  \end{pgfscope}
  \begin{pgfscope}
    \definecolor{fc}{rgb}{0.0000,0.0000,0.0000}
    \pgfsetfillcolor{fc}
    \pgftransformshift{\pgfqpoint{0.0000bp}{165.6000bp}}
    \pgftransformscale{1.0000}
    \pgftext[base,left]{$\mathbb{L}_A$}
  \end{pgfscope}
  \begin{pgfscope}
    \pgfsetlinewidth{0.6788bp}
    \definecolor{sc}{rgb}{0.0000,0.0000,0.0000}
    \pgfsetstrokecolor{sc}
    \pgfsetmiterjoin
    \pgfsetbuttcap
    \pgfpathqmoveto{16.0000bp}{168.0000bp}
    \pgfpathqlineto{14.4000bp}{168.0000bp}
    \pgfusepathqstroke
  \end{pgfscope}
  \begin{pgfscope}
    \pgfsetlinewidth{0.6788bp}
    \definecolor{sc}{rgb}{0.0000,0.0000,0.0000}
    \pgfsetstrokecolor{sc}
    \pgfsetmiterjoin
    \pgfsetbuttcap
    \pgfpathqmoveto{16.0000bp}{88.0000bp}
    \pgfpathqlineto{16.0000bp}{172.0000bp}
    \pgfusepathqstroke
  \end{pgfscope}
  \begin{pgfscope}
    \pgfsetlinewidth{0.6788bp}
    \definecolor{sc}{rgb}{0.0000,0.0000,0.0000}
    \pgfsetstrokecolor{sc}
    \pgfsetmiterjoin
    \pgfsetbuttcap
    \pgfpathqmoveto{16.0000bp}{88.0000bp}
    \pgfpathqlineto{200.0000bp}{88.0000bp}
    \pgfusepathqstroke
  \end{pgfscope}
\end{pgfpicture}
}
        \caption{\texttt{move-v} preconditions}\label{fig:res:pkmovev}
    \end{subfigure}
    \caption{Knowledge obtained about \emph{preconditions} of the \texttt{move-*} actions from solving the problems presented in~\figref{fig:results:sokoTraining}. See~\figref{fig:res:ekmove} for an explanation of how to read the graph. Note that the candidate data is not measured in ungrounded predicates, but set cardinality.}\label{fig:res:pkmove}
\end{figure}

\end{document}
