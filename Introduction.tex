\documentclass[Master.tex]{subfiles}
\begin{document}

	\begin{itemize}
		\item Introduce problem of learning (motivation)
		\item Context: Walsh (explain optimistic and pessimistic algos), Conditional effects (abstract introduction)
		\item Problem restatement: Sokoban
		\item Result restatement: Conditional effects, better precondition learning (negative preconditions), more general algorithm
		\begin{itemize}
			\item Limits: No disjunctive preconds, (no parameters)
		\end{itemize}
		\item Road map
	\end{itemize}

	Today most study within the field of agent learning, revolves around
	machine learning and statistical analysis. While that is a good approach
	to learning in a partially observable environment, it does not give
	us an insight into purely logical inductive learning. In this paper
	we will attempt to explain learning from a strictly logic based standpoint,
	to provide an insight of how learning is achieved and what it means
	for an agent to learn something.

	To understand the concept of learning for an agent we must understand
	what it means to learn. Learning as defined by the oxford dictionary
	is \emph{The acquisition of knowledge or skills through study, experience
		or being taught}. In the human world the first skills an infant learns
	is how to manipulate the enviroment through its motor functions, in
	a STRIPS world these motor functions would be the actions which is
	available to the agent. Understanding SRIPS actions therefore makes
	the perfect candidate as a first step to achieve an understanding
	of a learning agent.

	When you want to learn anything, you must put yourself in a situation
	in which what you want to learn becomes obvious. For instance assume
	you aquired a black box with two buttons on it, to learn what the
	black box does you must first press one of the buttons, this is what
	is meant by putting yourself in the situation. Once you have put yourself
	in that situation then the next step is to analyse the outcome. Analysing
	the outcome means to gain an understanding of how you affected the
	world, in inductive learning this is known as hypothesis construction.

	In this paper we will try to explain the dynamic between these two
	steps however we will focus on hypothesis construction, specifically
	we will show how it can be achieved where actions have conditional
	effects.
%%%%%%%%%%%%%%%%%%%%%%%%%
% Overview
%%%%%%%%%%%%%%%%%%%%%%%%%

% Introduction:
% 	- Introduce problem of learning (motivation)
% 	- Context: Walsh (explain optimistic and pessimistic algos), Conditional effects (abstract introduction)
% 	- Problem restatement: Sokoban
% 	- Result restatement: Conditional effects, better precondition learning (negative preconditions), more general algorithm
% 		* Limits: No disjunctive preconds, (no parameters)
% 	- Road map
%
% Introduction to Learning
% 	- Explain scientific learning, how to learn scientifically?
% 		* Strategy
% 	- Algorithm
% 	- Strategy examples
% 	- framework for worlds (pddl)
% 	- Sokoban as pddl
% 	- Algorithm for pddl
%
% Nonconditional(Strips-style) Learning
% 	- Non-conditional actions in pddl
% 	- Sokoban example
% 	- Effect Learning
% 	- Precondition Learning
% 	- Hypothesis construction
% 		* How to: optimistic algorithm
% 		* How to: pessimistic algorithm
%
% Conditional effects learning
% 	- What is conditional effects
% 	- Difficulities learning conditional effects
% 	- Explain our models
% 	- Hyper graph construction from state
% 	- Hyper graph merging algorithm
% 	- Hypothesis construction
% 		* Pessimistic strategy
% 		* Problems with using optimistic strategy
% 		* Algorithm for hypothesis from hyper graph
%
% Methods (Implementation)
% 	- Parsing domains
% 	- Problems with fast-downward
% 		* Preprocessing (SAS+)
% 		* A* implementation
% 	- bounded planner optimization
% 	- Nonconditional optimistic
% 	- Conditional pessimistic
%
% Results
%
% Discussion
% 	- Comparison to Walsh
% 	- Limitations
% 	- Expressive power
%
% Conclusion
% 	- more general algorithm
% 	- Conditional effects

\end{document}
